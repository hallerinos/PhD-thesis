%!TEX root = thesis.tex
%%%%%%%%%%%%%%%%%%%%%%%
%
%
%%%%%%%%%%%%%%%%%%%%%%%%%%%%%%%%%%%%%%%%%%%%%%%%%%
%%%%%%%%%%%%%%%%%%%%%%%%%%%%%%%%%%%%%%%%%%%%%%%%%%
\chapter{Topological phases of matter}
\label{ch:topological_phases_of_matter}
%%%%%%%%%%%%%%%%%%%%%%%%%%%%%%%%%%%%%%%%%%%%%%%%%%
%%%%%%%%%%%%%%%%%%%%%%%%%%%%%%%%%%%%%%%%%%%%%%%%%%
%
%
Interestingly, the meaning of the term ``topology'' changes with the research focus of the scientist.
For instance, chemists quite often understand topology as the geometric configuration of a molecule and biologists talk about the topology of knotted proteins.
A condensed matter physicist typically classifies certain structures, e.g. insulating materials or magnetic whirls, with the concepts of topological invariants~\cite{topocondmat}.
This approach is closely related to one of the most fundamental invariants in the mathematical branch of topology: the genus of a manifold which can be thought of as the ``number of holes''.
This number is known not to change under smooth deformations.
Consider the following three ``manifolds'', i.e. a donut, a coffee mug and a pretzel.
It may be delicate to do, but one could simply stretch and massage the donut to obtain a mug without ``cutting'' and ``glueing'', which is a necessity to go from donut to pretzel.
The ``stretching'' can be understood as smooth deformation, and as a consequence the two shapes have the same genus.
However ``cutting'' and ``glueing'' is something very sudden and irreversible, which changes the genus of the manifold.
The idea of explaining phenomena in quantum states of matter such as the quantum Hall effect or superfluid phase transitions using the mathematical concepts of topology was established by David J. Thouless, F. Duncan M. Haldane and J. Michael Kosterlitz and resulted in the 2016 Nobel prize~\cite{NP2016}.

Exploring, engineering and probing topological quantum matter, in my perspective, is still one of the most exciting research areas.
It is not only a delicate challenge due to the interdisciplinary methods that have been developed in the past decades, it is also highly relevant for its technical applications.
For instance, a so-called topological quantum computer exploits the rich exchange rules of exotic quasiparticles called anyons, which exist at the interfaces of topological quantum matter~\cite{Freedman2002}.
These particle ``braidings'' can be understood as logical gates that make up the processing unit of a hypothetical computer.
One of the major advantages of using these quasiparticles compared to other approaches is their incredible robustness against perturbations in the bulk of the material.
The foundation for the existence of such a class of quasiparticles is commonly accepted to be topological order: a certain long-range ordered pattern of quantum entanglement.
One example of a topologically ordered state would be the Moore-Read state~\cite{Moore1991,Read1996} which has a large wave-function overlap with the fractional quantum Hall state at filling factor $\nu=5/2$~\cite{Storni2010}.
Since fractional quantum Hall states are strongly correlated quantum many body systems, analytic attempts are particularly involved and quantitative statements like those in~\cite{Storni2010} are based on exact diagonalization of small systems or numerical simulations in general.

The ``trivial'' cousins of topologically ordered phases of matter are symmetry protected topological phases of matter which do not show long-range entanglement.
The fascination of this topic is -- similar to true topologically ordered phases -- based on the geometric ground state degeneracy and exponential localization of the interface excitations.
One famous example of such a system is a topological insulator, which, like an ordinary insulator, has a bulk energy gap separating the valence from the conduction band.
Unlike ordinary insulators, it hosts gapless surface states which are protected by symmetry.
The nature of topological insulators can be understood by simple non-interacting models, which makes the topic analytically amendable using standard band theory.
This lead to the full classification of symmetry protected topological phases, famously known as the periodic table of topological insulators and superconductors~\cite{Altland1997,Kitaev2009}.

Laboratory experiments based on theoretical predictions remained an exception, until the recent strive of synthetic quantum matter:
engineered systems in which the interactions between constituents are tunable to obtain effectively single-particle and even strongly correlated states of matter.
One of the most prominent examples are ultracold atoms trapped in optical lattices, which provide all kinds of ingredients theoreticians like to play with: magnetic fields~\cite{Lin2009}, spin-orbit coupling~\cite{Lin2011}, mixed particle statistics~\cite{Ferrari2002}, quantum simulation~\cite{Mazza2012} and even the realization of $4$ independent spacial coordinates exploiting the concept of synthetic dimensions~\cite{Lohse2018}, just to name a few.
%
%
%%%%%%%%%%%%%%%%%%%%%%%%%%%%%%%%%%%%%%%%
\section{Topological band theory}
\label{sec:topological_band_theory}
%%%%%%%%%%%%%%%%%%%%%%%%%%%%%%%%%%%%%%%%
%
%
To make the food example in the introduction more rigorous, the Gauss-Bonnet theorem is a beautiful statement on the integral of the Gaussian curvature $K$ over a surface $S$ which defines the Euler characteristic~\cite{Nakahara1990}
\begin{align}
    \chi = \frac1{2\pi}\int_S {\bm K}\rd {\bm S}.
    \label{eq:gauss_bonnet_theorem}
\end{align}
For instance, the Euler characteristic of a three-dimensional sphere of radius $R$ evaluates to
\begin{align}
    \chi_{S^3} = \frac1{2\pi}\int_0^R\rd r\int_0^{2\pi}\rd\phi \int_0^\pi\rd\theta K r^2\sin\theta = 2,
\end{align}
since $K=1/r^2$ in spherical coordinates.
In general, the Euler characteristic is quantized, independent of smooth deformations of the surface $S$ and relates to the genus $g$ (``number of holes'') by $\chi=2-2g$.
The topological invariants encountered in the next sections are similar, although they characterize more abstract objects.

The classification of topological insulators is possible within the band theory of solids: neglecting interactions and exploiting translational symmetry of crystals defines a periodicity in crystal-momentum space, which I already exploited excessively in \cref{ch:the_quantization_of_motion_and_fields}.
Each crystal momentum ${\bm k}$ is a good quantum number and linked to the Bloch states $\ket{\psi_s(\bm k)}=\re^{\ri\bm k}\ket{u_s(\bm k)}$ defined in a single unit-cell consisting of $s$ individual constituents.
Each Bloch state is an eigenstate of the Bloch Hamiltonian $\hat H(\bm k)$ which can be represented as a $\bm k$-dependent matrix $H(\bm k)$.
Its ordered spectrum $\varepsilon_s(\bm k)$ define what is known as the band structure.
For insulators, a finite energy gap $\Delta$ separates the occupied valence band states from the empty conduction band states.
Lattice translation symmetry implies the identity $H(\bm k + \bm G) = H(\bm k)$ for any reciprocal lattice vector $\bm G$, which defines the crystal momentum in the periodic Brillouin zone with $\bm k\equiv \bm k+\bm G$, which has the topology of a torus.
An insulating band structure can thus be viewed as a mapping between the torus to the space of Bloch Hamiltonians with a finite energy gap~\cite{Kane2013}.

Two systems I, II are called topologically equivalent, if their corresponding Bloch Hamiltonians $H_{\rm I}$, $H_{\rm II}$ can be adiabatically deformed without closing the energy gap.
Otherwise put, for topologically equivalent phases, there exists a parameter path $\gamma(t)$ adiabatically changing the Hamiltonian $H_{\gamma(t)}$ which connects the two phases $H_{\gamma(0)}=H_{\rm I}$ and $H_{\gamma(1)}=H_{\rm II}$ in such a way that the band gap $\Delta_{\gamma(t)}\neq0$ does not vanish along the path.
Consider now the special case of an interface where the crystal smoothly interpolates between two distinct topological phases.
Somewhere along the path the gap necessarily vanishes to not violate the assumption, which implies the presence of zero-energy states spacially localized around the gap-closing point.
This interplay between bulk and surface is a ubiquitous phenomenon in topological insulators called the bulk-boundary correspondence.

One of the key concepts in topological band theory is the Berry phase, for which the construction arises from the intrinsic phase ambiguity of quantum states.
Let me quickly recap the main results of the original work, presented in~\cite{Berry1984}.
Imagine a particle evolving along a closed path ${\bm r}(t)$ in an adiabatic way such that the evolution of the state follows Schrödinger's equation
\begin{align}
    \hat H(\bm r)\ket{\psi(\bm r(t))} = \ri\hbar\frac{\partial}{\partial t}\ket{\psi({\bm r(t)})},
    \label{eq:time_evolution}
\end{align}
and denote $\ket{n(\bm r)}$ the eigenstates with energies $\varepsilon_n(\bm r)$.
If the initial state is an eigenstate $\ket{n(\bm r(0))}$, it will evolve according to $\hat H$ and result in $\ket{n(\bm r(t))}=\re^{-\ri/\hbar\int_0^t \varepsilon_n(\bm r(t))\rd t'}\ket{n(\bm r(0))}$.
However, the eigenvalue equations do not relate the phases of $\ket{n(\bm r)}$ at different $\bm r$, and any (smooth) choice is allowed.
A state with ``arbitrary'' phase can thus be written as
\begin{align}
    \ket{\psi(t)} = \re^{-\ri\gamma_n(t)}\ket{n(\bm r(t))}.
    \label{eq:strange_gauge}
\end{align}
Although $\gamma$ is, in general, non-integrable, the phase is not allowed to fluctuate randomly in time, which is specified by direct substitution of \cref{eq:strange_gauge} in \cref{eq:time_evolution}
\begin{align}
    \dot\gamma_n(t) = \varepsilon_n(\bm r(t))/\hbar -\ri \braket{n({\bm r(t)})|\bm\nabla n({\bm r(t)})} \dot{\bm r}(t).
    \label{eq:phase_evolution}
\end{align}
By defining the Berry connection (sometimes called Berry potential) $\bm A_n(\bm R) = -\ri \braket{n(\bm R)|\bm\nabla n(\bm R)}$, the allowed phase change of $\ket\psi$ around a closed loop $C$ in $\bm r$-space is thus given by
\begin{align}
    \ket{\psi(T)} = \re^{-\ri \gamma_n(C)}\re^{-\ri/\hbar\int_0^T\varepsilon_n(\bm r(t))dt}\ket{\psi(0)},
    \quad
    \gamma_n(C) = \oint_C{\bm A_n}\rd \bm r.
    \label{eq:phase_change}
\end{align}
Note the transformation of $\bm A_n$ under the gauge transformation $\ket{n(\bm r)}\rightarrow\re^{\ri\mu(\bm r)}\ket{n(\bm r)}$, i.e.
\begin{align}
    \bm A \rightarrow {\bm A} + \bm\nabla\mu(\bm r),
\end{align}
which is similar to the electromagnetic vector potential.
Normalization of the state implies that $\bm\nabla \braket{n|n} = \braket{\bm\nabla n|n} + \braket{n|\bm\nabla n} = 0$, which shows that $\braket{n|\bm\nabla n}$ is imaginary and thus ${\bm A_n}$ real-valued.
This also demonstrates that $\bm A_n$ is indeed not gauge invariant, but the analog of a magnetic flux must be.
In particular, we can derive the Berry flux by rewriting $\gamma_n(C)$ using Stokes's theorem (here it is assumed that $\bm r\in\mathds R^3$ for simplicity, a discussion follows later),
\begin{align}
    \gamma_n(C) = -\ri\int_S\rd{\bm S}{\bm\nabla}\times{\bm A_n} = -\ri\int_S\rd{\bm S} \sum_{m\neq n}\braket{\bm\nabla n|m}\times\braket{m|\bm\nabla n}
    \label{eq:berry_phase_stokes_theorem}
\end{align}
in which $\rd{\bm S}$ denotes the area element in ${\bm r}$-space, the surface $S$ is spanned by the closed contour $C$, and the excluded terms are justified by $\bm A$ being imaginary.
The off-diagonal matrix elements are given by the eigenvalue equation and result to $\braket{m|{\bm\nabla} n} = \braket{m|\bm\nabla\hat H|n}/(\varepsilon_n-\varepsilon_m)$ ($m\neq n$).
It is thus possible to define the equivalent of a magnetic field, called Berry curvature
\begin{align}
    \bm F_n = -\ri\sum_{m\neq n}\braket{n|\bm\nabla\hat H|m}\times\braket{m|\bm\nabla\hat H|n}/(\varepsilon_m-\varepsilon_n)^2
\end{align}
such that we arrive at the beautiful result
\begin{align}
    \gamma_n(C) = \oint_C{\bm A_n}\rd \bm r = \int_S {\bm F_n} \rd{\bm S}.
\end{align}

To understand the general equations, let me compute the Berry phase for a generic $2\times 2$ Hamiltonian spanned by the vector of Pauli matrices $\bm\sigma$ as
\begin{align}
    H(\bm d(\bm k)) = \bm d(\bm k) \bm\sigma = \varepsilon(\bm k) \hat{\bm d}(\bm k)\bm\sigma
    =
    \varepsilon(\bm k)
    \begin{pmatrix}
        \hat d_z & \hat d_x - \ri \hat d_y \\
        \hat d_x + \ri \hat d_y & -\hat d_z
    \end{pmatrix},
    \label{eq:arbitrary_2by2}
\end{align}
in which $\hat{\bm d}={\bm d}/|\bm d|$ is a vector of unit norm and $\varepsilon_\pm = \pm\varepsilon = \pm|\bm d|$ spans the energy dispersion.
A term proportional to the identity, i.e. $d_0\mathbb 1$, can be safely ignored as the eigenvectors do not depend on it.
For completeness, the eigenvectors are given by
\begin{align}
    \ket{\pm} = \frac{1}{\sqrt{2}}
    \begin{pmatrix}
        \pm \sqrt{1 \pm \hat d_z}\re^{+\ri\vartheta/2}\\
        \phantom\pm\sqrt{1 \mp \hat d_z}\re^{-\ri\vartheta/2}
    \end{pmatrix}
    ,
    \quad
    \vartheta = \arg(\hat d_x-\ri \hat d_y).
\end{align}
Note that the parameter space is entirely spanned by $\bm d$, and \cref{eq:arbitrary_2by2} has the property $\bm\nabla\hat H \equiv \bm\nabla_{\bm d}\hat H = \bm\sigma$, such that the Berry curvature evaluates to ${\bm F}_+ = \frac12\bm d/d^3=\frac12\hat{\bm d}/d^2$\footnote{A temporary rotation to align $\bm d$ with $\sigma_z$ simplifies the computation a lot.}.
The final integration then yields the following fundamental result
\begin{align}
    \gamma_{\pm}(C) = \pm\frac12\Omega(C),
\end{align}
with $\Omega(C)$ the solid angle subtended by the path $C$ formed by $\hat {\bm d}$ from the degeneracy at $\varepsilon=0$~\cite{Berry1984}.
Note that the two bands have opposite Berry phases, and in general $\sum_n\gamma_n(C)=0$ must vanish.
If $C$ corresponds to a $2\pi q$ rotation of $\hat{\bm d}$ in a plane, the Berry phase is thus $\pm\pi q$.

This defines an integer-valued topological invariant known as the Chern number~\cite{Nakahara1990}
\begin{align}
    \chi_n = \frac1{2\pi}\int_S {\bm F_n} \rd{\bm S}
\end{align}
which has a stunning similarity to the Euler characteristic \cref{eq:gauss_bonnet_theorem}.
$\bm F_n$ can thus be viewed as a curvature with similarities to the Gaussian curvature $K$, and the quantization of the Chern number extends well beyond the two-band model.
In particular, it follows from the fact that for a loop $C$, the ``inside'' of $C$ is a bit ambiguous, and by reducing the inside of the closed loop to a surface integral, both equations necessarily agree up to a multiple of $2\pi$, in which the multiple $q$ is given by the amount of times $C$ loops around the surface $S$.

In parameter spaces of other than $3$ dimensions, Stoke's theorem cannot be employed to transform the loop integral in \cref{eq:berry_phase_stokes_theorem}.
The generalization, however, transforms \cref{eq:phase_change} into the surface integral of a $2$-form bounded by $C$.
For example, the two-dimensional Berry connection integral results in $\gamma_n(C) = \int_S\rd r_1\rd r_2 \brlr{\partial_{r_1}{\bm A}_{n,r2} - \partial_{r_2}{\bm A}_{n,r1}}$.
%
%
%%%%%%%%%%%%%%%%%%%%%%%%%%%%%%%%%%%%%%%%
\section{The Su-Schrieffer-Heeger chain}
\label{sec:the_SSH_chain}
%%%%%%%%%%%%%%%%%%%%%%%%%%%%%%%%%%%%%%%%
%
%
Let me clarify the previous statements by giving a pedagogical example.
Consider the following one-dimensional Bloch Hamiltonian, corresponding to the Su-Schrieffer-Heeger model~\cite{Heeger1988}
\begin{align}
    H(k)
    =
    \begin{pmatrix}
        0 & J+J'\re^{\ri k}\\
        J+J'\re^{\ri k} & 0
    \end{pmatrix}
\end{align}
in which the couplings $J,J'\leq0$ are assumed positive.
The $2\times2$ matrix can be easily diagonalized and yields the following band dispersion
\begin{align}
    \varepsilon_{\pm}(k) = \pm \sqrt{|J+J'\re^{\ri k}|^2} = \sqrt{J^2+J'^2+2JJ'\cos(k)}.
\end{align}
The valence/conduction band shows a maximum/minimum at $k=\pi$, which leads to the band gap $\Delta = 2\left|J-J'\right|$.
As long as $J\neq J'$, the band gap $\Delta\neq0$ is preserved, and, for reasons which are clear in the next sections, I call $J>J'$ the TRI and $J<J'$ the TOI phase.
Within each of those phases, all Hamiltonians are topologically equivalent, which is trivially satisfied by the definition of a smooth path which linearly interpolates the different values of the couplings.
It is even possible to define a family of Hamiltonians according to
\begin{align}
    H_{\gamma(t)}(k) =
    \begin{pmatrix}
        \sin(\pi t) & 1-t + t\re^{\ri k} \\
        1-t + t\re^{\ri k} & -\sin(\pi t)
    \end{pmatrix}
    \label{eq:hamiltonian_path}
\end{align}
in which, incidentally, $H_{\gamma(0)}$ is of the TRI and $H_{\gamma(1)}$ is of the TOI kind.
The spectrum of $H_{\gamma(t)}(k)$ is readily solved and evaluates to $\varepsilon_\pm(k,t)=\pm\sqrt{1-2(1-t)t+2(1-t)t\cos k+\sin(\pi t)}$.
It is easy to show that $\Delta_{\gamma(t)}\geq0$ for $0\leq t\leq1$, which demonstrates that TOI and TRI phases are, indeed, topologically equivalent.
However, if we require to keep $H_{\gamma(t)}$ entirely off-diagonal and symmetric (we will see later that this preserves chiral symmetry), an adiabatic deformation like \cref{eq:hamiltonian_path} does not exist and in order to evolve from a TOI to a TRI state, the band gap $\Delta$ necessarily vanishes at the critical point $J=J'$.

In conclusion, TOI and TRI states are topologically distinct under the assumption that a certain symmetry must be preserved at all times.
%
%
%%%%%%%%%%%%%%%%%%%%%%%%%%%%
\section{Edge zero states}
\label{sec:edge_zero_states}
%%%%%%%%%%%%%%%%%%%%%%%%%%%%
%
%
%
%
%%%%%%%%%%%%%%%%%%%%%
\section{Zak phase}
\label{sec:zak_phase}
%%%%%%%%%%%%%%%%%%%%%
%
%
%
%
%%%%%%%%%%%%%%%%%%%%%%%%
\section{Chern number}
\label{sec:chern_number}
%%%%%%%%%%%%%%%%%%%%%%%%
%
%
%
%
%%%%%%%%%%%%%%%%%%%%%%%%%%%%%%%%%%%%%%%%%%%%%%%%%
\section{Symmetry-protected topological phases}
\label{sec:symmetry_protected_topological_phases}
%%%%%%%%%%%%%%%%%%%%%%%%%%%%%%%%%%%%%%%%%%%%%%%%%
%
%
