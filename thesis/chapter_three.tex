%!TEX root = thesis.tex
%%%%%%%%%%%%%%%%%%%%%%%
%
%
%%%%%%%%%%%%%%%%%%%%%%%%%%%%%%%%%%%%%%%%%%%%%%%%%%
%%%%%%%%%%%%%%%%%%%%%%%%%%%%%%%%%%%%%%%%%%%%%%%%%%
\chapter{Topological phases of matter}
\label{ch:topological_phases_of_matter}
%%%%%%%%%%%%%%%%%%%%%%%%%%%%%%%%%%%%%%%%%%%%%%%%%%
%%%%%%%%%%%%%%%%%%%%%%%%%%%%%%%%%%%%%%%%%%%%%%%%%%
%
%
Interestingly, the meaning of the term ``topology'' changes with the research focus of the scientist.
For instance, chemists quite often understand topology as the geometric configuration of a molecule and biologists talk about the topology of knotted proteins.
A condensed matter physicist typically classifies certain structures, e.g. insulating materials or magnetic whirls, with the concepts of topological invariants~\cite{topocondmat}.
This approach is closely related to one of the most fundamental invariants in the mathematical branch of topology: the genus of a manifold which can be thought of as the ``number of holes''.
This number is known not to change under smooth deformations.
Consider the following three ``manifolds'', i.e. a donut, a coffee mug and a pretzel.
It may be delicate to do, but one could simply stretch and massage the donut to obtain a mug without ``cutting'' and ``glueing'', which is a necessity to go from donut to pretzel.
The ``stretching'' can be understood as smooth deformation, and as a consequence the two shapes have the same genus.
However ``cutting'' and ``glueing'' is something very sudden and irreversible, which changes the genus of the manifold.
The idea of explaining phenomena in quantum states of matter such as the quantum Hall effect or superfluid phase transitions using the mathematical concepts of topology was established by David J. Thouless, F. Duncan M. Haldane and J. Michael Kosterlitz and resulted in the 2016 Nobel prize~\cite{NP2016}.

Exploring, engineering and probing topological quantum matter, in my perspective, is still one of the most exciting research areas.
It is not only a delicate challenge due to the interdisciplinary methods that have been developed in the past decades, it is also highly relevant for its technical applications.
For instance, a so-called topological quantum computer exploits the rich exchange rules of exotic quasiparticles called anyons, which exist at the interfaces of topological quantum matter~\cite{Freedman2002}.
These particle ``braidings'' can be understood as logical gates that make up the processing unit of a hypothetical computer.
One of the major advantages of using these quasiparticles compared to other approaches is their incredible robustness against perturbations in the bulk of the material.
The foundation for the existence of such a class of quasiparticles is commonly accepted to be topological order: a certain long-range ordered pattern of quantum entanglement.
One example of a topologically ordered state would be the Moore-Read state~\cite{Moore1991,Read1996} which has a large wave-function overlap with the fractional quantum Hall state at filling factor $\nu=5/2$~\cite{Storni2010}.
Since fractional quantum Hall states are strongly correlated quantum many body systems, analytic attempts are particularly involved and quantitative statements like those in~\cite{Storni2010} are based on exact diagonalization of small systems or numerical simulations in general.

The ``trivial'' cousins of topologically ordered phases of matter are symmetry protected topological phases of matter which do not show long-range entanglement.
The fascination of this topic is -- similar to true topologically ordered phases -- based on the geometric ground state degeneracy and exponential localization of the interface excitations.
One famous example of such a system is a topological insulator, which, like an ordinary insulator, has a bulk energy gap separating the valence from the conduction band.
Unlike ordinary insulators, it hosts gapless surface states which are protected by symmetry.
The nature of topological insulators can be understood by simple non-interacting models, which makes the topic analytically amendable using standard band theory.
This lead to the full classification of symmetry protected topological phases, famously known as the periodic table of topological insulators and superconductors~\cite{Altland1997,Kitaev2009}.

Laboratory experiments based on theoretical predictions remained an exception, until the recent strive of synthetic quantum matter:
engineered systems in which the interactions between constituents are tunable to obtain effectively single-particle and even strongly correlated states of matter.
One of the most prominent examples are ultracold atoms trapped in optical lattices, which provide all kinds of ingredients theoreticians like to play with: magnetic fields~\cite{Lin2009}, spin-orbit coupling~\cite{Lin2011}, mixed particle statistics~\cite{Ferrari2002}, quantum simulation~\cite{Mazza2012} and even the realization of $4$ independent spacial coordinates exploiting the concept of synthetic dimensions~\cite{Lohse2018}, just to name a few.
%
%
%%%%%%%%%%%%%%%%%%%%%%%%%%%%%%%%%%%%%%%%
\section{Topological band theory}
\label{sec:topological_band_theory}
%%%%%%%%%%%%%%%%%%%%%%%%%%%%%%%%%%%%%%%%
%
%
To make the food example in the introduction more rigorous, the Gauss-Bonnet theorem is a beautiful statement on the integral of the Gaussian curvature $K$ over a surface $S$ which defines the Euler characteristic~\cite{Nakahara1990}
\begin{align}
    \chi = \frac1{2\pi}\int_S {\bm K}\rd {\bm S}.
    \label{eq:gauss_bonnet_theorem}
\end{align}
For instance, the Euler characteristic of a three-dimensional sphere of radius $R$ evaluates to
\begin{align}
    \chi_{S^3} = \frac1{2\pi}\int_0^R\rd r\int_0^{2\pi}\rd\phi \int_0^\pi\rd\theta K r^2\sin\theta = 2,
\end{align}
since $K=1/r^2$ in spherical coordinates.
In general, the Euler characteristic is quantized, independent of smooth deformations of the surface $S$ and relates to the genus $g$ (``number of holes'') by $\chi=2-2g$.
The topological invariants encountered in the next sections are similar, although they characterize more abstract objects.

The classification of topological insulators is possible within the band theory of solids: neglecting interactions and exploiting translational symmetry of crystals defines a periodicity in crystal-momentum space, which I already exploited excessively in \cref{ch:the_quantization_of_motion_and_fields}.
Each crystal momentum ${\bm k}$ is a good quantum number and linked to the Bloch states $\ket{\psi_s(\bm k)}=\re^{\ri\bm k}\ket{u_s(\bm k)}$ defined in a single unit-cell consisting of $s$ individual constituents.
Each Bloch state is an eigenstate of the Bloch Hamiltonian $\hat H(\bm k)$ which can be represented as a $\bm k$-dependent matrix $H(\bm k)$.
Its ordered spectrum $\varepsilon_s(\bm k)$ define what is known as the band structure.
For insulators, a finite energy gap $\Delta$ separates the occupied valence band states from the empty conduction band states.
Lattice translation symmetry implies the identity $H(\bm k + \bm G) = H(\bm k)$ for any reciprocal lattice vector $\bm G$, which defines the crystal momentum in the periodic Brillouin zone with $\bm k\equiv \bm k+\bm G$, which has the topology of a torus.
An insulating band structure can thus be viewed as a mapping between the torus to the space of Bloch Hamiltonians with a finite energy gap~\cite{Kane2013}.

Two systems I, II are called topologically equivalent, if their corresponding Bloch Hamiltonians $H_{\rm I}$, $H_{\rm II}$ can be adiabatically deformed without closing the energy gap.
Otherwise put, for topologically equivalent phases, there exists a parameter path adiabatically changing the Hamiltonian $H_{t}$ which connects the two phases $H_{0}=H_{\rm I}$ and $H_{1}=H_{\rm II}$ in such a way that the band gap $\Delta_{t}\neq0$ does not vanish along the path.
Consider now the special case of an interface where the crystal smoothly interpolates between two distinct topological phases.
Somewhere along the path the gap necessarily vanishes to not violate the assumption, which implies the presence of zero-energy states spacially localized around the gap-closing point.
This interplay between bulk and surface is a ubiquitous phenomenon in topological insulators called the bulk-boundary correspondence.

One of the key concepts in topological band theory is the Berry phase, for which the construction arises from the intrinsic phase ambiguity of quantum states.
Let me quickly recap the main results of the original work, presented in~\cite{Berry1984}.
Imagine a particle evolving along a closed path ${\bm r}(t)$ in an adiabatic way such that the evolution of the state follows Schrödinger's equation
\begin{align}
    \hat H(\bm r)\ket{\psi(\bm r(t))} = \ri\hbar\frac{\partial}{\partial t}\ket{\psi({\bm r(t)})},
    \label{eq:time_evolution}
\end{align}
and denote $\ket{n(\bm r)}$ the eigenstates with energies $\varepsilon_n(\bm r)$.
If the initial state is an eigenstate $\ket{n(\bm r(0))}$, it will evolve according to $\hat H$ and result in $\ket{n(\bm r(t))}=\re^{-\ri/\hbar\int_0^t \varepsilon_n(\bm r(t))\rd t'}\ket{n(\bm r(0))}$.
However, the eigenvalue equations do not relate the phases of $\ket{n(\bm r)}$ at different $\bm r$, and any (smooth) choice is allowed.
A state with ``arbitrary'' phase can thus be written as
\begin{align}
    \ket{\psi(t)} = \re^{-\ri\gamma_n(t)}\ket{n(\bm r(t))}.
    \label{eq:strange_gauge}
\end{align}
Although $\gamma$ is, in general, non-integrable, the phase is not allowed to fluctuate randomly in time, which is specified by direct substitution of \cref{eq:strange_gauge} in \cref{eq:time_evolution}
\begin{align}
    \dot\gamma_n(t) = \varepsilon_n(\bm r(t))/\hbar -\ri \braket{n({\bm r(t)})|\bm\nabla n({\bm r(t)})} \dot{\bm r}(t).
    \label{eq:phase_evolution}
\end{align}
By defining the Berry connection (sometimes called Berry potential) $\bm A_n(\bm R) = -\ri \braket{n(\bm R)|\bm\nabla n(\bm R)}$, the allowed phase change of $\ket\psi$ around a closed loop $C$ in $\bm r$-space is thus given by
\begin{align}
    \ket{\psi(T)} = \re^{-\ri \gamma_n(C)}\re^{-\ri/\hbar\int_0^T\varepsilon_n(\bm r(t))dt}\ket{\psi(0)},
    \quad
    \gamma_n(C) = \oint_C{\bm A_n}\rd \bm r.
    \label{eq:phase_change}
\end{align}
Note the transformation of $\bm A_n$ under the gauge transformation $\ket{n(\bm r)}\rightarrow\re^{\ri\mu(\bm r)}\ket{n(\bm r)}$, i.e.
\begin{align}
    \bm A \rightarrow {\bm A} + \bm\nabla\mu(\bm r),
\end{align}
which is similar to the electromagnetic vector potential.
Normalization of the state implies that $\bm\nabla \braket{n|n} = \braket{\bm\nabla n|n} + \braket{n|\bm\nabla n} = 0$, which shows that $\braket{n|\bm\nabla n}$ is imaginary and thus ${\bm A_n}$ real-valued.
This also demonstrates that $\bm A_n$ is indeed not gauge invariant, but the analog of a magnetic flux must be.
In particular, we can derive the Berry flux by rewriting $\gamma_n(C)$ using Stokes's theorem (here it is assumed that $\bm r\in\mathds R^3$ for simplicity, a discussion follows later),
\begin{align}
    \gamma_n(C) = -\ri\int_S\rd{\bm S}{\bm\nabla}\times{\bm A_n} = -\ri\int_S\rd{\bm S} \sum_{m\neq n}\braket{\bm\nabla n|m}\times\braket{m|\bm\nabla n}
    \label{eq:berry_phase_stokes_theorem}
\end{align}
in which $\rd{\bm S}$ denotes the area element in ${\bm r}$-space, the surface $S$ is spanned by the closed contour $C$, and the excluded terms are justified by $\bm A$ being imaginary.
The off-diagonal matrix elements are given by the eigenvalue equation and result to $\braket{m|{\bm\nabla} n} = \braket{m|\bm\nabla\hat H|n}/(\varepsilon_n-\varepsilon_m)$ ($m\neq n$).
It is thus possible to define the equivalent of a magnetic field, called Berry curvature
\begin{align}
    \bm F_n = -\ri\sum_{m\neq n}\braket{n|\bm\nabla\hat H|m}\times\braket{m|\bm\nabla\hat H|n}/(\varepsilon_m-\varepsilon_n)^2
\end{align}
such that we arrive at the beautiful result
\begin{align}
    \gamma_n(C) = \oint_C{\bm A_n}\rd \bm r = \int_S {\bm F_n} \rd{\bm S}.
\end{align}

To understand the general equations, let me compute the Berry phase for a generic $2\times 2$ Hamiltonian spanned by the vector of Pauli matrices $\bm\sigma$ as
\begin{align}
    H(\bm d(\bm k)) = \bm d(\bm k) \bm\sigma = \varepsilon(\bm k) \hat{\bm d}(\bm k)\bm\sigma
    =
    \varepsilon(\bm k)
    \begin{pmatrix}
        \hat{\bm d}_z & \hat{\bm d}_x - \ri \hat{\bm d}_y \\
        \hat{\bm d}_x + \ri \hat{\bm d}_y & -\hat{\bm d}_z
    \end{pmatrix},
    \label{eq:arbitrary_2by2}
\end{align}
in which $\hat{\bm d}={\bm d}/|\bm d|$ is a vector of unit norm and $\varepsilon_\pm = \pm\varepsilon = \pm|\bm d|$ spans the energy dispersion.
A term proportional to the identity, i.e. $d_0\mathbb 1$, can be safely ignored as the eigenvectors do not depend on it.
They are normalized vectors of the Bloch sphere and can be parametrized as
\begin{align}
    \ket{\pm} = \frac{1}{\sqrt{2}}
    \begin{pmatrix}
        \pm \sqrt{1 \pm \hat{\bm d}_z}\re^{+\ri\vartheta/2}\\
        \phantom\pm\sqrt{1 \mp \hat{\bm d}_z}\re^{-\ri\vartheta/2}
    \end{pmatrix}
    ,
    \quad
    \vartheta = \arg(\hat{\bm d}_x-\ri \hat{\bm d}_y).
\end{align}
Note that the explored parameter space is entirely spanned by the entries of the vector $\bm d$, and as such the Hamiltonian in \cref{eq:arbitrary_2by2} has the property that $\bm\nabla\hat H \equiv \bm\nabla_{\bm d}\hat H = \bm\sigma$.
The local Berry curvature evaluates to ${\bm F}_+ = \frac12\bm d/d^3=\frac12\hat{\bm d}/d^2$\footnote{A temporary rotation to align $\bm d$ with $\sigma_z$ simplifies the computation a lot.}.
Note that the curvature has the form of a point source with strength $1/2$ located at the degeneracy $\bm d=0$.
The surface integral then yields the following fundamental result
\begin{align}
    \gamma_{\pm}(C) = \pm\frac12\Omega(C),
\end{align}
with $\Omega(C)$ the solid angle subtended by the loop $C$ formed by $\hat {\bm d}$ from the degeneracy at $\bm d=0$~\cite{Berry1984}.
Note that the two bands have opposite Berry phases, and in general $\sum_n\gamma_n(C)=0$ must vanish.
For instance, if the loop is a $2\pi$ rotation of $\hat{\bm d}$ in a plane, the Berry phase equals to $\pm\pi$.
In case $\bm d$ covers the full Bloch sphere, the Berry phase corresponds to $2\pi$ since the solid angle of the unit-sphere is $4\pi$.

If the full Bloch sphere is covered (i.e. the surface $S$ is required to be closed), an integer-valued topological invariant known as the Chern number can be defined~\cite{Nakahara1990}
\begin{align}
    \chi_n = \frac1{2\pi}\int_S {\bm F_n} \rd{\bm S}.
    \label{eq:chern_number}
\end{align}
In particular, the integer-valuedness of $\chi_n$ follows from the fact that for a loop $C$, the ``inside'' of $C$ is a bit ambiguous -- by reducing the inside of the loop to a surface integral, the equations necessarily agree modulo $2\pi$.
Therefore, the Berry curvature integrated over a closed surface must be an element of $2\pi \mathds Z$.
Note that \cref{eq:chern_number} shares the same fundamental expression as the Euler characteristic in \cref{eq:gauss_bonnet_theorem}.
$\bm F_n$ can thus be understood as a curvature with similarities to a Gaussian one, and as such the quantization of the Chern number naturally extends well beyond the two-band model.

In parameter spaces of other than $3$ dimensions, Stoke's theorem cannot be employed to transform the loop integral in \cref{eq:berry_phase_stokes_theorem}.
The generalization, however, transforms \cref{eq:phase_change} into the surface integral of a $2$-form bounded by $C$.
For example, the two-dimensional Berry connection integral results in $\gamma_n(C) = \int_S\rd r_1\rd r_2 \brlr{\partial_{r_1}{\bm A}_{n,r2} - \partial_{r_2}{\bm A}_{n,r1}}$.
%
%
%%%%%%%%%%%%%%%%%%%%%%%%%%%%%%%%%%%%%%%%
\section{The Su-Schrieffer-Heeger chain}
\label{sec:the_SSH_chain}
%%%%%%%%%%%%%%%%%%%%%%%%%%%%%%%%%%%%%%%%
%
%
Let me clarify the previous statements by giving a pedagogical example.
Consider the following one-dimensional Bloch Hamiltonian, corresponding to the Su-Schrieffer-Heeger model~\cite{Heeger1988}
\begin{align}
    H_0(k)
    =
    \begin{pmatrix}
        0 & J+J'\re^{\ri k}\\
        J+J'\re^{-\ri k} & 0
    \end{pmatrix}
\end{align}
in which the couplings $J,J'\geq0$ are assumed positive.
The Hamiltonian falls under the category in \ref{eq:arbitrary_2by2}, and I identify
\begin{align}
    H_0(k) = {\bm d}{\bm\sigma},
    \quad
    {\bm d}_x = J+J'\cos(k),
    \quad
    {\bm d}_y = J'\sin(k),
    \quad
    {\bm d}_z = 0.
\end{align}
Let us explore more the topological equivalency I detailed in the previous section.
The valence/conduction band shows a maximum/minimum at $k=\pi$, which leads to the band gap $\Delta = 2\left|J-J'\right|$.
As long as $J\neq J'$, the band gap $\Delta\neq0$ is preserved, and, for reasons which are clear in the next sections, I call $J>J'$ the TRI and $J<J'$ the TOI phase.
Within each of those phases, all Hamiltonians are topologically equivalent, which is trivially satisfied by the definition of a smooth path which linearly interpolates the different values of the couplings.
It is also possible to define a family of Hamiltonians according to
\begin{align}
    H_{t}(k) =
    \begin{pmatrix}
        \sin(\pi t) & 1-t + t\re^{\ri k} \\
        1-t + t\re^{\ri k} & -\sin(\pi t)
    \end{pmatrix}
    \label{eq:hamiltonian_path}
\end{align}
in which, incidentally, $H_0$ is of the TRI and $H_1$ is of the TOI kind.
The spectrum of $H_{t}(k)$ is readily solved and evaluates to $\varepsilon_\pm(k,t)=\pm\sqrt{1-2(1-t)t+2(1-t)t\cos k+\sin(\pi t)}$.
It is easy to show that $\Delta_{t}\geq0$ for $0\leq t\leq1$, which demonstrates that TOI and TRI phases are, indeed, topologically equivalent.

This statement can be geometrically understood by the fact that the single-valued parametrization $\bm d(k)$ generates an arc of a circle or a full circle on the Bloch sphere.
In other words, the surface $S$ spanned by $C$ never wraps around the degeneracy, which is the source of ${\bm F}_n$.
As a consequence, $\gamma_n(C)$ is not quantized and adiabatic deformations like \cref{eq:hamiltonian_path} are possible to effectively remove the degeneracy from the surface integral, smoothly connecting all nonzero geometric phases to the trivial one.

If, however, we require to keep $H_{t}$ entirely off-diagonal and symmetric, an adiabatic deformation like \cref{eq:hamiltonian_path} does not exist and in order to evolve from a TOI to a TRI state, the band gap $\Delta$ necessarily vanishes at the critical point $J=J'$.
The Berry phase of such a system can be either $\pi\mathds Z$, depending if the path $C$ spanned by $\hat{\bm d}$ loops around the degeneracy or if $\hat{\bm d}$ retraces itself and so defines no surface.
By imposing the off-diagonal structure of the family of Hamiltonians, the degeneracy cannot be removed from the surface integral (which would indeed require a deformation along ${\bm d}_z$).
This justifies yet another topological index, called the winding number, in the equivalence classes of Hamiltonians which respect the imposed symmetry.
It is defined as the Berry phase divided $\pi$, i.e.
\begin{align}
    \mathcal W = |\gamma_n(C)/\pi|.
\end{align}
Since the winding counts the number of circles which $\hat{\bm d}$ performs while passing through the Brillouin zone, it can be straightforwardly defined as
\begin{align}
    \mathcal W = \frac{1}{2\pi}\int\rd k \arg(\hat{\bm d}_x - \ri \hat{\bm d}_y)
    =
    \frac{1}{4\pi\ri} \tr\int\rd k \sigma_z H^{-1}\partial_k H
    =
    \frac{1}{4\pi\ri} \tr\int\rd k \sigma_z g^{-1}\partial_k g,
\end{align}
in which $g(k)=G(k,\omega=0)$ is the propagator\footnote{The expression with the propagator naturally extends to interacting one-dimensional systems, as shown in~\cite{Gurarie2011}. This is particularly important for our work~\todo{[REF]}.}.
The winding number can most easily read-out by close observation of ${\bm d}$, which either traverses around the degeneracy at the origin, or not.
This will then translate to $\hat{\bm d}$ drawing circles around the Bloch sphere, or retracing itself, thus spanning no surface.
In particular, the former will be achieved if $J<J'$ ($\mathcal W=1$), while the latter occurs for $J>J'$ ($\mathcal W=0$).
From now on, I call a phase with nonzero winding number the topological insulator (TOI) and the other trivial insulator (TRI).

To understand the emergence of edge modes, we must obviously introduce an interface of some sort.
First of all, let me Fourier-transform the Bloch Hamiltonian to find the representation in real space, i.e.
\begin{align}
    H = \sum_x \brlr{J c^\dag_{x,A}c^\pdag_{x,B} + J' c^\dag_{x,B}c^\pdag_{x+1,A}} + \hc,
\end{align}
in which $c_{x,A/B}$ are the annihilators of a spinless fermionic particle at site $x$ and sublattice $A/B$.
An interface which hosts edge states with energy zero, following the logic of the previous section, is readily implemented by two semi-infinite crystals, one with bulk invariant $0$ (e.g. for all negative lattice positions $x<0$), the other with bulk invariant $1$ (for all lattice positions $x\geq0$).
For convenience, let me here construct this crystal from the two limiting cases (i) $J'=0$ and (ii) $J=0$.
Note that for such a system, the central $A$ site at position $x=0$ does not contribute to the Hamiltonian and hosts a perfectly localized mode of zero energy.
Obviously, this edge mode is preserved if I consider the vacuum in region (i).
Let me thus simplify the hypothetical crystal by replacing the trivial insulator with the vacuum, and proceed by relaxing the constraint $J=0$ to $J'>J$ on the semi-infinite crystal:
\begin{align}
    \hat H = \sum_{x\geq0} J c^\dag_{x,A}c^\pdag_{x,B} + J' c^\dag_{x,B}c^\pdag_{x+1,A} + \hc
\end{align}
In this case, the presence of an edge zero mode is not particularly obvious and requires further inspection.
One observation is that the fermionic Hamiltonian can be diagonalized by unitary transformations, which requires only a diagonalization of $\hat H = \bm c^\dag h \bm c$, in which $\bm c=(c_{0,A},c_{0,B},c_{1,A},c_{1,B},\dots)^T$ is the successive vector of all annihilation operators.
The matrix $h$ is quite sparse and has only alternating off-diagonal elements
\begin{align}
    h =
    \begin{pmatrix}
        0 & J & 0 & 0 &\dots \\
        J & 0 & J' & 0 & \dots \\
        0 & J' & 0 & J & \dots \\
        0 & 0 & J & 0 & \dots \\
        \vdots & \vdots & \vdots & \vdots & \ddots
    \end{pmatrix},
\end{align}
which makes an analytic treatment not too difficult, even if the entries are spacially dependent~\cite{Asbth2016}.
An exact boundary mode can be found by evaluating $hv:L=0$, with a vector $v_L = (a_0,b_0,a_1,b_1,\dots)^T$, resulting in the set of equations
\begin{align}
    J b_0 = 0,
    \quad
    Ja_m + J' a_{m+1} = 0,
    \quad
    J' b_m + J b_{m+1} = 0.
\end{align}
Defining $w=J'/J$, the solution reveals an exponentially localized zero energy mode which has no support on sublattice $B$, i.e.
\begin{align}
    a_k=(-w)^{1-k} a_0 \rightarrow |a_k| = \re^{-(k-1)/\log w}|a_0|,
    \quad
    b_k = 0.
    \label{eq:boundary_mode}
\end{align}
It is also transparent that the state generated by the vector $v_L$ would violate normalization in the TRI phase ($w<1$) and is thus allowed to exist only if the geometric series is convergent, such that $a_0 = (1-w^{-1})^{-1/2}$.

In case of a finite crystal with $N$ sites, a similar reasoning reveals the existence of two such boundary modes, generated by the vectors $v_L$ and $v_R$.
Similarly to $v_L$, $v_R$ is exponentially localized at the right interface, has no support on sublattice $A$ and decays exponentially into the bulk.
Thus, the two modes share exactly no overlap $v_L^T v_R = 0$.
However, the two modes are not eigenstates of the Hamiltonian since each one slightly violates the boundary conditions $b_0=0$ and $a_N=0$.
Note that $v_{L/R}^T h v_{L/R} = 0$ and $v_{L/R}^T h v_{R/L} = (1-w^{-1})^{-1}\re^{-(N-1)/\log w}\re^{\pm\varphi}$ with some phase $\varphi$.
This then yields ``state hybredization'', a pair of orthogonal eigenstates with energy $\pm(1-w^{-1})^{-1}\re^{-(N-1)/\log w}$ which read $v_\pm = 1/\sqrt2(v_L \pm v_R\re^{\pm\varphi})$.
Therefore, the hybredized states are still exponentially localized boundary modes and of ``almost'' zero energy with an exponential splitting in the system size.
%
%
%%%%%%%%%%%%%%%%%%%%%%%%%%%%%%%%%%%%%%%%%%%%%%%%%
\section{Symmetry-protected topological phases}
\label{sec:symmetry_protected_topological_phases}
%%%%%%%%%%%%%%%%%%%%%%%%%%%%%%%%%%%%%%%%%%%%%%%%%
%
%
\todo{arrived here...}
