\documentclass{svmono}
%DIF LATEXDIFF DIFFERENCE FILE
%DIF DEL thesis_old_flattened.tex   Mon Dec 28 12:32:51 2020
%DIF ADD thesis_new_flattened.tex   Mon Dec 28 12:32:28 2020

\usepackage{type1cm}
\usepackage{makeidx}         \usepackage{graphicx}        \usepackage{multicol}        \usepackage[bottom]{footmisc}\usepackage{amsmath}
\usepackage{bm}
\usepackage{amssymb}
\usepackage[export]{adjustbox}
\usepackage{mathtools}
\usepackage{bbold}
\usepackage{subfigure}
\usepackage{hyperref}
\usepackage{cleveref}
\usepackage{lipsum}
\usepackage[utf8]{inputenc}
\usepackage[english]{babel}
\usepackage{pdfpages}
\usepackage[top=3cm,bottom=4cm,outer=3.5cm,inner=3.5cm]{geometry}
\usepackage[normalem]{ulem}
\usepackage{braket}
\usepackage{tikz}
\usepackage{ifthen}
\usepackage{dsfont}
\usetikzlibrary{tikzmark}

%DIF 25a25-28
 %DIF > 
 %DIF > 
\usepackage{etoolbox} %DIF > 
\cslet{blx@noerroretextools}\empty %DIF > 
%DIF -------
\usepackage[backend=bibtex]{biblatex}
\addbibresource{biblio.bib}
%DIF 27a31
\usepackage{autonum} %DIF > 
%DIF -------

%DIF 28a33
\usepackage{csquotes} %DIF > 
%DIF -------

\newcommand{\todo}[1]{{\color{cyan} \scshape !!! #1 !!!}}
\newcommand{\todoil}[1]{{\color{cyan} \scshape #1}}

\setlength{\parskip}{0.2cm}

\def\ri{\mathrm i}
\def\re{\mathrm e}
\def\rd{\mathrm d}
\def\rD{\mathcal D}
\def\hc{{\rm h.c.}}
\def\tr{{\rm tr}}
\def\C{\mathcal{C}}
\def\DC{\Delta\C}
\def\GX{\Gamma X}
\def\DCav{\overline{\Delta\C}}
\def\up{\uparrow}
\def\down{\downarrow}
\def\pdag{{\vphantom\dag}}
\def\pp{\vphantom{n'}}
\def\Cav{\overline\C}
\def\MH{H}
\def\HS{\mathcal{H}}
\def\FS{\mathcal{F}}
\def\omegaT{\tilde{\omega}}
\def\tbc{{\\[1cm]\bf \color{red}[TO BE CONTINUED...]}}
\newcommand{\ave}[1]{\langle #1 \rangle}
\newcommand{\sign}[1]{\text{sign}\left( #1 \right)}
\newcommand{\commutator}[1]{\left[ #1 \right]}
\newcommand{\anticommutator}[1]{\left\{ #1 \right\}}
\newcommand{\brlr}[1]{\left( #1 \right)}
\newcommand{\abs}[1]{\left| #1 \right|}
 
\makeindex 
%DIF PREAMBLE EXTENSION ADDED BY LATEXDIFF
%DIF UNDERLINE PREAMBLE %DIF PREAMBLE
\RequirePackage[normalem]{ulem} %DIF PREAMBLE
\RequirePackage{color}\definecolor{RED}{rgb}{1,0,0}\definecolor{BLUE}{rgb}{0,0,1} %DIF PREAMBLE
\providecommand{\DIFaddtex}[1]{{\protect\color{blue}\uwave{#1}}} %DIF PREAMBLE
\providecommand{\DIFdeltex}[1]{{\protect\color{red}\sout{#1}}}                      %DIF PREAMBLE
%DIF SAFE PREAMBLE %DIF PREAMBLE
\providecommand{\DIFaddbegin}{} %DIF PREAMBLE
\providecommand{\DIFaddend}{} %DIF PREAMBLE
\providecommand{\DIFdelbegin}{} %DIF PREAMBLE
\providecommand{\DIFdelend}{} %DIF PREAMBLE
\providecommand{\DIFmodbegin}{} %DIF PREAMBLE
\providecommand{\DIFmodend}{} %DIF PREAMBLE
%DIF FLOATSAFE PREAMBLE %DIF PREAMBLE
\providecommand{\DIFaddFL}[1]{\DIFadd{#1}} %DIF PREAMBLE
\providecommand{\DIFdelFL}[1]{\DIFdel{#1}} %DIF PREAMBLE
\providecommand{\DIFaddbeginFL}{} %DIF PREAMBLE
\providecommand{\DIFaddendFL}{} %DIF PREAMBLE
\providecommand{\DIFdelbeginFL}{} %DIF PREAMBLE
\providecommand{\DIFdelendFL}{} %DIF PREAMBLE
%DIF HYPERREF PREAMBLE %DIF PREAMBLE
\providecommand{\DIFadd}[1]{\texorpdfstring{\DIFaddtex{#1}}{#1}} %DIF PREAMBLE
\providecommand{\DIFdel}[1]{\texorpdfstring{\DIFdeltex{#1}}{}} %DIF PREAMBLE
\newcommand{\DIFscaledelfig}{0.5}
%DIF HIGHLIGHTGRAPHICS PREAMBLE %DIF PREAMBLE
\RequirePackage{settobox} %DIF PREAMBLE
\RequirePackage{letltxmacro} %DIF PREAMBLE
\newsavebox{\DIFdelgraphicsbox} %DIF PREAMBLE
\newlength{\DIFdelgraphicswidth} %DIF PREAMBLE
\newlength{\DIFdelgraphicsheight} %DIF PREAMBLE
% store original definition of \includegraphics %DIF PREAMBLE
\LetLtxMacro{\DIFOincludegraphics}{\includegraphics} %DIF PREAMBLE
\newcommand{\DIFaddincludegraphics}[2][]{{\color{blue}\fbox{\DIFOincludegraphics[#1]{#2}}}} %DIF PREAMBLE
\newcommand{\DIFdelincludegraphics}[2][]{% %DIF PREAMBLE
\sbox{\DIFdelgraphicsbox}{\DIFOincludegraphics[#1]{#2}}% %DIF PREAMBLE
\settoboxwidth{\DIFdelgraphicswidth}{\DIFdelgraphicsbox} %DIF PREAMBLE
\settoboxtotalheight{\DIFdelgraphicsheight}{\DIFdelgraphicsbox} %DIF PREAMBLE
\scalebox{\DIFscaledelfig}{% %DIF PREAMBLE
\parbox[b]{\DIFdelgraphicswidth}{\usebox{\DIFdelgraphicsbox}\\[-\baselineskip] \rule{\DIFdelgraphicswidth}{0em}}\llap{\resizebox{\DIFdelgraphicswidth}{\DIFdelgraphicsheight}{% %DIF PREAMBLE
\setlength{\unitlength}{\DIFdelgraphicswidth}% %DIF PREAMBLE
\begin{picture}(1,1)% %DIF PREAMBLE
\thicklines\linethickness{2pt} %DIF PREAMBLE
{\color[rgb]{1,0,0}\put(0,0){\framebox(1,1){}}}% %DIF PREAMBLE
{\color[rgb]{1,0,0}\put(0,0){\line( 1,1){1}}}% %DIF PREAMBLE
{\color[rgb]{1,0,0}\put(0,1){\line(1,-1){1}}}% %DIF PREAMBLE
\end{picture}% %DIF PREAMBLE
}\hspace*{3pt}}} %DIF PREAMBLE
} %DIF PREAMBLE
\LetLtxMacro{\DIFOaddbegin}{\DIFaddbegin} %DIF PREAMBLE
\LetLtxMacro{\DIFOaddend}{\DIFaddend} %DIF PREAMBLE
\LetLtxMacro{\DIFOdelbegin}{\DIFdelbegin} %DIF PREAMBLE
\LetLtxMacro{\DIFOdelend}{\DIFdelend} %DIF PREAMBLE
\DeclareRobustCommand{\DIFaddbegin}{\DIFOaddbegin \let\includegraphics\DIFaddincludegraphics} %DIF PREAMBLE
\DeclareRobustCommand{\DIFaddend}{\DIFOaddend \let\includegraphics\DIFOincludegraphics} %DIF PREAMBLE
\DeclareRobustCommand{\DIFdelbegin}{\DIFOdelbegin \let\includegraphics\DIFdelincludegraphics} %DIF PREAMBLE
\DeclareRobustCommand{\DIFdelend}{\DIFOaddend \let\includegraphics\DIFOincludegraphics} %DIF PREAMBLE
\LetLtxMacro{\DIFOaddbeginFL}{\DIFaddbeginFL} %DIF PREAMBLE
\LetLtxMacro{\DIFOaddendFL}{\DIFaddendFL} %DIF PREAMBLE
\LetLtxMacro{\DIFOdelbeginFL}{\DIFdelbeginFL} %DIF PREAMBLE
\LetLtxMacro{\DIFOdelendFL}{\DIFdelendFL} %DIF PREAMBLE
\DeclareRobustCommand{\DIFaddbeginFL}{\DIFOaddbeginFL \let\includegraphics\DIFaddincludegraphics} %DIF PREAMBLE
\DeclareRobustCommand{\DIFaddendFL}{\DIFOaddendFL \let\includegraphics\DIFOincludegraphics} %DIF PREAMBLE
\DeclareRobustCommand{\DIFdelbeginFL}{\DIFOdelbeginFL \let\includegraphics\DIFdelincludegraphics} %DIF PREAMBLE
\DeclareRobustCommand{\DIFdelendFL}{\DIFOaddendFL \let\includegraphics\DIFOincludegraphics} %DIF PREAMBLE
%DIF LISTINGS PREAMBLE %DIF PREAMBLE
\RequirePackage{listings} %DIF PREAMBLE
\RequirePackage{color} %DIF PREAMBLE
\lstdefinelanguage{DIFcode}{ %DIF PREAMBLE
%DIF DIFCODE_UNDERLINE %DIF PREAMBLE
  moredelim=[il][\color{red}\sout]{\%DIF\ <\ }, %DIF PREAMBLE
  moredelim=[il][\color{blue}\uwave]{\%DIF\ >\ } %DIF PREAMBLE
} %DIF PREAMBLE
\lstdefinestyle{DIFverbatimstyle}{ %DIF PREAMBLE
	language=DIFcode, %DIF PREAMBLE
	basicstyle=\ttfamily, %DIF PREAMBLE
	columns=fullflexible, %DIF PREAMBLE
	keepspaces=true %DIF PREAMBLE
} %DIF PREAMBLE
\lstnewenvironment{DIFverbatim}{\lstset{style=DIFverbatimstyle}}{} %DIF PREAMBLE
\lstnewenvironment{DIFverbatim*}{\lstset{style=DIFverbatimstyle,showspaces=true}}{} %DIF PREAMBLE
%DIF END PREAMBLE EXTENSION ADDED BY LATEXDIFF

\begin{document}

\author{Andreas Haller}
\title{On the tuning of particle transport,\\ strongly correlated helical phases and\\ the measurement of topological invariants}
\subtitle{Dissertation for the award of the title\\[0.5cm] {\Large``Doctor of Natural Sciences''}\\[0.5cm] at the Faculty of Physics, Mathematics and Computer Science\\ of the Johannes Gutenberg-University in Mainz}

\maketitle

\frontmatter \newpage
\setcounter{page}{1}

\clearpage{}\DIFdelbegin %DIFDELCMD < \chapter*{Abstract}
%DIFDELCMD < %%%
\addcontentsline{toc}{chapter}{\DIFdel{Abstract}}
%DIFAUXCMD
\DIFdel{In this thesis, we investigate the effects of interactions in several different quasi-one-dimensional ladder models of fermions and bosons, such as those encountered in ongoing experiments of synthetic quantum matter, in particular in setups of ultracold atoms trapped in optical lattices.
}%DIFDELCMD < 

%DIFDELCMD < %%%
\DIFdel{In fermionic systems, we found a possibility to flexibly tune the zero frequency component of the conductivity at zero temperature (a.k.a. the Drude weight) by repulsive density-density interactions.
The enhancement of this quantity under repulsive interactions is contradicting the ``common wisdom'' that one-dimensional systems should always show a decreasing Drude weight.
The loophole is given through the slim 2D extension of quasi-one-dimensional ladders.
It allows to generate pseudospin polarized band structures which leads to the anomalous behavior of the Drude weight in interacting systems.
Our results are thus relevant for the modification of transport properties of coupled-wire systems in general.
The results are based on perturbation theory, Abelian bosonization, and numerical estimates in the non-perturbative regime through matrix product state (MPS) simulations.
}%DIFDELCMD < 

%DIFDELCMD < %%%
\DIFdel{We study the emergence of so-called helical liquids by repulsive interactions and density-assisted hoppings, resulting from a commensurability between particle density and flux generated from complex hopping elements.
Some of these phases have a natural extension to (fractional) quantum Hall phases in two spatial dimensions, e.g. the Laughlin-like states, and can thus be interpreted as the predecessors of (fractional) topological phases of matter.
The study of strongly correlated helical liquids is thus important for the bottom-up fabrication of (fractional) topological phases in setups of synthetic quantum matter.
For fermions, we study the emergence of an exotic phase located at particle filling factor $\nu=1/2$ which naturally extends to a fractional quantum Hall phase: the bosonic $K=8$ state.
We then study a similar phase in a ladder of hard-core bosons, located at filling factor $\nu=1$.
In both cases, we derive the phase diagram through renormalization group theory calculations and test our hypotheses through MPS simulations.
}%DIFDELCMD < 

%DIFDELCMD < %%%
\DIFdel{A natural quest in the context of topological insulators concerns the measurement of topological invariants, especially in the case of interacting systems, where a priori it does not amount to a band dispersion quantity.
For chiral symmetric systems, we propose a simple dynamical protocol based on the mean chiral displacement which provides a tomography of the topological index and indicates the presence of symmetry-broken phases through a characteristic divergence.
This protocol is tested using dynamical MPS simulations, and we provide a basic blueprint for an experiment of ultracold atoms trapped in optical lattices.
}%DIFDELCMD < \clearpage{}
%DIFDELCMD < \clearpage{}
%DIFDELCMD < 

%DIFDELCMD < \begin{dedication}
%DIFDELCMD <     %%%
\DIFdel{Let's call it a PhD }%DIFDELCMD < {\tiny %%%
\DIFdel{(please)}%DIFDELCMD < }
%DIFDELCMD < \end{dedication}
%DIFDELCMD < \clearpage{}
%DIFDELCMD < \clearpage{}%%%
\DIFdelend \chapter*{Acknowledgments}
\addcontentsline{toc}{chapter}{Acknowledgments}
Now that I am right at the finish line, besides being happy that the exhaustive part is finally over, I feel a great deal of gratitude.
I would simply not have made it through, were it not for the wonderful people that take part in my life.
\DIFdelbegin \DIFdel{It is impossible to mention all of those incredible individuals that shaped and decorated the path of my PhD, and I apologize if you do not find yourself among those I listed here.
}\DIFdelend 

First of all I want to thank the members of the examination board for taking the time to evaluate the dissertation.
I know that you are all incredibly busy, and truly appreciate your \DIFdelbegin \DIFdel{efforts }\DIFdelend \DIFaddbegin \DIFadd{tremendous effort }\DIFaddend in training the next generation of scientists.

I want to express my deep gratitude towards {\it Prof. Michele Burrello}\DIFdelbegin \DIFdel{, you are always }\DIFdelend \DIFaddbegin \DIFadd{.
You are }\DIFaddend one of the first persons I approach for professional advice, and I am always astonished by how much you know about physics.
Thank you so much for your mentoring, and for inviting me to the Niels Bohr Institute.
Brainstorming with smart and inspiring people is one of the best experiences during my PhD.
One of the best \DIFdelbegin \DIFdel{example }\DIFdelend \DIFaddbegin \DIFadd{examples }\DIFaddend out of of this ``equivalence class'' \DIFaddbegin \DIFadd{of individuals }\DIFaddend is {\it Prof. Pietro Massignan}.
I am looking forward to visiting the post-pandemic Primavera Sound Festival with you and Hélia.
\DIFdelbegin \DIFdel{All }\DIFdelend \DIFaddbegin \DIFadd{Most of }\DIFaddend my academic collaborators \DIFdelbegin \DIFdel{for some reason seem to be }\DIFdelend \DIFaddbegin \DIFadd{are }\DIFaddend Italian, and although I am not a fan of cliches, I must admit that they are \DIFdelbegin \DIFdel{very passionate in general}\DIFdelend \DIFaddbegin \DIFadd{indeed very passionate}\DIFaddend :
Thank you, {\it Michele Filippone}, for \DIFdelbegin \DIFdel{kicking my bottom on a regular basis }\DIFdelend \DIFaddbegin \DIFadd{your regular motivation speeches }\DIFaddend during the time we worked together, for having me around in Geneva, for showing me the Bains des Pâquis and warning me about drinking cold water after a Swiss Fondue.
Speaking of passionate Italians, I want to express my gratitude towards {\it Jun.-Prof. Jamir Marino}.
Thank you ``für den frischen Wind'' as we Germans like to say, for your feedback regarding my thesis and for putting so much effort in a lively scientific exchange, which is incredibly hard \DIFdelbegin \DIFdel{during this pandemic}\DIFdelend \DIFaddbegin \DIFadd{to maintain right now}\DIFaddend .
By now it is clear to me that \DIFdelbegin \DIFdel{the decision to strive for }\DIFdelend an academic career is unavoidably linked to a \DIFdelbegin \DIFdel{certain }\DIFdelend level of uncertainty\DIFaddbegin \DIFadd{: be it compensation, doubting yourself or feeling unrecognized and irrelevant, it results in a lot of unnecessary distraction from actual work}\DIFaddend .
I am grateful to the head of our group, {\it Prof. Peter van Dongen}, who managed to eliminate most of these uncertainties during the period of my PhD.
Thank you for your scientific advice, for being part of the examination board, and for carefully reading this thesis.
I \DIFaddbegin \DIFadd{am glad that you consider me part of your little ``bubble'', and I wish you the best for the future.
I }\DIFaddend want to thank {\it Prof. Patrick Windpassinger} for \DIFdelbegin \DIFdel{being part of the examination board and for }\DIFdelend writing a letter of recommendation to the graduate school \DIFdelbegin \DIFdel{.
It has always been a pleasure to chat with you when we met at the institute's staircase}\DIFdelend \DIFaddbegin \DIFadd{and for being part of the examination board}\DIFaddend .
My PhD would have been impossible without the \DIFdelbegin \DIFdel{fund from }\DIFdelend \DIFaddbegin \DIFadd{scholarship of }\DIFaddend the Graduate School Materials Science in Mainz (MAINZ) and the Max Planck Graduate Center (MPGC).
Special thanks to {\it Michael Fuchs}, {\it Katrin Klauer}, and to {\it Prof. Thomas Speck}.

A student's life would be pretty sad \DIFaddbegin \DIFadd{and lonely }\DIFaddend without a group of fellows\DIFdelbegin \DIFdel{suffering vicariously with him.
}\DIFdelend \DIFaddbegin \DIFadd{:
}\DIFaddend Special thanks to my dear friends and colleagues, {\it Matteo Campo}, {\it Álvaro Díaz Fernández}, {\it Achim Harzheim}, {\it William Janke}, {\it Johannes Jünemann}, {\it Philipp Schmoll} and {\it Niklas Tausendpfund}, for taking the journey with me, and for reading part of this thesis.
I will miss our joint conferences, lunches and coffee breaks.

Music is a big part of my life, and it hurts me very much that these past few months were exceptionally quiet.
To all my musician friends, in particular to {\it Simon Engelhardt}, {\it Carlos Wagner} and the Grundfunk family:
There will be a time when we enjoy once more the incredible joy of playing with real people, at a real stage, and in front of an audience.
Until then, hold on tight and keep on practicing.

Sometimes (mostly Tuesdays) I came in late due to the merciless morning squash matches between me and {\it Andreas Dittinger}, {\it Fabian Kolf} and {\it Maximilian Shaikh-Yousef}.
Before my passion for squash, I was happily climbing like an ape during my free evenings in the Blockwerk Mainz with {\it Johannes} and {\it Julia Hofmann}, {\it Ruben Hoffmann} and the \DIFdelbegin \DIFdel{BFE crew}\DIFdelend \DIFaddbegin \DIFadd{``BFE crew''}\DIFaddend , which again resulted in me coming in late to university.
At other times, I overslept because of a merry wine-drinking session with our favorite upstairs neighbors {\it Laura Hollegger} and {\it Max Henderson}.
To all of my dear friends: Thanks for the joy and happiness you bring into my life.

Family always comes first, although by now we live a bit detached from each other.
I want to thank my brother-in-law, {\it Stefan Stamer}, and my parents-in-law, {\it Helma Heinz-Stamer} and {\it Thomas Stamer}.
You know me for half of my life now, and I can only imagine how annoying it must have been for you to meet me \DIFdelbegin \DIFdel{during my teens, hitting on your sister/daughter}\DIFdelend \DIFaddbegin \DIFadd{while I was a long-haired teenager}\DIFaddend .
Joking aside, I \DIFdelbegin \DIFdel{'ve always felt }\DIFdelend \DIFaddbegin \DIFadd{always feel }\DIFaddend warm and welcome \DIFdelbegin \DIFdel{in your home}\DIFdelend \DIFaddbegin \DIFadd{when I am around you}\DIFaddend , and I thank you deeply for trusting me with the biggest treasure you have\DIFdelbegin \DIFdel{.
}\DIFdelend \DIFaddbegin \DIFadd{:
}\DIFaddend {\it Susanne}, you are the love of my life\DIFdelbegin \DIFdel{and always make me feel safe, even in times I doubt myself}\DIFdelend .
I don't know if you remember our chat nine years ago, but you are the reason why I pursued a career as a Physicist, \DIFdelbegin \DIFdel{and I will never forget that}\DIFdelend \DIFaddbegin \DIFadd{which, besides marrying you, was one of the best decisions of my life}\DIFaddend .
I want to thank my grandma, {\it Irmgard Heinz}, and I hope this message reaches \DIFdelbegin \DIFdel{her }\DIFdelend \DIFaddbegin \DIFadd{you }\DIFaddend somehow.
You were the sweetest person of age that I know, always happy when we passed by, and never showed your disappointed if we were busy otherwise.
I \DIFdelbegin \DIFdel{will never forget }\DIFdelend \DIFaddbegin \DIFadd{remember }\DIFaddend our joint trip to the canary islands with Susanne \DIFdelbegin \DIFdel{, which }\DIFdelend \DIFaddbegin \DIFadd{like it was yesterday, because it }\DIFaddend woke my curiosity to travel the world.

To my siblings, {\it Michael Haller} and {\it Astrid Fengler}.
Thank you for your patience with me.
\DIFdelbegin \DIFdel{I've not always been the best little brother, and certainly was a little rascal when I was younger.
}\DIFdelend Despite all our differences, I know that I can always count on you, and I am deeply grateful for having you in my life.
{\it Liebe Mama, lieber Papa}, ihr habt \DIFdelbegin \DIFdel{meine Entscheidung Physik zu studieren }\DIFdelend \DIFaddbegin \DIFadd{mein Studium }\DIFaddend stets unterstützt, auch wenn ihr keinen \DIFdelbegin \DIFdel{direkten }\DIFdelend Bezug zu dem Thema habt.
Euch \DIFdelbegin \DIFdel{ist }\DIFdelend \DIFaddbegin \DIFadd{war }\DIFaddend es stets das \DIFdelbegin \DIFdel{Wichtigste, dass eure Kinder eigene Entscheidungen treffen und diese dann auch realisieren}\DIFdelend \DIFaddbegin \DIFadd{oberste Gut, dass wir die wichtigen Entscheidungen unseres Lebens unvoreingenommen und selbstständig getroffen haben}\DIFaddend .
Nur durch dieses große Vertrauen in uns sind wir zu den Personen geworden, die wir heute sind.
Dafür liebe ich euch und bin euch unendlich dankbar.

\DIFdelbegin \DIFdel{Lastly, I want to thank ``il capo '' }\DIFdelend \DIFaddbegin \DIFadd{Al mio capo }\DIFaddend {\it \DIFaddbegin \DIFadd{Prof. }\DIFaddend Matteo Rizzi}\DIFdelbegin \DIFdel{.
You made }\DIFdelend \DIFaddbegin \DIFadd{:
You made all of }\DIFaddend this possible by shaping me into the scientist I am today.
I \DIFaddbegin \DIFadd{remember very well when I entered your office for the first time, not knowing what to expect, a little shy and intimidated by your wisdom.
Although I knew so little at the time we started working together, you never lost interest and aided me to the best of your abilities.
I }\DIFaddend could not have been guided better \DIFdelbegin \DIFdel{, }\DIFdelend and I am proud to call myself your student.
With all of my heart, I \DIFaddbegin \DIFadd{deeply thank you for our joint time and }\DIFaddend wish the best to you, Michela, and Giovanni.
\DIFaddbegin 

\DIFadd{I am grateful to everyone that shaped and decorated the path of my PhD.
It has been a wonderful experience, and I would do it all over again.
}\DIFaddend \clearpage{}
\DIFaddbegin \clearpage{}\chapter*{Abstract}
\addcontentsline{toc}{chapter}{\DIFadd{Abstract}}
\DIFadd{In this thesis, we investigate the effects of interactions in several different quasi-one-dimensional ladder models of fermions and bosons, such as those encountered in ongoing experiments of synthetic quantum matter, in particular in setups of ultracold atoms trapped in optical lattices.
}\DIFaddend 

\DIFaddbegin \DIFadd{In fermionic systems, we find a possibility to flexibly tune the zero frequency component of the conductivity at zero temperature (the Drude weight) by repulsive density-density interactions.
The enhancement of this quantity under repulsive interactions is contradicting the ``common wisdom'' that one-dimensional systems should always show a decreasing Drude weight.
The loophole is given through the slim 2D extension of quasi-one-dimensional ladders.
It allows to generate pseudospin polarized band structures which lead to the anomalous behavior of the Drude weight in interacting systems.
Our results are thus relevant for the modification of transport properties of coupled-wire systems in general.
The results are based on perturbation theory, Abelian bosonization, and numerical estimates in the non-perturbative regime through matrix product state (MPS) simulations.
}


\DIFadd{We then study the emergence of so-called helical liquids by repulsive interactions and density-assisted hoppings, resulting from a commensurability between particle density and flux generated from complex hopping elements.
Some of these phases have a natural extension to (fractional) quantum Hall phases in two spatial dimensions, e.g. the Laughlin-like states, and can thus be interpreted as the predecessors of (fractional) topological phases of matter.
The study of strongly correlated helical liquids is therefore important for the bottom-up fabrication of (fractional) topological phases in setups of synthetic quantum matter.
For fermions, we study the emergence of an exotic phase located at particle filling factor $\nu=1/2$ which naturally extends to a fractional quantum Hall phase: the bosonic $K=8$ state.
We then study a similar phase in a ladder of hard-core bosons, located at filling factor $\nu=1$.
In both cases, we derive the phase diagram through renormalization group theory calculations and test our hypotheses through MPS simulations.
}

\DIFadd{A natural concern in the context of topological insulators is the measurement of topological invariants, especially in the case of interacting systems, where a priori it does not amount to a band dispersion quantity.
For systems possessing chiral symmetry, we propose a simple dynamical protocol based on the mean chiral displacement which provides a tomography of the topological index and indicates the presence of symmetry-broken phases through a characteristic divergence.
This protocol is tested using dynamical MPS simulations, and we provide a basic blueprint for an experiment of ultracold atoms trapped in optical lattices.
}\clearpage{}

\DIFaddend \tableofcontents




\mainmatter 

\clearpage{}\chapter*{Introduction}
\addcontentsline{toc}{chapter}{Introduction}
\DIFdelbegin \DIFdel{As an undergraduate student studying elementary physics, I always thought of quantum particles as plane waves living in the differential-geometric world of }\DIFdelend \DIFaddbegin \DIFadd{During the typical lectures of a physicist, quantum mechanics is initially approached in }\DIFaddend ``first quan\-ti\-zation''\DIFdelbegin \DIFdel{which (at least for me) was }\DIFdelend \DIFaddbegin \DIFadd{, where the wave function is explicitly solved using differential geometry.
It can be }\DIFaddend both beautiful and frustrating at the same time as \DIFdelbegin \DIFdel{the task of solving the most simple problems may become quite }\DIFdelend \DIFaddbegin \DIFadd{even simple problems are rather }\DIFaddend involved, if not even impossible \DIFdelbegin \DIFdel{.
The modern approach is not a straightforward attempt to solve the real-space wave function by integrating a given Schrödinger equation: It rather sticks to an alternative representation in the algebraic world of ``second quantization'' where the action of operators dictates all physical consequences.
I must admit that I find the }\DIFdelend \DIFaddbegin \DIFadd{to solve analytically.
``Second quantization'' is more powerful in that respect, which allows to reformulate the same theory in a way that relies on operator actions.
The }\DIFaddend nomenclature of ``first'' and ``second'' \DIFaddbegin \DIFadd{is }\DIFaddend somewhat confusing, as there is no such thing as two \DIFdelbegin \DIFdel{consecutive }\DIFdelend \DIFaddbegin \DIFadd{alternative }\DIFaddend ways of quantization -- it's just two \DIFdelbegin \DIFdel{alternative }\DIFdelend \DIFaddbegin \DIFadd{equivalent }\DIFaddend and consistent ways of describing the same theory.
\DIFdelbegin \DIFdel{Ultimately, I've learned to accept this issue as a result of chronology: the original formulation of quantum mechanics is commonly called first quantization, in which the (motion of the) particle is quantized and possible electromagnetic fields or potentials are considered classical, whereas quantized fields have been formulated in the language of second quantization.
}\DIFdelend However, as we will see soon, the advantage of second quantization manifests itself in a simpler and more efficient way to describe many-body systems such that its development can be seen as the first major cornerstone in the emergence of quantum field theory.

Independent of the formulation, be it first or second, all quantum theories require certain basic concepts:
all quantum states are represented by state vectors $\{|q\rangle\}$ forming a complete basis of the Hilbert space and observables are defined through Hermitian operators acting on that space.
The states are given through a set of good quantum numbers $q$, e.g. for the electron of hydrogen $q=(n,l,m)$ associated with the total energy, angular momentum and its projection along the primary axis.
In \cref{part:theory}, we review these concepts in more detail.
In particular, after a general introduction to second quantization in \DIFdelbegin \DIFdel{\mbox{%DIFAUXCMD
\cref{sec:creation_and_annihilation_operators,sec:representation_of_generic_operators}}\hspace{0pt}%DIFAUXCMD
}\DIFdelend \DIFaddbegin \DIFadd{\mbox{%DIFAUXCMD
\cref{sec:creation_and_annihilation_operators}}\hspace{0pt}%DIFAUXCMD
}\DIFaddend , we motivate the existence of (electronic) bands and the tight binding approximation in \cref{sec:tight_binding_systems}, derived from the Kronig-Penney model in the strong potential limit (see \cref{sec:periodic_potentials}).
As such, these sections are mainly aimed at newcomers in the field.

We then illustrate the treatment of interactions on top of a quadratic/non-interacting Hamiltonian \DIFdelbegin \DIFdel{which is a crucial tool }\DIFdelend \DIFaddbegin \DIFadd{using crucial tools }\DIFaddend in the understanding of \DIFaddbegin \DIFadd{many-body }\DIFaddend quantum matter.
\DIFdelbegin \DIFdel{This becomes particularly simple in the context }\DIFdelend \DIFaddbegin \DIFadd{We introduce the concept }\DIFaddend of Luttinger liquids (see \cref{sec:tomonaga_LL}), from which we derive the quantum field theory satisfying the algebra of a quantum harmonic oscillator.
The beauty of Luttinger liquids is based on the effective theory: it emerges from a class of intrinsically interacting microscopic models and remains quadratic at the operator level.
Therefore, the effective theory is analytically solvable, despite the fact that it describes interacting quantum systems.

We then use the formalism of path integrals (see \cref{sec:properties_of_real_scalar_fields_and_their_correlations}) to derive analytic expressions of the correlation functions.
A common perturbation \DIFdelbegin \DIFdel{which potentially melts the }\DIFdelend \DIFaddbegin \DIFadd{potentially melting a }\DIFaddend Luttinger liquid phase is a sine-Gordon interaction \DIFdelbegin \DIFdel{, which }\DIFdelend \DIFaddbegin \DIFadd{that }\DIFaddend appears naturally in \DIFdelbegin \DIFdel{a system of two coupled Luttinger liquid }\DIFdelend \DIFaddbegin \DIFadd{systems of coupled }\DIFaddend wires (see~\cref{sec:LL_with_spin}).
In order to treat such higher-order interactions analytically, we introduce the concept of renormalization group theory (see \cref{sec:renormalization_group_theory}), \DIFdelbegin \DIFdel{which provides a platform to }\DIFdelend \DIFaddbegin \DIFadd{and }\DIFaddend study if and when the gapless phase evolves to a gapped, massive field theory.

One major drawback is that renormalization group theory relies on many simplifications along the way to make the calculation \DIFdelbegin \DIFdel{amendable}\DIFdelend \DIFaddbegin \DIFadd{amenable}\DIFaddend , and microscopic details of the model are often lost in the approximations.
These details are exchanged in favor of ``universality''\DIFdelbegin \DIFdel{, }\DIFdelend \DIFaddbegin \DIFadd{: }\DIFaddend an effective theory which describes the physical properties of a family of different microscopic models at long scales.
In this framework, it is thus not intended to provide quantitative predictions for the microscopic theories, such that numerical tools are of the essence to probe the qualitative predictions from RG in reality.

A brute force calculation of a full solution scales exponentially with the number of constituents, such that exact diagonalization of interacting models is limited to small system sizes (e.g. \DIFdelbegin \DIFdel{$\sim 20$ }\DIFdelend \DIFaddbegin \DIFadd{$\sim 30$ }\DIFaddend sites for interacting spin-1/2 models in typical studies).
This stresses the need for more efficient numerical techniques in the study of non-perturbative regimes which cannot be \DIFdelbegin \DIFdel{amended }\DIFdelend \DIFaddbegin \DIFadd{tackled }\DIFaddend analytically.
In this thesis, we mainly use matrix product states (MPS), and illustrate the necessary concepts in \cref{ch:matrix_product_states}.
In \cref{sec:tensor_networks,sec:reduced_density_matrix_and_renyi_entropy}, we begin with a general introduction into the field of tensor networks and MPS.
A discussion of two scaling relations of the entanglement entropy is used to explain the effectivity, and the limitations of the MPS \DIFdelbegin \DIFdel{Ansatz }\DIFdelend \DIFaddbegin \DIFadd{ansatz }\DIFaddend (see \cref{sec:scaling_relations_of_the_entanglement_entropy}).
As a natural extension to MPS, we then introduce matrix product operators (MPO) in \cref{sec:matrix_product_operators} which are most useful to express the Hamiltonian, and to calculate observables in general.
After the preliminary introductions, we introduce the basic strategy to approximate ground states based on an MPS \DIFdelbegin \DIFdel{Ansatz }\DIFdelend \DIFaddbegin \DIFadd{ansatz }\DIFaddend in \cref{sec:variational_ground_state_search}.
This algorithm is quite similar to the \DIFdelbegin \DIFdel{(imaginary) }\DIFdelend time evolution presented in \cref{sec:time_evolution} \DIFdelbegin \DIFdel{, which }\DIFdelend \DIFaddbegin \DIFadd{and it }\DIFaddend can also be used to simulate finite temperature systems, presented in \cref{sec:finite_temperature}.
We conclude the chapter by a basic introduction into symmetry invariant MPS formulations, discussed in \cref{sec:exploiting_symmetries}.

\DIFdelbegin \DIFdel{Perhaps the most interesting behavior of quantum many-body physics is that of so-called emergent phenomena -- e.g. the appearance of quasi-particles which extend the canonical statistical properties of fermions and bosons in low dimensions, or the presence of robust quasi-particles at the boundaries of a topological insulator.
For a basic }\DIFdelend \DIFaddbegin \DIFadd{One rapidly growing field in modern physics is topological quantum matter, which falls outside Landau's paradigm to characterize phases of matter with local order parameters.
They are instead associated with more fundamental and global entities that are restricted to integer numbers, called topological invariants.
For an }\DIFaddend overview of this topic, we review the classification of \DIFdelbegin \DIFdel{non-interacting }\DIFdelend topological matter in \cref{ch:topological_phases_of_matter}.
The key concept \DIFdelbegin \DIFdel{for the full classification }\DIFdelend is based on topological band theory, presented in \cref{sec:topological_band_theory}\DIFdelbegin \DIFdel{, a }\DIFdelend \DIFaddbegin \DIFadd{: A }\DIFaddend genuinely non-interacting formulation of symmetry protected topological phases of matter.
The notion of \DIFdelbegin \DIFdel{the topological index }\DIFdelend \DIFaddbegin \DIFadd{topological invariants }\DIFaddend arises from the Berry phase and the Berry connection, \DIFdelbegin \DIFdel{and we recap the original paper.
As an example, }\DIFdelend \DIFaddbegin \DIFadd{which are introduced and explained.
We then explore }\DIFaddend the Su-Schrieffer-Heeger chain \DIFdelbegin \DIFdel{is then explored }\DIFdelend in \cref{sec:the_SSH_chain} to apply the introduced concepts.
We conclude the third chapter by explaining the classification of topological insulators and superconductors in \cref{sec:Periodic_table_of_topological_insulators_and_superconductors}.

In \cref{part:results}, \DIFdelbegin \DIFdel{we summarize our }\DIFdelend \DIFaddbegin \DIFadd{I summarize my }\DIFaddend original contributions to our published works \DIFdelbegin \DIFdel{, which are then presented }\DIFdelend \DIFaddbegin \DIFadd{and present them }\DIFaddend in a thematic manner.
The flexible tuning of transport properties, i.e. the Drude weight, is presented in \cref{drude_increased1}.
The next three works presented in \cref{one_half1,integer1,chiral1} regard the emergence and understanding of strongly correlated helical phases of matter.
In \cref{mcd1}, we propose a dynamical experiment based on the mean chiral displacement to read-out the topological invariant in one-dimensional topological phases protected by chiral symmetry.
This thesis is concluded by a summary and perspectives about interesting future directions.
\clearpage{}

\clearpage{}

\begin{partbacktext}
    \part{Theoretical Prerequisites}
    \label{part:theory}
\end{partbacktext}
\clearpage{}
\clearpage{}\chapter{The quantization of motion and fields}
\label{ch:the_quantization_of_motion_and_fields}
The present chapter is inspired by a number of excellent books and lectures such as~\cite{AshcroftMermin1978,AltlandSimons2010,BruusFlensberg2004,Czycholl2016,FetterWalecka2003,Giamarchi2003,Rizzi2016,Burrello2020} extending the basic aspects of quantum mechanics to a modern way of understanding quantum field theory in general.
\section{\DIFdelbegin \DIFdel{Creation and annihilation operators}\DIFdelend \DIFaddbegin \DIFadd{Quantum field theories for condensed matter}\DIFaddend }
\label{sec:creation_and_annihilation_operators}
Consider a complete set of quantum numbers $\{\alpha\}$ which label a normalized set of states $\{\ket{\alpha}\}$ spanning the full Hilbert space $\HS^1$ of a generic single particle system described by the (time-independent) Schrödinger equation
\begin{align}
    \hat H \ket{\alpha} = \varepsilon_\alpha\ket{\alpha}.
\end{align}
The single particle wave function $\Psi_\alpha(r)$ of a quantum state occupying $\alpha$ is defined as the inner product of the vector $\ket{\alpha}$ with the real-space covector, i.e.
\begin{align}
    \Psi_\alpha(r) \coloneqq \braket{r|\alpha}.
\end{align}
It is \DIFdelbegin \DIFdel{is thus }\DIFdelend understood as the coefficients for the basis transform $\ket{\alpha}\rightarrow\ket{r}$, i.e.
\begin{align}
    \ket{\alpha} = \int{\rm d}{r}\, \Psi_\alpha(r) \ket{r}.
\end{align}
According to the \DIFdelbegin \DIFdel{basic postulate of quantum mechanics}\DIFdelend \DIFaddbegin \DIFadd{spin-statistics theorem}\DIFaddend , the wave function of two indistinguishable particles with quantum numbers $\alpha_1$ and $\alpha_2$ is given by the (anti-)symmetrized product
\begin{align}
    \Psi_{\alpha_1,\alpha_2,\nu}(r_1,r_2) = \frac1{\sqrt2}\left(\braket{r_1|\alpha_1}\braket{r_2|\alpha_2} + \nu\braket{r_2|\alpha_1}\braket{r_1|\alpha_2}\right),
\end{align}
depending on the particle statistics upon exchange of their position, i.e. \DIFdelbegin \DIFdel{$\nu=\pm1$ }\DIFdelend \DIFaddbegin \DIFadd{$\nu=+1$ and $\nu=-1$ }\DIFaddend for bosons and fermions, respectively.
The two-particle wave function can \DIFdelbegin \DIFdel{thus }\DIFdelend be represented by \DIFdelbegin \DIFdel{a more simple }\DIFdelend \DIFaddbegin \DIFadd{an }\DIFaddend inner product
\begin{align}
    \Psi_{\alpha_1,\alpha_2,\nu}(r_1,r_2) = \left(\bra{r_1}\otimes\bra{r_2}\right)\ket{\alpha_1 \alpha_2}_\nu,
\end{align}
which is given by the (anti-)symmetric Kronecker product
\begin{align}
    \ket{\alpha_1 \alpha_2}_\nu = \frac1{\sqrt2}\left(\ket{\alpha_1}\otimes\ket{\alpha_2} + \nu\ket{\alpha_2}\otimes\ket{\alpha_1}\right).
\end{align}
In general, the symmetric $N$-particle state vector is an element of the $N$-particle Hilbert space $\HS^N = \bigotimes_{i=1}^N\HS^1$ and reads
\begin{align}
    \ket{\alpha_1,\alpha_2,\dots,\alpha_N}_\nu = \frac1{\sqrt{N!\prod_{\alpha}(n_\alpha!)}}\sum_P \nu^{(1-\sign{P})/2}\ket{\alpha_{P(1)}}\otimes\ket{\alpha_{P(2)}}\otimes\dots\otimes\ket{\alpha_{P(N)}}.
    \label{eq:symmetric_many_body_state}
\end{align}
In the above equation, we assume ordered quantum numbers (e.g. increasing positions along a wire, or increasing energies), denote the total number of particles with quantum number $\alpha$ as $n_\alpha$ and $\sign{P}$ the sign of the permutation $P\in S^N$ of the permutation group [$\sign{P}=\pm1$ if the permutation is even/odd].

The representation in the ordered expression of \cref{eq:symmetric_many_body_state} is not particularly compact since equal values of $\alpha$ may appear $n_\alpha$ times in the $N$-letter long ket -- the occupation number representation removes this redundancy.
The states in this representation are then given by
\begin{align}
    \ket{n_1, n_2, \dots}_\nu = \ket{\underbrace{\alpha_1,\alpha_1,\dots,\alpha_1}_{n_1}, \underbrace{\alpha_2, \alpha_2,\dots, \alpha_2}_{n_2}, \alpha_3, \dots}
\end{align}
and they span the space $\FS^N$ of the (anti-)symmetrized $N$-particle states $\sum_{i} n_i = N$.
\DIFdelbegin \DIFdel{Thus, the }\DIFdelend \DIFaddbegin \DIFadd{The }\DIFaddend subset $\FS^N\subset \HS^N$ contains all $N$-particle states \DIFdelbegin \DIFdel{which transform according to the basic postulate of quantum mechanics }\DIFdelend \DIFaddbegin \DIFadd{satisfying the spin-statistics theorem }\DIFaddend such that any physical state $\ket{\Psi}\in\HS^N$ can be written as a linear superposition of the Fock states
\begin{align}
    \ket{\Psi}_\nu = \sum_{\sum_i n_i = N}C(n_1,n_2,\dots)\ket{n_1,n_2,\dots}_\nu.
\end{align}
The full Fock space $\FS$ is defined as a direct sum of vector spaces with fixed $N$, i.e.
\begin{align}
    \FS = \bigoplus_{N=0}^\infty \FS^N,
\end{align}
including the one-dimensional vacuum space commonly denoted by $\{\ket{0}\}=\FS^0$.

Let us now impose a linear map $\hat a^\dag_i:\FS\rightarrow\FS$ connecting the individual subsets through
\begin{align}
    \hat a^\dag_i\ket{n_1,\dots,n_i,\dots}_\nu = \sqrt{n_i+1}\nu^{\sum_{j<i}n_j}\ket{n_1,\dots,n_i+1,\dots}_\nu
    \label{eq:creation}
\end{align}
in which fermionic states must be understood ${\rm mod}_2$ such that the Pauli exclusion principle is explicitly satisfied: $\hat a^\dag{}^2\ket{0}=\hat a^\dag\ket{1}={\rm mod}_2(1+1)\ket{{\rm mod}_2(1+1)} = 0\ket{0} = 0$.
Notice that through the linear map we can express the canonical basis of any subset $\FS^N\subset\FS$ as
\begin{align}
    \ket{n_1,n_2,\dots}_\nu = \prod_i\frac1{\sqrt{n_i!}}\left(\hat a^\dag_i\right)^{n_i}\ket{0}_\nu,
    \label{eq:Fock_basis}
\end{align}
and as such have a tool which promotes the vacuum to any state of the full Fock space.
\begin{figure}
    \centering
    \subfigure[]{\includegraphics{figures/connected_fock_space_bosons.pdf}}
    \hfil
    \subfigure[]{\includegraphics{figures/connected_fock_space_fermions.pdf}}
    \caption{(a) Subspaces $\FS^N$ of $N$-particle bosonic states $\ket{N}$ characterized by a single quantum number. Adjacent spaces are connected through the linear maps $a^{(\dag)}$ defined in \cref{eq:creation,eq:annihilation}. (b) Subspaces for a fermionic system characterized by a single quantum number.}
    \label{fig:fock_spaces}
\end{figure}
Notice the absence of the phase $\nu$ on the right hand side which is due to the fact that the product is ordered.
For this reason (and in the following), the linear maps $\hat a^\dag_i$ are commonly called creation operators.
Two different linear maps $j<i$ satisfy the following equation
\begin{align}
    \hat a^\dag_i \hat a^\dag_j\ket{n_1,n_2,\dots}_\nu = \nu \hat a^\dag_j\hat a^\dag_i\ket{n_1,n_2,\dots}_\nu,
\end{align}
and \DIFdelbegin \DIFdel{thus span the famous }\DIFdelend \DIFaddbegin \DIFadd{span what is known as the }\DIFaddend (anti-)commutation relation
\begin{align}
    \commutator{\hat a^\dag_i, \hat a^\dag_j}_\nu \coloneqq \hat a^\dag_i \hat a^\dag_j - \nu \hat a^\dag_j \hat a^\dag_i = 0.
\end{align}
From the Hermitian adjoint of the equations before we get the condition
\begin{align}
   \braket{n_1,\dots,n_i,\dots | \hat a^\pdag_i | m_1, \dots, m_i,\dots}_\nu =
   \sqrt{n_i+1}\nu^{\sum_{j<i}n_j} \delta_{n_1,m_1}\dots\delta_{n_i+1,m_i}\dots
\end{align}
and \DIFdelbegin \DIFdel{thus
}\DIFdelend \DIFaddbegin \DIFadd{as such
}\DIFaddend \begin{align}
   \hat a^\pdag_i \ket{n_1, \dots, n_i,\dots}_\nu =
   \sqrt{n_i}\nu^{\sum_{j<i}n_j} \ket{n_1, \dots, n_i-1,\dots}_\nu,
   \label{eq:annihilation}
\end{align}
from which we obtain the algebra relations of the creation/annihilation operators
\begin{align}
    \commutator{\hat a^\pdag_i,\hat a^\dag_j}_\nu = \delta_{i,j},
    \quad
    \commutator{\hat a^\pdag_i,\hat a^\pdag_j}_\nu = 0,
    \quad
    \commutator{\hat a^\dag_i,\hat a^\dag_j}_\nu = 0.
    \label{eq:Heisenberg_algebra}
\end{align}
In reverse,~\cref{eq:Fock_basis} is a consequence of the Stone-von Neumann theorem which states that, given the Heisenberg algebra defined through~\cref{eq:Heisenberg_algebra}, the action of the operators and the representation of the Fock basis is unique (up to unitary transformations)~\cite{Hall2013}.

To conclude this first section, a change of basis $\{\ket\alpha\}\rightarrow\{\ket\beta\}$ yields\footnote{Remember that $\mathbb1 = \sum_\alpha\ket{\alpha}\bra{\alpha}$, $\ket{\beta} = \sum_\alpha\ket{\alpha}\braket{\alpha|\beta}$ and $\ket{\beta}=\hat a^\dag_\beta\ket{0}$, leading to \cref{eq:creation_annihilation_basis_rotation}.} a unitary transformation of the operators
\begin{align}
    \hat a^\dag_\beta = \sum_\alpha \braket{\alpha|\beta}\hat a^\dag_\alpha,
    \quad
    \hat a^\pdag_\beta = \sum_\alpha \braket{\beta|\alpha}\hat a^\pdag_\alpha
    \label{eq:creation_annihilation_basis_rotation}
\end{align}
which requires only a computation of the single-particle matrix elements $\braket{\alpha|\beta}$.
Before we move on to the representation of observables, a word on common notations: \DIFdelbegin \DIFdel{Quite often the authors }\DIFdelend \DIFaddbegin \DIFadd{authors frequently }\DIFaddend assume a certain particle statistics and operator algebra at the beginning of their work which implies a constant (and \DIFdelbegin \DIFdel{thus }\DIFdelend \DIFaddbegin \DIFadd{therefore often }\DIFaddend dropped) subscript $\nu$.
Fermionic annihilation operators are mostly identified through the letter $\hat c$ whereas $\hat b$ often corresponds to bosonic operators.
Furthermore, a common convention identifies the commutator through \DIFdelbegin \DIFdel{crotchets
}\DIFdelend \DIFaddbegin \DIFadd{square brackets
}\DIFaddend \begin{align}
    \left[\hat A,\hat B\right]\coloneqq\commutator{\hat A,\hat B}_+ = \hat A\hat B - \hat B\hat A
\end{align}
and the anticommutator through curly braces
\begin{align}
    \anticommutator{\hat A,\hat B}\coloneqq\commutator{\hat A,\hat B}_- = \hat A\hat B + \hat B\hat A.
\end{align}
Let me also provide a useful expression to evaluate the commutation relation of operator products recursively
\begin{align}
    \commutator{\hat A, \hat B \hat C}_\pm
    &= \hat A\hat B\hat C \mp \hat B\hat C\hat A + \hat B\hat A\hat C - \hat B\hat A\hat C
    \\
    &= \commutator{\hat A, \hat B}_\pm\hat C \mp \hat B\hat C\hat A \pm \hat B\hat A\hat C
    = \commutator{\hat A, \hat B}_\pm\hat C \mp \hat B\commutator{\hat A,\hat C}.
    \label{eq:recursive_commutation}
\end{align}
In \DIFdelbegin \DIFdel{many cases , the sets of quantum numbers are }\DIFdelend \DIFaddbegin \DIFadd{cases where the quantum number is }\DIFaddend continuous (e.g. position $x$) \DIFdelbegin \DIFdel{and as such a }\DIFdelend \DIFaddbegin \DIFadd{the }\DIFaddend sum in \cref{eq:creation_annihilation_basis_rotation} is promoted to an integral
\DIFdelbegin \DIFdel{expression:
}\DIFdelend \begin{align}
    \hat a(x) = \sum_\alpha \braket{x | \alpha} \hat a_\alpha,
    \quad
    \hat a_\alpha = \int{\rd x} \braket{\alpha | x} \hat a(x).
\end{align}
This is commonly highlighted through a bracket notation of the continuous quantum number.
In conclusion, using the Fock state representation automatically assures the (anti-)symmetric properties of many-body quantum states in real space.
It provides thus an ideal basis for analytical and numerical simulations of many-body states of matter.
\DIFdelbegin \section{\DIFdel{Representation of generic operators}}
%DIFAUXCMD
\addtocounter{section}{-1}%DIFAUXCMD
%DIFDELCMD < \label{sec:representation_of_generic_operators}
%DIFDELCMD < %%%
\DIFdel{Let us start with a general operator acting }\DIFdelend \DIFaddbegin 

\DIFadd{Let us provide the operator form which act }\DIFaddend on a single particle of the full $N$-particle state (usually dubbed ``one-body operator'').
Familiar examples are the momentum or the position operator $\hat x_i$, $\hat p_i$ acting on the $i$th particle, or compositions of these operators like single particle potentials $\hat V(\hat x_i)$.
It is \DIFdelbegin \DIFdel{thus }\DIFdelend not surprising that the general expression of such operators can be given in terms of the particle creation and annihilation operators we introduced in \cref{sec:creation_and_annihilation_operators}.
\DIFdelbegin %DIFDELCMD < 

%DIFDELCMD < %%%
\DIFdelend A one-body operator $\hat o$ diagonal in an arbitrary single-particle basis is \DIFdelbegin \DIFdel{trivially }\DIFdelend \DIFaddbegin \DIFadd{then }\DIFaddend extended to the $N$-particle states written in the same basis\DIFaddbegin \DIFadd{, i.e.
}\DIFaddend \begin{align}
    \hat O_1
    = \sum_{\alpha} o_{\alpha}\hat n_{\alpha}.
\end{align}
This is most easily understood: one-body operators act on only a single entity of the full set of particles, leaving the others untouched.
In a more general basis, the one-body operator transforms according to \cref{eq:creation_annihilation_basis_rotation}, resulting in
\begin{align}
    \hat O_1 = \sum_{\alpha, \beta}\braket{\alpha | \hat o | \beta} \hat a^\dag_\alpha \hat a^\pdag_\beta = \sum_{\alpha,\beta}o_{\alpha,\beta}\hat a^\dag_\alpha \hat a^\pdag_\beta.
    \label{eq:one_body_operator}
\end{align}
Note that in the above, the indices $\alpha$ and $\beta$ denote a full set of quantum numbers of the many-body system.
It is now straightforward to introduce generic two-body operators $\hat O_2$,
\begin{align}
    \hat O_2 = \sum_{\alpha,\beta,\gamma,\delta}O_{\alpha,\beta,\gamma,\delta}\hat a^\dag_\alpha \hat a^\dag_\beta \hat a^\pdag_\gamma \hat a^\pdag_\delta,
\end{align}
in which the expectation value reads $O_{\alpha,\beta,\gamma,\delta}\coloneqq \braket{\alpha,\beta | \hat o | \gamma,\delta}$.
For example, a generic two-point interaction $\hat V\ket{r_1,\dots,r_N}=1/2\sum_{n\neq m}V(r_n-r_m)\ket{r_1,\dots,r_N}$ in continuous space takes the second quantized form
\begin{align}
    \hat V = \frac12\int{\rd^d r}\int{\rd^d r'}V({\bm r}-{\bm r'})\hat a^\dag({\bm r})\hat a^\dag({\bm r'})\hat a^\pdag({\bm r'})\hat a^\pdag({\bm r}).
    \label{eq:two_point_interaction}
\end{align}
The generalization to generic $M$-body operators is now straightforward
\begin{align}
    \hat O_M = \sum_{\alpha_1,\dots,\alpha_M}\sum_{\beta_1,\dots,\beta_M}O_{\alpha_1,\dots,\alpha_M,\beta_1,\dots,\beta_M}\hat a^\dag_{\alpha_1}\dots \hat a^\dag_{\alpha_M}\hat a^\pdag_{\beta_1}\dots \hat a^\pdag_{\beta_M}.
\end{align}
\section{Periodic potentials}
\label{sec:periodic_potentials}
To really see when second quantization becomes useful, we go one step back and review basic properties of a single particle moving in a periodic potential which is effectively characterized by a Hamiltonian composed of generic one-body operators of the form of \cref{eq:one_body_operator}.
In particular, the Hamiltonian considered here reads
\begin{align}
    \hat H_0 = \int\rd^d r\, \hat a^\dag({\bm r})\brlr{\frac{\hat{\bm p}^2}{2m}+V_{ae}({\bm r})}\hat a^\pdag({\bm r})
\end{align}
with operators $\hat a^\pdag({\bm r})$ annihilating a particle at position $\bm r$.
The local potential $V_{ae}$ is a collection of $N_a$ identical potentials,
\begin{align}
    V_{ae}({\bm r}) = \sum_{i=1}^{N_a}v_{ae}({\bm R}_i - {\bm r}),
\end{align}
and the value of $v_{ae}$ is determined by the relative distance from the lattice position ${\bm R}_i$.
If the creation operators satisfy the anticommutator algebra in real space
\begin{align}
    \anticommutator{\hat a^\pdag({\bm r}),\hat a^\dag({\bm r'})}=\delta({\bm r}-{\bm r'}),
\end{align}
the Hamiltonian defines a spinless fermionic system embedded in an arbitrary lattice.
A regular (Bravais) lattice structure in $d$ dimensions is in general spanned by $d$ linearly independent (not necessary mutually perpendicular and normalized) vectors ${\bm x}_i$ and can \DIFdelbegin \DIFdel{thus }\DIFdelend be defined as the set
\begin{align}
    \mathcal{R} = \anticommutator{\sum_{i=1}^d n_i {\bm x}_i,\ n_i\in\mathds Z}.
\end{align}
The beauty of this approach becomes visible as soon as we define the lattice translation operator
\begin{align}
    \hat T_{\bm n} : \psi_\alpha({\bm r}) \mapsto \psi_\alpha\brlr{{\bm r}+T_{\bm n}},\ T_{\bm n}\coloneqq\sum_i n_i {\bm x}_i={\bm n}\underline{\bm x}
\end{align}
which translates every function from one position to a distance parametrized through the $d$-dimensional vector ${\bm n}\in\mathds Z^d$ and the collection of all lattice vectors $\underline{\bm x}\coloneqq ({\bm x}_1,\dots,{\bm x}_d)^T$.
The local potential $V_{ae}({\bm r})$ is by definition invariant under lattice translations, and two-body interactions $V({\bm r}-{\bm r'})$ are clearly invariant under continuous translations, such that the full Hamiltonian $\hat H$ commutes with the translation operator and we can write the eigenfunctions of $\hat H$ as eigenfunctions of $\hat T_{\bm n}$.
The eigenfunctions of all translation operators $\hat T_{\bm n}$ are called Bloch waves and have the following properties\footnote{Note here that a further restriction on the values of the vector ${\bm k}$ arise from imposing boundary conditions. For instance, Born-von Karman boundary conditions imply a periodicity after $L_j$ unit translations $\psi_\alpha({\bm r}+ L_j{\bm R}_j)=\psi_\alpha({\bm r})$, \DIFdelbegin \DIFdel{which confines }\DIFdelend \DIFaddbegin \DIFadd{confining }\DIFaddend the allowed values of $k_j$ to integer multiples of $\frac{\bm G_j}{L_j}$ in which $\bm G_j$ is the reciprocal vector of ${\bm R}_j$, i.e. ${\bm G}_j {\bm R}_j = 2\pi/a$.}
\DIFdelbegin %DIFDELCMD < \begin{align}
%DIFDELCMD <     \hat T_{\bm n}\psi_{\alpha}({\bm r}) &= c_{\bm n}\psi_{\alpha}({\bm r})\\
%DIFDELCMD <     c_{{\bm n}_1}c_{{\bm n}_2} &= c_{{\bm n}_1+{\bm n}_2} \Rightarrow c_{\bm n} = \re^{{\bm s}\cdot{\bm n}\underline{\bm x}},\ {\bm s}\in\mathds C^d\\
%DIFDELCMD <     1=\int_V\rd^dr\,\abs{\psi_{\alpha}({\bm r})}^2 &= \int_V\rd^dr\,\abs{\hat T_{\bm n}\psi_{\alpha}({\bm r})}^2 \Rightarrow 1=\abs{c_{\bm n}}^2 \Rightarrow {\bm s}=\ri{\bm k},\ {\bm k}\in\mathds R^d
%DIFDELCMD < \end{align}%%%
\DIFdelend \DIFaddbegin \begin{align}
    \hat T_{\bm n}\psi_{\alpha}({\bm r}) &= c_{\bm n}\psi_{\alpha}({\bm r}),\\
    c_{{\bm n}_1}c_{{\bm n}_2} &= c_{{\bm n}_1+{\bm n}_2} \Rightarrow c_{\bm n} = \re^{{\bm s}\cdot{\bm n}\underline{\bm x}},\ {\bm s}\in\mathds C^d,\\
    1=\int_V\rd^dr\,\abs{\psi_{\alpha}({\bm r})}^2 &= \int_V\rd^dr\,\abs{\hat T_{\bm n}\psi_{\alpha}({\bm r})}^2 \Rightarrow 1=\abs{c_{\bm n}}^2 \Rightarrow {\bm s}=\ri{\bm k},\ {\bm k}\in\mathds R^d.
\end{align}\DIFaddend 
It is now possible to phrase Bloch's theorem~\cite{Bloch1929}, which states that the eigenfunctions of a particle moving in periodic potentials assume the simple structure
\begin{align}
    \psi_\alpha({\bm r}) = \psi_{\alpha{\bm k}}({\bm r}) = \re^{\ri{\bm k}\cdot {\bm r}}u_{\alpha{\bm k}}({\bm r}).
    \label{eq:bloch_theorem}
\end{align}
Note that we introduced the translation operators eigenvalue argument $\bm k$ in the list of good quantum numbers.
In particular, $\psi$ is a weighted plane wave in which $u$ inherits the lattice periodicity
\begin{align}
    \hat T_{\bm n} u_{\alpha{\bm k}}({\bm r})
    =
    \re^{-\ri{\bm k}\cdot \brlr{{\bm r}+{\bm n}\underline{\bm x}}}c_{\bm n}\psi_{\alpha{\bm k}}({\bm r})
    =
    \re^{-\ri{\bm k}\cdot {\bm r}}\re^{-\ri{\bm k}\cdot {\bm n}\underline{\bm x}}\re^{\ri{\bm k}\cdot {\bm n}\underline{\bm x}}\psi_{\alpha{\bm k}}({\bm r})
    =
    u_{\alpha{\bm k}}({\bm r}).
\end{align}
In general, the wave vector $\bm k$ seems to play the same role as the particle momentum $\bm p$ in the Sommerfeld theory of a free particle.
However, this is not true as is clear from the momentum operator being $\hat{\bm p}=-\ri\hbar\bm \nabla$, and as such
\begin{align}
  \hat{\bm p}\psi_{\alpha{\bm k}}({\bm r}) = \hbar{\bm k}\psi_{\alpha{\bm k}} - \ri\hbar\re^{\ri{\bm k}{\bm r}}\bm\nabla u_{\alpha \bm k}(\bm r).
\end{align}
The vector ${\bm k}$ is \DIFdelbegin \DIFdel{thus }\DIFdelend dubbed crystal momentum, remembering the fact that $\psi_{\alpha{\bm k}}$ is only an eigenstate of momentum if the potential is constant\DIFdelbegin \DIFdel{and thus }\DIFdelend \DIFaddbegin \DIFadd{, i.e. }\DIFaddend $\bm\nabla V_{ae}=0$.

As we will see in the next section, \DIFdelbegin \DIFdel{the }\DIFdelend Bloch waves can be used to define a complete set of tightly localized ``Wannier'' states \DIFdelbegin \DIFdel{, which allows to work with }\DIFdelend \DIFaddbegin \DIFadd{that allow to introduce }\DIFaddend effective ``tight binding\DIFdelbegin \DIFdel{approximations'' }\DIFdelend \DIFaddbegin \DIFadd{'' models }\DIFaddend in which the coupling parameters of the model are implicitly defined through overlap integrals of the Bloch functions.
\DIFdelbegin \DIFdel{This way, the }\DIFdelend \DIFaddbegin \DIFadd{A basic understanding of tight binding approximations from strong periodic potentials is essential, as we rely on such models in all of our works as a starting point to study it's }\DIFaddend many-body \DIFdelbegin \DIFdel{phases of the effective model can be studied in a more efficient way without the explicit knowledge of the coupling constants.
Nonetheless, to get an idea of the form of the Bloch functions, let us solve the Schrödinger equation of a particle moving along one dimension in the potential
}%DIFDELCMD < \begin{align}
%DIFDELCMD <     V_{ae}(x) = V_0 \Theta\brlr{b-{\rm mod}_{a}\commutator{x+b}},\ V_0,a,b\in\mathds R_+
%DIFDELCMD <     \label{eq:kronig_penney_potential}
%DIFDELCMD < \end{align}%%%
\DIFdelend \DIFaddbegin \DIFadd{phases.
With that purpose in mind, let us study the Kronig-Penney model~\mbox{%DIFAUXCMD
\cite{KronigPenney1931}}\hspace{0pt}%DIFAUXCMD
, which is defined in one spatial dimension with the potential
}\begin{align}
    V_{ae}(x) = V_0 \Theta\brlr{b-{\rm mod}_{a}\commutator{x+b}},\ V_0,a,b\in\mathds R_+,
    \label{eq:kronig_penney_potential}
\end{align}\DIFaddend 
such that the nontrivial contours are of size $b$ and displaced by a factor $a$ called the lattice spacing.
\DIFdelbegin \DIFdel{This model is called the Kronig-Penney model~\mbox{%DIFAUXCMD
\cite{KronigPenney1931} }\hspace{0pt}%DIFAUXCMD
and }\DIFdelend \DIFaddbegin \DIFadd{It }\DIFaddend provides analytic solutions of \DIFaddbegin \DIFadd{tightly }\DIFaddend bound electrons in the limit $b\rightarrow0$ and $V_0\rightarrow\infty$ such that $bV_0={\rm const.}$, which is depicted in \cref{fig:kronig_penney_potential}.
\begin{figure}
    \centering
    \includegraphics{figures/kronig_penney_potential.png}
    \caption{Different contours of the periodic potential \cref{eq:kronig_penney_potential} for $bV_0=1/4$. The gray boxes corresponds to the fixed values $V_0=1/3$ and $b=3/4$, while the blue gradient visualizes the limit $b\rightarrow0$ and $V_0\rightarrow\infty$ while preserving the product $bV_0=P$.}
    \label{fig:kronig_penney_potential}
\end{figure}

To simplify \DIFdelbegin \DIFdel{our }\DIFdelend \DIFaddbegin \DIFadd{the }\DIFaddend problem, we can consider two different regions, i.e. (i) a free particle $-\frac{\hbar^2}{2m}\partial_x^2\psi_{\rm (i)}(x) = E\psi_{\rm (i)}(x)$ and (ii) a particle moving in a constant potential $-\frac{\hbar^2}{2m}\partial_x^2\psi_{\rm (ii)}(x) = (E-V_0)\psi_{\rm (ii)}(x)$.
In order to read out the Bloch weights we conveniently factor a crystal momentum from the wave function
\begin{align}
    \psi_{\alpha k}^{\rm (i)}(x) = \re^{\ri kx}u_{\alpha k}^{\rm (i)}(x),
    \
    \psi_{\beta k}^{\rm (ii)}(x) = \re^{\ri kx}u_{\beta k}^{\rm (ii)}(x),
    \\
    u_{\alpha k}^{\rm (i)}(x) = A\re^{(\tau\alpha-\ri k) x} + A'\re^{-(\tau\alpha+\ri k) x},
    \
    u_{\beta k}^{\rm (ii)}(x) = B\re^{(\tau'\beta-\ri k) x} + B'\re^{-(\tau'\beta+\ri k) x},
\end{align}
with $\hbar^2\alpha^2 = 2m\abs{E}$, $\hbar^2\beta^2 = 2m\abs{(E-V_0)}$, $\tau^2=-\sign{E}$ and $\tau'^2=-\sign{E-V_0}$.
Plane waves are \DIFdelbegin \DIFdel{thus }\DIFdelend obtained in case of $E>V_0$, such that $\tau=\tau'=\ri$.
The wave functions are supposed to be smooth at the boundaries\footnote{We hereby use the notation of limits from above or below, i.e. $f(0^\pm)=\lim_{\epsilon\rightarrow0}f(\pm\epsilon)$ for $\epsilon>0$.}, which results in
\begin{align}
\psi_{\alpha k}^{\rm (i)}(-b+0^-)=\psi_{\beta k}^{\rm (ii)}(-b+0^+),
    \quad
    {\psi_{\alpha k}^{\rm (i)}}'(-b+0^-)={\psi_{\beta k}^{\rm (ii)}}'(-b+0^+),
\end{align}
and the Bloch functions inherit the potential's periodicity
\begin{align}
    u_{\alpha k}^{\rm (i)}(a+0^+) = u_{\alpha k}^{\rm (ii)}(0^-),
    \quad
    u_{\alpha k}^{\rm (i)}{}'(a+0^+) = u_{\alpha k}^{\rm (ii)}{}'(0^-).
\end{align}
In summary, the following matrix \DIFdelbegin \DIFdel{equation is derived
}%DIFDELCMD < \begin{align}
%DIFDELCMD <     &\qquad M =\\
%DIFDELCMD <     &\begin{pmatrix}
%DIFDELCMD <         1 & 1 & -1 & -1\\
%DIFDELCMD < \tau\alpha & -\tau\alpha & -\tau'\beta & \tau'\beta\\
%DIFDELCMD < \re^{(\tau \alpha-\ri k)(a-b)}  & \re^{-(\tau \alpha+\ri k) (a-b)} &
%DIFDELCMD <         -\re^{-(\tau'\beta -\ri k)b}     & -\re^{ (\tau'\beta+\ri k)b} \\
%DIFDELCMD < (\tau \alpha-\ri k)\re^{(\tau \alpha-\ri k)(a-b)}  & -(\tau \alpha+\ri k)\re^{-(\tau \alpha+\ri k) (a-b)} &
%DIFDELCMD <         -(\tau'\beta -\ri k)\re^{-(\tau'\beta -\ri k)b}     & (\tau'\beta+\ri k)\re^{ (\tau'\beta+\ri k)b} \nonumber\\
%DIFDELCMD <     \end{pmatrix},
%DIFDELCMD < \end{align}%%%
\DIFdelend \DIFaddbegin \DIFadd{is derived
}{\footnotesize
\begin{align}
    M =
    \begin{pmatrix}
        1 & 1 & -1 & -1\\
\tau\alpha & -\tau\alpha & -\tau'\beta & \tau'\beta\\
\re^{(\tau \alpha-\ri k)(a-b)}  & \re^{-(\tau \alpha+\ri k) (a-b)} &
        -\re^{-(\tau'\beta -\ri k)b}     & -\re^{ (\tau'\beta+\ri k)b} \\
(\tau \alpha-\ri k)\re^{(\tau \alpha-\ri k)(a-b)}  & -(\tau \alpha+\ri k)\re^{-(\tau \alpha+\ri k) (a-b)} &
        -(\tau'\beta -\ri k)\re^{-(\tau'\beta -\ri k)b}     & (\tau'\beta+\ri k)\re^{ (\tau'\beta+\ri k)b} \nonumber\\
    \end{pmatrix}
\end{align}\DIFaddend 
\DIFdelbegin \DIFdel{satisfying }\DIFdelend \DIFaddbegin }\DIFadd{which satisfies }\DIFaddend $M(A,A',B,B')^T=0$.
For nontrivial results, the determinant of $M$ should be equal to zero, which is satisfied for solutions of the transcendental equation
\begin{align}
    \cos(ka)
    =
    \cosh(\alpha\tau(a-b))\cosh(b\beta\tau')
    +
    \frac{\alpha^2\tau^2+\beta^2\tau'^2}{2\alpha\beta\tau\tau'}\sinh(\alpha\tau(a-b))\sinh(b\beta\tau).
    \label{eq:kronig_penney_transcendental_equation}
\end{align}
To approximate the right hand side in the aforementioned limits, application of
\begin{align}
    b\rightarrow0,
    \quad
    V_0\rightarrow\infty,
    \quad
    bV_0 = {\rm const.}
    \\
    \Rightarrow
    b\beta^2\rightarrow 2mbV_0/\hbar^2,
    \quad
    \cosh(b\beta\tau')\rightarrow 1,
    \quad
    \sinh(b\beta\tau')\rightarrow b\beta\tau'
\end{align}
provides an exact rewriting of \cref{eq:kronig_penney_transcendental_equation} in the limit of narrow and strong periodic potentials\footnote{This limit is actually equivalent to a potential composed by delta-distributions.}
\begin{align}
    f(\alpha a) \coloneqq \cosh(\alpha\tau a) + \tau'^2\frac{P}{\alpha\tau a}\sinh(\alpha\tau a),
    \quad
    P \coloneqq mabV_0/\hbar^2.
    \label{eq:kronig_penney_transcendental_equation_approx}
\end{align}
If we assume bound states between the potential wells $0<E<V_0$, the signs become $\tau^2=-1$ and $\tau'^2=+1$, such that \cref{eq:kronig_penney_transcendental_equation_approx} evaluates to
\begin{align}
    f(\alpha a) = \cos(\alpha a) + \frac{P}{\alpha a}\sin(\alpha a).
    \label{eq:kronig_penney_transcendental_equation_approx_bound}
\end{align}
For nonzero $P$, the right-hand-side of \cref{eq:kronig_penney_transcendental_equation_approx_bound} is not bound to the interval $[-1,+1]$ spanned by the left-hand-side of \cref{eq:kronig_penney_transcendental_equation}, and establishes values of $\alpha$ (\DIFdelbegin \DIFdel{thus, }\DIFdelend the square-root of the energy $E$) for which no (real) momentum exists (see \cref{fig:kronig_penney_dispersion} (a)).
Such energies are called forbidden and allow for a first understanding of band-gaps induced by the nonzero lattice potential $V_0>0$.
\begin{figure}
    \centering
    \subfigure[]{\includegraphics{figures/kronig_penney_transcendental.png}}
    \subfigure[]{\includegraphics{figures/kronig_penney_dispersion_2.png}}
    \subfigure[]{\includegraphics{figures/kronig_penney_dispersion_1.png}}
    \caption{
    (a) Plot of the right-hand-side of the transcendental equation. Solutions do not exist in the red regions.
    (b)-(c) Shapes of the dispersion relation $(\alpha a/\pi)^2$ versus crystal momentum $ka/\pi$ given by \cref{eq:kronig_penney_transcendental_equation_approx_bound}.
    Different opacity (transparent to colors) represent increasing values of $P\in\{0.1,1,5,20,50,1000\}$.
    (c) The properties of the wave functions allow to uniquely relate the crystal momentum $ka$ to an energy $\alpha a$ in which the limit of free electrons [i.e. $P\rightarrow0$] is pronounced.}
    \label{fig:kronig_penney_dispersion}
\end{figure}

If $P=0$, we are left with the energy-momentum relation of free electrons $k=\alpha$ and \DIFdelbegin \DIFdel{thus }\DIFdelend $E_0={\hbar^2k^2}/({2m})$.
If we approach $P\rightarrow\infty$, the allowed energies are formed by the roots of $\sin(\alpha a)$, i.e.
\begin{align}
    E_{\infty,n_b}=(\hbar n_b\pi)^2/(2ma^2),
    \label{eq:kronig_penney_energy_tb}
\end{align}
for which the level spacing is given by the squares of integer numbers ($n_b=1,2,\dots$).
The intermediate regimes $0<P<\infty$ can be solved by numerical evaluation of \cref{eq:kronig_penney_transcendental_equation_approx_bound} and are plotted in \cref{fig:kronig_penney_dispersion} (a).
Straightforward evaluation of $\arccos(f)$ yields the so-called ``reduced zone scheme'' displayed in \cref{fig:kronig_penney_dispersion} (b).

To get an analytic understanding of the wave functions in the above limit, let's focus on the region without potential (i) [remember that one is interested in the limit $b\rightarrow0$]
\begin{align}
    \psi_{\alpha k}^{\rm (i)}(x) = \re^{\ri kx}u_{\alpha k}^{\rm (i)}(x),
    \quad
    u_{\alpha k}^{\rm (i)}(x) = A\re^{\ri(\alpha-k) x} + A'\re^{-\ri(\alpha+k) x},
    \quad
    u_{\alpha k}^{\rm (i)}(x+na)=u_{\alpha k}^{\rm (i)}(x)
    \label{eq:kronig_penney_wavefunctions}
\end{align}
in which the prefactors are related by\footnote{An additional constraint for $AA^*$ is found by requiring normalization of the wave functions, which allows to compute expectation values. However, it is not needed for the purpose of this section and we refer to \cite{KronigPenney1931}.}
\begin{align}
    A' = -A\frac{1-\re^{\ri(\alpha-k)a}}{1-\re^{-\ri(\alpha+k)a}}.
    \label{eq:kronig_penney_constants}
\end{align}

Starting from \cref{eq:kronig_penney_transcendental_equation_approx_bound}, we see that for any solution $\alpha a$ satisfying $f(\alpha a)=\cos(k a)$ there is an infinite number of symmetric points in momentum space which fulfill the same equation, i.e. (a) $ka + 2n\pi$ and (b) $-ka+2n\pi$.
Close inspection of \cref{eq:kronig_penney_wavefunctions,eq:kronig_penney_constants} reveals that a transformation according to (a) leaves the wave function invariant, while (b) maps it to the solutions of $-ka$, $-\alpha a$.

In other words, (a) is merely a shift by a unit reciprocal vector and (b) flips the direction of the propagating wave, hence establishes a mirror symmetry at the $ka=0$ axis.
The properties (a) and (b) allow to define the ``unfolding'' of the reduced zone scheme:
without loss of generality, a crystal momentum $n_b\pi\leq ka<(n_b+1)\pi$ can be uniquely connected to an energy value $n_b\pi\leq \alpha a<(n_b+1)\pi$ by introducing a band index $(n_b=0,1,\dots)$.
Here, the limit of free particles ($P=0$) is readily restored, since \cref{eq:kronig_penney_constants} vanishes for $\alpha a = ka$ (except for the special points $ka=n_b\pi$).

At the special points $ka=n_b\pi$, \cref{eq:kronig_penney_constants} is actually ill defined and evaluates to the two viable solutions $A'=\pm A$, which can be understood from the two non-commuting limits
\begin{align}
    -\lim_{\alpha a \searrow n_b\pi}\frac{1-\re^{\ri(\alpha a-n_b\pi))}}{1-\re^{-\ri(\alpha a+n_b\pi)}}
    =
    -\lim_{\epsilon\searrow0}\frac{1-\re^{+\ri\epsilon}}{1-\re^{-\ri\epsilon}}
    =
    \lim_{\epsilon\searrow0}\re^{+\ri\epsilon}\frac{1-\re^{-\ri\epsilon}}{1-\re^{-\ri\epsilon}}
    =
    +1,
    \\
    -\lim_{k a \searrow n_b\pi}\frac{1-\re^{\ri(n_b\pi - ka))}}{1-\re^{-\ri(n_b\pi+ka)}}
    =
    -\lim_{k a \searrow n_b\pi}\frac{1-\re^{-\ri(n_b\pi + ka))}}{1-\re^{-\ri(n_b\pi+ka)}}
    =
    -1.
\end{align}
The two allowed eigenfunctions are standing waves $\propto \cos(kx),\sin(kx)$.
By reformulating the special points in terms of the particles' de Broglie wavelength $\lambda=2\pi/k$, the formation of standing waves can be understood as a residual effect of (constructive) Bragg reflection on a periodic grid structure.
Standing waves are obtained if the particle's de Broglie wavelength and the spacing of the potential satisfy $2a=n_b\lambda$.

In the limit $P\rightarrow\infty$, the energy $E_{\infty,n_b}$ assumes the discrete values in \cref{eq:kronig_penney_energy_tb} resulting in motionless eigenstates -- the resulting waves are tightly bound to the potential minimum.
If we relax this limit a bit, i.e. $P\gg 1$, a reasonable approximation of the lowest band curvature is obtained by a first order Taylor series of \cref{eq:kronig_penney_transcendental_equation_approx_bound} resulting in the typical dispersion relation for tight binding systems, i.e.
\begin{align}
    E_{P\gg1,1}\approx t_0 + t_1 \cos(k a),
    \quad
    t_0 = + E_{\infty,1} - \frac{\pi^2 \hbar^4}{a^3 m^2 b V_0},
    \quad
    t_1 = -\frac{\pi^2\hbar^4}{a^3 m^2 bV_0}.
    \label{eq:kronig_penney_tight_binding_dispersion}
\end{align}
This concludes the pedagogic \DIFdelbegin \DIFdel{recap }\DIFdelend \DIFaddbegin \DIFadd{review }\DIFaddend of the Kronig-Penney model, and we proceed by \DIFdelbegin \DIFdel{giving a more generic recipe to solve arbitrary periodic potentials.
}\DIFdelend \DIFaddbegin \DIFadd{commenting on the tight binding approximation in more detail.
}\DIFaddend 

\DIFdelbegin \DIFdel{In order to tackle generic potentials, let us expand the Schrödinger equation in reciprocal space through the following identities}\footnote{\DIFdel{The reciprocal space provides a way to Fourier-transform as the functions $\re^{\ri {\bm G r}}$ form a basis on the primitive cell of the real lattice over the square-integrable functions. In particular, the functions satisfy the orthogonality equation $\delta_{\bm G, \bm G'}=\frac1{V}\int\rd^dr\,\re^{\ri({\bm G-\bm G'}){\bm r}}$.}}
%DIFAUXCMD
\addtocounter{footnote}{-1}%DIFAUXCMD
%DIFDELCMD < \begin{align}
%DIFDELCMD <     V_{ae}({\bm r}) = \sum_{\bm G}V_{ae}{}_{\bm G}\re^{\ri {\bm G r}},
%DIFDELCMD <     \quad
%DIFDELCMD <     V_{ae}{}_{\bm G} = \frac1{V}\int\rd^dr\,V_{ae}({\bm r})\re^{-\ri {\bm G r}},
%DIFDELCMD <     \\
%DIFDELCMD <     u_{n{\bm k}}({\bm r}) = \sum_{\bm G}u_{n{\bm k}}{}_{\bm G}\re^{\ri {\bm G r}},
%DIFDELCMD <     \quad
%DIFDELCMD <     u_{n{\bm k}}{}_{\bm G} = \frac1{V}\int\rd^dr\,u_{n{\bm k}}({\bm r})\re^{-\ri {\bm G r}},
%DIFDELCMD < \end{align}
%DIFDELCMD < %%%
\DIFdel{in which $V$ is the volume of the primitive unit cell.
Straightforward evaluation yields the algebraic eigenvalue problem for the unknown functions $u_{n{\bm k}}{}_{\bm G}$
}%DIFDELCMD < \begin{align}
%DIFDELCMD <     \frac{\hbar^2}{2m}({\bm G}+{\bm k})^2u_{n{\bm k}}{}_{\bm G}+\sum_{\bm G'}V_{ae}{}_{\bm G-\bm G'}u_{n{\bm k}}{}_{\bm G'} = E_{n{\bm k}}.
%DIFDELCMD <     \label{eq:periodic_lattices_numerics}
%DIFDELCMD < \end{align}
%DIFDELCMD < %%%
\DIFdel{The dimension of the linear equation is infinite due to the sum over all reciprocal lattice vectors and has to be truncated if one wants to solve the equation numerically (the convergence of such a truncation has to be carefully checked).
Clearly, strongly confined potentials such as delta functions studied in the Kronig-Penney model are particularly bad candidates to solve numerically through evaluation of \mbox{%DIFAUXCMD
\cref{eq:periodic_lattices_numerics} }\hspace{0pt}%DIFAUXCMD
because the resulting matrix equation will not be sparse and any truncation will result in a significant error.
}\DIFdelend \section{Tight binding systems}
\label{sec:tight_binding_systems}
The systems considered here are those of tightly bound constituents to the lattice centers.
Such types can be found in traditional solid state scenarios where the nuclei are well separated beyond the typical Bohr radius of the valence electrons, in setups of ultracold atoms trapped in optical lattices~\cite{Bloch2008}, in photonic waveguides~\cite{Lu2014} or polaritons~\cite{Amo2016}.
In all our works, we cover theoretical aspects of interacting tight binding models which can be experimentally realized in a variety of different setups.
For this reason, it will be useful to review briefly how these models are motivated from first principles, and how they can be understood in second quantization.

\DIFdelbegin %DIFDELCMD < \begin{figure}
%DIFDELCMD <     \centering
%DIFDELCMD <     \subfigure[]{\includegraphics{figures/wannier1_1.png}}
%DIFDELCMD <     \subfigure[]{\includegraphics{figures/wannier1_10.png}}
%DIFDELCMD <     \subfigure[]{\includegraphics{figures/wannier1_100.png}}\\
%DIFDELCMD <     \subfigure[]{\includegraphics{figures/wannier2_1.png}}
%DIFDELCMD <     \subfigure[]{\includegraphics{figures/wannier2_10.png}}
%DIFDELCMD <     \subfigure[]{\includegraphics{figures/wannier2_100.png}}\\
%DIFDELCMD <     %%%
%DIFDELCMD < \caption{%
{%DIFAUXCMD
\DIFdelFL{Example Wannier functions of the first two bands for a periodic potential of the form $V_{ae}(x)=V_0\cos(2\pi/ax)$ for different potential depth. The integration of the Bloch states was performed by assuming a periodicity over $L=50$ lattice translations.}}
    %DIFAUXCMD
%DIFDELCMD < \label{fig:tight_binding_wanniers}
%DIFDELCMD < \end{figure}
%DIFDELCMD < 

%DIFDELCMD < %%%
\DIFdel{For this purpose}\DIFdelend \DIFaddbegin \DIFadd{Here}\DIFaddend , we assume the problem of the single-particle Hamiltonian $\hat H_0$ to be fully solved, such that the Bloch states $\psi_{\alpha{\bm k}}$ diagonalize $\hat H_0$ and have energy eigenvalues $\varepsilon_{\alpha{\bm k}}$.
This allows to introduce the Wannier basis -- a localized basis composed of the Bloch states and defined as
\begin{align}
    \ket{w_{\alpha{\bm R}}} = \frac1{\sqrt N}\sum_{\bm k}\re^{-\ri{\bm k}{\bm R}}\ket{\psi_{\alpha{\bm k}}}.
    \label{eq:wannier_states}
\end{align}
Note that every momentum-resolved Bloch function can be multiplied with a complex phase without changing its properties.
This naturally provides a gauge freedom to optimize the Wannier function's properties -- for instance, the construction of a maximally localized basis~\cite{Marzari2012}.
\DIFdelbegin \DIFdel{Without going into detail about optimizing Wannier functions, we want to present a basic visualization of these localized states.
For this purpose, let us consider a cosine periodic potential
}%DIFDELCMD < \begin{align}
%DIFDELCMD <     V_{ae}(x) = V_0\cos\brlr{\frac{2\pi}a x}
%DIFDELCMD < \end{align}
%DIFDELCMD < %%%
\DIFdel{which yields a particular easy (a tridiagonal Toeplitz) matrix equation for the Bloch vectors presented in \mbox{%DIFAUXCMD
\cref{eq:periodic_lattices_numerics}}\hspace{0pt}%DIFAUXCMD
.
To obtain the Wannier functions, we gauge every Bloch function to be purely real at $x=0$, resulting in the examples displayed in \mbox{%DIFAUXCMD
\cref{fig:tight_binding_wanniers}}\hspace{0pt}%DIFAUXCMD
.
}\DIFdelend 


The existence of localized Wannier states is translated to the language of second quantization by the notion that transformations between a Bloch and Wannier basis are always unitary.
Hence, annihilation and creation operators of Wannier and Bloch states are set in relation by
\begin{align}
    \hat a^\dag_{\alpha{\bm R}} = \frac1{\sqrt N}\sum_{\bm k}\re^{-\ri{\bm k}{\bm R}}\hat a^\dag_{\alpha{\bm k}},
    \quad
    \hat a^\dag_{\alpha{\bm k}} = \frac1{\sqrt N}\sum_{\bm R}\re^{+\ri{\bm k}{\bm R}}\hat a^\dag_{\alpha{\bm R}}.
    \label{eq:wannier_states_2}
\end{align}
Since the non-interacting Hamiltonian is diagonal in the Bloch basis, they are the eigenfunctions with energies $\varepsilon_{\alpha{\bm k}} = \int\rd^dr\, \psi_{\alpha{\bm k}}^* \hat H_0 \psi_{\alpha{\bm k}} = \frac1N\sum_{\bm R,\bm R'}\re^{\ri\bm k(\bm R'-\bm R)}\int\rd^dr\, w_{\alpha{\bm R}}^* \hat H_0 w_{\alpha{\bm R'}}$.
In particular, the Hamiltonian is readily cast into the localized basis according to
\begin{align}
    \hat H_0
    =
    \varepsilon_{\alpha{\bm k}}\hat a^\dag_{\alpha{\bm k}}\hat a^\pdag_{\alpha{\bm k}}
    \overset{\text{\cref{eq:wannier_states_2}}}{=}
    \frac1N
    \re^{\ri{\bm k}\brlr{{\bm R}-{\bm R'}}}
    \varepsilon_{\alpha{\bm k}}
    \hat a^\dag_{\alpha{\bm R}}\hat a^\pdag_{\alpha{\bm R'}}
    =
    T_{\alpha,i,j}
    \hat a^\dag_{\alpha{\bm R}_i}\hat a^\pdag_{\alpha{\bm R}_j},
    \label{eq:tight_binding_hamiltonian}
\end{align}
for which we conveniently use the sum convention.
The matrix $T_{\alpha}$ contains all amplitudes of transition processes between two lattice centers, to be determined through the dispersion relation of the $\alpha$-band
\begin{align}
    T_{\alpha,i,j} \coloneqq \frac1N\sum_{\bm k}\re^{\ri{\bm k}\brlr{{\bm R}_i-{\bm R}_j}}\varepsilon_{\alpha{\bm k}} = \frac1N\sum_{\bm k}\re^{\ri{\bm k}\brlr{{\bm R}_i-{\bm R}_j}}\int\rd^dr\, \psi_{\alpha{\bm k}}^* \hat H_0 \psi_{\alpha{\bm k}}.
\end{align}
On an intuitive level, the tight binding Hamiltonian denotes quantum particles hopping between lattice sites, connected through the matrix elements $T_{\alpha,i,j}$.
Without quantitative computations of the transition probabilities, we can fix the matrix to $T_{\alpha,i,j}=\sum_rt_{\alpha r}\delta_{i-r,j} + \hc$ with some constants $t_{\alpha r}$.
The relation \cref{eq:tight_binding_hamiltonian} has strong implications on the analytic form of the dispersion relation $\varepsilon_{\alpha{\bm k}}$ -- it is fully determined through the geometry of the crystal lattice.
For instance, one-dimensional lattices with single atom unit cells and lattice spacing $a$ have the particularly easy solution
\begin{align}
    \varepsilon_{\alpha k} = \sum_r 2t_{\alpha r}\cos(r ka).
    \label{eq:1D_tight_binding_dispersion}
\end{align}
Note that this equation is consistent with the Kronig-Penney dispersion relation in the tight-binding limit, presented in \cref{eq:kronig_penney_tight_binding_dispersion}.
In higher dimensions, the evaluation of the dispersion relation may become lengthy, but remains always analytic.

Oftentimes, a single-band approximation is established to fix (and drop) the band index $\alpha$.
This situation may be achieved in case the bottom band is sufficiently separated from the second.
In the Kronig-Penney model, this corresponds to $V_0\gg \hbar^2/(mab)$ with particle mass $m$, lattice constant $a$ and potential width \DIFdelbegin \DIFdel{$b$}\DIFdelend \DIFaddbegin \DIFadd{$b\ll a$}\DIFaddend .
For the emergence and discussion of such an approximation in case of ultracold atoms trapped in optical lattices, we refer to~\cite{Jaksch1998,Bloch2008,Buechler2010,Mazza2012}.
Note that the single band approximation naturally fails \DIFdelbegin \DIFdel{do }\DIFdelend \DIFaddbegin \DIFadd{to }\DIFaddend describe effects like orbital-selective Mott transitions which require at least two inequivalent bands~\cite{Anisimov2002,vanDongen2005}.

A corresponding two-body interaction $\hat V$ in the single-band approximation reads
\begin{align}
    \hat V = \frac12\sum_{i,i',j,j'}V_{i,i',j,j'}\hat a^\dag_{{\bm R}_i}\hat a^\dag_{{\bm R}_{i'}}\hat a^\pdag_{{\bm R}_j}\hat a^\pdag_{{\bm R}_{j'}}.
\end{align}
The matrix elements of the interaction are given by the integral expressions
\begin{align}
    V_{i,i',j,j'} = \int\rd^dr\int\rd^dr'\,w^*_{{\bm R}_i}({\bm r})w^*_{{\bm R}_{i'}}({\bm r}')w_{{\bm R}_j}({\bm r}')w_{{\bm R}_{j'}}({\bm r})V({\bm r-\bm r'}).
    \label{eq:two_body_interaction_transition_rates}
\end{align}
The explicit determination of the matrix elements requires knowledge of the form of the Wannier states which is an active field of research on its own.
However, these states are localized and as such the transition rate integrals are short-ranged in most cases.
In one \DIFdelbegin \DIFdel{dimension, it is often sufficient to account }\DIFdelend \DIFaddbegin \DIFadd{spatial dimension, the simplest nontrivial tight binding model accounts }\DIFaddend for transitions and interactions up to nearest neighbors \DIFaddbegin \DIFadd{only, i.e.
}\DIFaddend \begin{align}
    T_{i,j} \approx t_0\delta_{i,j} + t_1\delta_{i-1,j}+t_1^*\delta_{i+1,j},
    \quad
    V_{i,i',j,j'} \approx U\delta_{i,j'}\delta_{i',j}\delta_{i,i'} + V\delta_{i,j'}\delta_{i',j}(\delta_{i-1,i'}+\delta_{i+1,i'}).
\end{align}
Note that in case the particle has some flavor (e.g. spin), the kinetic terms of $\hat H_0$ typically act in a diagonal manner.
On the contrary, the two-body interaction allows for both intra-flavor and inter-flavor scattering processes.

We are at liberty to slightly simplify the notation and arrive at the family of single-orbital Hubbard-type models, written in second quantization
\DIFdelbegin %DIFDELCMD < \begin{align}
%DIFDELCMD <     \hat H_{\rm Hubbard} = \sum_{i,s}\brlr{t_1 \hat a^\dag_{i,s}\hat a^\pdag_{j+1,s} + \hc} + \frac U2\sum_{i,s,s'}\hat n_{i,s}\hat n_{i,s'} + V\sum_{i,s,s'}\hat n_{i,s}\hat n_{i+1,s'}.
%DIFDELCMD <     \label{eq:hubbard_hamiltonian}
%DIFDELCMD < \end{align}%%%
\DIFdelend \DIFaddbegin \begin{align}
    \hat H_{\rm Hubbard} = \sum_{i,s}\brlr{t_1 \hat a^\dag_{i,s}\hat a^\pdag_{j+1,s} + \hc} + U\sum_{i,s,s'}\hat n_{i,s}\hat n_{i,s'} + V\sum_{i,s,s'}\hat n_{i,s}\hat n_{i+1,s'}.
    \label{eq:hubbard_hamiltonian}
\end{align}\DIFaddend 
The operator $\hat a_{i,s}$ annihilates a Wannier state of flavor $s$, localized around the $i$'th lattice position.
In most cases, $t_1 = -t$ with $t>0$, which is also physically motivated to confine the minimum of the dispersion to $k=0$.
The validity of the single-band approximation for real materials requires small interaction amplitudes compared to the band gap between the first and second Bloch band.

If the matrix $T$ defines a lattice grid with loops, complex transition amplitudes give rise to nontrivial fluxes penetrating the lattice which may alter the underlying physics.
\DIFdelbegin \DIFdel{The famous Peierls' substitution approximates the effect of a uniform magnetic field}\DIFdelend \DIFaddbegin \DIFadd{Such a scenario may occur in case of additional magnetic fields}\DIFaddend , in which case the Hamiltonian follows the principle of minimal coupling
\begin{align}
    \hat H_0 = \frac{\hat{\bm p}^2}{2m}+V_{ae}(\bm r)\rightarrow \frac{\brlr{\hat{\bm p} - q\bm A(\bm r)}^2}{2m}+V_{ae}(\bm r),
\end{align}
and $q$ denotes the electric charge of the particle.
Consider now the \DIFdelbegin \DIFdel{rescaled Wannier functions }\DIFdelend \DIFaddbegin \DIFadd{Wannier functions multiplied with a phase factor
}\DIFaddend \begin{align}
    w'_{\alpha\bm R} = \re^{\ri\frac q\hbar G_{\bm R}}w_{\alpha\bm R},
    \quad
    \psi'_{\alpha\bm k} = \frac1N\sum_{\bm R}\re^{\ri\bm k\bm R}w'_{\alpha\bm R}
\end{align}
in which $G_{\bm R}(\bm r) = \int_{\bm R}^{\bm r}\bm A(\bm r')\cdot\rd \bm r'$ is a lattice-position and position dependent phase.
In certain situations, this provides a neat substitution to treat the troublesome $\hat{\bm p}\bm A$ term.
In fact, $\bm\nabla G_{\bm R}(\bm r) = \bm A(\bm r) + \int_0^1\rd\lambda\lambda (\bm r-\bm R)\times\bm B(\bm R+\lambda(\bm r-\bm R))$ with magnetic field $\bm B=\bm\nabla\times\bm A$~\cite{Luttinger1951}.
To observe the influence of the magnetic field, the transition elements between different Wannier states have to be computed, i.e.
\DIFdelbegin %DIFDELCMD < \begin{align}
%DIFDELCMD < T_{\alpha,i,j}
%DIFDELCMD <     =
%DIFDELCMD <     \int\rd^d r
%DIFDELCMD <         w_{\alpha\bm R_i}^*
%DIFDELCMD <         \re^{-\ri\frac q\hbar G_{\bm R_i}}
%DIFDELCMD <         \brlr{\frac{\brlr{\hat{\bm p}-q\bm A}^2}{2m}+V_{ae}}
%DIFDELCMD <         w_{\alpha \bm R'}
%DIFDELCMD <         \re^{\ri\frac q\hbar G_{\bm R_j}}
%DIFDELCMD <     \\
%DIFDELCMD <     =
%DIFDELCMD <     \re^{-\ri\frac q\hbar\int_{\bm R_i}^{\bm R_j}\bm A(\bm r')\cdot\rd \bm r'}
%DIFDELCMD <     \int\rd^d r
%DIFDELCMD <         \re^{\ri\frac q\hbar\Phi_{\bm R_i,\bm R_j}(\bm r)}
%DIFDELCMD <         w_{\alpha\bm R_i}^*
%DIFDELCMD <         \brlr{\frac{\brlr{\hat{\bm p}-q\bm A+q\bm\nabla G_{\bm R_j}}^2}{2m}+V_{ae}}
%DIFDELCMD <         w_{\alpha \bm R_j},
%DIFDELCMD < \end{align}%%%
\DIFdelend \DIFaddbegin \begin{align}
T_{\alpha,i,j}
    =
    \int\rd^d r
        w_{\alpha\bm R_i}^*
        \re^{-\ri\frac q\hbar G_{\bm R_i}}
        \brlr{\frac{\brlr{\hat{\bm p}-q\bm A}^2}{2m}+V_{ae}}
        w_{\alpha \bm R'}
        \re^{\ri\frac q\hbar G_{\bm R_j}}
    \\
    =
    \re^{-\ri\frac q\hbar\int_{\bm R_i}^{\bm R_j}\bm A(\bm r')\cdot\rd \bm r'}
    \int\rd^d r
        \re^{\ri\frac q\hbar\Phi_{\bm R_i,\bm R_j}(\bm r)}
        w_{\alpha\bm R_i}^*
        \brlr{\frac{\brlr{\hat{\bm p}-q\bm A+q\bm\nabla G_{\bm R_j}}^2}{2m}+V_{ae}}
        w_{\alpha \bm R_j},
    \label{eq:pre_peierls_approximation}
\end{align}\DIFaddend 
in which $\Phi_{\bm R_i,\bm R_j}(\bm r) = \oint_{\bm R_j\rightarrow\bm r\rightarrow\bm R_i}\bm A(\bm r')\cdot\rd \bm r'$ is the flux through the plaquette spanned by $\bm R_i, \bm R_j$ and $\bm r$.
Expanding to first order in $B=|\bm B|$, the transition elements are dressed with a complex phase factor and result \DIFdelbegin \DIFdel{to the famously known Peierls approximation }%DIFDELCMD < \begin{align}
%DIFDELCMD <     T_{\alpha,i,j}'=\re^{\ri\frac q\hbar\int_{\bm R_j}^{\bm R_i}\bm A(\bm r')\cdot\rd\bm r'}T_{\alpha,i,j}+\mathcal O(B).
%DIFDELCMD < \end{align}%%%
\DIFdelend \DIFaddbegin \DIFadd{in
}\begin{align}
    T_{\alpha,i,j}'=\re^{\ri\frac q\hbar\int_{\bm R_j}^{\bm R_i}\bm A(\bm r')\cdot\rd\bm r'}T_{\alpha,i,j}+\mathcal O(B).
    \label{eq:peierls_substitution}
\end{align}\DIFaddend 
\DIFdelbegin \DIFdel{The transition elements correspond to the change in the Berry phase on the lattice scale~(for a discussion and implications of this effect, see \mbox{%DIFAUXCMD
\cref{ch:topological_phases_of_matter}}\hspace{0pt}%DIFAUXCMD
) }\DIFdelend \DIFaddbegin \DIFadd{On a more abstract level, the approximation is often implemented and explored as the so-called ``Peierls substitution method'' (fully ignoring the error) to study intriguing phases of matter known as topological insulators}\footnote{\DIFadd{The complex transition elements induce a change of the Berry phase on the lattice scale. For a discussion and implications of this effect, see \mbox{%DIFAUXCMD
\cref{ch:topological_phases_of_matter}}\hspace{0pt}%DIFAUXCMD
.}}\DIFadd{~\mbox{%DIFAUXCMD
\cite{Ozawa20192}}\hspace{0pt}%DIFAUXCMD
.
It is important to note that the phase space of these studies is largely inaccessible by a realistic vector potential~\mbox{%DIFAUXCMD
\cite{Alexandrov1991,Julen2014}}\hspace{0pt}%DIFAUXCMD
, and other methods like Raman-assisted tunneling~\mbox{%DIFAUXCMD
\cite{Garcia2012} }\hspace{0pt}%DIFAUXCMD
must be used to achieve the desired complex phase modulation of the tunneling elements}\DIFaddend .
\DIFdelbegin \DIFdel{Note that the amplitudes of the transitions are altered as well, which must be considered in the implementation of real experiments~\mbox{%DIFAUXCMD
\cite{Alexandrov1991}}\hspace{0pt}%DIFAUXCMD
.
}\DIFdelend 


\DIFdelbegin \DIFdel{Perhaps }\DIFdelend \DIFaddbegin \DIFadd{One of }\DIFaddend the most successful description of electrons in solids is band theory -- based on screened many-body interactions described by effective one-body potentials.
This results in a form of \cref{eq:hubbard_hamiltonian} without density-density interactions such that a diagonalization of the hopping matrix $T$ solves the problem.
However, due to its intrinsic single-particle character, band theory cannot reliably capture truly many-body features such as band magnetism or Mott-to-metal-insulator transitions.
This motivates the study of Hubbard-type models: despite being a brutal simplification of the true two-body interaction, they are the simplest systems that provide an explanation of such interaction-driven features.
Despite its apparent innocence, there is no universal treatment of Hubbard models in general.
In one spatial dimension, the Hamiltonian can be solved analytically using the Bethe \DIFdelbegin \DIFdel{Ansatz and thus }\DIFdelend \DIFaddbegin \DIFadd{ansatz and }\DIFaddend falls in the category of being ``integrable''~\cite{Essler2005}.
In general, Bethe \DIFdelbegin \DIFdel{Ansatz }\DIFdelend \DIFaddbegin \DIFadd{ansatz }\DIFaddend integrability is a fragile property and even slight perturbations will break it.
The models we study in the next chapters, albeit closely related to the one-dimensional Hubbard model, do not categorize as integrable due to the presence of additional terms that break some of its fundamental symmetries (e.g. the conservation of the particle flavor).
This motivates the use of alternative methods to study the properties of interacting tight binding models.
One of the most prominent and useful analytic concepts in one dimension is \DIFdelbegin \DIFdel{the theory of Luttinger liquids (}\DIFdelend \DIFaddbegin \DIFadd{bosonization, }\DIFaddend paired with renormalization group theory\DIFdelbegin \DIFdel{) }\DIFdelend \DIFaddbegin \DIFadd{, }\DIFaddend which approximates the microscopic model in the low temperature limit.
\DIFaddbegin \DIFadd{In many of our works (see \mbox{%DIFAUXCMD
\cref{drude_increased1,chiral1,one_half1,integer1}}\hspace{0pt}%DIFAUXCMD
), we rely on (abelian) bosonization and perform a renormalization group theory study, which is why we introduce these concepts in more detail in the next sections.
}\DIFaddend \section{Tomonaga-Luttinger liquids}
\label{sec:tomonaga_LL}
Tomonaga-Luttinger liquids are gapless interacting states of matter appearing in many one-dimensional quantum systems.
Their understanding was formalized at the beginning of the '80s by Duncan Haldane~\cite{Haldane1981} through the application of a technique called \DIFdelbegin \DIFdel{Abelian }\DIFdelend \DIFaddbegin \DIFadd{abelian }\DIFaddend bosonization.
It was understood that the low-lying excitations of these fermionic interacting models can be approximated with free bosons.
In this section, we give a basic overview of the mapping from spinless fermions to bosonic particle-hole excitations in the vicinity of the Fermi points.
\DIFaddbegin \DIFadd{This mapping can be straightforwardly applied in many different contexts, e.g. in our work in~\mbox{%DIFAUXCMD
\cref{drude_increased1}}\hspace{0pt}%DIFAUXCMD
, which characterizes the anomalous zero temperature transport properties of a partially gapped interacting fermionic ladder model.
}\DIFaddend 

The main interest here is to achieve an effective model in one dimension capturing the relevant degrees of freedom at low temperatures and low energies.
For that purpose, let \DIFdelbegin \DIFdel{me start with }\DIFdelend \DIFaddbegin \DIFadd{us start from }\DIFaddend the free fermionic 1D Hamiltonian $\hat H_0$ denoted by
\DIFdelbegin \DIFdel{the following (diagonal) representation in momentum space
}%DIFDELCMD < \begin{align}
%DIFDELCMD <     \hat H_0 = \sum_k \frac{(\hbar k)^2}{2m}\hat n_k
%DIFDELCMD <     \label{eq:hamiltonian_free_particles}
%DIFDELCMD < \end{align}%%%
\DIFdelend \DIFaddbegin \begin{align}
    \hat H_0 = \sum_k \frac{(\hbar k)^2}{2m}\hat n_k,
    \label{eq:hamiltonian_free_particles}
\end{align}\DIFaddend 
with dispersion relation $\varepsilon_k =\frac{k^2}{2m}$ depicted in \cref{fig:1D_quadratic_dispersion}.
\begin{figure}
    \centering
    \includegraphics{figures/1D_quadratic_dispersion.png}
    \caption{Quadratic dispersion relation with approximations close to the Fermi energy $\varepsilon_F$.}
    \label{fig:1D_quadratic_dispersion}
\end{figure}
Close to the Fermi energy $\varepsilon_F\coloneqq\varepsilon_{\pm k_F}$, one may linearize the free dispersion to obtain a system of two different species
\begin{align}
    \hat H_0
    &\approx \sum_k \hbar^2\brlr{\varepsilon_F \pm \frac{k_F}{m}(k\mp k_F)}\hat n_k
    \\
    &= \sum_q \hbar^2\brlr{\varepsilon_F + \frac{k_F}{m}q\brlr{\hat n_{q+k_F}-\hat n_{q-k_F}}}
    \approx \sum_{q} v_F \hbar q \brlr{\hat c^\dag_{q,R}\hat c^\pdag_{q,R} - \hat c^\dag_{q,L}\hat c^\pdag_{q,L}},
    \label{eq:dispersion_linearization}
\end{align}
which implies a restriction of $q$ to a small window $|q|<\Gamma\ll k_F$ beyond which \cref{eq:dispersion_linearization} is considered invalid.
Note the introduction of the so-called right and left operators $\hat c_{q,R/L}$ which annihilate particles propagating to the left/right with Fermi velocity $\pm v_F=\hbar k_F/m$\DIFaddbegin \footnote{\DIFadd{Curvature effects can (and in some cases must) be considered, which improves the quality of the effective low energy description. A detailed discussion of these ``nonlinear Luttinger liquids'' is beyond the scope of this thesis, and we refer to~\mbox{%DIFAUXCMD
\cite{Imambekov2009}}\hspace{0pt}%DIFAUXCMD
.}}\DIFaddend .

For the next part, it is convenient to investigate the local density in momentum space, i.e.
\begin{align}
    \hat n(x) = \hat c^\dag_x \hat c^\pdag_x = \frac1L\sum_{k,q}\re^{-\ri x q}\hat c^\dag_{k+q}\hat c^\pdag_{k}.
    \label{eq:local_density}
\end{align}
$\hat n(x)$ \DIFdelbegin \DIFdel{thus }\DIFdelend creates a superposition of particle-hole plane waves with characteristic wavelength $q^{-1}$.
The bare creation of particle-hole pairs is defined through the density operator
\begin{align}
    \hat \rho_{-q}\coloneqq \sum_k \hat c^\dag_{k-q}\hat c^\pdag_{k}.
\end{align}
Note \DIFdelbegin \DIFdel{further }\DIFdelend that $\hat\rho_q^\dag = \hat\rho_{-q}$.
By confining the theory close to the Fermi points, there are only two different classes of particle-hole excitations with $q\approx 0$ and $q\approx\pm2k_F$.
The long-wavelength excitations $q\approx0$ are particle-hole pairs of the same species (left or right movers), and excitations of $q\approx\pm2k_F$ are particle-hole pairs of a mixture of the two.
This implies drastic consequences for the relevant action of operators, which we will see in the following.
Note that the density operator of the left/right species $\hat\rho_{\tau,q}$, $\tau\in\{L,R\}$ applied to a Fermi sea creates stable particle-hole excitations (i.e. particles and holes propagate with the same velocity $\pm v_F$) and can thus be used to construct a complete basis of the subspace $\FS^N$ -- for this discussion, we refer to~\cite{vonDelft1998}.
The consequences of the approximation in \cref{eq:dispersion_linearization} is easily understood for the single-particle operators
\DIFdelbegin %DIFDELCMD < \begin{align}
%DIFDELCMD <     \hat c^\dag_x = \frac1{\sqrt L}\sum_k \re^{-\ri k x}\hat c^\dag_k \approx \frac1{\sqrt L}\sum_{|q|<\Gamma}\re^{-\ri (q+k_F) x}\hat c^\dag_{q,R} + \re^{-\ri (q-k_F) x}\hat c^\dag_{q,L}
%DIFDELCMD <     \label{eq:confinement_creation_annihilation}
%DIFDELCMD < \end{align}%%%
\DIFdelend \DIFaddbegin \begin{align}
    \hat c^\dag_x = \frac1{\sqrt L}\sum_k \re^{-\ri k x}\hat c^\dag_k \approx \frac1{\sqrt L}\sum_{|q|<\Gamma}\re^{-\ri (q+k_F) x}\hat c^\dag_{q,R} + \re^{-\ri (q-k_F) x}\hat c^\dag_{q,L},
    \label{eq:confinement_creation_annihilation}
\end{align}\DIFaddend 
which is then used to find the local density
\DIFdelbegin %DIFDELCMD < \begin{align}
%DIFDELCMD <     \hat n(x)
%DIFDELCMD <     &\approx \frac1L\sum_{q,q'}\brlr{\re^{-\ri (q+k_F) x}\hat c^\dag_{q,R} + \re^{\ri (k_F-q) x}\hat c^\dag_{q,L}}\brlr{\re^{\ri (k_F+q') x}\hat c^\pdag_{q',R} + \re^{-\ri (k_F-q') x}\hat c^\pdag_{q',L}},
%DIFDELCMD <     \\
%DIFDELCMD <     &= \hat \rho_R(x) + \hat \rho_L(x) + \re^{-2\ri k_Fx}\hat c^\dag_{R}(x)\hat c^\pdag_{L}(x) + \re^{2\ri k_Fx}\hat c^\dag_{L}(x)\hat c^\pdag_{R}(x).
%DIFDELCMD <     \label{eq:local_density_approximation}
%DIFDELCMD < \end{align}%%%
\DIFdelend \DIFaddbegin \begin{align}
    \hat n(x)
    &\approx \frac1L\sum_{q,q'}\brlr{\re^{-\ri (q+k_F) x}\hat c^\dag_{q,R} + \re^{\ri (k_F-q) x}\hat c^\dag_{q,L}}\brlr{\re^{\ri (k_F+q') x}\hat c^\pdag_{q',R} + \re^{-\ri (k_F-q') x}\hat c^\pdag_{q',L}}
    \\
    &= \hat \rho_R(x) + \hat \rho_L(x) + \re^{-2\ri k_Fx}\hat c^\dag_{R}(x)\hat c^\pdag_{L}(x) + \re^{2\ri k_Fx}\hat c^\dag_{L}(x)\hat c^\pdag_{R}(x).
    \label{eq:local_density_approximation}
\end{align}\DIFaddend 
The first two terms correspond to the $q\approx0$ part of the density, and scattering occurs on the same side of the dispersion relation.
The last two terms scatter right with left movers and transfer particles from one side to the other, which appear at $q\approx\pm2k_F$.

We now turn to an arbitrary two-body interaction of the form~\cref{eq:two_point_interaction} which reads
\DIFdelbegin %DIFDELCMD < \begin{align}
%DIFDELCMD <     \hat V
%DIFDELCMD < &= \frac12\int\rd r\int\rd x V(r)\hat c^\dag(r+x)\hat c^\dag(x)\hat c^\pdag(x)\hat c^\pdag(r+x),
%DIFDELCMD <     \\
%DIFDELCMD <     &= \frac1{2L^2}\sum_{kk'll'}\int\rd r\int\rd x V(r)\re^{-\ri r(k-l')}\re^{-\ri x(k-l'+k'-l)}\hat c^\dag_k\hat c^\dag_{k'}\hat c^\pdag_l\hat c^\pdag_{l'},
%DIFDELCMD < \\
%DIFDELCMD <     &= \frac1{2L}\sum_{kk'q}V(q)\hat c^\dag_k\hat c^\dag_{k'}\hat c^\pdag_{k'+q}\hat c^\pdag_{k-q}
%DIFDELCMD <     = \frac1{2L}\sum_{q}V(q)\hat\rho_q\hat\rho^\dag_{q} - \mu.
%DIFDELCMD <     \label{eq:two_body_interaction_momentum_space}
%DIFDELCMD < \end{align}%%%
\DIFdelend \DIFaddbegin \begin{align}
    \hat V
&= \frac12\int\rd r\int\rd x V(r)\hat c^\dag(r+x)\hat c^\dag(x)\hat c^\pdag(x)\hat c^\pdag(r+x)
    \\
    &= \frac1{2L^2}\sum_{kk'll'}\int\rd r\int\rd x V(r)\re^{-\ri r(k-l')}\re^{-\ri x(k-l'+k'-l)}\hat c^\dag_k\hat c^\dag_{k'}\hat c^\pdag_l\hat c^\pdag_{l'}
\\
    &= \frac1{2L}\sum_{kk'q}V(q)\hat c^\dag_k\hat c^\dag_{k'}\hat c^\pdag_{k'+q}\hat c^\pdag_{k-q}
    = \frac1{2L}\sum_{q}V(q)\hat\rho_q\hat\rho^\dag_{q} - \mu.
    \label{eq:two_body_interaction_momentum_space}
\end{align}\DIFaddend 
The last term is just a constant $\mu = \frac N{2L}\sum_qV(q)$ and can \DIFdelbegin \DIFdel{thus }\DIFdelend be neglected.
By imposing that relevant contributions act close to the Fermi energy involving only momenta in the interval $|k|\in[k_F-\Gamma,k_F+\Gamma]$, we can split the sum in two contributions, one involving scattering processes at small and the other scattering at large momenta
\begin{align}
    \hat V \approx \frac1{2L}\brlr{\sum_{q\approx0}V(q)\hat\rho_{q}\hat\rho^\dag_{q} + \sum_{q\approx2k_F}V(q)\hat\rho_{q}\hat\rho^\dag_{q}}.
\end{align}
An intuitive classification of the scattering processes is given in~\cref{fig:scattering_processes}.
\begin{figure}
    \centering
    \subfigure[]{\includegraphics[width=0.328\textwidth]{figures/g4_right.pdf}}
    \subfigure[]{\includegraphics[width=0.328\textwidth]{figures/g4_left.pdf}}
    \subfigure[]{\includegraphics[width=0.328\textwidth]{figures/g2.pdf}}
    \subfigure[]{\includegraphics[width=0.328\textwidth]{figures/g1.pdf}}
    \subfigure[]{\includegraphics[width=0.328\textwidth]{figures/g13.pdf}}
    \subfigure[]{\includegraphics[width=0.328\textwidth]{figures/g22.pdf}}
    \caption{Relevant scattering processes of a generic density-density interaction in one-dimensional quantum systems. (a)/(b) The depicted scattering is commonly referred to as ``forward scattering'' $g_4$ process ($4$ right/left operators, $q\approx0$), (c) as ``backscattering'' $g_2$ process (containing $2$ pairs of right and left operators, $q\approx0$) and (d) as $g_1$ process with momentum transfer $q\approx 2k_F$. Other possible scatterings like the ones depicted in (e) and (f) require the existence of high-energy excitations and are \DIFdelbeginFL \DIFdelFL{thus }\DIFdelendFL exponentially suppressed at low temperatures.}
    \label{fig:scattering_processes}
\end{figure}
This simple argumentation allows to consider only the most relevant processes at low temperatures/energies, i.e. those presented in panels (a) - (d).
We will call those processes forward scattering
\begin{align}
    g_4\sum_{k,k',q}\hat c^\dag_{k+q,\tau}\hat c^\dag_{k'-q,\tau}\hat c^\pdag_{k',\tau}\hat c^\pdag_{k,\tau} = g_4 \sum_q\hat\rho^\pdag_{q,\tau}\hat\rho^\dag_{q,\tau},
\end{align}
backscattering
\DIFdelbegin %DIFDELCMD < \begin{align}
%DIFDELCMD <     g_2\sum_{k,k',q}\hat c^\dag_{k+q,\tau}\hat c^\dag_{k'-q,\overline\tau}\hat c^\pdag_{k',\overline\tau}\hat c^\pdag_{k,\tau} = g_2\sum_q \hat\rho^\pdag_{q,\tau}\hat\rho^\dag_{q,\overline\tau}
%DIFDELCMD < \end{align}%%%
\DIFdelend \DIFaddbegin \begin{align}
    g_2\sum_{k,k',q}\hat c^\dag_{k+q,\tau}\hat c^\dag_{k'-q,\overline\tau}\hat c^\pdag_{k',\overline\tau}\hat c^\pdag_{k,\tau} = g_2\sum_q \hat\rho^\pdag_{q,\tau}\hat\rho^\dag_{q,\overline\tau},
\end{align}\DIFaddend 
and $g_1$ scattering
\begin{align}
    g_1\sum_{k,k',q} \hat c^\dag_{k+q,\tau}\hat c^\dag_{k'-q,\overline\tau}\hat c^\pdag_{k',\tau}\hat c^\pdag_{k,\overline\tau}.
\end{align}
Note that constants on the right hand side of the previous equations are neglected.
The $g_1$ scattering of spinless fermions can be rewritten according to
\begin{align}
    \sum_{k,k',q}V(q)
    \hat c^\dag_{k+q,\tau}\hat c^\dag_{k'-q,\overline\tau}\hat c^\pdag_{k',\tau}\hat c^\pdag_{k,\overline\tau}
    =
    \sum_{k,k',q,q'}V(q) \delta_{q,q'-k+k'}
    \hat c^\dag_{k'+q',\tau}\hat c^\dag_{k-q',\overline\tau}\hat c^\pdag_{k',\tau}\hat c^\pdag_{k,\overline\tau}
    \\
    =
    -\sum_{k,k',q}V(k'-k+q)\hat c^\dag_{k+q,\tau}\hat c^\dag_{k'-q,\overline\tau}\hat c^\pdag_{k',\overline\tau}\hat c^\pdag_{k,\tau}
    \approx -V(2k_F)\sum_q\hat\rho^\pdag_{q,\tau}\hat\rho^\dag_{q,\overline\tau},
    \label{eq:umklapp_backscattering_equivalence}
\end{align}
which proofs that $g_1$ terms are up to a sign equivalent to backscattering terms in case of indiscernible particles.
Interestingly, the $2k_F$ components of forward scatterings can be reordered in the same manner\footnote{This is interesting because one is inclined to neglect the $2k_F$ forward scattering processes at first sight. The reordering demonstrates that $q\approx0$ are equivalent to $2k_F$ scatterings such that both have to be accounted for on equal grounds.}, such that we find the identity $g_4 = g_2 \approx V(0)-V(2k_F)$.
Other processes like those depicted in \cref{fig:scattering_processes} (e) and (f) require the existence of high-energy excitations and can \DIFdelbegin \DIFdel{thus }\DIFdelend be neglected for the effective low temperature theory developed here.
In summary, we can express the interaction in terms of the density-density operators as
\begin{align}
    \hat V \approx \frac{\hbar}{2L}\sum_{q,\tau}\brlr{g_4\hat\rho^\pdag_{q,\tau}\hat\rho^\dag_{q,\tau} + g_2\hat\rho^\pdag_{q,\tau}\hat\rho^\dag_{q,\overline\tau}}
    =
    \frac{\hbar}{L}\sum_{q>0}
    \begin{pmatrix}
        \hat\rho^\pdag_{q,R} & \hat\rho^\pdag_{q,L}
    \end{pmatrix}
    \begin{pmatrix}
        g_4 & g_2 \\
        g_2 & g_4
    \end{pmatrix}
    \begin{pmatrix}
        \hat\rho_{-q,R} \\ \hat\rho_{-q,L}
    \end{pmatrix}
    .
    \label{eq:interaction_densities}
\end{align}
To proceed further, it will be useful to compute the commutation of the density operators
\DIFdelbegin %DIFDELCMD < \begin{align}
%DIFDELCMD <     \commutator{\hat\rho_{q,\tau},\hat\rho_{q',\tau'}}
%DIFDELCMD <     =
%DIFDELCMD <     \delta_{\tau,\tau'}\sum_k\brlr{\hat c^\dag_{k+q,\tau}\hat c^\pdag_{k-q',\tau}-\hat c^\dag_{k+q+q',\tau}\hat c^\pdag_{k,\tau}}
%DIFDELCMD <     \approx
%DIFDELCMD <     -\sigma_\tau\delta_{\tau,\tau'}\delta_{q,-q'}\frac{qL}{2\pi}
%DIFDELCMD <     \label{eq:chiral_density_commutation}
%DIFDELCMD < \end{align}%%%
\DIFdelend \DIFaddbegin \begin{align}
    \commutator{\hat\rho_{q,\tau},\hat\rho_{q',\tau'}}
    =
    \delta_{\tau,\tau'}\sum_k\brlr{\hat c^\dag_{k+q,\tau}\hat c^\pdag_{k-q',\tau}-\hat c^\dag_{k+q+q',\tau}\hat c^\pdag_{k,\tau}}
    \approx
    -\sigma_\tau\delta_{\tau,\tau'}\delta_{q,-q'}\frac{qL}{2\pi},
    \label{eq:chiral_density_commutation}
\end{align}\DIFaddend 
in which $\sigma_\tau=\pm1$ for $\tau=R/L$, respectively, and the right hand side is obtained through a projection on the ground state\footnote{This is a reasonable approximation for small interactions only (as a result from a first order perturbative expansion).
For a more thorough discussion of \cref{eq:chiral_density_commutation}, see~\cite{Giamarchi2003}.
\DIFdelbegin \DIFdel{Although this approximation may turn out to be too rough for real systems, it allows to obtain the Luttinger liquid field theory.}\DIFdelend }.
We are now at liberty to define canonical bosonic operators representing the interaction degrees of freedom for $q>0$
\begin{align}
    \hat b^\dag_{+q} \coloneqq \sqrt\frac{2\pi}{qL}\hat\rho_{-q,L},
    \quad
    \hat b^\pdag_{+q} \coloneqq \sqrt\frac{2\pi}{qL}\hat\rho_{+q,L},
    \\
    \hat b^\dag_{-q} \coloneqq \sqrt\frac{2\pi}{qL}\hat\rho_{+q,R},
    \quad
    \hat b^\pdag_{-q} \coloneqq \sqrt\frac{2\pi}{qL}\hat\rho_{-q,R},
    \label{eq:canonical_bosonic_operators}
\end{align}
that satisfy the commutation relation $\commutator{\hat b_q^\pdag,\hat b_{q'}^\dag} = \delta_{q,q'}$.
Using \cref{eq:canonical_bosonic_operators} results in a familiar expression for the interaction written in \cref{eq:interaction_densities}, i.e.
\begin{align}
    \hat V \approx\sum_{q>0}\frac{\hbar q}{2\pi}
    \begin{pmatrix}
        \hat b_q & \hat b^\dag_{-q}
    \end{pmatrix}
    \begin{pmatrix}
        g_4 & g_2 \\
        g_2 & g_4
    \end{pmatrix}
    \begin{pmatrix}
        \hat b^\dag_q \\ \hat b_{-q}
    \end{pmatrix}
    .
    \label{eq:quadratic_interactions}
\end{align}
The interaction, originally quartic in the fermionic degrees of freedom, can be cast into a sum of bosonic operators which are the low-energy excitations of the original model.
All that is left to do is to expand the kinetic term into this new basis, which is actually a quite lengthy calculation if we were to approach it by brute-force.
There is an indirect reasoning through Schur's lemma~\cite{schur_neue_1905}: if two operators $\hat H$ and $\hat H'$ have identical commutation relations with all $\{a^\pdag_\alpha,a^\dag_\alpha\}$, then the two operators are equal up to an overall constant, which in this case can be interpreted as a chemical potential.
One can easily compute the commutator of the mover-density with the kinetic Hamiltonian
\begin{align}
    \commutator{\hat H_0, \hat\rho_{q,\tau}}
    =
    \sum_{p,\tau'}v_F\sigma_{\tau'}\hbar \commutator{\hat n_{p,\tau'},\hat\rho_{q,\tau}}
    =
    \sigma_\tau v_F \hbar q \hat\rho_{q,\tau}
\end{align}
and find its equivalent expression in bosonic degrees of freedom to be
\begin{align}
    \hat H_0' = \frac{\hbar \pi v_F}L\sum_{q,\tau}\hat\rho_{q,\tau}\hat\rho_{-q,\tau} = \frac{2\hbar\pi v_F}L\sum_{q>0,\tau}\hat\rho_{q,\tau}\hat\rho_{-q,\tau},
\end{align}
which can be verified through evaluation of
\begin{align}
    \commutator{\hat H_0',\hat\rho_{q,\tau}}
    \overset{\text{\cref{eq:recursive_commutation}}}{=}
    -\frac{\hbar \pi v_F}L
    \sum_{p,\tau'}
    \brlr{
    \commutator{\hat\rho_{q,\tau},\hat\rho_{p,\tau'}}\hat\rho_{-p,\tau'}
    -
    \hat\rho_{p,\tau'}\commutator{\hat\rho_{q,\tau},\hat\rho_{-p,\tau'}}
    }
    \overset{\text{\cref{eq:chiral_density_commutation}}}{=}
    \sigma_\tau v_F \hbar q\hat\rho_{q,\tau}
\end{align}
and \DIFdelbegin \DIFdel{thus }\DIFdelend we conclude our previous statement $\hat H_0' = \hat H_0 + \mu$ with an irrelevant constant $\mu$.
The effective low energy Hamiltonian containing kinetic and interaction energy satisfies the following matrix equation
\begin{align}
    \hat H = \hat H_0 + \hat V \approx
    \sum_{q>0}\frac{\hbar q}{2\pi}
    \begin{pmatrix}
        \hat b_q & \hat b^\dag_{-q}
    \end{pmatrix}
    \begin{pmatrix}
        2\pi v_F + g_4 & g_2 \\
        g_2 & 2\pi v_F + g_4
    \end{pmatrix}
    \begin{pmatrix}
        \hat b^\dag_q \\ \hat b_{-q}
    \end{pmatrix}
    .
    \label{eq:luttinger_hamiltonian_nondiagonal}
\end{align}
As a final step, we want to find the spectrum of the previous Hamiltonian through a basis transformation\footnote{The matrix coupling the dot product of the operator spinors $\hat B_q$ is independent on the momentum $q$ and as such the basis transformation $T$ will not depend on $q$ either.} $\hat B_q = T \hat B'_q$ with $\hat B_q = (\hat b_q^\dag, \hat b_{-q})^T$ such that
\begin{align}
    \hat H = \sum_{q>0}\hbar q\hat B^\dag_q H \hat B^\pdag_q = \sum_{q>0}\hbar q\hat B'^\dag_q T^\dag H T \hat B'^\pdag_q
\end{align}
is in its diagonal form.
Naive (unitary) rotations do not preserve the commutators of the spinor $\hat B$, defined through
\begin{align}
    \commutator{\hat B_{q,i}^\pdag,\hat B^\dag_{q,j}} = \commutator{\hat B_{q,i}'^\pdag,\hat B'^\dag_{q,j}} = (-\sigma_z)_{i,j},
\end{align}
which imposes an additional constraint on the transformation $T$ according to
\begin{align}
    T^\dag\sigma_z T = \sigma_z,
    \quad
    T^\dag = \sigma_zT^{-1}\sigma_z.
\end{align}
We \DIFdelbegin \DIFdel{thus }\DIFdelend find the similarity relation between the original and the rotated basis according to
\begin{align}
    H'=T^\dag HT =\sigma_z T^{-1}\sigma_z H T,
\end{align}
which allows us to solve the eigenvalue equation of $\sigma_z H$ without knowing the explicit form of $T$.
Since $\sigma_zH$ has vanishing trace, its spectrum is symmetric $\pm E$ and we arrive at the appealing result $H' = u\mathbb1$ with the $2\times2$ unit matrix $\mathbb1$ and scalar eigenvalue
\begin{align}
    u = \frac1{2\pi}\sqrt{\brlr{2\pi v_F + g_4}^2 - g_2^2},
\end{align}
leading to the identity
\begin{align}
    \hat H = \sum_{q>0} u\hbar q \hat B'^\dag_q \hat B'^\pdag_q = \sum_{q>0} \hbar \omega_q \hat B'^\dag_q \hat B'^\pdag_q = \sum_q \hbar  \omega_q \hat b'^\dag_q\hat b'^\pdag_q.
\end{align}
Note that we succeeded to rewrite the original problem as a sum of decoupled harmonic oscillators with frequencies $\omega_q\coloneqq u|q|$.
Assuming two conjugate fields $\commutator{\hat \phi_{k},\hat \pi_{k'}} = \ri\pi\hbar\delta_{k,k'}$, the substitution of ladder operators,
\begin{align}
    \hat b'_q = \sqrt{\frac{mw_q}{2\pi\hbar}}\brlr{\hat\phi_q + \frac{\ri}{mw_q}\hat\pi_{-q}},
    \quad
    \hat b'^\dag_q = \sqrt{\frac{mw_q}{2\pi\hbar}}\brlr{\hat\phi_{-q} - \frac{\ri}{mw_q}\hat\pi_{q}},
\end{align}
brings the diagonalized Hamiltonian to the more traditional form
\begin{align}
    \hat H = \sum_q \frac{\hat\pi_q\hat\pi_{-q}}{2\pi m} + \frac{m\omega_q^2}{2\pi}\hat\phi_q\hat\phi_{-q}
    =
    \int\frac{\rd x}{2\pi}\, \brlr{\hat\pi^2/m + {Da^2}(\partial_x\hat\phi)^2}.
\end{align}
In the above, we neglected a constant term ($-\sum_q\omega_q/2\pi$) and introduced an effective spring constant through the velocity and lattice spacing $D=mu^2/a^2$.

The field $\partial_x\hat\phi$ can \DIFdelbegin \DIFdel{thus }\DIFdelend be interpreted as the ``position offset'' from the equilibrium, which in this case translates to density fluctuations around the equilibrium density $n=k_F/\pi$.
We \DIFdelbegin \DIFdel{thus }\DIFdelend make the connection
\begin{align}
    \int\rd x\hat n = \frac1\pi\brlr{k_Fx - \hat\phi(x)},
\end{align}
which captures the non-oscillatory behavior of \cref{eq:local_density_approximation}.
$\hat\pi$ is the canonical momentum of the density fluctuations, i.e., a charge current.
Moving to dimensionless units (by imposing $\hbar=c=1$ and requiring a unit mass) and rescaling the fields $\hat\phi\rightarrow \frac1{\sqrt{uK}}\hat\phi$, $\hat\pi\rightarrow \sqrt{uK}\hat\pi$ results in the standard Luttinger liquid Hamiltonian
\begin{align}
    \hat H = \int\frac{\rd x}{2\pi}\, \brlr{uK\hat\pi^2 + \frac uK(\partial_x\hat\phi)^2},
    \label{eq:ll_hamiltonian}
\end{align}
in which the couplings are encoded in the dimensionless Luttinger parameters
\begin{align}
    u = v_F\sqrt{\brlr{1 + y_4}^2 - y_2^2},
    \quad
    K = \sqrt{\frac{1+y_4-y_2}{1+y_4+y_2}},
    \quad
    y_4=g_4/(2\pi v_F),
    \quad
    y_2=g_2/(2\pi v_F).
\end{align}
On an intuitive level (given the demonstrated equivalence to the quantum harmonic oscillator), the relation to the local densities is now straightforward:
\begin{align}
    \partial_x\hat\phi(x)=-\pi[\hat\rho_R(x)+\hat\rho_L(x)],
    \quad
    \hat\pi(x)\coloneqq\partial_x\hat\theta=\pi[\hat\rho_R(x)-\hat\rho_L(x)].
\end{align}
Note that the chosen prefactor $\pm\pi$ is only a matter of convention and convenience.
A rigorous derivation of the field operators in terms of the local creation and annihilation operators is presented in several excellent books on the topic, e.g.~\cite{Bruus2004,Giamarchi2003,Gogolin2004}.

These identities, combined with \cref{eq:local_density_approximation} are the basic ingredient to implement the concept of Luttinger liquids to a given (density-density) interaction and find the model-specific coupling constants $g_{4/2}$ on a practical level.
The relation to the annihilation operators (in the continuum) is maybe less obvious,
\begin{align}
    \hat c_{R/L}(x) = \frac1{\sqrt{2\pi\alpha}}\exp\brlr{\pm\ri k_F x\mp\ri\hat \phi(x)}\re^{\ri\hat\theta(x)}\hat \eta_{R/L},
    \label{eq:bosonization_identity}
\end{align}
in which the operator identity is to be understood in the limit $\alpha\rightarrow0$, corresponding to a regularized momentum cutoff~\cite{Bruus2004}.
Note that the field operators commute with the total density, and as such the so-called Klein factors $\hat\eta_{R/L}$ are introduced to connect different Fock spaces.

On a more intuitive level, the identity above shares a fundamental equivalence to the \DIFdelbegin \DIFdel{famous }\DIFdelend Jordan-Wigner transformation~\cite{Jordan1928}.
The exact identity
\begin{align}
    \hat c^\pdag_x=\exp\brlr{\ri\pi\sum_{x'<x}\hat \sigma^+_{x'}\hat \sigma^-_{x'}}\hat \sigma^-_x
    ,\quad
    \anticommutator{\hat c^\dag_x,\hat c^\pdag_{x'}}=\delta_{x,x'}
    ,\quad
    \commutator{\hat \sigma^+_x,\hat \sigma^-_{x'\neq x}}=0
    ,
\end{align}
establishes the equality between spinless fermions and spin-1/2 on a lattice.
Preservation of the (anti-)commutation relation is ensured by the exponential factor, the so-called Jordan-Wigner string.
This allows to interpret $\pm\pi\int\rd x\hat n(x) = \pm k_F x\mp\hat\phi(x)$ as the argument of the Jordan-Wigner string.
The Jordan-Wigner transformation is regularly used to (i) find representations of fermionic operators in terms of spin matrices (see \cref{eq:jordan_wigner_trafo}), and (ii) to treat one-dimensional interacting spin models in the fermionic language, in which case they become exactly solvable under some circumstances~\cite{Lieb1961}.
In this sense bosonization is more powerful and extends to a universal treatment of interacting one-dimensional fermions and bosons.
For a more elaborate discussion and the bosonization of bosons see e.g.~\cite{Cazalilla2004}.


Note that the results presented here are actually independent of the original starting point, i.e. \cref{eq:hamiltonian_free_particles}.
Therefore, the fermi velocity $v_F$ can be replaced by arbitrary group velocities evaluated on the Fermi surface $v_F=(\partial_k\varepsilon_k)_{k=k_F}$.
Significant deviations are then expected in case of flat bands (to be more precise, in case of strong interactions comparable with the bandwidth of the kinetic Hamiltonian), in which case all scattering processes of \cref{fig:scattering_processes} have to be considered.
\DIFdelbegin \DIFdel{An attempt which takes into account such curvature effects solves many of the related issues~\mbox{%DIFAUXCMD
\cite{Imambekov2009}}\hspace{0pt}%DIFAUXCMD
.
}\DIFdelend 

The evaluation of all scattering processes is a tough task and regularly neglected, which then leads to expected quantitative deviations.
However, from a qualitative viewpoint, the neglected scatterings (whenever they are \DIFdelbegin \DIFdel{irrelevant }\DIFdelend \DIFaddbegin \DIFadd{truly ``irrelevant'' }\DIFaddend in the sense of not opening energy gaps, see \cref{sec:renormalization_group_theory}) change only the effective couplings such that physical consequences predicted from \DIFdelbegin \DIFdel{the Luttinger liquid theory }\DIFdelend \DIFaddbegin \DIFadd{abelian bosonization }\DIFaddend (like the asymptotic decay of correlation functions) remain valid.
\section{Properties of real scalar fields and their correlations}
\label{sec:properties_of_real_scalar_fields_and_their_correlations}
\DIFdelbegin \DIFdel{For the evaluation of correlation functions }\DIFdelend \DIFaddbegin \DIFadd{Correlation functions are essential to probe many-body phase on a practical level, and we evaluate a number of such observables throughout all of our works.
For the analytic computation of correlations }\DIFaddend and the later employed renormalization group theory presented in \cref{sec:renormalization_group_theory}, it is beneficial to \DIFdelbegin \DIFdel{recap shortly }\DIFdelend \DIFaddbegin \DIFadd{first review briefly }\DIFaddend the language of path integrals\DIFdelbegin \DIFdel{(which Feynman invented as a graduate student)}\DIFdelend .
During the process of derivation, the Luttinger parameter $K$ is absorbed in the definition of the fields and appears again in the discussion at the end of this section.
The starting point \DIFaddbegin \DIFadd{here is a Luttinger liquid Hamiltonian which }\DIFaddend is \DIFdelbegin \DIFdel{thus a Hamiltonian that is }\DIFdelend separable in the fields, i.e. $\hat H(\hat\phi,\hat\pi) = \hat H(\hat \phi) + \hat H(\hat \pi)$.

The goal is to evaluate the so-called time evolution kernel
\begin{align}
    U(x_f,x_i,T) = \braket{\phi(x_f,T)|\phi(x_i,t_i=0)}=\braket{\phi_f|\re^{-\ri\hat H(\hat\phi,\hat\pi) T}|\phi_i}
    \label{eq:time_evolution_kernel}
\end{align}
in which $\ket{\phi}$ is the simultaneous eigenstate of the field operator $\hat\phi(x)\ket{\phi}=\phi(x)\ket{\phi}$.
Instead of taking the full exponential, one may split the time integration into $N$ parts
\begin{align}
    \re^{-\ri\hat HT}=\brlr{\re^{-\ri\hat H\Delta t}}^N
    ,
    \quad
    \Delta t = \frac TN
    ,
\end{align}
and then insert a complete set of states in between each of the factors, i.e. $\mathbb 1=\int\rd\bm\phi\ket{\bm\phi}\bra{\bm\phi}$
\footnote{
    The identity is understood as a continuum limit of the discretized fields $\hat\phi(x,t)\rightarrow{\hat{\bm\phi}}(t)$ in which the vector $\hat{\bm\phi}=(\hat\phi_1,\hat\phi_2,\dots,\hat\phi_{N_L})^T$ contains the lattice support of the continuum field operator $\hat\phi(x_j,t)=\hat\phi_j(t)$ such that $x_j=ja$ and $a=L/N_L$ (same for $\hat\pi(x,t)\rightarrow\hat{\bm\pi}(t)$).
    As such, one obtains a finite set of quantum numbers $\ket{\bm\phi(t)}=\ket{\phi_1(t),\dots,\phi_{N_L}(t)}$ for which each $\ket{\phi_j(t)}$ encodes a ``position'' state and $\ket{\pi_j}$ it's conjugate ``momentum'' state.
    From this analog picture, and for all practical purposes, one may borrow the identities $\braket{\bm\phi(t)|\bm\pi(t')}=\frac1{2\pi}\exp\brlr{\ri\bm\phi(t)\cdot\bm\pi(t')}$ and $\mathbb1=\brlr{\prod_{j=1}^{N_L}\int\rd\phi_j}\ket{\phi_1,\dots,\phi_j}\bra{\phi_1,\dots,\phi_j}=\int\rd\bm\phi\ket{\bm\phi}\bra{\bm\phi}=$ (same for $\hat{\bm\pi}$).
    The continuum is then restored at the end of the calculation by the simultaneous limits $a\rightarrow0$ and $N_L\rightarrow\infty$ such that $aN_L=L$ is preserved at all times.
    To keep this in mind, we now proceed by using bold letters in case of explicit discretization of the corresponding continuum field.
    For a more universal derivation of the many-body path integral using coherent state representations, we refer to \cite{AltlandSimons2010}.
}.
The insertion of this set of states then corresponds to unconstrained integrations over smaller time slices of the full time evolution, i.e.
\begin{align}
    U(x_f,x_i,T) =
    \int{\rd{\bm\phi}_{N-1}\rd{\bm\phi}_{N-2}\dots\rd{\bm\phi}_2,\rd{\bm\phi}_1}
    \braket{\bm\phi_f|\re^{-\ri\hat H\Delta t}|\bm\phi_{N-1}}
    \braket{\bm\phi_{{N-1}}|\re^{-\ri\hat H\Delta t}|\bm\phi_{{N-2}}}
    \dots
    \nonumber\\
    \braket{\bm\phi_{2}|\re^{-\ri\hat H\Delta t}|\bm\phi_{1}}
    \braket{\bm\phi_{1}|\re^{-\ri\hat H\Delta t}|\bm\phi_i}.
\end{align}
The integration variables $\{{\bm\phi}_{N-1},\dots,{\bm\phi}_1\}$ can be viewed as the field configuration at times $n\Delta t$ and in this sense, the previous expression assumes already an integral over paths.
Defining the transfer ``matrix'' $T_{{\bm\phi}',{\bm\phi}}=\braket{{\bm\phi}'|\re^{-\ri\hat H\Delta t}|{\bm\phi}}$, the time evolution kernel reads
\begin{align}
    U(x_f,x_i,T) =
    \int{\rd{\bm\phi}_{N-1}\rd{\bm\phi}_{N-2}\dots\rd{\bm\phi}_2\rd{\bm\phi}_1}
    T_{{\bm\phi}_f,{\bm\phi}_{{N-1}}}
    T_{{\bm\phi}_{{N-1}},{\bm\phi}_{{N-2}}}
    \dots
    T_{{\bm\phi}_{{2}},{\bm\phi}_{{1}}}
    T_{{\bm\phi}_{{1}},{\bm\phi}_i}.
\end{align}
For the next steps we need the Baker-Campbell-Hausdorff formula
\begin{align}
    \re^{\hat A+\hat B} = \re^{\hat A + \hat B + \frac12\commutator{\hat A, \hat B}+\dots}
\end{align}
which for a pair of canonically conjugate operators is exact at second order.
The separability of the Hamiltonian $\hat H = \hat H(\bm\phi,\bm\pi) = \hat H(\bm\phi)+\hat H(\bm\pi)$ is exploited to expand the exponential to
\begin{align}
    \re^{-\ri\hat H\Delta t}
= \re^{-\ri\Delta t\hat H(\bm\pi)}\re^{-\ri\Delta t\hat H(\bm\phi)}+\mathcal O(\Delta t^2).
\end{align}
The error in the above is well under control by taking $N$ sufficiently large, and vanishes in the continuum limit.
Insertion of $\mathbb1=\int\rd\bm\pi\ket{\bm\pi}\bra{\bm\pi}$ between the two exponentials yields the approximate form of the transfer matrix,
\begin{align}
    T_{{\bm\phi}',{\bm\phi}} =
    \int\rd\bm\pi
    \braket{{\bm\phi}'|\re^{-\ri\Delta t\hat H(\bm\pi)}\ket{\bm\pi}\bra{\bm\pi}\re^{-\ri\Delta t\hat H(\bm\phi)}|{\bm\phi}} + \mathcal O(\Delta t^2),
    \\
=
    \int\frac{\rd\bm\pi}{2\pi}\re^{\ri\bm\pi^T(\bm\phi'-\bm\phi)}\re^{-\ri\Delta t H(\bm\phi,\bm\pi)} + \mathcal O(\Delta t^2).
\end{align}
Note that we succeeded to replace the operator form of the Hamiltonian by its scalar eigenvalues.
The full time evolution kernel then evaluates to
\begin{align}
    U(x_f,x_i,T) =
    \int
    {\rd\{\bm\phi\}}\int{\rd\{\bm\pi\}}
    \re^{\ri\Delta t\sum_{n=1}^{N}\brlr{\bm\pi_n^T\frac{\bm\phi_{n}-\bm\phi_{n-1}}{\Delta t}-H(\bm\phi_{n-1},\bm\pi_n)}},
    \label{eq:path_integral_pre_integration}
\end{align}
where we conveniently use $\bm\phi_{0/N}\coloneqq \bm\phi_{i/f}$, $\rd\{\bm\phi\}=\prod_{n=1}^{N-1}\rd\bm\phi_n$ and $\rd\{\bm\pi\}=\prod_{n=1}^N\rd\bm\pi_n/2\pi$.
Note that the argument in the exponential clearly corresponds to a discretized version of the Lagrangian.
Now we exploit the fact that $H(\bm\phi,\bm\pi)=au/(2\pi)\bm\pi^T\bm\pi + V(\bm\phi)$ is separable in a quadratic kinetic and a potential term, which allows to perform a Gaussian integration of the momentum fields.
A Gaussian integral of two real vectors $\bm v$, $\bm j$ and a real and invertible matrix $\bm A$ is
\begin{align}
    \int\rd\bm v\re^{-\frac12\bm v^T\bm A\bm v + \bm j^T\bm v} = \frac{(2\pi)^{{\rm rank}(\bm A)/2}}{\sqrt{\det\bm A}}\re^{\frac12\bm j^T\bm A^{-1}\bm j}.
    \label{eq:multi_gaussian}
\end{align}
The argument of \cref{eq:path_integral_pre_integration} is purely imaginary, and as such the integral is convergent with a regularization $\Delta t\rightarrow\Delta t(1+\ri\epsilon)$ by taking the limit $\epsilon\rightarrow0$.
More importantly, the (arbitrary!) vector $\bm j$ acts as a ``source'' term to define an expectation value.
Consider the derivative $\partial_{j_m}\partial_{j_n}|_{\bm j=0}$ acting on the \DIFdelbegin \DIFdel{gaussian, it will pull }\DIFdelend \DIFaddbegin \DIFadd{Gaussian, which targets }\DIFaddend the $m,n$ component of $\bm A^{-1}$ up to an overall constant, i.e.
\begin{align}
    \partial^2_{j_m,j_n}\int\rd\bm v\re^{-\frac12\bm v^T\bm A\bm v + \bm j^T \bm v}\Bigg|_{\bm j=0}
    =
    \int\rd\bm v\re^{-\frac12\bm v^T\bm A\bm v}v_mv_n
    =
    \frac{(2\pi)^{{\rm rank}(\bm A)/2}}{\sqrt{\det(\bm A)}}A^{-1}_{mn}
    \\
    \longrightarrow \braket{v_mv_n}\coloneqq (2\pi)^{-{\rm rank}(\bm A)/2}\sqrt{\det(\bm A)}\int\rd\bm v\re^{-\frac12\bm v^T\bm A\bm v}v_mv_n = A_{mn}^{-1}.
    \label{eq:multi_source}
\end{align}
This suggests an interpretation of the Gaussian weight as a probability distribution.
Iteration of the differentiation operation four times yields $\braket{v_mv_nv_qv_p}=A^{-1}_{mn}A^{-1}_{qp}+A^{-1}_{mq}A^{-1}_{np}+A^{-1}_{mp}A^{-1}_{nq}$ and one obtains the general formula
\DIFdelbegin %DIFDELCMD < \begin{align}
%DIFDELCMD <     \braket{v_{i_1}v_{i_2}\dots v_{i_{2n}}} =
%DIFDELCMD <     \sum_{\tiny\begin{array}{c}\text{pairings of}\\ \{i_1,\dots,i_{2n}\}\end{array}}
%DIFDELCMD <     A^{-1}_{i_{k_1}i_{k_2}}\dots A^{-1}_{i_{k_{2n-1}}i_{k_{2n}}}.
%DIFDELCMD < \end{align}%%%
\DIFdelend \DIFaddbegin \begin{align}
    \braket{v_{i_1}v_{i_2}\dots v_{i_{2n}}} =
    \sum_{\begin{array}{c}\text{pairings of}\\ \{i_1,\dots,i_{2n}\}\end{array}}
    A^{-1}_{i_{k_1}i_{k_2}}\dots A^{-1}_{i_{k_{2n-1}}i_{k_{2n}}}.
\end{align}\DIFaddend 
In the continuum limit, the set $\bm v$ translates to a function $v(x)$ and the matrix $\bm A$ is replaced by the propagator $A(x,x')$.
Therefore, the natural generalization becomes
\begin{align}
    \int\rD[v]\re^{-\frac12\int\rd x\rd x' v(x)A(x,x')v(x')+\int\rd x j(x) v(x)}
    \propto
    (\det A)^{-1/2}\re^{\frac12\int\rd x\rd x' j(x) A^{-1}(x,x')j(x')}.
    \label{eq:field_source}
\end{align}
When the variables entering the Gaussian integration were discrete, the interpretation of the determinant was straightforward.
In the present case, one must interpret $A$ as a Hermitian operator with an infinite set of eigenvalues, and $\det A$ denotes the product over this infinite set.
Although the constant of proportionality $(2\pi)^{{\rm rank}(\bm A)/2}$ is formally divergent in the continuum limit, it does not affect the averages (by definition), i.e.
\begin{align}
    \braket{v(x)v(x')} = A^{-1}(x,x').
\end{align}

Finally, application of \cref{eq:multi_gaussian} on \cref{eq:path_integral_pre_integration} and then taking the continuum limits yields the path integral
\begin{align}
    U(x_f,x_i,T)
    =
    \int_{\phi(x_i)}^{\phi(x_f)}{\rD[\phi]}
    \re^{\ri\int_0^T\rd tL[\phi,\dot\phi]}
    ,\quad
    L = \int\frac{\rd x}{2\pi u}(\partial_t\phi)^2 - V(\phi).
    \label{eq:time_evolution_kernel_path_integral}
\end{align}
Since constants of proportionality do not enter in the expectation values by definition, they are absorbed into the definition of $\rD$ (which is given implicitly here).
The integration covers all paths through the classical coordinate space spanned by $\phi$ which begin and end at the same initial and final points $\phi(x_i)$ and $\phi(x_f)$, respectively.
Each path is weighted by the corresponding classical action -- note the absence of hats which would denote a quantum mechanical operator.
However, quantum mechanics is still at full presence, as the integration is not restricted to solutions of the classical equations of motion.

The standard definition of the partition function is given as $Z=\tr(\exp(-\beta \hat H))$, and the trace amounts to a summation over all possible configurations of the system.
This makes apparent the interpretation of $T=-\ri \beta$ in \cref{eq:time_evolution_kernel}.
A similar derivation then results in the ``Euclidean'' action such that the path integral becomes a quantum mechanical partition function
\begin{align}
    Z = \tr(\exp(-\beta \hat H)) = \int\rD[\phi]\re^{-S[\phi]}
    ,\quad
    S = \int\rd\tau\brlr{\frac1{2\pi u}\int\rd x(\partial_\tau\phi)^2+V(\phi)}.
\end{align}

We now impose the Luttinger liquid potential $V(\phi) = \frac u{2\pi}\int\rd x(\partial_x\phi)^2$, and proceed with
\begin{align}
    S[\phi]=\frac1{2\pi}\int\rd\tau\rd x\brlr{ \frac{(\partial_\tau\phi)^2}{u} + u(\partial_x\phi)^2}.
\end{align}
The partition function is readily recast into frequency and momentum space (using the notation ${\bm k}=(\omega, k)$).
In addition, we introduce an additional source term similar to \cref{eq:field_source},
\begin{align}
    Z[\eta] = \int\rD_\phi \re^{-S[\phi]} = \int\rD_\phi \exp\brlr{-\int\frac{\rd k\rd\omega}{2\pi}\phi(-{\bm k})\commutator{\frac{w^2+(uk)^2}{u}}\phi({\bm k}) +\int\rd k\rd\omega\eta(-{\bm k})\phi({\bm k})},
    \label{eq:full_partition_function}
\end{align}
with the definition of the Green's function $G^{-1}({\bm k})=\frac{\omega^2+(uk)^2}{\pi u}$.
One arrives at
\begin{align}
    \braket{\phi({\bm k})\phi(-{\bm k})} = G({\bm k})
    ,\quad
    \braket{\phi(x,\tau)\phi(x',\tau')} = G(x,\tau,x',\tau').
    \label{eq:KG_greens_functions_equality}
\end{align}
A subsequent Fourier transformation, combined with the definition of the polar coordinates ${\bm p}=(\omega/u,-k)$, ${\bm r} = (u(\tau-\tau'), x-x')$, $r = \sqrt{(x-x')^2+u^2(\tau-\tau')^2}$ yields an analytic expression of the correlation function
\DIFdelbegin %DIFDELCMD < \begin{align}
%DIFDELCMD <     \braket{\phi(x,\tau)\phi(x',\tau')}
%DIFDELCMD <     = \int\frac{\rd k\rd\omega}{4\pi^2}G({\bm k})\re^{\ri\omega(\tau-\tau')-\ri k(x-x')}
%DIFDELCMD <     = \int_0^\infty\rd p\, p\int_0^{2\pi}\frac{\rd\alpha}{4\pi}\frac{\re^{\ri pr\cos\alpha}}{p^2}
%DIFDELCMD <     \\
%DIFDELCMD <     = \int_0^\infty\rd p\frac{J_0(pr)}{2p}
%DIFDELCMD <     \approx
%DIFDELCMD <     \int_{\Lambda_{\rm min}}^{\Lambda_{\rm max}}\rd p \frac{J_0(pr)}{2p}
%DIFDELCMD <     \approx
%DIFDELCMD <     -\frac14\log\brlr{\frac{r^2+\Lambda_{\rm max}^{-2}}{\Lambda_{\rm min}^{-2}}},
%DIFDELCMD <     \label{eq:kg_correlations_approximation}
%DIFDELCMD < \end{align}%%%
\DIFdelend \DIFaddbegin \begin{align}
    \braket{\phi(x,\tau)\phi(x',\tau')}
    = \int\frac{\rd k\rd\omega}{4\pi^2}G({\bm k})\re^{\ri\omega(\tau-\tau')-\ri k(x-x')}
    = \int_0^\infty\rd p\, p\int_0^{2\pi}\frac{\rd\alpha}{4\pi}\frac{\re^{\ri pr\cos\alpha}}{p^2}
    \\
    = \int_0^\infty\rd p\frac{J_0(pr)}{2p}
    \approx
    \int_{\Lambda_{\rm min}}^{\Lambda_{\rm max}}\rd p \frac{J_0(pr)}{2p}
    \approx
    -\frac14\ln\brlr{\frac{r^2+\Lambda_{\rm max}^{-2}}{\Lambda_{\rm min}^{-2}}},
    \label{eq:kg_correlations_approximation}
\end{align}\DIFaddend 
with $J_0$ the Bessel function of the first kind.
The first approximation considers the fact that the Green's function should be the effective description of a system on a lattice which provides momentum cutoffs $\Lambda_{\rm max} = 2\pi/a$ and $\Lambda_{\rm min} = 2\pi/L$.
The second approximation considers a smooth upper momentum cutoff rather than the sharp lattice limit, which is discussed in more detail in \cref{sec:renormalization_group_theory}.

To apply the statements for the interacting theory, only the fields must be rescaled $\phi\rightarrow\phi/\sqrt K$ and $\theta\rightarrow\theta\sqrt K$.
In conclusion, the equal-time correlations for the interacting theory satisfy
\DIFdelbegin %DIFDELCMD < \begin{align}
%DIFDELCMD <     \frac1K\braket{\phi(x)\phi(x')} = \braket{\phi'(x)\phi'(x')} \approx -\frac12\log(|x-x'|).
%DIFDELCMD <     \label{eq:greens_1}
%DIFDELCMD < \end{align}%%%
\DIFdelend \DIFaddbegin \begin{align}
    \frac1K\braket{\phi(x)\phi(x')} = \braket{\phi'(x)\phi'(x')} \approx -\frac12\ln(|x-x'|).
    \label{eq:greens_1}
\end{align}\DIFaddend 
The Lagrangian of the $\theta$ field is the same as for the $\phi$ field, \DIFdelbegin \DIFdel{thus }\DIFdelend the same correlations hold, i.e.
\DIFdelbegin %DIFDELCMD < \begin{align}
%DIFDELCMD <     K\braket{\theta(x)\theta(x')} = \braket{\theta'(x)\theta'(x')} \approx -\frac12\log(|x-x'|).
%DIFDELCMD <     \label{eq:greens_2}
%DIFDELCMD < \end{align}%%%
\DIFdelend \DIFaddbegin \begin{align}
    K\braket{\theta(x)\theta(x')} = \braket{\theta'(x)\theta'(x')} \approx -\frac12\ln(|x-x'|).
    \label{eq:greens_2}
\end{align}\DIFaddend 
\section{Luttinger liquids with spin}
\label{sec:LL_with_spin}
To proceed further to the class of Luttinger liquids with (pseudo-)spin degree of freedom, the kinetic Hamiltonian is assumed to be decoupled\footnote{In our research, the spin conservation is explicitly broken in most cases. The analytic treatment employed in \cref{one_half1,integer1,chiral1} starts from a decoupled system, which is understood as two independent wires each hosting a Luttinger liquid. We then study the ``most relevant'' emergent terms resulting from a (perturbative) coupling of the two wires in the presence of interactions. For the definition of relevant and irrelevant terms, see \cref{sec:renormalization_group_theory}.}, i.e.
\begin{align}
    \hat H_0 = \hat H_{0,\uparrow}+\hat H_{0,\downarrow}
    = \frac{2\hbar\pi v_F}L\sum_{q>0,\tau\in\{L,R\},s\in\{\uparrow,\downarrow\}}\hat\rho^\pdag_{q,\tau,s}\hat\rho^\dag_{q,\tau,s},
\end{align}
which allows to introduce a pair of conjugate fields $(\hat \phi_s,\hat \pi_s)$ for each spin flavor $s$.
The scattering processes $g_4$ and $g_2$ can be generalized in a straightforward manner, and the associated interaction terms are given by
\begin{align}
  \sum_{q,\tau,s}
  g_{4\parallel}\hat\rho^\pdag_{q,\tau,s}\hat\rho^\dag_{q,\tau,s}
  +
  g_{4\perp}\hat\rho^\pdag_{q,\tau,s}\hat\rho^\dag_{q,\tau,\overline s}
  +
  g_{2\parallel}\hat\rho^\pdag_{q,\tau,s}\hat\rho^\dag_{q,\overline\tau,s}
  +
  g_{2\perp}\hat\rho^\pdag_{q,\tau,s}\hat\rho^\dag_{q,\overline\tau,\overline s}.
\end{align}
To account for the scattering among different spins, it is customary to write the model in terms of a charge and spin degree of freedom, defined as
\begin{align}
    \hat f_\pm=\frac1{\sqrt2}\brlr{\hat f_\uparrow \pm \hat f_\downarrow},
    \quad
f\in\{\theta,\phi\}.
\end{align}
This rotation is trivial on the level of the kinetic Hamiltonian and
\begin{align}
    \hat H_0 = \frac{2\hbar\pi v_F}L\sum_{q>0,\tau,s\in\{+,-\}}\hat\rho^\pdag_{q,\tau,s}\hat\rho^\dag_{q,\tau,s}.
\end{align}
Things are different for the forward- and backscattering processes $g_{2/4}$.
In the spinful scenario, it is necessary to distinguish between intra-spin and inter-spin scattering, such that
\begin{align}
  \sum_{s\in\{\uparrow,\downarrow\}}
  \brlr{
  g_{4\parallel}\hat\rho^\pdag_{q,\tau,s}\hat\rho^\dag_{q,\tau,s}
  +
  g_{4\perp}\hat\rho^\pdag_{q,\tau,s}\hat\rho^\dag_{q,\tau,\overline s}
  +
  g_{2\parallel}\hat\rho^\pdag_{q,\tau,s}\hat\rho^\dag_{q,\overline\tau,s}
  +
  g_{2\perp}\hat\rho^\pdag_{q,\tau,s}\hat\rho^\dag_{q,\overline\tau,\overline s}
  }
  \\
  =
  \frac12
  \sum_{i\in\{2,4\}}
  \brlr{
  [g_{i\parallel}+g_{i\perp}]\hat\rho^\pdag_{q,\tau,+}\hat\rho^\dag_{q,\tau,+}
  +
  [g_{i\parallel}-g_{i\perp}]\hat\rho^\pdag_{q,\tau,-}\hat\rho^\dag_{q,\tau,-}
  }.
\end{align}
The $g_1$ terms require more attention in this case:
While inter-spin terms $g_{1\parallel}$ are equivalent to backscattering terms (up to a sign, see \cref{eq:umklapp_backscattering_equivalence}), the $g_{1\perp}$ term is resulting in
\begin{align}
    g_{1\perp}\sum_{s\in\{\uparrow,\downarrow\}}\hat c^\dag_{L,s}(x)\hat c^\pdag_{R,s}(x)\hat c^\dag_{R,\overline s}(x)\hat c^\pdag_{L,\overline s}(x)
    =
    \frac{g_{1\perp}}{4\pi^2\alpha^2 }\sum_{s\in\{\uparrow,\downarrow\}}\re^{-2\ri\hat \phi_s(x)}\re^{2\ri\hat \phi_{\overline s}(x)}
    =
    \frac{g_{1\perp}}{2\pi^2\alpha^2}\cos(2\sqrt2\hat \phi_-(x))
\end{align}
and therefore, the total Hamiltonian is of the form (again, setting $\hbar=1$)
\begin{align}
    \hat H = \sum_{s\in\{+,-\}}\int\frac{\rd x}{2\pi}\brlr{u_s K_s\hat\pi_s^2 + \frac{u_s}{K_s}(\partial_x\hat\phi_s)^2}
    +
    \frac{2g_{1\perp}}{(2\pi\alpha)^2}\int\rd x\cos(2\sqrt2\hat\phi_-(x))
    \label{eq:ll_hamiltonian_spin}
\end{align}
with coupling constants
\begin{align}
    u_sK_s = v_F(1+y_{4s}-y_{2s}),
    \quad
    \frac{u_s}{K_s} = v_F(1+y_{4s}+y_{2s}),
    \\
    y_{is} = \frac{g_{i\parallel}+sg_{i\perp}}{2\pi v_F},
    \quad
    i\in\{2,4\},
    \quad
    s\in\{+,-\}.
\end{align}
The first observation regards the separation of the Hamiltonian into two quasi-independent sectors (the overall particle number must still be conserved), namely those of charge and spin excitations propagating with different velocities.
This is famously known as spin-charge separation and a feature of all one-dimensional metallic systems.
It describes the fractionalization of particles into two distinct quasi-particles, called spinons with zero charge and spin 1/2 and chargeons with charge minus one and no spin.
As is clear from the discussion above, the two sectors remain independent and therefore move in general with different velocities~\cite{Tomonaga1950,Luttinger1963,Haldane1981,Kim2006}.

The second observation regards the presence of a non-quadratic term in the spin sector of the Hamiltonian.
In general, $\hat O_{\rm sG} = g\int\rd x\cos(\beta\hat f)$
are called sine-Gordon terms of the field $\hat f$, with $\beta$ an arbitrary number, and are potentially ``relevant'' for a Luttinger liquid~\cite{Giamarchi2003,Gogolin2004,AltlandSimons2010}.
The ``relevancy'' of such terms is determined by its flow under Wilsonian renormalization group theory which we perform explicitly in \cref{sec:renormalization_group_theory}.
If $g$ is large enough (depending on $\beta$), it drives the system to an ordered phase by ``pinning'' its argument's eigenvalues to semiclassical minima $\beta f=(2n+1)\pi$.
In the context of \cref{eq:ll_hamiltonian_spin}, it opens a gap in the spin sector and establishes an ordered spin density associated to the argument of $\hat O_{\rm sG}$.
\section{Renormalization group theory}
\label{sec:renormalization_group_theory}
In \cref{sec:LL_with_spin}, we motivated an interaction $\hat O_{\rm sG}$ which is typically encountered in the effective field theory of (spinful) one-dimensional models that may drive the gapless Luttinger liquid to a different phase of matter.
This interaction is commonly used to describe a particular class of phase transitions from a gapless to a gapped system, called Berezinskii-Kosterlitz-Thouless (BKT) transitions.
It is peculiar in the sense that it does not involve any spontaneous symmetry breaking and can thus be considered as an example of a topological phase transition~\cite{AltlandSimons2010}.
Traditionally, the BKT transition was first encountered in 2D classical systems, but due to the close analogy between (1+1)D quantum field theories and classical statistical mechanics in two spatial dimensions, it appears naturally in one-dimensional quantum systems as well.
The emergence of the sine-Gordon type field theory from a 2D classical model requires a rather involved treatment of the classical partition function, and we refer to~\cite{Kosterlitz1974,AltlandSimons2010} for details on the derivation.
\DIFdelbegin %DIFDELCMD < 

%DIFDELCMD < %%%
\DIFdel{We now proceed by explaining the basic }\DIFdelend \DIFaddbegin \DIFadd{We encounter very similar sine-Gorden terms in the effective low energy theory of our works presented in~\mbox{%DIFAUXCMD
\cref{one_half1} }\hspace{0pt}%DIFAUXCMD
and \mbox{%DIFAUXCMD
\cref{integer1}}\hspace{0pt}%DIFAUXCMD
, and treat them using second order }\DIFaddend renormalization group (RG) \DIFdelbegin \DIFdel{analysis of the sine-Gordon model, based on~\mbox{%DIFAUXCMD
\cite{Gogolin2004}}\hspace{0pt}%DIFAUXCMD
.
}\DIFdelend \DIFaddbegin \DIFadd{theory, which is explained in the remaining section.
}

\DIFaddend The action of the sine-Gordon model reads $S = S_0 + S_I$, with
\begin{align}
    S_0 = \frac1{2\pi}\int\rd x\rd\tau\brlr{\frac1{uK}(\partial_\tau\phi)^2 + \frac uK (\partial_x\phi)^2}
    ,\quad
    S_I = g\int\rd x\rd\tau\cos(\beta\phi)
    .
\end{align}
Remember that the action can be thought of the continuum limit of a lattice model with well defined Brillouin zone, which connects the long wavelength with the small crystal momentum components of $\phi$.
The idea of renormalization is to let the system flow towards larger distances by sequentially integrating the short-distance (fast) components of the fields by partial path integration and representing the result in terms of an effective model for the long-wavelength field.
The final step is then to recover the original size of the momentum shell by a rescaling of \DIFdelbegin \DIFdel{lengthscales}\DIFdelend \DIFaddbegin \DIFadd{length scales}\DIFaddend .
For technical convenience, it is best to assume a circular cutoff constraint $|{\bm k}|<\Lambda\sim2\pi/a$.
If the model is of a sine-Gordon type, the effective model will have the same form as the original one (up to a rescaled set of coupling constants and extra ``irrelevant'' terms which tend to disappear), from which the so-called ``RG-flow equations'' are derived.

Let's start by splitting the fields into the aforementioned slow and fast components, where the ``fast'' components are contained in an infinitesimal momentum shell $\rd\Lambda=\Lambda-\Lambda'$
\begin{align}
    \phi_\Lambda({\bm x}) = \phi_{\Lambda'}({\bm x}) + h({\bm x}),
    \quad
    \phi_{\Lambda'}({\bm x})\coloneqq\frac1{\sqrt L}\sum_{k<\Lambda'}\re^{\ri {\bm k}{\bm x}}\phi_{\bm k},
    \quad
    h({\bm x}) \coloneqq \frac1{\sqrt L}\sum_{\Lambda'<k<\Lambda}\re^{\ri {\bm k}{\bm x}}\phi_{\bm k}.
\end{align}
We will see shortly, that instead of using $\rd\Lambda$ as a measure of the flow it is more useful to consider the quantity $\brlr{\Lambda'/\Lambda}^\alpha=1-\alpha\frac{\rd\Lambda}\Lambda + O(\rd\Lambda^2) = 1-\alpha\rd l + O(\rd l^2)$ with \DIFdelbegin \DIFdel{$\rd l =\frac{\rd\Lambda}\Lambda = \rd\log\Lambda$}\DIFdelend \DIFaddbegin \DIFadd{$\rd l =\frac{\rd\Lambda}\Lambda = \rd\ln\Lambda$}\DIFaddend .
Note that these definitions imply $\Lambda'=\Lambda\re^{-l}$, such that the momentum shell flows exponentially fast to smaller values.
The Luttinger liquid part of the \DIFdelbegin \DIFdel{Euclidian }\DIFdelend \DIFaddbegin \DIFadd{Euclidean }\DIFaddend action is linear under such decomposition and therefore
\begin{align}
    Z_\Lambda
    = \int\rD\phi_{\Lambda'}\rD h \re^{-S_0[\phi_{\Lambda'}]-S_0[h]-S_I[\phi_{\Lambda'}({\bm x}) + h({\bm x})]}
    = Z_h\int\rD\phi_{\Lambda'}\re^{-S_0[\phi_{\Lambda'}]}\braket{\re^{-S_I[\phi_{\Lambda'}({\bm x}) + h({\bm x})]}}_h,
\end{align}
which results in the definition of an effective action
\DIFdelbegin %DIFDELCMD < \begin{align}
%DIFDELCMD <     S_{\rm eff}[\phi_{\Lambda'}] = S_0[\phi_{\Lambda'}] - \log\braket{\re^{-S_I[\phi_{\Lambda'}+h]}}_h.
%DIFDELCMD < \end{align}%%%
\DIFdelend \DIFaddbegin \begin{align}
    S_{\rm eff}[\phi_{\Lambda'}] = S_0[\phi_{\Lambda'}] - \ln\braket{\re^{-S_I[\phi_{\Lambda'}+h]}}_h.
\end{align}\DIFaddend 
The analytic evaluation of the expectation on the right hand side requires further simplifications -- one prominent possibility is the perturbative expansion up to second order in the coupling amplitude $g$ of the interaction part of the action,
\begin{align}
    S^{(2)}_{\rm eff}[\phi_{\Lambda'}] = S_0[\phi_{\Lambda'}] + \braket{S_I[\phi_{\Lambda'}+h]}_h - \frac12\braket{S^2_I[\phi_{\Lambda'}+h]}_{h,{\rm conn.}} + O(g^3),
\end{align}
in which $\braket{F^2}_{\rm conn.} = \braket{F^2}-\braket{F}^2$ denotes the ``connected'' part of the expectation value~\cite{Gogolin2004}.
The renormalization group approach carried out in the following is thus reliable for small \DIFdelbegin \DIFdel{renormalized }\DIFdelend couplings $g\ll \frac u{2\pi K}$.

The first order term can be rewritten as
\begin{align}
    \braket{S_I[\phi_{\Lambda'}+h]}_h = g\int\rd^2x\braket{\cos(\beta\anticommutator{\phi_{\Lambda'}+h})}_h
    = \frac g2\int\rd^2x\brlr{\re^{\ri\beta \phi_{\Lambda'}}\braket{\re^{\ri\beta h}}_h+\re^{-\ri\beta \phi_{\Lambda'}}\braket{\re^{-\ri\beta h}}_h},
\end{align}
in which the expectation value of the exponential can be rewritten by using
\begin{align}
    \braket{\re^{\ri\sum_kb_k\phi(x_k)}} = \re^{-\frac12\sum_{k,k'}b_kb_{k'}\braket{\phi(x_k)\phi(x_{k'})}}.
    \label{eq:expectation_value_exponential_fields}
\end{align}
In this case the correlations are finite up to first order in $\rd\Lambda$, i.e.
\begin{align}
    \braket{h^2}_h = K\int_{\Lambda-\rd\Lambda}^\Lambda\rd p\frac{J_0(0)}{2p} = \frac K2\frac{\rd\Lambda}\Lambda = \frac{K\rd l}2.
    \label{eq:rg_hsq}
\end{align}
This way, we can easily expand the expectation value from above to
\begin{align}
    \braket{S_I[\phi_{\Lambda'}+h]}_h
    = g\int\rd^2x\anticommutator{\cos(\beta \phi_{\Lambda'})\re^{-\frac{K\beta^2}4\rd l}}
    \\
    = g\brlr{1 - \frac{K\beta^2}4\rd l}\int\rd^2x\cos(\beta \phi_{\Lambda'}).
\end{align}
As a next step, we would like to recover the size of the original momentum shell by $|{\bm k}|\rightarrow |{\bm k}'|=\frac{\Lambda}{\Lambda'}|{\bm k}| = (1+\rd l)|{\bm k}|+O(\rd l^2)$.
In order to preserve the Fourier transform, it is a necessity to keep the dot product ${\bm k'}{\bm x'}$ invariant under the flow $\rd l$.
Therefore, space-time must be rescaled in the opposite manner to crystal momentum, i.e. $|{\bm x}|\rightarrow|{\bm x'}|=(1-\rd l)|{\bm x}|$.
Differentials are \DIFdelbegin \DIFdel{thus }\DIFdelend transformed according to the equality
\begin{align}
    \rd^2x = \rd^2 x'(1-\rd l)^{-2} = \rd^2 x'(1+2\rd l) + O(\rd l^2).
\end{align}
Due to the presence of gradients, the Gaussian part $S_0$ is left invariant and we arrive at the first-order approximation of the effective action
\begin{align}
    S^{(1)}_{\rm eff}[\phi_\Lambda]=S_0[\phi_\Lambda] + g'\int\rd^2x\cos(\beta \phi_{\Lambda}),
    \quad
    g' = g\brlr{1 + \commutator{2-D_g}\rd l},
    \quad
    D_g=\frac{K\beta^2}4.
    \label{eq:bkt_first_order}
\end{align}
In the first-order approximation, the so-called scaling dimension $D_g$ appears and is associated with the coupling $g$ of the perturbation $\cos(\beta\phi)$.
From \cref{eq:bkt_first_order} the differential equation of the coupling constant under the renormalization flow $\rd l$ is identified as
\begin{align}
    \frac{\rd g}{\rd l} = \frac{g'-g}{\rd l} = (2-D_g)g.
\end{align}
Its solutions are $g(l) = g_0\re^{(2-D_g)l}$, where $g_0\coloneqq g(0)$ is the initial condition of the coupling before the RG flow.
It is \DIFdelbegin \DIFdel{thus }\DIFdelend evident that the flow of the coupling is fully determined by the scaling dimension: if $D_g>2$, then $g(l)\rightarrow0$ vanishes exponentially fast and the coupling is dubbed ``irrelevant''.
If however $D_g<2$, then $g(l)\rightarrow\infty$ and the interaction is called ``relevant''.
Due to the perturbative character of this study, one must however consider that $g(l)$ is not allowed to exceed the initial values of the Gaussian part $g(l)<g^*$ with $g^*=\min(uK,u/K)$.
This value is \DIFdelbegin \DIFdel{thus }\DIFdelend related to an energy scale of the microscopic system, e.g. the bandwidth of the non-interacting model.
An upper stop value for the flow $l^*$ is thus given by $g^*\coloneqq g(l^*)$ and one may assume that the system is driven sufficiently away from the Luttinger liquid critical point, characterized by the scale invariant action $S_0$.

An estimate for the scaling of the correlation length is obtained by
\begin{align}
    \xi(l) \propto a(l) \propto \Lambda^{-1}(l) = \Lambda^{-1}_0\re^{l},
    \label{eq:correlation_length_rg_flow}
\end{align}
in which $a(l)$ is the renormalized lattice spacing, which naturally scales like the inverse of the momentum shell, i.e. $\Lambda\sim2\pi/a$.
In conclusion, the energy gap associated with the correlation length $\xi^{-1}$ scales like the momentum shell itself, i.e. $\Delta(l)\propto\xi(l)^{-1}\propto\re^{-l}$.
By combining the solutions $\Delta^*$ with $g^*$, one can \DIFdelbegin \DIFdel{thus }\DIFdelend estimate the size of the gap induced by the relevant sine-Gordon operator~\cite{Gogolin2004}, i.e.
\begin{align}
    {\Delta^*}\propto\re^{-l^*}=\brlr{\frac{g_0}{g^*}}^{1/(2-D_g)}.
\end{align}
Note that the estimate is given in arbitrary units and as such makes no quantitative prediction.
Instead, it predicts a qualitative behavior -- the smaller the scaling dimension $D_g$ the larger the related gap will be.
This is in agreement with the previous discussion of the relevancy of the sine-Gordon interaction for $D_g<2$: if the scaling dimension evolves to smaller values, the associated operator becomes ``more relevant''.

Now for the second order correction -- the first term in the connected \DIFdelbegin \DIFdel{correlator }\DIFdelend \DIFaddbegin \DIFadd{correlation function }\DIFaddend evaluates to
\begin{align}
    \braket{S^2_I[\phi_{\Lambda'}+h]}=\frac{g^2}2\re^{-\beta^2\braket{h^2}}\int\rd^2x\rd^2x'
    \left(
        \cos\commutator{\phi_{\Lambda'}({\bm x})+\phi_{\Lambda'}({\bm x'})}\re^{-\beta^2\braket{h({\bm x})h({\bm x'})}}
        \right.\\
        \left.+
        \cos\commutator{\phi_{\Lambda'}({\bm x})-\phi_{\Lambda'}({\bm x'})}\re^{+\beta^2\braket{h({\bm x})h({\bm x'})}}
    \right)
    \label{eq:second_order_rg_term1}
\end{align}
and contains the decaying two-point correlations of the $h$-fields.
For convenience, let me proceed with the assumption that the two-point correlations are of the form $\braket{h({\bm x})h({\bm x'})} = \frac K2 C(r=|{\bm x}-{\bm x'}|)\rd l$ such that the action is rewritten to
\DIFaddbegin {\small
\DIFaddend \begin{align}
    \braket{S^2_I[\phi_{\Lambda'}+h]}=\frac{g^2}2\brlr{1 - \frac{K\beta^2}2\rd l}\int\rd^2x\rd^2x'
    \left(
        \cos\commutator{\beta\phi_{\Lambda'}({\bm x})+\beta\phi_{\Lambda'}({\bm x'})}\anticommutator{1-\frac{K\beta^2}2C(|{\bm x}-{\bm x'}|)\rd l}
        \right.\\
        \left.+
        \cos\commutator{\beta\phi_{\Lambda'}({\bm x})-\beta\phi_{\Lambda'}({\bm x'})}\anticommutator{1+\frac{K\beta^2}2C(|{\bm x}-{\bm x'}|)\rd l}
    \right).
\end{align}
\DIFaddbegin }\DIFaddend The disconnected part of the correlation function is evaluated using the identity $2\cos(a)\cos(b)=\cos(a+b)+\cos(a-b)$ and $(1-\alpha\rd l)^2 = 1-2\alpha\rd l + O(\rd l^2)$ and reads
\begin{align}
    \braket{S_I[\phi_{\Lambda'}+h]}_h^2
    = g^2\brlr{1 - \frac{K\beta^2}4\rd l}^2\int\rd^2x\rd^2x'\cos(\beta \phi_{\Lambda'}({\bm x}))\cos(\beta \phi_{\Lambda'}({\bm x'}))
    \\
    = \frac{g^2}2\brlr{1 - \frac{K\beta^2}2\rd l}\int\rd^2x\rd^2x'
        \left(
            \cos\commutator{\beta \phi_{\Lambda'}({\bm x})+\beta \phi_{\Lambda'}({\bm x'})}
            +
            \cos\commutator{\beta \phi_{\Lambda'}({\bm x})-\beta \phi_{\Lambda'}({\bm x'})}
        \right).
\end{align}
Therefore, the $C$-independent terms cancel in the connected part of the correlation function and the final result is
\DIFdelbegin %DIFDELCMD < \begin{align}
%DIFDELCMD <     -\frac12\braket{S^2_I[\phi_{\Lambda'}+h]}_{h,{\rm conn.}} = \frac{g^2K\beta^2}8\rd l\int\rd^2x\rd^2x'
%DIFDELCMD <     C(|{\bm x}-{\bm x'}|)
%DIFDELCMD <     \left(
%DIFDELCMD <         \cos\commutator{\beta\phi_{\Lambda'}({\bm x})+\beta\phi_{\Lambda'}({\bm x'})}
%DIFDELCMD <         \right.
%DIFDELCMD <         \\
%DIFDELCMD <         \left.-
%DIFDELCMD <         \cos\commutator{\beta\phi_{\Lambda'}({\bm x})-\beta\phi_{\Lambda'}({\bm x'})}
%DIFDELCMD <     \right)
%DIFDELCMD < \end{align}%%%
\DIFdelend \DIFaddbegin \begin{align}
    -\frac12\braket{S^2_I[\phi_{\Lambda'}+h]}_{h,{\rm conn.}} = \frac{g^2K\beta^2}8\rd l\int\rd^2x\rd^2x'
    C(|{\bm x}-{\bm x'}|)
    \left(
        \cos\commutator{\beta\phi_{\Lambda'}({\bm x})+\beta\phi_{\Lambda'}({\bm x'})}
        \right.
        \\
        \left.-
        \cos\commutator{\beta\phi_{\Lambda'}({\bm x})-\beta\phi_{\Lambda'}({\bm x'})}
    \right).
\end{align}\DIFaddend 


To proceed in solving the integral, the function $C$ needs to be discussed further.
For a sharp momentum cutoff, one obtains from \cref{eq:kg_correlations_approximation} $C(r) = J_0(\Lambda r)$ with asymptotic expression $J_0(\Lambda r)\approx\sqrt{2/(\pi \Lambda r)}\cos(\Lambda r-\pi/4)$.
This function has a long algebraic tail and is \DIFdelbegin \DIFdel{thus }\DIFdelend not a sharp function in $r$ (see also \cref{fig:rg_cutoff}).
The origin of this long tail resides in the choice of the momentum cutoff in the integration scheme.
Recall that we aim to integrate an infinitesimal shell of large momenta, which allows a certain degree of freedom in the form of the shell itself.
The momentum integration can be rewritten as
\begin{align}
    \int_{0}^{\Lambda}\rd p \rightarrow \int_{0}^\infty\rd p f_n(p,\Lambda),
    \quad
    f_n(p,\Lambda) = \frac{\Lambda^n}{p^n+\Lambda^n},
    \quad
    n\in\mathds N,
    \label{eq:integral_cutoff}
\end{align}
which implements a smooth cutoff around $\Lambda$.
The sharp situation is recovered for $n\rightarrow\infty$ (see \cref{fig:rg_cutoff}).
In practice, the integration of the fast modes in \cref{eq:kg_correlations_approximation} evaluates to
\begin{align}
    \braket{h({\bm x}),h(\bm x')}_h = \frac K2\int_{\Lambda'}^\Lambda\rd p\frac{J_0(p r)}{p} = \frac K2\int_{0}^\Lambda\rd p\frac{J_0(p r)}{p}
    -\frac K2\int_{0}^{\Lambda'}\rd p\frac{J_0(p r)}{p}
    \\
    \longrightarrow
    \frac K2\int_{0}^\infty\rd p J_0(p r)p^{n-1}\brlr{\frac1{p^n+\Lambda'^n}-\frac1{p^n+\Lambda^n}}
    \\
    =
    \frac K2\int_{0}^\infty\rd p J_0(p r)p^{n-1}\frac{n\Lambda^n}{\left(\Lambda^n+p^n\right)^2}\rd l + O(\rd l^2),
    \label{eq:rg_hxhxpr}
\end{align}
leading to the modified function $C_n$ which depends on the smoothness $n$, i.e.
\begin{align}
    C_n(r)
    =
    \int_{0}^\infty\rd p J_0(p r)p^{n-1}\frac{n\Lambda^n}{\left(\Lambda^n+p^n\right)^2}.
    \label{eq:rg_cn_def}
\end{align}
\begin{figure}
    \centering
    \subfigure[]{\includegraphics{figures/cutoff_function.png}}
    \subfigure[]{\includegraphics{figures/rg_correlations.png}}
    \subfigure[]{\includegraphics{figures/rg_correlations_log.png}}
    \caption{Panel (a) shows the chosen cutoff $f_n(p,\Lambda)$ with $n\in\{2,4,6,8,\infty\}$ for the integral expression in \cref{eq:integral_cutoff}, which results in various approximations of $C(r)$ for altering $n$ plotted in (b). Panel (c) highlights the exponentially sharp function $C_2(r)=\Lambda rK_1(r\Lambda)$ compared to the sharp cutoff result $C_\infty(r)=J_0(\Lambda r)$.}
    \label{fig:rg_cutoff}
\end{figure}
For the special case $n=2$, the integral of the $h$-fields evaluates to $C_2(r)=\Lambda r K_1(\Lambda r)$ that decays exponentially fast (see \cref{fig:rg_cutoff}(c)).
In particular, the function follows the asymptotic decay $z K_1(z) \sim \sqrt{\pi z/2}\exp(-z)$ and is already negligible for $z=1$, i.e. $K_1(1)\approx0.0062$.
Therefore, the integration of $C_2(r)$ can be confined to a small interval $r<\alpha$ where $\alpha\sim2\pi/\Lambda=a$ is a small length scale comparable with the lattice spacing $a$.

In order to utilize the strong confinement of $C_2$, it is beneficial to introduce relative coordinates ${\bm R} = 1/2({\bm x}+{\bm x'})$ and ${\bm r} = {\bm x}-{\bm x'}$.
The integral can then be rewritten as
\DIFaddbegin {\small
\DIFaddend \begin{align}
    -\frac12\braket{S^2_I[\phi_{\Lambda'}+h]}_{h,{\rm conn.}}
    =
    \frac{g^2K\beta^2}8\rd l\int\rd^2R\int\rd^2r
    C_2(r)
    \left(
        \cos\commutator{\beta\phi_{\Lambda'}({\bm R}+{\bm r})+\beta\phi_{\Lambda'}({\bm R}-{\bm r})}
        \right.
        \label{eq:rg_int_1}\\
        \left.-
        \cos\commutator{\beta\phi_{\Lambda'}({\bm R}+{\bm r})-\beta\phi_{\Lambda'}({\bm R}-{\bm r})}
    \right)
    \label{eq:rg_int_2}
    \\
    \approx
    \frac{g^2K\beta^2}{8}\rd l\int\rd^2R\rd^2rC_2(r)
    \left(
        \cos\commutator{2\beta\phi_{\Lambda'}({\bm R})}
        -
        \cos\commutator{\beta\partial_{\bm R}\phi_{\Lambda'}({\bm R}){\bm r})}
    \right).
    \label{eq:rg_int_rel_coords}
\end{align}
\DIFaddbegin }
\DIFaddend 

The first $\cos$ term (\cref{eq:rg_int_1}) did not exist in the original Hamiltonian\footnote{
    Note that this term generation is continuous, and to account for it, we should start from a more generic interaction containing all the higher harmonics, i.e. $S_{I'} = S_I + \sum_{j=1}^\infty \tilde g_j\int\rd x\rd\tau\cos(2j\beta\phi)$.
    The first-order corrections of the \DIFdelbegin \DIFdel{Euclidian }\DIFdelend \DIFaddbegin \DIFadd{Euclidean }\DIFaddend action then yield a coupled system of differential equations in which the amplitudes of less relevant operators depend on those of more relevant ones (but {\it not} vice-versa).
    This allows for a practical and well-justified simplification of dropping the sum in the previous expression, because the dominant coupling is always independent of the amplitudes of less-relevant operators.
} -- it is a new sine-Gordon type interaction with larger scaling dimension $D_{\tilde g} = K\beta^2$ compared to $D_g=K\beta^2/4$.
The operator associated to $\tilde g$ is thus less relevant than the original term and can be disregarded.
The second $\cos$ term in (\cref{eq:rg_int_2}) yields a renormalization of the quadratic part,
\begin{align}
    -\frac12\braket{S^2_I[\phi_{\Lambda'}+h]}_{h,{\rm conn.}}
    \approx
    \frac{\alpha^4 g^2K\beta^4}{16u^2}\rd l\int\rd x\rd\tau\frac1{u^2}(\partial_
    \tau\phi_{\Lambda'})^2+(\partial_x\phi_{\Lambda'})^2.
    \label{eq:RG_second_order_approximation}
\end{align}
In the above, we neglect a constant and gave for granted the harmonic approximation and integration of $\bm r$ with $r=\sqrt{(x-x')^2+u^2(\tau-\tau')^2}$ in polar coordinates.
The constant $\alpha^4/u^2$ is determined by the cutoff of the integrals over the relative coordinates, and $\alpha=\mathcal O(a)$ is on the order of the lattice spacing.
In summary, we obtain the effective action
\begin{align}
    S_{\rm eff}^{(2)}[\phi_\Lambda] = \frac1{2\pi}\int\rd\brlr{ x\rd\tau\frac{1}{uK'}(\partial_\tau\phi_\Lambda)^2 + \frac{u}{K'}(\partial_x\phi_\Lambda)^2} +  g'\int\rd^2 x\cos(\beta\phi_\Lambda),
\end{align}
which is self-similar to the original action up to the renormalized couplings
\begin{align}
    \frac1{uK'}=\frac1{uK}+\frac{\alpha^4 g^2 K\beta^4\pi}{8 u^4}\rd l,
    \quad
    \frac u{K'}=\frac u{K}+\frac{\alpha^4 g^2 K\beta^4\pi}{8u^2}\rd l,
    \quad
    g' = g\brlr{1 + \commutator{2-\frac{\beta^2 K}4}\rd l},
    \\
    \Rightarrow
    K' = \brlr{\commutator{\frac{1}{uK}+\frac{\alpha^4 g K\beta^4\pi}{8u^4}}\commutator{\frac{u}{K}+\frac{\alpha^4 g K\beta^4\pi}{8u^2}}}^{-1/2}
    =
    K-\frac{\pi \alpha^4 \beta^4 g^2 K^3}{8u^2}\rd l + O(\rd l^2).
\end{align}

This concludes the derivation of the so-called RG flow which is described by the system of differential equations we just derived
\begin{align}
    \frac{\rd K}{\rd l} = -\frac{\pi \alpha^4 \beta^4 g^2 K^3}{8u^2},
    \quad
    \frac{\rd g}{\rd l} = g\brlr{2-\frac{\beta^2K}4}.
\end{align}
Viable estimates are already obtained far from the phase transitions, and the second order approximation provides more insights close to the phase transition at $D_g=2$, in particular at $K=8/\beta^2$.
It is beneficial to define the flow equations close to the critical point, e.g. through the parameters
\begin{align}
    x = \frac{\beta^2 K}4 - 2,
    \quad
    y = 4g\sqrt{\frac{\pi\alpha^4}{u^2}},
\end{align}
which dictate the modified RG equations
\begin{align}
    \frac{\rd x}{\rd l} = -\frac{y^2}8(x+2)^3,
    \quad
    \frac{\rd y}{\rd l} = - xy.
\end{align}
This set of variables is particularly useful to investigate the equations in proximity of $x=0$.
Note that $(x+2)^3\approx8$, such that $\rd x/\rd l = -y^2$ and $\rd y/\rd l=-xy$ describes the effective RG flow close to the Luttinger liquid fix point at $x=0$ and $y=0$, which are known as the ``Kosterlitz-Thouless equations''~\cite{Kosterlitz1974,Kosterlitz1973}.

\begin{figure}
    \centering
    \subfigure[]{\includegraphics{figures/BKT_RG_flow1.png}}
    \subfigure[]{\includegraphics{figures/BKT_RG_flow2.png}}
    \subfigure[]{\includegraphics{figures/BKT_RG_flow3.png}}
    \caption{Plot of the modified RG equations far (a) and close (b) to the critical point $x=0$. Panel (b) close to the transition at $x=0$ shows four different scenarios for $g>0$: (i) $x>0$ and $y<x$ drives the coupling $y\rightarrow0$. This is the regime in which $g$ is irrelevant such that the system remains in the Luttinger liquid phase. (ii) $x>0$ and $y=x$ describes the critical BKT line which flows to the critical fix point $x=y=0$. (iii) $y>x$, the system always flows to $y\rightarrow\infty$ independently of $x$. (iv) $x<0$, the system flows to $y\rightarrow\infty$. Panel (c) contrasts the results of the first order RG approximation which neglects the flow of the Luttinger parameter $K$.}
    \label{fig:bkt_flow_equations}
\end{figure}

We obtain four scenarios (assuming positive couplings $y>0$ in general):

(i) If $x>0$, and $y<x$, the system flows towards $y=0$ and a finite value of $x$ at which the flow ends.
This is a situation in which the interaction is irrelevant and the coupling vanishes.
Note that a vanishing coupling $y$ implies that the flow ends, and the system \DIFdelbegin \DIFdel{thus }\DIFdelend assumes a fix point.
Although the interaction is irrelevant and as such the system remains in a Luttinger liquid phase, it ``renormalizes'' the effective Luttinger liquid parameter during the flow.
The effective interaction is then given by the final value of $x$ (thus $K$) at which the flow terminates.

(ii) If $x>0$ and $y=x$ (taken in proximity of $x$ and $y$ small), the trajectory follows a critical BKT line to the fix point $x=y=0$.
The effective Luttinger parameter assumes the ``marginal'' value $K=8/\beta^2$ for which the scaling dimension results to $D_g=2$.

(iii) In case of $y>|x|$, the coupling $y$ first flows towards smaller couplings, but then towards $y\rightarrow\infty$, independently on the value of $x$.
This corresponds to a case in which the interaction becomes dominating.
The Luttinger liquid description breaks down and the system ends up in a gapped phase, pinning the field to the semiclassical minima of the interaction.

(iv) For $x<0$ and $y<|x|$, the system flows towards $y\rightarrow\infty$, leading to the same gapped phase as in case (iii).
The difference is given by the starting point: \DIFdelbegin \DIFdel{Contrary }\DIFdelend \DIFaddbegin \DIFadd{contrary }\DIFaddend to (iii), the dominance of $y$ is entirely determined by the initial value of $x$, and even an infinitesimal interaction amplitude results in the breakdown of the Luttinger liquid description and the formation of an energy gap.

To characterize the trajectories of the flow chart, one must notice that $x\rd x = y\rd y$ in the proximity of $x=0$, which describes the conserved quantity $c = x^2-y^2$.
The constant $c$ thus describes hyperbolas cutting the $x$ and $y$ axis for $c>0$ and $c<0$, respectively (cf. \cref{fig:bkt_flow_equations} (b)).
We focus now on the critical line $c=0, x>0$ separating relevant and irrelevant $y$.
Assume now that the microscopic Hamiltonian is described by an internal parameter $\lambda>0$ which drives the system to the BKT critical line at $\lambda_c$.
Assume further that $y$ is irrelevant for $\lambda<\lambda_c$, and relevant for $\lambda>\lambda_c$.
There are two cases:

(a) For $c>0$ and $x>0$, $y$ is irrelevant and the trajectory flows to the fix point at $x_{\rm eff} = \lim_{l\rightarrow\infty} x(l)$ and $y=0$.
We can \DIFdelbegin \DIFdel{thus }\DIFdelend linearize $c=b^2(\lambda_c-\lambda)$ in the vicinity of the critical line, in which $\lambda$ encodes an internal parameter of the microscopic Hamiltonian that drives the phase transition (see previous paragraph).
Correlations of the fields $\phi/\theta$ still decay logarithmically and the coupling is being renormalized to the asymptotic value $K_{\rm eff} = \frac4{\beta^2}\brlr{b\sqrt{\lambda_c-\lambda}+2}$, which describes a square-root singularity close to the critical point.

(b) For $c<0$ and $x>0$, $y$ is relevant and pins the field configuration to the semiclassical minima of the sine-Gordon interaction.
The differential equation can be recast to
\begin{align}
    \frac{\rd x}{\rd l} = -y^2 = c-x^2 = -(|c|+x^2)
    \Rightarrow
    \int_{x(0)}^{x(l)}\rd x(x^2+|c|)^{-1} = -l
    .
\end{align}
The solution is thus given by $x(l) = \sqrt{|c|}\tan(\arctan(x(0)/\sqrt{|c|})-l\sqrt{|c|})$ and the integration must be terminated when $|x(l^*)|\sim 1$ (for which the perturbative calculation is invalid), i.e.
\begin{align}
    l^*=\frac{\pi}{2\sqrt{|c|}}+\frac{\arctan\brlr{\frac{x(0)}{\sqrt{|c|}}}}{\sqrt{|c|}}\approx \frac{\pi}{2b\sqrt{\lambda-\lambda_c}}.
\end{align}
An estimate of the correlation length in the system after the RG flow is given by \cref{eq:correlation_length_rg_flow}, and evaluates to $\xi\propto a\exp(l^*)=a\exp[{{\pi}/({2b\sqrt{\lambda-\lambda_c}})}]$.
Any attempt to realize systems close to the phase transition \DIFdelbegin \DIFdel{thus }\DIFdelend requires system lengths in excess of the exponentially divergent correlation length.
The requirement for numerical techniques is \DIFdelbegin \DIFdel{thus }\DIFdelend two-fold: resolving the phase transition requires excessively long systems, and must capture a divergent correlation length.
One simulation technique which was successfully applied for such tough systems is the density matrix renormalization group (DMRG) based on matrix product states (MPS), which we formalize in \cref{ch:matrix_product_states}.
This is to say that the bottleneck of MPS to be efficient is based on simulating systems with a short correlation length (see \cref{sec:scaling_relations_of_the_entanglement_entropy}).
However, MPS provides a possibility to extrapolate the results to both infinite system sizes and correlation lengths\DIFdelbegin \footnote{\DIFdel{To be explicit here, through a finite system size scaling, followed by a finite bond dimension scaling, which controls the effective correlation length captured by the MPS approximation.}}%DIFAUXCMD
\addtocounter{footnote}{-1}%DIFAUXCMD
\DIFdelend , which overcomes the related issues and allows to extract quantitative results with great success\DIFdelbegin \DIFdel{, even if the correlation length is large}\DIFdelend .


In \cref{one_half1,integer1}, we encounter various types of sine-Gordon terms embedded in a two-component Luttinger liquid.
Among those which are potentially relevant, we characterize the phase diagram by a second order RG analysis equivalent to the example detailed in this section.
In general, the resulting RG equations are not the simple two-variable BKT equations and they form a higher dimensional flow chart.
We thus rely on numerical integration of coupled differential equations through Runge-Kutta methods to determine the relevancy of operators.
For more quantitative estimates, we use MPS simulations.
\clearpage{}
\clearpage{}\chapter{Matrix product states}
\label{ch:matrix_product_states}
The analytical treatment presented in the previous chapter relies on a heavy line of approximations.
One of the most crucial \DIFdelbegin \DIFdel{consequence }\DIFdelend \DIFaddbegin \DIFadd{consequences }\DIFaddend of these simplifications is that quantitative deviations occur if the interactions are comparable to the kinetic bandwidth.
This restricts heavily the validity region of the low-energy effective field theory of a particular model.
To make an example, a sine-Gordon term with $\beta=4$ is encountered in the effective low-energy field theory of the spin-$1/2$ XXZ chain and the interaction will be relevant in the RG sense for $K<1/2$.
Such a small Luttinger parameter typically requires very strong nearest neighbor interactions that are compatible with the \DIFdelbegin \DIFdel{system's bandwidth }\DIFdelend \DIFaddbegin \DIFadd{bandwidth of the system}\DIFaddend \footnote{The XXZ model is integrable and the phase transition occurs when all couplings are equal to each other.}, which thus implicitly breaks the assumptions to derive the effective low-energy field theory in the first place.
Although the field theoretic description might not break down entirely, heavy quantitative deviations from the Luttinger liquid predictions of the effective coupling parameters are to be expected.
This motivates the use of numerical tools to verify the analytic predictions in the strongly interacting cases.
In this chapter, we will provide the essentials to get acquainted with the concepts of tensor networks, in particular with matrix product states (MPS).

First, we give a basic introduction to tensor networks and explain the intuitive concept of renormalization through a truncation of the auxiliary dimension of these structures.
Second, we review the concept of ``area laws'' -- a statement about the structure of quantum correlations of gapped and short-ranged Hamiltonians -- and connect it to the properties of tensor networks.
We then present renormalization strategies which provide variational search algorithms to target ground (and low-lying excited) states and continue by presenting a simple yet effective time-evolution algorithm in the framework of MPS.
At the end of the chapter, we explain how temperature and mixed states can be encoded in the framework of MPS, and review shortly the exploit of Abelian symmetries in the numerical techniques which in general reduces the overall computational complexity.
\section{Tensor networks}
\label{sec:tensor_networks}
\DIFdelbegin \DIFdel{A tensor is a collection of complex numbers in multidimensional arrays}\DIFdelend \DIFaddbegin \DIFadd{In the context of tensor networks, a tensor is understood as a multidimensional array containing complex elements}\DIFaddend .
Its rank is \DIFdelbegin \DIFdel{given by }\DIFdelend \DIFaddbegin \DIFadd{defined as }\DIFaddend the number of indices: \DIFdelbegin \DIFdel{for instance, }\DIFdelend scalar values are of rank zero, vectors of rank one and matrices of rank two\footnote{
    Note that \DIFdelbegin \DIFdel{the rank }\DIFdelend \DIFaddbegin \DIFadd{this definition }\DIFaddend of \DIFdelbegin \DIFdel{a tensor }\DIFdelend \DIFaddbegin \DIFadd{``rank'' }\DIFaddend differs from the usual understanding \DIFdelbegin \DIFdel{of a rank in terms of the }\DIFdelend \DIFaddbegin \DIFadd{as }\DIFaddend image space \DIFdelbegin \DIFdel{dimensions}\DIFdelend \DIFaddbegin \DIFadd{dimension}\DIFaddend .
}.
Tensor networks are in general contractions of many tensors and as such it is customary to define graphical representations which display its fundamental structure, i.e. its rank.
The most common convention is to represent a tensor object as a circle, triangle, hourglass or rectangle with as many legs as the tensor has indices.
It is then easy to sketch the tensor contraction of a simple matrix-matrix multiplication as
\begin{align}
    C_{i,j}=A_{ik}B_{kj}\equiv\includegraphics[valign=c]{figures/MatrixMultiplication.pdf},
\end{align}
in which we conveniently use the sum convention.
Switching from formulas to graphical notation may become interesting when considering larger tensor networks.
One easy example is a trace of many matrices, which can be sketched as a circle contraction.
\begin{align}
    \tr\brlr{ABCDEF} \equiv \includegraphics[valign=c]{figures/MatrixTrace.pdf}.
\end{align}
For more complex contractions containing tensors of higher ranks, it is important to optimize the sequence of contractions to reduce the overall computational cost to a minimum.
For instance, in \cref{fig:contraction_sequences} two different sequences of contractions obviously result in the same outcome, but the overall complexity class is different.
Assume each leg has dimension $m$, then sequence $S_1$ is $O(m^4)$ whereas sequence $S_2$ is $O(m^5)$.
\begin{figure}
    \centering
    \includegraphics[width=0.8\textwidth]{figures/ComplexContraction1.pdf}
    \caption{Different complexity classes for the same contraction, but different sequences. Assume each index has dimension $m$, then the green contractions are $O(m^4)$, but red is $O(m^5)$. Hence, sequence $1$ requires less multiplications compared to $2$ and is \DIFdelbeginFL \DIFdelFL{thus }\DIFdelendFL less prone to numerical truncation errors.}
    \label{fig:contraction_sequences}
\end{figure}

A triangle representation may make sense to indicate a tensor being an isometry.
For instance, consider the two isometries $U$, $V^\dag$ of a generic singular value decomposition (SVD) $M = U \Lambda V^\dag$: they contain the left- and right-singular vectors of $M$ and are thus semi-unitaries, i.e.
\begin{align}
    U^\dag U^\pdag = \mathbb1,
    \quad
    V^\dag V^\pdag = \mathbb1.
\end{align}
In graphical notation, the identity matrix is represented as a straight line as it does neither stretch nor rotate the entries of a tensor.
The previous equation can \DIFdelbegin \DIFdel{thus }\DIFdelend be recast to
\begin{align}
    U^\dag U^\pdag \equiv \includegraphics[valign=c]{figures/right_isometry.pdf} = \includegraphics[valign=c]{figures/right_identity.pdf}\,,
    \qquad
    V^\dag V^\pdag \equiv \includegraphics[valign=c]{figures/left_isometry.pdf} = \includegraphics[valign=c]{figures/left_identity.pdf}\,.
    \label{eq:isometries}
\end{align}
Although the two pictures represent the same multiplication at the present stage, a distinction of the two will be useful in the context of canonical MPS at a later point (in particular, in \cref{eq:isometry_tensors_diagram}).
A simple squared unitary matrix is orthogonal with respect to both left/right multiplication with its adjoint and as such can be represented as a hourglass.
The SVD of a normal matrix $M$ can be recast to the graphical notation
\begin{align}
    \includegraphics[valign=c]{figures/matrix.pdf}
    \equiv M = U \Lambda V^\dag \equiv
    \includegraphics[valign=c]{figures/svd.pdf}\,.
\end{align}
Let me now introduce probably the most important decomposition applied in tensor networks -- the generalized version of the SVD for rank $n$ tensors\footnote{For tensor network experts, do not confuse with the higher order SVD (HOSVD) or Tucker decomposition.}:
(i) the rank $n$ array has to be reduced to a rank $2$ object, i.e. a matrix.
This transformation (which must be a bijection) is commonly called tensor reshaping and is understood in graphical notation as a fusion of multiple tensor legs.
(ii) The matrix then is decomposed through a standard SVD, followed by (iii) the restoration of the original rank $n$ object through the inverse reshaping applied in step (i).
The identity relation between the original and decomposed tensor is assured by the inverse of (i).
For a practical example, consider a rank $n$ tensor $T$ which is graphically depicted by $n$ legs (labelled by increasing numbers from $1,...,n$ from left to right).
To perform a SVD between the bipartition after leg $j$, the tensor is reshaped to a matrix through a fusion of legs $1,...,j$ and $j+1,...,n$ prior to the performed SVD, after which the original rank is restored.
In particular, the sequence of identities reads
\DIFdelbegin %DIFDELCMD < \begin{align}
%DIFDELCMD <     T_{i_1,i_n}^{i_2,\dots,i_j,i_{j+1},\dots,i_{n-1}}
%DIFDELCMD <     \equiv&\includegraphics[valign=c]{figures/rankntensor.pdf},
%DIFDELCMD <     \\
%DIFDELCMD <     \overset{(i)}{=}&
%DIFDELCMD <     f^{-1}\circ
%DIFDELCMD <     \includegraphics[valign=c]{figures/reshaping.pdf}
%DIFDELCMD <     \equiv
%DIFDELCMD <     f^{-1}\circ U^\pdag_{(i_1,\dots,i_j),k}\Lambda^\pdag_{k}V^\dag_{k,(i_{j+1},\dots,i_n)},
%DIFDELCMD <     \\
%DIFDELCMD <     \overset{(ii)}{=}&
%DIFDELCMD <     f^{-1}\circ\includegraphics[valign=c]{figures/bigsvd.pdf}
%DIFDELCMD <     \equiv
%DIFDELCMD <     f^{-1}\circ U^\pdag_{(i_1,\dots,i_j),k}\Lambda^\pdag_{k}V^\dag_{k,(i_{j+1},\dots,i_n)},
%DIFDELCMD <     \\
%DIFDELCMD <     \overset{(iii)}{=}&
%DIFDELCMD <     \includegraphics[valign=c]{figures/rankntensor_decomposed.pdf}
%DIFDELCMD <     \equiv
%DIFDELCMD <     U_{i_1,k}^{\pdag i_2,\dots,i_j}\Lambda^\pdag_{k}{V^\dag}^{i_{j+1},\dots,i_{n-1}}_{k,i_n},
%DIFDELCMD <     \label{eq:SVD_generalized}
%DIFDELCMD < \end{align}%%%
\DIFdelend \DIFaddbegin \begin{align}
    T_{i_1,i_n}^{i_2,\dots,i_j,i_{j+1},\dots,i_{n-1}}
    \equiv&\includegraphics[valign=c]{figures/rankntensor.pdf}
    \\
    \overset{(i)}{=}&
    f^{-1}\circ
    \includegraphics[valign=c]{figures/reshaping.pdf}
    \equiv
    f^{-1}\circ U^\pdag_{(i_1,\dots,i_j),k}\Lambda^\pdag_{k}V^\dag_{k,(i_{j+1},\dots,i_n)}
    \\
    \overset{(ii)}{=}&
    f^{-1}\circ\includegraphics[valign=c]{figures/bigsvd.pdf}
    \equiv
    f^{-1}\circ U^\pdag_{(i_1,\dots,i_j),k}\Lambda^\pdag_{k}V^\dag_{k,(i_{j+1},\dots,i_n)}
    \\
    \overset{(iii)}{=}&
    \includegraphics[valign=c]{figures/rankntensor_decomposed.pdf}
    \equiv
    U_{i_1,k}^{\pdag i_2,\dots,i_j}\Lambda^\pdag_{k}{V^\dag}^{i_{j+1},\dots,i_{n-1}}_{k,i_n},
    \label{eq:SVD_generalized}
\end{align}\DIFaddend 
in which $f^{-1}$ denotes the inverse of the tensor reshaping / leg fusion, and Einstein notation implies the tensor contraction \DIFdelbegin \DIFdel{contraction }\DIFdelend with summed index $k$.
In the equations above, it is assumed that the applied SVD is compact: $\Lambda$ is a square and diagonal matrix containing the nonzero singular values, such that $\Lambda_{ij}\equiv s_i\delta_{ij}$.

To get acquainted with the use of tensors in a more practical sense, let us aim to understand some of the described concepts in the context of two-dimensional pictures.
If we assume the picture to be rasterized in an equidistant manner using square pixels, the RGB-entries of the pixel can be collected in a 3D array of the form
\begin{align}
    P_{y,x}^{c}\equiv\includegraphics[valign=c]{figures/tensor_picture.pdf}.
\end{align}
The position of the pixel is encoded by the indices $y,x$ and $c$ represents a leg containing the RGB-values (e.g., $c\in\{1,2,3\}\equiv\{r,g,b\}$).
Using the concept of the generalized SVD, we can decompose the tensor $P$ into two semi-unitaries $U,V^\dag$ and a vector $S_k$ containing the singular values of the bipartition.
There exist three different bipartitions: (a) $P_{(y,x),c}$, (b) $P_{(y,c),x}$ and (c) $P_{y,(x,c)}$.
(b) and (c) are equivalent up to a transposition of the original object, therefore leaving two truly different decompositions, (a) and (b).
The bipartition (a) targets a decomposition between $(y,x)$ and $c$ whereas (b) targets a decomposition between $y$ and $(x,c)$.
Strategy (a) thus aims to truncate the color degree of freedom, which is not very useful due to the low dimensionality of the $c$ leg (there are only $3$ nonzero singular values to begin with), such that we will focus entirely on the decomposition of (b).

\begin{figure}
    \centering
    \subfigure[]{\includegraphics{figures/SVD_uncompressed.png}}
    \hfil
    \subfigure[]{\includegraphics{figures/beach_singular_values.png}}
    \hfil
    \subfigure[]{\includegraphics{figures/SVD_compressed.png}}
    \hfil
    \subfigure[]{\includegraphics{figures/SVD_compressed2.png}}
    \hfil
    \subfigure[]{\includegraphics{figures/SVD_compressed3.png}}
    \hfil
    \subfigure[]{\includegraphics{figures/SVD_compressed4.png}}
    \caption{(a) Uncompressed picture of Castelsardo (Sardinia, IT).
    The digital picture consists of $n_y\times n_x=3024\times2276$ pixels. (b) Singular values of the matrix $P_{i,(jc)}$ representing the picture. (c) Keeping the $m\equiv260$ largest singular values still yields a good approximation of the uncompressed picture and requires only $20\%$ of the original storage space. Panels (d)-(f) visualize the impact of progressively lowering $m$ (later we will call this quantity the ``bond dimension''): while details are still visible in picture (d), they gradually disappear in panel (e) until even the dominant shapes become unrecognizable in figure (f).}
    \label{fig:svd_image_compression}
\end{figure}
If the singular/spectral values are ordered, a truncation of the $\Lambda$ matrix neglects subdominant left and right eigenvectors of the original object and as such approximates the original matrix.
In case of a subsequent normalization to the original trace, the truncation process can be understood as a compression or tensor renormalization scheme.
The number of singular values kept in the approximation is \DIFdelbegin \DIFdel{thus }\DIFdelend naturally linked to the accuracy of the numerical renormalization.
Assume the picture to be of dimensions $\dim(P)=p_y\times p_x\times 3$, then keeping $m$ singular values implies that we need to store only a fraction of the matrices $U$ and $V^\dag$ to approximate the original picture $P$.
Assuming that the matrix $M\equiv \{P_{y,(x,c)}\}_{y,x,c}$ is of dimension $n_y=p_y$ times $n_x=3p_x$, the following visualization highlights the SVD
\begin{align}
    \includegraphics[valign=c]{figures/svd_mmat.pdf}
    =
    \includegraphics[valign=c]{figures/svd_umat.pdf}
    \
    \includegraphics[valign=c]{figures/svd_smat.pdf}
    \
    \includegraphics[valign=c]{figures/svd_vmat.pdf}.
    \label{eq:compression_scheme}
\end{align}
If we limit to $m$ spectral values, we only need to store $n_y\cdot m$ elements from $U$, $m$ singular values and $m\cdot n_x$ elements from $V^\dag$.
In total, we \DIFdelbegin \DIFdel{thus }\DIFdelend arrive at a compression rate
\begin{align}
    C_m = \frac{(n_x+n_y+1)m}{n_x n_y},
\end{align}
which allows for a high compression in case of a large and dense matrix $M$.
To give an example, we present the impact of compression in \cref{fig:svd_image_compression}.
Panel (a) denotes an uncompressed picture, the singular values of the digital picture presented in (b), and the result of keeping $m=260$ singular values as an approximation to the full picture, presented in (c).
The impact of gradually decreasing the value of $m$ is visualized in panels (d) - (f).

In a second example, we consider a fraction of the Mandelbrot set embedded in Gaussian noise presented in \cref{fig:svd_image_compression_mandelbrot}.
The visualization aims to compare the original picture for different levels of noise (panels (b) and (d-f)) with the approximation (panels (c) and (g-i)) in which the 100 smallest Schmidt values are kept (indicated by the black grid line in panel (a)).
The distribution of the spectral values is controlled by the fraction of random noise mixed into the picture: the decay is weaker for a larger fraction of noise\footnote{The spectral properties of random matrices are well-known, beautifully summarized through the Wigner ``semicircle'' distribution~\cite{Wigner1958OnTD}.}, causing a larger truncation error in the approximation, which ultimately leads to significant deviations to the original contour details and overall color.

\begin{figure}
    \centering
    \subfigure[]{\includegraphics{figures/mandelbrot_svds.png}}
    \hfil
    \subfigure[]{\includegraphics{figures/mandelbrot_noise_original1.png}}
    \hfil
    \subfigure[]{\includegraphics{figures/mandelbrot_noise_approx1.png}}
    \\
    \subfigure[]{\includegraphics{figures/mandelbrot_noise_original4.png}}
    \hfil
    \subfigure[]{\includegraphics{figures/mandelbrot_noise_original5.png}}
    \hfil
    \subfigure[]{\includegraphics{figures/mandelbrot_noise_original6.png}}
    \\
    \subfigure[]{\includegraphics{figures/mandelbrot_noise_approx4.png}}
    \hfil
    \subfigure[]{\includegraphics{figures/mandelbrot_noise_approx5.png}}
    \hfil
    \subfigure[]{\includegraphics{figures/mandelbrot_noise_approx6.png}}
    \caption{(a) SVD of an image of a portion of the Mandelbrot set, mixed with different levels of white noise (black to bright gray: $0$, $0.2$, $0.4$ and $0.8$), plotted in panels (b,d-f).
    The original picture shows a fast decay of the Schmidt values compared to the higher noise levels, leading to a reasonable approximation (compare panel (b) with panel (c)). The amount of noise alters the algebraic decay of the Schmidt values, ultimately resulting in a larger error, and more noticeable deviations to the original picture (compare panels (d-f) with panels (g-i)). The number of Schmidt values kept in the approximations is fixed to $100$.}
    \label{fig:svd_image_compression_mandelbrot}
\end{figure}

The depicted compression scheme can actually be applied to quantum states.
Recall that any state can be expressed by a set of $L$ good quantum numbers, such that
\begin{align}
    \ket{\psi} = \sum_{i_1,i_2,\dots,i_L}C({i_1,i_2,\dots,i_L})\ket{i_1,i_2,\dots,i_L}
    \label{eq:generic_state}
\end{align}
and the collection of all $\ket{i_1,i_2,\dots,i_L}$ form a complete and orthonormal basis.
Note that the $L$-dimensional array $C$ depends on all internal quantum numbers and is \DIFdelbegin \DIFdel{thus }\DIFdelend exponentially large in the system size $L$ and the dimension of the quantum numbers $i_j$.
By the visual conventions, we may imagine the coefficients $C$ as a rank $L$ tensor, similar to the tensor $T$ discussed in \cref{eq:SVD_generalized}, i.e.
\begin{align}
    C(i_1,i_2,\dots,i_L) \rightarrow C^{[i_1,i_2,\dots,i_L]} \equiv  \includegraphics[valign=c]{figures/arbitrary_wavefunction.pdf}.
\end{align}
To account for the fact that the indices ${\bm i} = i_1,i_2,\dots,i_L$ are not fixed to specific values here, we will from now on denote the dependence of the tensor $C$ on the set of quantum numbers with square brackets.
We now proceed by applying the generalized SVD in a sequential manner:
Starting with the decomposition spanned by the bipartition of the first (last) leg, keeping the left (right) isometry and shifting the right (left) isometry to the adjacent leg position, is possible to decompose $C$ into a product of isometries with a single matrix in between two arbitrary leg positions $\ell,\ell+1$.
The decomposition sequence is easily reformulated in graphical notation
\DIFdelbegin %DIFDELCMD < \begin{align}
%DIFDELCMD <     C^{[i_1,i_2,\dots,i_L]} \equiv \includegraphics[valign=c]{figures/arbitrary_wavefunction.pdf},
%DIFDELCMD <     =
%DIFDELCMD <     \includegraphics[valign=c]{figures/arbitrary_wavefunction_partially_decomposed_1.pdf},
%DIFDELCMD <     \\
%DIFDELCMD <     =
%DIFDELCMD <     \includegraphics[valign=c]{figures/arbitrary_wavefunction_partially_decomposed_2.pdf},
%DIFDELCMD <     \\
%DIFDELCMD <     =
%DIFDELCMD <     \includegraphics[valign=c]{figures/arbitrary_wavefunction_partially_decomposed_3.pdf},
%DIFDELCMD <     \\
%DIFDELCMD <     =\includegraphics[valign=c]{figures/arbitrary_wavefunction_decomposed.pdf},
%DIFDELCMD <     \label{eq:move_the_lambda_matrix}
%DIFDELCMD < \end{align}%%%
\DIFdelend \DIFaddbegin \begin{align}
    C^{[i_1,i_2,\dots,i_L]} \equiv \includegraphics[valign=c]{figures/arbitrary_wavefunction.pdf}
    =
    \includegraphics[valign=c]{figures/arbitrary_wavefunction_partially_decomposed_1.pdf}\phantom{,}
    \\
    =
    \includegraphics[valign=c]{figures/arbitrary_wavefunction_partially_decomposed_2.pdf}\phantom{,}
    \\
    =
    \includegraphics[valign=c]{figures/arbitrary_wavefunction_partially_decomposed_3.pdf}\phantom{,}
    \\
    =\includegraphics[valign=c]{figures/arbitrary_wavefunction_decomposed.pdf},
    \label{eq:move_the_lambda_matrix}
\end{align}\DIFaddend 
in which at each step the three gray tensors are contracted before performing the next SVD.
This particular decomposition of a generic quantum state, i.e.
\begin{align}
    C^{[i_1,i_2,\dots,i_L]} \equiv \includegraphics[valign=c]{figures/arbitrary_wavefunction_decomposed.pdf},
    \label{eq:mixed_canonical_form}
\end{align}
is called the (mixed) canonical form of an MPS\footnote{One could even perform the sequence \cref{eq:move_the_lambda_matrix} by leaving the Schmidt weights on every link, thus obtaining another canonical form -- the so-called $\Gamma-\lambda$ form~\cite{Vidal2007,Orus2008}.}.
In other words, it represents the Schmidt decomposition of the state $\ket{\psi}$ in the subspaces $\HS_{A(\ell)}$ and $\HS_{B(\ell)}$ spanned by the first $\ell$ and last $L-\ell$ quantum numbers i.e.
\begin{align}
    \ket\psi = \sum_{\bm i} C^{[\bm i]}\ket{\bm i} = \sum_{k}s_{k}(\ell)\ket{\psi_{A(\ell),k}}
    \ket{\psi_{B(\ell),k}},
    \label{eq:schmidt_decomposition}
\end{align}
in which the transformed states $\ket{\psi_{A/B,k}}$ are given by the SVD-decomposed coefficient tensor
\begin{align}
    \ket{\psi_{A(\ell),k}}
    &=
    U^{[i_1]}_{k_1} U^{[i_2]}_{k_1,k_2}U^{[i_3]}_{k_2,k_3}\dots U^{[i_\ell]}_{k_{\ell-1},k}\ket{i_1,i_2,i_3,\dots,i_j}
    =\prod_{k=1}^\ell U^{[i_k]}\ket{i_1,\dots,i_\ell},\\
\ket{\psi_{B(\ell),k}}
    &=
    {V^\dag}^{[i_{\ell+1}]}_{k,k_{\ell+1}} {V^\dag}^{[i_{\ell+2}]}_{k_{\ell+1},k_{\ell+2}}\dots {V^\dag}^{[i_L]}_{k_{L-1}}\ket{i_{\ell+1},\dots,i_L}
    =
    \prod_{k=\ell+1}^L V^{\dag[i_k]}\ket{i_{\ell+1},\dots,i_L}.
    \label{eq:bipartition_orthonormal_basis}
\end{align}
In the two expressions above, we conveniently use the sum convention.
Note that the position of the central matrix in \cref{eq:mixed_canonical_form} is not fixed to the center of the chain.
It can be moved arbitrarily to the right (left) by a sequence of singular value decompositions
\begin{align}
    \Lambda(\ell) V^{\dag[i_{\ell+1}]}V^{\dag[i_{\ell+2}]} &= U^{[i_{\ell+1}]}\Lambda(\ell+1) \tilde V^{\dag[i_{\ell+1}]}
    \equiv
    \includegraphics[valign=c]{figures/fMPS_3_1.pdf}
    =
    \includegraphics[valign=c]{figures/fMPS_3_2.pdf},
    \\
    U^{[i_{\ell-1}]}U^{[i_{\ell}]}\Lambda(\ell) &= \tilde U^{[i_{\ell-1}]}\Lambda(\ell-1) \tilde V^{\dag[i_{\ell}]}
    \equiv
    \includegraphics[valign=c]{figures/fMPS_4_1.pdf}
    =
    \includegraphics[valign=c]{figures/fMPS_4_2.pdf}.
\end{align}
Due to its fundamental importance for the remaining section, we will from now on use the notation
\begin{align}
    \ket\psi = \brlr{\prod_{k=1}^\ell U^{[i_k]}}\Lambda(\ell)\brlr{\prod_{k=\ell+1}^L V^{\dag[i_k]}}\ket{i_1,i_2,\dots,i_L}
\end{align}
to denote the mixed canonical decomposition of the coefficient tensor $C$ with rank-$3$ isometries ${U^{[i_k]}/V^{\dag[i_k]}}$ and a matrix $\Lambda(\ell)$ containing the Schmidt values of the bipartition after site $\ell$.
It is now obvious that (i) any quantum state can be decomposed into an MPS, (ii) that each MPS can be brought into the mixed canonical form and (iii) that MPS have a large gauge degree of freedom.
Point (iii) is clear from the fact that one may sandwich infinitely many identities in the MPS (each written as a product of two unitary matrices) which may in general rotate the colored isometry tensors, but leave the overall state invariant.
\section{Reduced density matrix and Rényi entropy}
\label{sec:reduced_density_matrix_and_renyi_entropy}
To understand the influence of the compression scheme, consider the density matrix of an MPS in canonical form
\DIFdelbegin \DIFdel{:
}\DIFdelend \begin{align}
    \hat\rho = \sum_{k,k'}s_ks_{k'}\ket{\psi_{A,k},\psi_{B,k}}\bra{\psi_{A,k'},\psi_{B,k'}}.
\end{align}
\DIFdelbegin \DIFdel{Now imagine to compute the reduced density matrix of a bipartition formed by a block $A(\ell)$ of the first $\ell$ }\DIFdelend \DIFaddbegin \DIFadd{Imagine a bipartition $A|B$ with $A(\ell)$ formed by the first $1,\dots,\ell$ }\DIFaddend quantum numbers and \DIFdelbegin \DIFdel{a block }\DIFdelend $B(\ell)$ \DIFdelbegin \DIFdel{of }\DIFdelend \DIFaddbegin \DIFadd{containing }\DIFaddend the remaining ones \DIFaddbegin \DIFadd{$\ell+1,\dots,L$ ($L$ denotes the total number of sites)}\DIFaddend .
A trace over the complement states of \DIFdelbegin \DIFdel{the partition $A/B$ }\DIFdelend \DIFaddbegin \DIFadd{one of the partitions }\DIFaddend gives the reduced density matrix \DIFdelbegin \DIFdel{in block$A/B$}\DIFdelend \DIFaddbegin \DIFadd{of the block}\DIFaddend , i.e.
\begin{align}
    \hat\rho_A = \tr_B(\hat\rho)
    &=
    \sum_{k,k',k''}
    s_ks_{k'}
    \braket{\psi_{B,k''}|\psi_{A,k},\psi_{B,k}}\braket{\psi_{A,k'},\psi_{B,k'}|\psi_{B,k''}}
    =
    \sum_k s_k^2 \ket{\psi_{A,k}}\bra{\psi_{A,k}},\nonumber\\
\hat\rho_B = \tr_A(\hat\rho)
    &=
    \sum_k s_k^2 \ket{\psi_{B,k}}\bra{\psi_{B,k}}.
\end{align}
Note that normalization of the quantum state $\ket\psi$ implies $\tr\rho=\sum_k s_k^2=1$ and $s_k^2$ can be interpreted as the probability of state $\ket{\psi_{A/B,k}}$ contributing to the reduced density matrix $\hat\rho_{A/B}$.
The reduced density matrix is thus diagonal in the Schmidt decomposition, and solely determined by the spectral values $s_k$.
Imagine now to approximate the state $\ket\psi$ with a state $\ket{\psi'}$ by limiting the number of different $s_k$ in the bipartition $A,B$ such that the $m$ largest spectral values are kept in $\ket{\psi'}$ and the rest is set to zero.
Normalization of the state $\ket{\psi'}$ requires that $\sum_{k=1}^m {s'_k}^2$ must be renormalized to unity, which is satisfied if $s_k'=s_k/\sqrt{\sum_{l=1}^m s_l^2}$.
The threshold $m$ is \DIFdelbegin \DIFdel{thus }\DIFdelend directly related to approximating the reduced density matrix $\hat \rho_{A/B}$ with the $m$ most likely states in the Schmidt decomposition of a given bipartition $A/B$.
Please note that $s_k\equiv s_k(\ell)$ refers to the $k$'th singular value of a particular bipartition $A(\ell)/B(\ell)$, which considers the first/last $\ell/L-\ell$ quantum numbers.
If we keep the $m$ largest singular values $s_k(\ell)$, $k\in\{1,2,\dots,m\}$ for any block size $\ell$, $m$ is a global property of MPS called the bond dimension.

Density matrices are naturally related to the generalized Gibbs entropy for quantum states
\DIFdelbegin %DIFDELCMD < \begin{align}
%DIFDELCMD <     S_1(\ell) = -\tr(\hat\rho_A\log\hat\rho_A) = -\tr(\hat\rho_B\log\hat\rho_B) = -\sum_k s^2_k\log(s^2_k)
%DIFDELCMD < \end{align}%%%
\DIFdelend \DIFaddbegin \begin{align}
    S_1(\ell) = -\tr(\hat\rho_A\ln\hat\rho_A) = -\tr(\hat\rho_B\ln\hat\rho_B) = -\sum_k s^2_k\ln(s^2_k),
\end{align}\DIFaddend 
which is a straightforward measure for the shared information between the bipartition, in information theory perspectives often called Shannon entropy of communication.
In this context, \DIFdelbegin \DIFdel{$-\log(s^2_k)$ }\DIFdelend \DIFaddbegin \DIFadd{$-\ln(s^2_k)$ }\DIFaddend is understood as the information content (or surprisal) of the Schmidt state, and $S_1$ is simply the expectation value of surprisal.
Such a definition is understood intuitively: measuring a state of very low probability is \DIFdelbegin \DIFdel{surprizing}\DIFdelend \DIFaddbegin \DIFadd{surprising}\DIFaddend , whereas likely quantum states bear a low surprise.
In a quantum physics audience, the entropy $S_1$ is commonly referred to as von Neumann entropy, and we denote it as $S_1$ because of its relation to the Rényi entropy -- a generalized version of the Shannon entropy
\DIFdelbegin %DIFDELCMD < \begin{align}
%DIFDELCMD <     S_\alpha(\ell) = \frac{1}{1-\alpha}\log\tr\hat\rho^\alpha_A
%DIFDELCMD < \end{align}%%%
\DIFdelend \DIFaddbegin \begin{align}
    S_\alpha(\ell) = \frac{1}{1-\alpha}\ln\tr\hat\rho^\alpha_A
\end{align}\DIFaddend 
in the limit $\alpha\rightarrow1$.
\DIFdelbegin \DIFdel{The derivation of the limit is infrequently presented in the literature, but straightforward by using the Schmidt decomposition and L'Hôpital's rule
}%DIFDELCMD < \begin{align}
%DIFDELCMD <     S_1(\ell)
%DIFDELCMD <     =
%DIFDELCMD <     \lim_{\alpha\rightarrow1}\frac1{1-\alpha}\log\tr\hat\rho_A^\alpha
%DIFDELCMD <     =
%DIFDELCMD <     -\lim_{\alpha\rightarrow1}\partial_\alpha\log\sum_k s_k^{2\alpha}
%DIFDELCMD <     =
%DIFDELCMD <     -\lim_{\alpha\rightarrow1}\frac{\sum_k\partial_\alpha s_k^{2\alpha}}{\sum_k s_k^{2\alpha}}
%DIFDELCMD <     =
%DIFDELCMD <     -\lim_{\alpha\rightarrow1}\sum_k \partial_{\alpha}\re^{\alpha\log s_k^2}
%DIFDELCMD <     \\
%DIFDELCMD <     =
%DIFDELCMD <     -\lim_{\alpha\rightarrow1}\sum_k \re^{\alpha\log s_k^2}\log s_k^2
%DIFDELCMD <     =
%DIFDELCMD <     -\sum_k s_k^2\log s_k^2
%DIFDELCMD <     =
%DIFDELCMD <     -\tr \hat\rho_A\log\hat\rho_A.
%DIFDELCMD < \end{align}
%DIFDELCMD < %%%
\DIFdelend In conclusion, using MPS techniques in the canonical form provides immediate access to the information content of a given bipartition.
Note that the maximum entropy of an MPS is always restricted by the bond dimension $m$, i.e. \DIFdelbegin \DIFdel{$S_1(\ell)<\log(m)$}\DIFdelend \DIFaddbegin \DIFadd{$S_1(\ell)<\ln(m)$}\DIFaddend .
\section{Scaling relations of the entanglement entropy}
\label{sec:scaling_relations_of_the_entanglement_entropy}
All of the discussion so far would not be very exciting for a wholehearted physicist, were it not for its close relation to quantum entanglement.
The phenomenon was firstly noted in $1935$ by Albert Einstein, Boris Podolsky, and Nathan Rosen~\cite{EPR1935}, also by Erwin Schrödinger~\cite{Schrdinger1935,Schrdinger1936}, describing what is famously known as the EPR paradox~\cite{Reid2009}.
At its heart, it describes a group of particles interacting such that the quantum state of each individual particle cannot be described independently of the state of the others in the group.
The paradox part in the aforementioned works is described by quantum measurements performed on individual particles, which leads to irreversible collapses of the wave function and an alteration of the original quantum state of the other particles in the group.
Surprisingly, the phenomenon is independent of the distance between the particles and forms \DIFdelbegin \DIFdel{thus }\DIFdelend a unique form of correlations exclusively present in quantum systems.

To make an example, consider the two-particle pure state $\ket{\psi}=\frac1{\sqrt2}\brlr{\ket{\uparrow\downarrow}+\ket{\downarrow\uparrow}}$ and the two particles being localized at two distant spacial positions.
In case we prepare a projective measurement of the first particle being in the $\uparrow$-state, the probability of the outcome is obviously $1/2$ and we end up with a collapsed post-measurement state of the form $\ket{\tilde\psi} = \ket{\uparrow\downarrow}$.
At the time of the measurement, the second particle collapses to the perfectly correlated alignment, and it does so immediately, independent on the spacial distance between the two parties.
A subsequent measurement of the second particle will thus yield $\downarrow$ with probability $1$.
Note that this instantaneous action does not violate causality since the outcome of the first measurement is random and as such cannot be used to transmit information\footnote{Until the outcome of the first measurement is communicated, the density matrix of the second spin is that of a mixed state, i.e. $1/2$ times the identity.}.
With this example in mind, let us now encapsulate the definition of $N$-partite separable and entangled for states with $N$ individual particles.

A state $\ket{\psi}$ is called $N$-partite separable \DIFdelbegin \DIFdel{iff. }\DIFdelend \DIFaddbegin \DIFadd{if and only if }\DIFaddend it can be written as a product of $N$ subsystems~\cite{Horodecki2009}
\begin{align}
    \ket{\psi} = \ket{\psi_1}\otimes\ket{\psi_2}\otimes\dots\otimes\ket{\psi_N}.
    \label{eq:separable_states}
\end{align}
Note that this expression is fundamentally different from the coherent superposition of exponentially many state vectors encountered in generic states (compare to ~\cref{eq:generic_state}).
As such, we call a state $N$-partite entangled, if it cannot be decomposed according to~\cref{eq:separable_states}.
The earlier example of a bipartite system is an element of the Bell basis, which are \DIFdelbegin \DIFdel{sometimes }\DIFdelend \DIFaddbegin \DIFadd{also }\DIFaddend called EPR states
\DIFdelbegin %DIFDELCMD < \begin{align}
%DIFDELCMD <     \ket{\psi^\pm}=\frac 1{\sqrt2}\brlr{\ket{\uparrow\downarrow}\pm\ket{\downarrow\uparrow}}
%DIFDELCMD <     \quad
%DIFDELCMD <     \ket{\phi^\pm}=\frac 1{\sqrt2}\brlr{\ket{\uparrow\uparrow}\pm\ket{\downarrow\downarrow}}
%DIFDELCMD < \end{align}%%%
\DIFdelend \DIFaddbegin \begin{align}
    \ket{\psi^\pm}=\frac 1{\sqrt2}\brlr{\ket{\uparrow\downarrow}\pm\ket{\downarrow\uparrow}},
    \quad
    \ket{\phi^\pm}=\frac 1{\sqrt2}\brlr{\ket{\uparrow\uparrow}\pm\ket{\downarrow\downarrow}},
\end{align}\DIFaddend 
and they all have remarkable properties, first recognized by Schrödinger -- if one measures only one of the states, one finds it with equal probability in either $\ket{0}$ or $\ket{1}$.
As such, \DIFdelbegin \DIFdel{they give }\DIFdelend \DIFaddbegin \DIFadd{the reduced density matrices represent mixed states and reveal }\DIFaddend no information about the subsystems, \DIFdelbegin \DIFdel{whereas they are }\DIFdelend \DIFaddbegin \DIFadd{although the full state is }\DIFaddend still pure and \DIFdelbegin \DIFdel{thus reveal full information about the global system}\DIFdelend \DIFaddbegin \DIFadd{therefore fully characterized}\DIFaddend .
The Bell states are special cases of bipartite maximally entangled states~\cite{Horodecki2009} and as such great examples to explore the entropic manifestation of entanglement.

In particular, the von Neumann entropy of a product state vanishes\footnote{This is actually a consequence of a much more general statement for open quantum systems and mixed states: the entropy of a subsystem can be greater than the entropy of the global system \DIFdelbegin \DIFdel{iff. }\DIFdelend \DIFaddbegin \DIFadd{if and only if }\DIFaddend the state is entangled~\cite{Horodecki1994}.
The inequality for separable states can \DIFdelbegin \DIFdel{thus }\DIFdelend be expressed as $S(\hat\rho)-S(\hat\rho_{A/B})\geq0$ with $\hat\rho$ the density matrix of the mixed state and $\hat\rho_{A/B}$ the reduced density matrix of bipartition $A/B$.
As such, the von Neumann entropy serves as a scalar separability criteria, in analogy to the Bell inequalities~\cite{Horodecki2006}.}.
Note that the reduced density matriy of the Bell states is $\hat\rho_{A/B}=\frac12\mathbb1$ and as such \DIFdelbegin \DIFdel{$S_1=\log2$ }\DIFdelend \DIFaddbegin \DIFadd{$S_1=\ln2$ }\DIFaddend which provides an upper limit for this two-partite system.
In general, a maximally entangled $N$-partite pure state is maximally entangled if the reduced density matrix is proportional to the identity.
Assuming the local internal dimension is $D$, it assumes the upper bound of the entropy \DIFdelbegin \DIFdel{$S_1(\ell)=\ell\log(D)$}\DIFdelend \DIFaddbegin \DIFadd{$S_1(\ell)=\ell\ln(D)$}\DIFaddend .
On the contrary, a product state has vanishing entropy.

Coming back to the connection to MPS with finite bond dimension, we now know that the upper bound restriction of the entropy limits the available entanglement of the approximation.
This, in turn, does not provide an explanation why such an approximation should not ``go wrong quickly'' in a sense that the important properties of the approximated state are lost, even if the bond dimension is large.
Incidentally, at the time DMRG was introduced by Steven White~\cite{White1992}, the success of the renormalization scheme remained a mystery.
Some insights were concluded by the derivation of universal scaling laws of the entropy provided e.g. by Calabrese and Cardy in 2004~\cite{Calabrese2004} in proximity of a quantum critical point.
In 2007, Hastings published his famous proof of the area law for one-dimensional quantum systems~\cite{Hastings2007}.
\DIFdelbegin \DIFdel{In its essence, it provides an upper bound to the von Neumann entropy for }\DIFdelend \DIFaddbegin \DIFadd{It is a remarkable result concerning the low-energy excitations of }\DIFaddend gapped and local \DIFdelbegin \DIFdel{(i.e. no long-ranged interactions) Hamiltoniansscaling as the area ($\equiv$ a constant) of one-dimensional systems}\DIFdelend \DIFaddbegin \DIFadd{one-dimensional Hamiltonians, for which the von Neumann entropy scales as the surface area of the bipartition}\DIFaddend .
Essentially, the works can be summarized \DIFdelbegin \DIFdel{by the following two statements}\DIFdelend \DIFaddbegin \DIFadd{as follows}\DIFaddend ~\cite{Eisert2010}:
Consider a \DIFdelbegin \DIFdel{block of $1,2,\dots,\ell$ sites in a system of }\DIFdelend \DIFaddbegin \DIFadd{finite regular lattice of size }\DIFaddend $N$ \DIFdelbegin \DIFdel{sites in total, then the entanglement }\DIFdelend \DIFaddbegin \DIFadd{and with spacing $a$, and a bipartition $1,2,\dots,\ell$ and $\ell+1,\ell+2,\dots,N$.
Then the von Neumann }\DIFaddend entropy of the physically relevant part (i.e. the low-energy/temperature subspace of $\HS$) scales as
\DIFdelbegin %DIFDELCMD < \begin{align}
%DIFDELCMD <     S_1(\ell) &=
%DIFDELCMD <     \frac c{3b}
%DIFDELCMD <     \left\{
%DIFDELCMD <     \begin{array}{l}
%DIFDELCMD <         \log[d(\ell)],\text{ if the system is gapless, and if $N\gg\ell\gg 1$,}\\[0.25cm]
%DIFDELCMD <         \log[\xi/a],\,\text{ if the correlation length $\xi$ is finite, and if $N\gg\ell\gg\xi/a$.}
%DIFDELCMD <     \end{array}
%DIFDELCMD <     \right.
%DIFDELCMD <     \label{eq:entropy_scaling}
%DIFDELCMD < \end{align}%%%
\DIFdelend \DIFaddbegin \begin{align}
    S_1(\ell) &=
    \frac c{3b}
    \left\{
    \begin{array}{l}
        \ln[d(\ell)],\text{ if the system is gapless, and if $N\gg\ell\gg 1$,}\\[0.25cm]
        \ln[\xi/a],\,\text{ if the correlation length $\xi$ is finite, and if $N\gg\ell\gg\xi/a$.}
    \end{array}
    \right.
    \label{eq:entropy_scaling}
\end{align}\DIFaddend 
In the above, $b$ encodes periodic ($b=1$) and open ($b=2$) boundary conditions, $c$ is the central charge of the conformal field theory (in a fermionic model, it corresponds to the number of Fermi momenta) and $d(\ell)$ is the ``chord distance'', which describes the length of a semi-circle
\begin{align}
    d(\ell) = \left|\frac{N}{\pi}\sin(\pi\ell/N)\right|.
\end{align}
The main message of \cref{eq:entropy_scaling} is that, for gapped systems, physically relevant states live in the low-entangled subspace of the full Hilbert space.
This implies that the MPS Ansatz is an optimal choice to target the low-lying excitations of the full many body Hilbert space, which, in turn, follow an area law of entanglement.

The generalization of MPS to higher dimensional lattices are called projected entangled pair states (PEPS), which, like the MPS, have the area law already built into their very construction.
In particular, they satisfy $S(\rho_A)=\mathcal O(|\partial A|)$ in which $\partial A$ denotes the boundary area of any subset $A$ of lattice sites.
Although a general proof as in the case of 1D gapped Hamiltonians is presently not established, area laws have been proven for non-interacting and gapped models (both bosonic and fermionic), and have also been checked \DIFdelbegin \DIFdel{vor }\DIFdelend \DIFaddbegin \DIFadd{for }\DIFaddend various strongly correlated interacting systems~\cite{Eisert2010,Laflorencie2016}.
\section{Matrix product operators}
\label{sec:matrix_product_operators}
Let us now understand the action of operators in the basis of MPS.
Consider a local operator acting on site $q$, which can be written in general as $\hat o_q = \sum_{i_q',i_q''}o_{i_q',i_q''}\ket{i_q'}\bra{i_q''}$.
Its action on the canonical form of an MPS can be expressed as
\begin{align}
    \hat o_q\ket\psi &= \sum_{i_1,i_2,\dots,i_q,\dots,i_L}\sum_{i_q'}o_{i_q',i_q} C^{i_1,i_2,\dots,i_q,\dots,i_L}\ket{i_1,i_2,\dots,i_{q-1},i_q',i_{q+1},\dots,i_L}
    \\
    &= \sum_{i_1,i_2,\dots,i_L}\tilde C^{i_1,i_2,\dots,i_L}\ket{i_1,i_2,\dots,i_L}
\end{align}
and we note that it leaves the overall form of the MPS invariant.
In particular, it acts as a contraction between the $o$-matrix and the $q$'th vertical leg of the $C$-tensor, according to
\begin{align}
    \tilde C^{i_1,i_2,\dots,i_q,i_{q+1},\dots,i_{L-1},i_L} = \sum_{i_q'}o_{i_q,i_q'}C^{i_1,i_2,\dots,i_q',i_{q+1},\dots,i_{L-1},i_L}\equiv\includegraphics[valign=c]{figures/operator_mps.pdf}.
\end{align}
Therefore, we can represent any local operator as a rank-$2$ tensor composed by the matrix elements of $\hat o$.
The same reasoning applies for operators acting on $p$ sites -- they are represented by rank-$2p$ tensors containing the entries of the matrix elements, contracted with the $p$ corresponding vertical links of the $C$-tensor.

The most straightforward example of a matrix product operator \DIFaddbegin \DIFadd{(MPO) }\DIFaddend is given through the Hamiltonian\DIFdelbegin \DIFdel{, and the derivation straightforward.
}\DIFdelend \DIFaddbegin \DIFadd{:
}\DIFaddend Consider for example the Ising model
\begin{align}
    \hat H = \hat H_{h} + \hat H_{J},
    \quad
    \hat H_{J} = J\sum_{i=\ell}^{L-1}\hat X_\ell \hat X_{\ell+1},
    \quad
    \hat H_{h} = h\sum_{\ell=1}^{L}\hat Z_\ell,
\end{align}
in which $\hat X/\hat Y/\hat Z$ denote the spin-1/2 operators.
They have the following representations
\begin{align}
    \hat X_\ell/\hat Y_\ell/\hat Z_\ell = \sum_{i_\ell,i'_\ell\in\{\uparrow,\downarrow\}}X_{i_\ell,i_\ell'}/Y_{i_\ell,i_\ell'}/Z_{i_\ell,i_\ell'}\ket{i_\ell}\bra{i'_\ell},
\end{align}
in which $X/Y/Z$ are the spin-1/2 Pauli matrices
\begin{align}
    X =
    \begin{pmatrix}
        0 & 1 \\
        1 & 0
    \end{pmatrix}
    ,
    \quad
    Y =
    \begin{pmatrix}
        0 & -\ri \\
        \ri & 0
    \end{pmatrix}
    ,
    \quad
    Z =
    \begin{pmatrix}
        1 & 0 \\
        0 & -1
    \end{pmatrix}.
\end{align}
The additional subscript $\ell$ denotes an isolated action of $\hat X_\ell/\hat Y_\ell/\hat Z_\ell$ on the $\ell$'th spin of the system.

Note that the Hamiltonian can be recast into a product of operators according to
\begin{align}
    \hat H = \hat W_1\brlr{\prod_{\ell=2}^{L-1}\hat W_{\ell}}\hat W_L,
\end{align}
in which we conveniently use a set of bulk $\hat W_{\ell\neq1,L}$, and two additional boundary operators $\hat W_1$, $\hat W_L$.
The bulk operators have the form
\DIFdelbegin %DIFDELCMD < \begin{align}
%DIFDELCMD <     \hat W_{\ell\neq1,L} =
%DIFDELCMD <     \begin{pmatrix}
%DIFDELCMD <         \hat{\mathbb 1}_\ell & \hat X_\ell & h \hat Z_\ell\\
%DIFDELCMD <          0 & 0 & J \hat X_\ell\\
%DIFDELCMD <         0 & 0 & \hat{\mathbb 1}_\ell\\
%DIFDELCMD <     \end{pmatrix}
%DIFDELCMD < \end{align}%%%
\DIFdelend \DIFaddbegin \begin{align}
    \hat W_{\ell\neq1,L} =
    \begin{pmatrix}
        \hat{\mathbb 1}_\ell & \hat X_\ell & h \hat Z_\ell\\
         0 & 0 & J \hat X_\ell\\
        0 & 0 & \hat{\mathbb 1}_\ell\\
    \end{pmatrix},
\end{align}\DIFaddend 
representing a set of local operators acting locally on a single site $\ell$.
Similarly, the boundary operators are given by
\begin{align}
    \hat W_1 =
    \begin{pmatrix}
        1 & 0 & 0
    \end{pmatrix}
    \begin{pmatrix}
        \hat{\mathbb 1}_1 & \hat X_1 & h \hat Z_1\\
        0 & 0 & J \hat X_1\\
        0 & 0 & \hat{\mathbb 1}_1\\
    \end{pmatrix}
    =
    \begin{pmatrix}
        \hat{\mathbb 1}_1 & \hat X_1 & h \hat Z_1
    \end{pmatrix}
    ,
    \quad
    \hat W_L =
    \begin{pmatrix}
        \hat{\mathbb 1}_L & \hat X_L & h \hat Z_L\\
        0 & 0 & J \hat X_L\\
        0 & 0 & \hat{\mathbb 1}_L\\
    \end{pmatrix}
    \begin{pmatrix}
        0 \\ 0 \\ 1
    \end{pmatrix}
    \begin{pmatrix}
        h \hat Z_L\\
        J \hat X_L\\
        \hat{\mathbb 1}_L\\
    \end{pmatrix}.
    \label{eq:boundary_W}
\end{align}
Since $\hat W_\ell$ spans a matrix with entries composed of local operators, which each have an MPO representation, the action of the Hamiltonian on a state $\ket\psi$ can be understood as a product of rank-$4$ tensors $W^{[i_\ell',i_\ell]}$ containing the matrix elements of $\hat W_{\ell}$.
In particular, one obtains the following tensorial representation, acting on a single site $\ell\neq1,L$ of the bulk
\begin{align}
    W^{[i_\ell',i_\ell]} =
    \begin{pmatrix}
        \delta_{i_\ell',i_\ell} & X_{i_\ell',i_\ell} & h Z_{i_\ell',i_\ell} \\
        0 & 0 & J X_{i_\ell',i_\ell} \\
        0 & 0 & \delta_{i_\ell',i_\ell}
    \end{pmatrix}
    \equiv\includegraphics[valign=c]{figures/mpo.pdf}
    .
\end{align}
It is straightforward to introduce two additional boundary tensors, in equivalence to \cref{eq:boundary_W}
\begin{align}
    W^{[i_1',i_1]} =
    \begin{pmatrix}
        \delta_{i_1',i_1} & X_{i_1',i_1} & h Z_{i_1',i_1}
    \end{pmatrix}
    \equiv\includegraphics[valign=c]{figures/mpo_left.pdf}
    ,\quad
    W^{[i_L',i_L]} =
    \begin{pmatrix}
         h Z_{i_L',i_L} \\
         J X_{i_L',i_L} \\
         \delta_{i_L',i_L}
    \end{pmatrix}
    \equiv\includegraphics[valign=c]{figures/mpo_right.pdf}.
\end{align}
The MPO representation of the Hamiltonian is quite versatile -- any Hamiltonian can be recast into a set of matrix product operators that are upper triangular (or lower triangular), even if the interactions are long-ranged or spatially dependent, e.g. in the case of disorder.
Note that the MPO form is not limited to spin systems due to the famous Jordan-Wigner transformation.
It is defined in terms of the ladder operators with matrix representations $\hat\sigma^\pm = \frac12(\hat X\pm\ri \hat Y)$, i.e.
\begin{align}
    \hat c_i^\pdag = \brlr{\prod_{j<i}-\hat Z}\hat\sigma^-_i,
    \qquad
    \hat c_i^\dag = \brlr{\prod_{j<i}-\hat Z}\hat\sigma^+_i.
    \label{eq:jordan_wigner_trafo}
\end{align}
This way, the fermionic creation and annihilation operators are represented by spin ladder operators.
To restore the anticommutation relations of fermions, the additional phase string is required.
Note that the presence of the string of $\hat Z$'s in general imposes long-ranged interactions in two-dimensional systems, such that a direct application of the Jordan-Wigner transformation is primarily used in (quasi-)one-dimensional systems~\cite{Franchini2017}\footnote{For higher-dimensional lattices, the Bravyi-Kitaev mapping may be more suited~\cite{Bravyi2002}.}.
\section{Variational ground state search}
\label{sec:variational_ground_state_search}
The \DIFdelbegin \DIFdel{original }\DIFdelend variational ground state search \DIFdelbegin \DIFdel{is formalized in the }\DIFdelend \DIFaddbegin \DIFadd{in the context of MPS was initially formalized by the }\DIFaddend density matrix renormalization group (DMRG), introduced by Steven White in 1992~\cite{White1992}.
In it's essence, it is a numerical optimization scheme targetting the low energy sector of a given system.
Nowadays, it is understood as an MPS Ansatz of the trial wavefunction in which at each iteration a few local tensors (in most cases one or two) are variationally optimized.
For the numerical simulations employed in all of our articles, we implemented two optimization schemes, which are then combined to find an optimal MPS approximation for finite systems.
The first one is the growing algorithm (sometimes dubbed iDMRG), which is most useful to construct the trial wavefunction of the second scheme, namely traditional DMRG reformulated in the language of MPS~\cite{Schollwoeck2011,Silvi2019}.

Consider the way the energy expectation value is written in the MPS language: it is a sandwich of the product of $W$ tensors (the local MPO representing the Hamiltonian) with the trial MPS, and it's conjugate.
To find the \DIFdelbegin \DIFdel{groundstate}\DIFdelend \DIFaddbegin \DIFadd{ground state}\DIFaddend , we extremize the energy
\begin{align}
    E = \min_{\ket{\psi}}\braket{\psi |\hat H | \psi}
\end{align}
under the constraint that $\braket{\psi|\psi}=1$.
Using a Lagrange multiplier, the equation becomes
\begin{align}
    \braket{\psi|\hat H|\psi} - \lambda_\psi\brlr{\braket{\psi|\psi}-1} = 0.
    \label{eq:lagrangian_multiplier}
\end{align}
Now the MPS structure of $\ket\psi$ is exploited, and we decompose the state into the canonical form
\begin{align}
    \ket\psi
    =
    \brlr{\prod_{k=1}^{\ell}U^{[i_k]}}\Lambda(\ell)\brlr{\prod_{k=\ell+1}^{L}V^{\dag[i_k]}}\ket{\bm i}
    \equiv
    \includegraphics[valign=c]{figures/arbitrary_wavefunction_decomposed_decolorized.pdf}\ket{i_1,i_2,\dots,i_L}.
\end{align}
Please note that we use the sum convention in the above, and the tensors $U^{[i_k]}$ and $V^{\dag[i_k]}$ satisfy the isometry conditions
\begin{align}
    \sum_{i_k}U^{\dag[i_k]}U^{[i_k]} = \mathbb 1,
    \quad
    \sum_{i_k}V^{\dag[i_k]}V^{[i_k]} = \mathbb 1.
    \label{eq:isometry_tensors}
\end{align}
Diagrammatically, the two conditions exactly correspond to the two graphical contractions
\begin{align}
    \includegraphics[valign=c]{figures/left_isometry_tensor.pdf}
    =
    \includegraphics[valign=c]{figures/identity_tensor_left.pdf}\,,
    \quad
    \includegraphics[valign=c]{figures/right_isometry_tensor.pdf}
    =
    \includegraphics[valign=c]{figures/identity_tensor_right.pdf}\,.
    \label{eq:isometry_tensors_diagram}
\end{align}
To cast \cref{eq:lagrangian_multiplier} into a variational problem for the tensors at position $\ell$ and $\ell+1$, it is convenient to use the graphical notation
\begin{align}
    \includegraphics[valign=c]{figures/lagrangian_multiplier_equation_1.pdf}
    -
    \lambda_\psi
    \brlr{\includegraphics[valign=c]{figures/lagrangian_multiplier_equation_2.pdf}-1}
    =
    0.
    \label{eq:lagrangian_multiplier_graphical}
\end{align}
In the above, we use square tensors to denote the local MPOs of the Hamiltonian.
The aim is now to optimize all tensors in order to extremize \cref{eq:lagrangian_multiplier_graphical}.
To obtain something feasible, we now derive an equation which considers a local optimization of the two colored tensors.
This is achieved by the application of a gradient on both sides of \cref{eq:lagrangian_multiplier}, in which we derive with respect to the entries of the contraction of the red, black and blue tensor, i.e.
\begin{align}
    \nabla(\ell,\ell+1)_{kn}\equiv\frac\partial{\partial\brlr{U^{i_\ell}_{k,l}S_l V^{\dag i_{\ell+1}}_{l,n}}^*}.
\end{align}
Since the tensors enter in the equation in a linear fashion, the application of the gradient simply corresponds to removing the three tensors from the conjugate contraction (and of course the scalar $1$) in \cref{eq:lagrangian_multiplier_graphical}.
One arrives at the tensorial expression
\begin{align}
    \includegraphics[valign=c]{figures/lagrangian_multiplier_equation_3.pdf}
    =
    \lambda_\psi
    \includegraphics[valign=c]{figures/lagrangian_multiplier_equation_4.pdf}
    .
    \label{eq:lagrangian_multiplier_graphical_2}
\end{align}
This can be better understood after reshaping $U^{[i_\ell]}\Lambda(\ell) V^{\dag[i_{\ell+1}]}$ (the red, black and blue tensors in \cref{eq:lagrangian_multiplier_graphical_2}) to a vector ${\bm A}_C^{[i_\ell,i_{\ell+1}]}=f\circ U^{[i_\ell]}\Lambda V^{\dag[i_{\ell+1}]}$ of dimension $m^2D^2$, in which $f$ denotes the reshaping.
The contracted network of grey tensors should then be reshaped to a matrix $\bm H_{\rm eff.}^{[i_\ell',i_{\ell+1}',i_\ell,i_{\ell+1}]}$ of dimension $m^2D^2\times m^2D^2$.
The earlier applied gradient then simply corresponds to $\nabla_{\bm A_C}$ and the tensorial expression in \cref{eq:lagrangian_multiplier_graphical_2} reduces to
\begin{align}
    \bm H_{\rm eff.}^{[i_\ell',i_{\ell+1}',i_\ell,i_{\ell+1}]}{\bm A}_C^{[i_\ell,i_{\ell+1}]} = \lambda_\psi {\bm A}_C^{[i_\ell',i_{\ell+1}']}.
    \label{eq:MPS_optimization}
\end{align}
It is now obvious that (i) the Lagrange multiplier $\lambda_\psi$ corresponds to the energy of the MPS and (ii) the equations span a standard eigenvalue problem for the vector $\bm A_C$ which can be solved efficiently for the end of the spectrum with sparse eigenvalue solvers based on an iterative matrix-vector multiplication.
This is achieved by, e.g. the Arnoldi algorithm~\cite{Arnoldi1951}, implemented in open-source libraries like ARPACK~\cite{Lehoucq1998}.
After the optimal low-energy vector is computed, it must be reshaped back to a tensor, and then decomposed through the generalized SVD to obtain the original canonical form of the MPS.
In particular, one performs the sequence
\begin{align}
    f^{-1}\circ {\bm A}_C^{[i_\ell,i_{\ell+1}]}
    \equiv
    \includegraphics[valign=c]{figures/growing_4_4.pdf}
    =
    \includegraphics[valign=c]{figures/growing_5.pdf}
    \equiv
    U^{[i_\ell]} \Lambda V^{\dag[i_{\ell+1}]}.
\end{align}
Note that the standard eigenvalue problem in \cref{eq:MPS_optimization} is ``downgraded'' to a generalized eigenvalue problem if we would not rely on the normalized and canonical form of the MPS~\cite{Silvi2019}.
The aim now is to find the globally optimal vectors ${\bm A}_C$ for each site, which can be achieved in an iterative manner by the growing and DMRG algorithm.

A minimal example of the growing MPS algorithm is formalized as follows:
\begin{enumerate}
    \item Choose an even number $L$ corresponding to the final number of quantum numbers in the resulting quantum state.
    It is equal to twice the number of iterations.
    \item Set $n=1$. Compute the lowest energy vector ${\bm A}_C^{[i_1,i_2]}$ for a system of two sites only.
    The eigenvalue equation and graphical representation reads
    \DIFdelbegin %DIFDELCMD < \begin{align}
%DIFDELCMD <         \bm H_{\rm eff.}^{[i_1',i_{2}',i_1,i_2]} {\bm A}_C^{[i_1,i_2]} &= \lambda_\psi {\bm A}_C^{[i_1',i_2']}\\
%DIFDELCMD <         \includegraphics[valign=c]{figures/two_site_problem_1.pdf}
%DIFDELCMD <         &=
%DIFDELCMD <         \lambda_\psi
%DIFDELCMD <         \includegraphics[valign=c]{figures/two_site_problem_2.pdf}\ ,
%DIFDELCMD <     \end{align}%%%
\DIFdelend \DIFaddbegin \begin{align}
        \bm H_{\rm eff.}^{[i_1',i_{2}',i_1,i_2]} {\bm A}_C^{[i_1,i_2]} &= \lambda_\psi {\bm A}_C^{[i_1',i_2']},\\
        \includegraphics[valign=c]{figures/two_site_problem_1.pdf}
        &=
        \lambda_\psi
        \includegraphics[valign=c]{figures/two_site_problem_2.pdf}\ ,
    \end{align}\DIFaddend 
    in which the green tensor represents ${\bm A}_C$ and the two grey rank-$3$ tensors correspond to the boundary MPOs of the Hamiltonian.
\item Use the generalized singular value decomposition depicted in \cref{eq:SVD_generalized} to decouple the two-site tensor into a product of two isometries and a matrix $\Lambda$ containing the spectral values
    \begin{align}
        f^{-1}\circ {\bm A}_C^{[i_1',i_2']} &= U^{[i_1']}\Lambda V^{\dag[i_{L}']},
        \\
        \includegraphics[valign=c]{figures/two_site_problem_2.pdf}
        &=
        \includegraphics[valign=c]{figures/two_site_problem_2_decomp.pdf}\ .
    \end{align}
    Renormalize the truncated matrix $\Lambda$ such that $\tr(\Lambda^2)=1$.
    \item Grow the size of the MPS from $2n$ to $2n+2$ sites by replacing the central matrix $\Lambda$ with a rank-$4$ tensor of fitting dimensions
    \begin{align}
        \brlr{\prod_{k=1}^nU^{[i_k']}}\Lambda \brlr{\prod_{k=1}^{n}V^{\dag[i_{L+1-n}']}}
        &\longrightarrow
        \brlr{\prod_{k=1}^nU^{[i_k']}} T^{[i_{n+1},i_{L-n}]}\brlr{\prod_{k=1}^{n}V^{\dag[i_{L+1-n}']}},
        \\
        \includegraphics[valign=c]{figures/growing_4_1.pdf}
        &\longrightarrow
        \includegraphics[valign=c]{figures/growing_4_2.pdf}\ .
    \end{align}
    Then use \cref{eq:MPS_optimization} to variationally solve for the lowest energy vector ${\bm A}_C$ corresponding to the inserted (green) tensor
    \begin{align}
        {\bm H}_{\rm eff.}^{[i_{n+1}',i_{n+2}',i_{n+1},i_{n+2}]} {\bm A}_C^{[i_{n+1},i_{n+2}]} &= \lambda_\psi {\bm A}_C^{[i_{n+1}',i_{n+2}']}
        ,\\
        \includegraphics[valign=c]{figures/growing_4_3.pdf}
        &=
        \lambda_\psi
        \includegraphics[valign=c]{figures/growing_4_4.pdf}.
    \end{align}
    \item Use the generalized singular value decomposition to decouple the two quantum numbers into a product of two isometries and a matrix $\Lambda$ containing the spectral values
    \begin{align}
        f^{-1}\circ \bm A_C^{[i_{n+1}',i_{n+2}']} &= U^{[i_{n+1}']}\Lambda V^{\dag[i_{L-n}']},
        \\
        \includegraphics[valign=c]{figures/growing_4_4.pdf}
        &=
        \includegraphics[valign=c]{figures/growing_5.pdf}.
    \end{align}
    Apply the compression scheme by keeping at most the $m$ largest singular values and renormalize the truncated matrix $\Lambda$.
    Increase $n\rightarrow n+1$. Go to step 4 or stop if $n=L/2$.
\end{enumerate}
The common application of the growing algorithm is two-fold.
It can be used naively to approximate the central tensors of a fully translationally invariant chain by sending $L\gg2$ to very large values.
The convergence is reached when the overlap between the optimal vectors $v(n,n+1)$ and $v(n-1,n)$ of the previous iteration step approaches a user-specified threshold.
This approach however requires the state to be fully translationally invariant over a single site.
A workaround of this issue is to relax the insertion of two sites in the center of the MPS to arbitrary many.
However, the convergence to the best thermodynamic MPS state is never achieved.
This is due to the intrinsically finite structure of the MPS in the growing algorithm which breaks translational symmetry at every step, and another Ansatz is required.

One viable option is to approximate the (infinite) environments to the left and right of the central tensors with the dominant eigenstates of the transfer matrices.
The corresponding algorithm is then called infinite boundary conditioned MPS~\cite{Phien2012} or variational uniform MPS (VUMPS)~\cite{ZaunerStauber2018}.
In our works, we were mostly interested in the simulation of finite systems and we omit the description of such an implementation.

The second application concerns the MPS obtained after $L/2$ steps -- it can serve as a trial state for a finite system of length $L$, which is refined by the DMRG algorithm:
\begin{enumerate}
    \item Set $n=1$. Start with a trial MPS in the canonical form (e.g. acquired by the growing algorithm) and transform the MPS such that the $\Lambda$ matrix is between sites $1$ and $2$, i.e.
    \begin{align}
        \includegraphics[valign=c]{figures/fMPS_1.pdf}.
    \end{align}
    \item Compute the new lowest energy vector $\bm A_C$ according to \cref{eq:MPS_optimization}\footnote{At the \DIFdelbegin \DIFdel{system's }\DIFdelend boundaries \DIFaddbegin \DIFadd{of the system}\DIFaddend , ${\bm A}_C$ is effectively a rank-$3$ \DIFdelbegin \DIFdel{tensors }\DIFdelend \DIFaddbegin \DIFadd{tensor }\DIFaddend and one of the isometries will \DIFaddbegin \DIFadd{necessarily }\DIFaddend be \DIFdelbegin \DIFdel{of }\DIFdelend \DIFaddbegin \DIFadd{a }\DIFaddend rank-$2$ \DIFaddbegin \DIFadd{tensor under the decompositions schematized above}\DIFaddend .} and decompose it to two isometries and a matrix $\Lambda$ containing the singular values
    \DIFdelbegin %DIFDELCMD < \begin{align}
%DIFDELCMD <         f^{-1}\circ \bm A_C^{[i_{n},i_{n+1}]} &= U^{[i_n]}\Lambda(n) V^{\dag[i_{n+1}]}\\
%DIFDELCMD <         \includegraphics[valign=c]{figures/growing_4_4.pdf}
%DIFDELCMD <         &=
%DIFDELCMD <         \includegraphics[valign=c]{figures/growing_5.pdf}
%DIFDELCMD <         .
%DIFDELCMD <     \end{align}%%%
\DIFdelend \DIFaddbegin \begin{align}
        f^{-1}\circ \bm A_C^{[i_{n},i_{n+1}]} &= U^{[i_n]}\Lambda(n) V^{\dag[i_{n+1}]},\\
        \includegraphics[valign=c]{figures/growing_4_4.pdf}
        &=
        \includegraphics[valign=c]{figures/growing_5.pdf}
        .
    \end{align}\DIFaddend 
    Apply the compression scheme by keeping at most the $m$ largest singular values and renormalize the truncated matrix $\Lambda$.
    Proceed then with step $3$ (sweeping from left to the right) or step $4$ (sweeping from right to the left).
    \item If $n=L-1$ continue with step 4. Shift the matrix $\Lambda(n)$ to the next site by the decomposition
    \DIFdelbegin %DIFDELCMD < \begin{align}
%DIFDELCMD <         \Lambda(n) V^{\dag[i_{n+1}]} V^{\dag[i_{n+2}]} &= U^{[i_{n+1}]}\Lambda(n+1) \tilde V^\dag V^{\dag[i_{n+2}]}
%DIFDELCMD <         \\
%DIFDELCMD <         \includegraphics[valign=c]{figures/fMPS_3_1.pdf}
%DIFDELCMD <         &=
%DIFDELCMD <         \includegraphics[valign=c]{figures/fMPS_3_2.pdf}
%DIFDELCMD <         .
%DIFDELCMD <     \end{align}%%%
\DIFdelend \DIFaddbegin \begin{align}
        \Lambda(n) V^{\dag[i_{n+1}]} V^{\dag[i_{n+2}]} &= U^{[i_{n+1}]}\Lambda(n+1) \tilde V^\dag V^{\dag[i_{n+2}]},
        \\
        \includegraphics[valign=c]{figures/fMPS_3_1.pdf}
        &=
        \includegraphics[valign=c]{figures/fMPS_3_2.pdf}
        .
    \end{align}\DIFaddend 
    Absorb $\tilde V^\dag$ in $V^{\dag[i_{n+2}]}$, then increase $n\rightarrow n+1$ and go back to step 2.
    This concludes a single sweeping step from left to right.
    \item If $n=1$ continue with step 3. Shift the matrix $\Lambda(n)$ to the previous site by the decomposition
    \DIFdelbegin %DIFDELCMD < \begin{align}
%DIFDELCMD <         U^{[i_{n-1}]}U^{[i_n]}\Lambda(n) &= U^{[i_{n-1}]}\tilde U\Lambda(n-1) V^{\dag[i_n]}
%DIFDELCMD <         \\
%DIFDELCMD <         \includegraphics[valign=c]{figures/fMPS_4_1.pdf}
%DIFDELCMD <         &=
%DIFDELCMD <         \includegraphics[valign=c]{figures/fMPS_4_2.pdf}
%DIFDELCMD <         .
%DIFDELCMD <     \end{align}%%%
\DIFdelend \DIFaddbegin \begin{align}
        U^{[i_{n-1}]}U^{[i_n]}\Lambda(n) &= U^{[i_{n-1}]}\tilde U\Lambda(n-1) V^{\dag[i_n]},
        \\
        \includegraphics[valign=c]{figures/fMPS_4_1.pdf}
        &=
        \includegraphics[valign=c]{figures/fMPS_4_2.pdf}
        .
    \end{align}\DIFaddend 
    Absorb $\tilde U$ in $U^{[i_{n-1}]}$, then decrease $n\rightarrow n-1$ and go back to step 2.
    This concludes a single sweeping step from right to left.
\end{enumerate}
\begin{figure}[ht]
    \centering
    \includegraphics[height=6cm]{figures/dmrg_flow_chart.pdf}
    \caption{Flow chart of usual two-site DMRG. After a trial MPS is put in canonical form, the tensor optimization is performed in ``sweeps'': the optimization of all local tensors performed sequentially from left to right, followed by the optimization of all local tensors from right to left.}
    \label{fig:DMRG_flow_chart}
\end{figure}
The DMRG optimization is usually performed in so-called ``sweeps'', i.e. a full sweeping cycle from left to right is done for each site $n=1,2,\dots,L-1$, after which a full sweeping cycle from right to left $n=L-1,L-2,\dots,1$ is applied.
Iteratively optimizing the local tensors leads to a variational trajectory in the family of MPS with bond dimension $m$.

The error to an exact eigenstate of the Hamiltonian can be estimated in a straightforward manner by computing the variance of the approximation.
For cases in which the MPS is equivalent to an eigenstate of the full Hamiltonian, the variance will reduce to $0$. In general, however, the approximation is not a true eigenstate of the Hamiltonian and as such we arrive at
\begin{align}
    \varepsilon_H(m) = \sqrt{\braket{\psi(m)|\hat H^2|\psi(m)} - \braket{\psi(m)|\hat H|\psi(m)}^2}\geq 0.
\end{align}
The indication of the dependence on $m$ is reminiscent of the fact that $\varepsilon_H(m)$ typically depends on the bond dimension of the MPS.
It can then be checked if the variance converges to machine precision by performing a so-called bond dimension scaling analysis~\cite{Hubig2018}.
In case of a finite bond dimension analysis, $\varepsilon_H(m)$ serves as a measure of the approximation error.
Alternatively, the discarded weight of the singular values can serve as a measure of the error.
It is defined as
\begin{align}
    \varepsilon_\rho(m) = \sup_{\ell}\sum_{k=m+1}^\infty s_k^2(\ell)
\end{align}
in which $s_k(\ell)$ denotes the $k$'th singular value of the mixed-canonical representation of the MPS after the optimization and just before the application of the compression scheme (i.e., before the truncation and renormalization of the singular values).

The presented algorithm can actually be implemented in a more efficient way by fixing the internal bond dimension from the start, e.g. through the growing algorithm and using a single-site tensor $A_C$ as the variational center, similarly to the time-evolution method which we introduce in the next section.
Compared to two-site DMRG, the dimensionality of the eigenvalue problem $A_C$ is reduced by a factor $D$, leading to a speed-up of at least $\mathcal O(D)$ per iteration~\cite{White2005}.
Although the overall optimization is then computationally more favorable, this leads to drawback of the single-site DMRG algorithm: the inaccessibility of the larger portion of singular values and as such the lack of the discarded weight $\varepsilon_\rho(m)$\footnote{This drawback evolves to a major disadvantage when exploiting symmetries in the tensor network, since single-site DMRG does not vary the established symmetry sectors of the MPS.}.
An elegant solution for the resolution of this issue is the subspace expansion~\cite{Hubig2015}, featuring a flexible interpolation between single-site and two-site DMRG.
This concept of reducing the variational space can be taken even further by the optimization of the local link tensors only~\cite{Fernandez2020}.
\section{Time evolution}
\label{sec:time_evolution}
There are several well-established methods used to simulate the time evolution of an initial MPS~\cite{Paeckel2019} and here we focus on the time-dependent variational principle (TDVP), mostly because it is easily formalized in the established MPO and MPS terminology\footnote{The crucial difference to other time-evolution algorithms is that it utilizes a backward evolution of the singular values in order to project the tensor onto its tangent space.
This way it aligns the variational trajectory to the tangent space of the MPS, leading to a particularly robust and efficient implementation compared to other algorithms~\cite{Haegeman2011}.}.
In the limit of infinite imaginary time, TDVP is known to be identical to traditional DMRG, which leads to a unified framework for the time evolution and optimization of matrix product states~\cite{Haegeman2016}.

Before the algorithm is introduced, we need a couple of prerequisites.
First, the MPS will be written in the mixed canonical gauge with a tensor $A_C$ (matrix $C$) which is not an isometry, i.e.
\DIFdelbegin %DIFDELCMD < \begin{align}
%DIFDELCMD <     C^{[i_1,i_2,\dots,i_\ell,\dots,i_L]} &= \brlr{\prod_{k=1}^{\ell-1} U^{[i_k]}} A_C^{[i_\ell]} \brlr{\prod_{k=\ell+1}^{L} V^{\dag[i_k]}}
%DIFDELCMD <     \equiv
%DIFDELCMD <     &\includegraphics[valign=c]{figures/center_site_gauge.pdf},
%DIFDELCMD <     \\
%DIFDELCMD <     &=\brlr{\prod_{k=1}^{\ell} U^{[i_k]}} C(\ell) \brlr{\prod_{k=\ell+1}^{L} V^{\dag[i_k]}}
%DIFDELCMD <     \equiv
%DIFDELCMD <     &\includegraphics[valign=c]{figures/center_site_gauge_2.pdf}
%DIFDELCMD <     .
%DIFDELCMD < \end{align}%%%
\DIFdelend \DIFaddbegin \begin{align}
    C^{[i_1,i_2,\dots,i_\ell,\dots,i_L]} &= \brlr{\prod_{k=1}^{\ell-1} U^{[i_k]}} A_C^{[i_\ell]} \brlr{\prod_{k=\ell+1}^{L} V^{\dag[i_k]}}
    \equiv
    &\includegraphics[valign=c]{figures/center_site_gauge.pdf}\phantom{,}
    \\
    &=\brlr{\prod_{k=1}^{\ell} U^{[i_k]}} C(\ell) \brlr{\prod_{k=\ell+1}^{L} V^{\dag[i_k]}}
    \equiv
    &\includegraphics[valign=c]{figures/center_site_gauge_2.pdf}
    .
\end{align}\DIFaddend 
As usual, the tensors $U$ and $V^\dag$ are isometries which satisfy \cref{eq:isometry_tensors,eq:isometry_tensors_diagram}.
The projector $\hat P_{T\ket\psi}$ onto the tangent space of the MPS of $\ket\psi$ for a single site is written as~\cite{Haegeman2016}
\DIFdelbegin %DIFDELCMD < \begin{align}
%DIFDELCMD <     \hat P_{T\ket\psi} = \sum_{\ell=1}^L \hat P_{A(\ell-1)}\hat{\mathbb1}_\ell \hat P_{B(\ell+1)} - \sum_{\ell=1}^{L-1}\hat P_{A(\ell)} \hat P_{B(\ell+1)}
%DIFDELCMD < \end{align}%%%
\DIFdelend \DIFaddbegin \begin{align}
    \hat P_{T\ket\psi} = \sum_{\ell=1}^L \hat P_{A(\ell-1)}\hat{\mathbb1}_\ell \hat P_{B(\ell+1)} - \sum_{\ell=1}^{L-1}\hat P_{A(\ell)} \hat P_{B(\ell+1)},
\end{align}\DIFaddend 
in which $\hat P_{A/B(\ell)}$ are the corresponding projections onto the vectors of $\ket\psi$ forming the partition $A/B(\ell)$ of the first/last $\ell$ quantum numbers\footnote{Note that we conveniently use the convention $\hat P_{A(0)}=\hat P_{B(L+1)}=1$.}.
In the gauge chosen for $\ket\psi$, the action of the tangent space projector onto the MPS representation of $\ket \psi$ can be written as a sum of rank-$2L$ tensors of the form
\DIFdelbegin %DIFDELCMD < \begin{align}
%DIFDELCMD <     P_{T\ket\psi}
%DIFDELCMD <     \equiv
%DIFDELCMD <     \sum_{\ell=1}^{L}
%DIFDELCMD <     \includegraphics[valign=c]{figures/tangent_space_projection_1.pdf}
%DIFDELCMD <     -
%DIFDELCMD <     \sum_{\ell=1}^{L-1}
%DIFDELCMD <     \includegraphics[valign=c]{figures/tangent_space_projection_2.pdf}
%DIFDELCMD < \end{align}%%%
\DIFdelend \DIFaddbegin \begin{align}
    P_{T\ket\psi}
    \equiv
    \sum_{\ell=1}^{L}
    \includegraphics[valign=c]{figures/tangent_space_projection_1.pdf}
    -
    \sum_{\ell=1}^{L-1}
    \includegraphics[valign=c]{figures/tangent_space_projection_2.pdf},
\end{align}\DIFaddend 
where the tensors at site $\ell$ are highlighted in green.
Note that the two terms correspond to a projection onto the reduced density matrices of the isometric bipartition spanned by the central tensor and matrix, respectively.
The first term corresponds to the projection onto all MPS which differ at most by a local tensor from $\ket\psi$ and the second removes all of those which are identical to $\ket\psi$.
Therefore, it is trivial to see that $\braket{\psi|\hat P_{T\ket\psi}|\psi}=0$.
Time evolution can now be understood as an orthogonal projection of the evolved state onto the tangent space of the corresponding MPS manifold~\cite{Haegeman2011}
\begin{align}
    \frac{\rd\ket\psi}{\rd t} = -\ri \hat P_{T\ket\psi}\hat H\ket\psi.
    \label{eq:MPS_time_evolution}
\end{align}
By writing the action of $\hat P_{T\ket\psi}$ onto $\hat H\ket\psi$ explicitly, it is clear that the right-hand-side of \cref{eq:MPS_time_evolution} reduces to a sum of single-site contributions.
The authors of~\cite{Haegeman2016} demonstrated that the full integration of the problem can be approximated with a decoupled set of differential equations for the individual single-site contributions, which can be integrated exactly.
The derivation relies on the assumption that the state $\ket\psi$ is encoded by an MPS in which the time-dependence is captured in $A_C^{[i_\ell]}$ (or $C(\ell)$).
One can \DIFdelbegin \DIFdel{thus }\DIFdelend identify a set of $2L-1$ independent differential equations
\begin{align}
    \frac{\rd\ket\psi}{\rd t} = -\ri \hat P_{A(\ell-1)}\hat {\mathbb1}_\ell \hat P_{B(\ell+1)}\hat H\ket\psi
    ,\quad
    \frac{\rd\ket\psi}{\rd t} = -\ri \hat P_{A(\ell)} \hat P_{B(\ell+1)}\hat H\ket\psi
    .
\end{align}
A slight modification then yields the differential equations for the local tensors
\begin{align}
    {\dot A}_C^{[i_\ell']} = +\ri H_{\rm eff.}^{[i_\ell',i_\ell]} A_C^{[i_\ell]},
    \quad
    {\dot C}(\ell) = -\ri K_{\rm eff.}(\ell) C(\ell).
    \label{eq:MPS_time_evolution_center}
\end{align}
The tensors $H_{\rm eff.}$ and $K_{\rm eff.}$ are given through the contractions
\begin{align}
    H_{\rm eff.}^{[i_\ell',i_\ell]}
    &\equiv
    \includegraphics[valign=c]{figures/Heff_1.pdf}\
    ,
    \\
    K_{\rm eff.}(\ell)
    &\equiv
    \includegraphics[valign=c]{figures/Keff_1.pdf}\
    ,
\end{align}
for which the site $\ell$ is again highlighted in green.
The integration is then performed explicitly for the vector/matrix expressions of \cref{eq:MPS_time_evolution_center} (denoted by bold symbols), and one arrives at
\begin{align}
    {\bm A}_C^{[i_\ell']}(t) &= \re^{-\ri t{\bm H}_{\rm eff.}^{i_\ell',i_\ell}} {\bm A}_C^{[i_\ell]}(t=0),
    \label{eq:TDVP_integration_AC}
    \\
    {{\bm C}}(\ell,t) &= \re^{+\ri t{\bm K}_{\rm eff.}(\ell)} {\bm C}(\ell,t=0).
    \label{eq:TDVP_integration_C}
\end{align}
Note that for each local integration of $A_C$, the corresponding matrix $C$ is evolved backwards in time.
Exponentials of matrices are readily computed with the Lanczos method, which is part of the ARPACK package~\cite{Lehoucq1998}.

The full TDVP algorithm is then depicted as follows:
\begin{enumerate}
    \item Set $n=1$ and choose a small integration timestep $\Delta t/2$. Start with an MPS in the right-canonical form, i.e.
    \begin{align}
        C^{[i_1,i_2,i_3,\dots,i_{L-1},i_L]}= A_C^{[i_1]}\brlr{\prod_{k=2}^LV^{\dag[i_k]}}\equiv\includegraphics[valign=c]{figures/right_canonical_gauge.pdf}.
    \end{align}
    \item Evolve $A_C^{[i_n]}$ according to \cref{eq:TDVP_integration_AC} and factorize it to $A_C^{[i_n]} = U^{[i_n]}C(n)$.
    Back-evolve $C(n)$ according to \cref{eq:TDVP_integration_C}, then continue with step 3. (sweeping from left to right) or 4. (sweeping from right to left).
    \item If $n=L$, continue with 4. Absorb the matrix $C(n)$ into the next tensor to create $A_C^{i_{n+1}}=C(n)V^{\dag[i_{n+1}]}$. Increase $n\rightarrow n+1$ and continue with 1.
    \item If $n=1$, continue with 3. Absorb the matrix $C(n)$ into the previous tensor to create $A_C^{i_{n-1}}=U^{[i_{n-1}]}C(n)$. Decrease $n\rightarrow n-1$ and continue with 1.
\end{enumerate}
Similarly to DMRG, TDVP is performed in completed sequences of full sweeping cycles $n=1,2,\dots,L$ and then $n=L,L-1,\dots,1$.
After one sweeping from left to right cycle ($n=1,2,\dots,L$), the MPS is integrated from time $t$ to $t+\Delta t/2$ with a local integration error of $\mathcal O(\Delta t^2)$.
Completing the left to right sequence ($n=L,L-1,\dots,1$) then constitutes to composing the integration with its adjoint, resulting in a second-order symmetric method.
A single integration step from $t\rightarrow t+\Delta t$ is \DIFdelbegin \DIFdel{thus }\DIFdelend achieved after a full sweeping cycle, and the error of the integration is more favorable $\mathcal O(\Delta t^3)$~\cite{Haegeman2016}.
Similarly to two-site DMRG, the TDVP algorithm can be easily implemented by time-evolving a two-site instead of a single-site tensor $A_C^{[i_\ell]}\rightarrow A_C^{[i_\ell,i_{\ell+1}]}$ and the center matrix is promoted to a single-site tensor $C^{[i_\ell]}$~\cite{Haegeman2016}.
The flow chart of two-site TDVP is equivalent to \cref{fig:DMRG_flow_chart}.
\section{Finite temperature}
\label{sec:finite_temperature}
By definition, MPS can only represent pure quantum states which have the generic form of~\cref{eq:generic_state}.
In order to represent some mixed state, it is therefore necessary to implement an additional ingredient like matrix product density operators (MPDO) minimally entangled typical thermal states (METTS) or matrix product purification schemes~\cite{Binder2015}.
The purification scheme is the most straightforward method, as it implements auxiliary degrees of freedom which act as thermofield doublet states~\cite{Barnett1987}.
In particular, the Hilbert space will be doubled $\HS = \HS_p\otimes\HS_a$, in which $\HS_p$ and $\HS_a$ denote the physical and auxiliary space.
The density matrix of the thermal system is then a trace of the pure density matrix with respect to the auxiliary states, i.e.
\begin{align}
    \hat\rho = \tr_a\ket\psi\bra\psi.
\end{align}
In infinite temperature systems, the density matrix is the identity and such states can be constructed as a product state of maximally entangled states between physical and auxiliary subspace.
For instance, \DIFdelbegin \DIFdel{in }\DIFdelend \DIFaddbegin \DIFadd{for }\DIFaddend the Hubbard model \DIFdelbegin \DIFdel{at filling }\DIFdelend \DIFaddbegin \DIFadd{with }\DIFaddend $N$ \DIFaddbegin \DIFadd{particles}\DIFaddend , a canonical infinite-temperature state is $\ket{\psi_0}=\hat C^{\dag N}\ket0$~\cite{Barthel2016} with
\begin{align}
    \hat C^\dag = \sum_{j=1}^L c^\dag_{j,\uparrow,p}c^\dag_{j,\uparrow,a} + c^\dag_{j,\downarrow,p}c^\dag_{j,\downarrow,a}.
\end{align}
To obtain a finite-temperature state at $\beta\neq0$, the initial state can then be time-evolved along the imaginary axis over a range $\beta/2$~\cite{Barnett1987}, i.e.
\begin{align}
    \hat\rho(\beta) \propto \tr_a\brlr{\re^{-\frac\beta2\hat H}\ket{\psi_0}\bra{\psi_0}\re^{\frac\beta2\hat H}}.
\end{align}
The constant is readily fixed by normalizing the purified state after tracing the auxiliary degrees of freedom.
Note that the trace of the auxiliary states leads to an additional gauge degree of freedom.
In particular, the application of any unitary gate onto the ancilla subspace will leave the physical state untouched.
This opens up the possibility to explore ``minimal entanglement'' gauges, resulting in particularly efficient tensor network schemes~\cite{Barthel2013,Hauschild2018,Wolff2020}.
\section{Exploiting symmetries}
\label{sec:exploiting_symmetries}
The existence of symmetries implies a corresponding conserved quantity according to the Noether theorem from 1918~\cite{Noether1971}.
This in general allows to decouple a full problem into simpler sub-problems, such as solving a translationally invariant quadratic Hamiltonian for each value of the conserved quantity (crystal momentum) in Fourier space.

In computational physics, this opens up the possibility to implement the preservation of conservation laws, which typically leads to a significant overall speed-up and better convergence.
Take for instance the preservation of the number of particles $N$ in closed quantum systems.
Implementing the preservation of $N$ in MPS allows to immediately select a target Fock space $\FS^N$ without the need of scanning a chemical potential in the full many-body Hilbert space.

In quantum mechanis, symmetry elements $g$ of a group $G$ are typically unitary and stationary representations, i.e. they satisfy the commutation
\begin{align}
    \commutator{U(g),\hat H} = 0\ \forall g\in G.
\end{align}
As a consequence there exists an eigenbasis of $\hat H$ such that the corresponding energy eigenstates belong to the subspace of a single irreducible representation (irrep)\footnote{An irrep of a group is a group representation which maps only trivial subspaces of the full vector space into itself and, in this sense, is indecomposable into ``smaller'' pieces.}.
One such example would be a quantum state of the form $\ket{\psi_{\alpha,\beta,\gamma}}$ which satisfy $\hat H\ket{\psi_{\alpha,\beta,\gamma}} = E_{\alpha,\gamma}\ket{\psi_{\alpha,\beta,\gamma}}$.
The index $\alpha$ labels the primary quantum number (or sector, or charge), and therefore the subspace spanned by a single irreducible representation.
The second index $\beta$ labels the contributing states to this subspace (secondary quantum number), and $E$ does not depend on that as a consequence of Schur's lemma.
The last index $\gamma$ (symmetry degeneracy label) represents all remaining information not captured by the primary quantum number $\alpha$~\cite{Silvi2019}.

A representation of a symmetry group element acts \DIFdelbegin \DIFdel{onto }\DIFdelend \DIFaddbegin \DIFadd{on }\DIFaddend the quantum states as
\DIFdelbegin %DIFDELCMD < \begin{align}
%DIFDELCMD <     U(g)\ket{\psi_{\alpha,\beta,\gamma}} = \sum_{\beta'}W^{\alpha}_{\beta,\beta'}(g)\ket{\psi_{\alpha,\beta',\gamma}}
%DIFDELCMD < \end{align}%%%
\DIFdelend \DIFaddbegin \begin{align}
    U(g)\ket{\psi_{\alpha,\beta,\gamma}} = \sum_{\beta'}W^{\alpha}_{\beta,\beta'}(g)\ket{\psi_{\alpha,\beta',\gamma}},
\end{align}\DIFaddend 
in which $W^\alpha(g)$ is the $\alpha$-irrep matrix of the group element $g$.
The representation of symmetries are called Abelian when $\commutator{U(g),U(g')}=0$ and non-Abelian when ($\commutator{U(g),U(g')}\neq0$).
Although an implementation of the latter is certainly possible, the former are computationally way easier to handle.
For example, the simple fusion rules in typical Abelian symmetries like particle number conservation ($\alpha\oplus\alpha'=\alpha+\alpha'$) generalize to Clebsch-Gordan coefficients in the case of $SU(2)$ leading to a significant development overhead~\cite{Schmoll2020}.
Abelian symmetries \DIFdelbegin \DIFdel{in general }\DIFdelend have one-dimensional irreps \DIFdelbegin \DIFdel{, and as such are of the form
}%DIFDELCMD < \begin{align}
%DIFDELCMD <     W^\alpha(g) = \re^{\ri\varphi_\alpha(g)}
%DIFDELCMD < \end{align}%%%
\DIFdelend \DIFaddbegin \DIFadd{in general, i.e.
}\begin{align}
    W^\alpha(g) = \re^{\ri\varphi_\alpha(g)},
\end{align}\DIFaddend 
in which the phase $\varphi$ depends on the primary quantum number and the group element itself\DIFdelbegin \DIFdel{, and }\DIFdelend \DIFaddbegin \DIFadd{.
In this case, }\DIFaddend a secondary quantum number does not emerge~\cite{Silvi2019}.
The symmetric Hamiltonian is thus block diagonal in the $\alpha$ sectors without further constraints.
For instance, the continuous planar rotation group, denoted by $U(1)$, is the symmetry group related to the conservation laws of integer quantities like the total number of particles.
In this case, the irrep label $\alpha\in\mathds Z$, the element $g\in[0,2\pi)$, compositions are additive $g\circ g' = {(g+g')\mod 2\pi}$, and $\varphi_\alpha(g) = g\alpha$.
For later use, we also define the inverse irrep $\alpha^\dag=-\alpha$.
Other examples of typical symmetry groups and their corresponding description listed in~\cite{Silvi2019}.

From the different classes of symmetries, pointwise symmetries are particularly efficient in the framework of tensor networks.
As the name suggests, the global representation of a pointwise symmetry satisfies $U(g)=\bigotimes_i V_i(g)$ and is composed of local representations $V_i(g)$ of the group element $g$ acting on site $i$.
There are two main families of symmetries which are pointwise, (i) global pointwise and (ii) lattice gauge symmetries.
The former (i) are described by local elements $V_i(g)$ which are nontrivial for every site, and they are usually uniform such that $V_i(g)=V(g)$.
The latter (ii) have nontrivial support for a restricted number of sites and are a powerful tool for the simulation of lattice gauge models~\cite{Banuls2014,Buyens2015}.
We focus here on the first one only.

The locality of pointwise symmetries can be exploited by implementing a tensor network which remains invariant under the action of the symmetry.
It is then straightforward to conclude that any MPS build from the ``symmetric tensors'' is also invariant under the action of the symmetry~\cite{Singh2010}.
But the reverse is also true, as a generic symmetry invariant state decomposes into the symmetric tensor network formulation~\cite{Singh2013}.
This framework can be established by two elementary building blocks, namely symmetric links (i) and symmetric tensors (ii).
To implement (i), the legs of the tensor are promoted to symmetric links by the association of the irrep label $\alpha$ to each leg index value $i$.
Index values with the same $\alpha$ are counted by the degeneracy number $\gamma$.
It is \DIFdelbegin \DIFdel{thus }\DIFdelend possible to generate a bijection between the leg index value $i$ and symmetric link tuple $(\alpha,\gamma)$.
To simplify notation, we will drop the indication of the bijection, and replace the occurrence of the leg index value with the symmetric link tuple, i.e. $i\equiv(\alpha,\gamma)$.
This way, symmetric links associate a specific unitary representation of the group $G$, and each tuple $(\alpha,\gamma)$ labels a specific irrep.

A tensor $T$ is called Abelian symmetric if it is invariant under the simultaneous application of $W_i(g)$ associated to the symmetric links $r$, i.e.
\begin{align}
    T^{[(\alpha_1,\gamma_1),\dots,(\alpha_n,\gamma_n)]} = T^{[(\alpha_1,\gamma_1),\dots,(\alpha_n,\gamma_n)]}\prod_{r=1}^n{\re^{\ri \varphi_{\alpha_r}(g)}}.
    \label{eq:symmetric_tensor}
\end{align}
In conclusion, the tensor $T$ is symmetric if nonzero elements exist where the string of charges satisfies $\prod_r {e^{\ri\varphi_{\alpha_r}(g)}} = 1$.
This condition can be brought to the more convenient computational rule $\sum_r\alpha_r = 0$, which suggests the use of a structural tensor~\cite{Silvi2019}
\begin{align}
    S^{[\alpha_1,\dots,\alpha_n]} = \delta_{0,\sum_r\alpha_r}.
\end{align}
In particular, the full symmetric tensor can then be decomposed to a product of the structural tensor and a degeneracy tensor
\begin{align}
    T^{[(\alpha_1,\gamma_1),\dots,(\alpha_n,\gamma_n)]} = S^{[\alpha_1,\dots,\alpha_n]} R^{[\gamma_1,\dots,\gamma_n]}.
\end{align}
The number of elements in $T$ is thus given by the product of elements in $R$ and $S$, which is typically much lower than the number of elements in the non-symmetric version~\cite{Silvi2019}.

It is, as usual, quite convenient to convey the equations to the graphical notation.
To satisfy \cref{eq:symmetric_tensor}, the amount of negative charges must compensate the positive charges.
For example, a rank-$3$ symmetric tensor would be of the form
\DIFdelbegin %DIFDELCMD < \begin{align}
%DIFDELCMD <     T^{[i_1,i_2,i_3]} \equiv T^{[(\alpha_1,\gamma_1),(\alpha_2,\gamma_2),(\alpha_3,\gamma_3)]}
%DIFDELCMD < \end{align}%%%
\DIFdelend \DIFaddbegin \begin{align}
    T^{[i_1,i_2,i_3]} \equiv T^{[(\alpha_1,\gamma_1),(\alpha_2,\gamma_2),(\alpha_3,\gamma_3)]}.
\end{align}\DIFaddend 
Let us, for convenience, consider only three different quantum numbers $\alpha_1\geq0$, $\alpha_2\geq0$ and $\alpha_3\leq0$, for which the tensor decomposes to
\begin{align}
    T^{[(\alpha_1,\gamma_1),(\alpha_2,\gamma_2),(\alpha_3,\gamma_3)]} = \delta_{\alpha_1+\alpha_2,-\alpha_3} R^{[\gamma_1,\gamma_2,\gamma_3]}.
\end{align}
It is \DIFdelbegin \DIFdel{thus }\DIFdelend a necessity that $\alpha_3$ compensates the sum of charges $\alpha_1$ and $\alpha_2$.
This can be depicted through a set of ``incoming'' and ``outgoing'' arrows with respect to $T$ representing the structural tensor $S$, i.e.
\begin{align}
    T^{[(\alpha_1,\gamma_1),(\alpha_2,\gamma_2),(\alpha_3,\gamma_3)]} \equiv \includegraphics[valign=c]{figures/symmetric_tensor_example.pdf}
    =
    \includegraphics[valign=c]{figures/symmetric_tensor_example_2.pdf}
    \times
    \includegraphics[valign=c]{figures/symmetric_tensor_example_3.pdf}.
\end{align}
All of the algorithms introduced in this chapter are compatible with the use of symmetric tensors, and can be promoted to symmetric algorithms in a straightforward manner.
For a careful implementation guide, we refer to the detailed steps in~\cite{Silvi2019}.
\clearpage{}
\clearpage{}\chapter{Topological phases of matter}
\label{ch:topological_phases_of_matter}
Interestingly, the meaning of the term ``topology'' changes with the research focus of the scientist.
For instance, chemists quite often understand topology as the geometric configuration of a molecule and biologists talk about the topology of knotted proteins.
A condensed matter physicist typically classifies certain structures, e.g. insulating materials or magnetic whirls, with the concepts of topological invariants.
This approach is closely related to one of the most fundamental invariants \DIFdelbegin \DIFdel{in the mathematical branch of topology}\DIFdelend \DIFaddbegin \DIFadd{for manifolds}\DIFaddend : the genus\DIFdelbegin \DIFdel{of a manifold }\DIFdelend \DIFaddbegin \DIFadd{, }\DIFaddend which can be thought of as the ``number of holes''.
This number is known not to change under smooth deformations.
Consider the following three ``manifolds'': a donut, a coffee mug and a pretzel.
It may be delicate to do, but one could simply stretch and massage the donut to obtain a mug without ``cutting'' and ``glueing'', which is a necessity to go from donut to pretzel.
The ``stretching'' can be understood as smooth deformation, and as a consequence the two shapes have the same genus.
However ``cutting'' and ``glueing'' is something very sudden and irreversible, which changes the genus of the manifold.
The idea of \DIFdelbegin \DIFdel{explaining phenomena in quantum states of matter such as the quantum Hall effect or superfluid phase transitions using the mathematical concepts of topology }\DIFdelend \DIFaddbegin \DIFadd{characterizing phases of matter through topological invariants }\DIFaddend was established, among others, by David J. Thouless, F. Duncan M. Haldane and J. Michael Kosterlitz and resulted in the 2016 Nobel prize~\cite{NP2016}.

Exploring, engineering and probing topological quantum matter \DIFdelbegin \DIFdel{, in my perspective, is still }\DIFdelend \DIFaddbegin \DIFadd{is }\DIFaddend one of the most \DIFdelbegin \DIFdel{exciting research areas }\DIFdelend \DIFaddbegin \DIFadd{striving research areas of our present time}\DIFaddend .
It is not only a delicate challenge due to the interdisciplinary methods that have been developed in the past decades, it is also highly relevant for its technical applications.
For instance, a so-called topological quantum computer is \DIFdelbegin \DIFdel{forseen }\DIFdelend \DIFaddbegin \DIFadd{foreseen }\DIFaddend to exploit the rich exchange rules of exotic quasi-particles called anyons, which are located at the defects of topological quantum matter~\cite{Freedman2002}.
The controlled particle exchange, called ``braiding'', can be understood as logical gates that make up the processing unit of a hypothetical computer.
One of the major advantages of using these quasi-particles compared to other approaches is their robustness against perturbations in the bulk of the material.
The foundation for the existence of such a class of quasi-particles is commonly accepted to be topological order: a certain long-range ordered pattern of quantum entanglement.
\DIFdelbegin \DIFdel{One example of a topologically ordered state would be the Moore-Read state~\mbox{%DIFAUXCMD
\cite{Moore1991,Read1996} }\hspace{0pt}%DIFAUXCMD
(to be precise, the ``Antipfaffian'' state defined in~\mbox{%DIFAUXCMD
\cite{Levin2007,Lee2007}}\hspace{0pt}%DIFAUXCMD
) which has a large wave-function overlap with the fractionalquantum Hall state at filling factor $\nu=5/2$~\mbox{%DIFAUXCMD
\cite{Storni2010,Rezayi2017}}\hspace{0pt}%DIFAUXCMD
}\footnote{\DIFdel{The statement only serves as a streamline here, as the analytic form of the experimentally observed $\nu=5/2$ quantum Hall state is actually still an open debate~\mbox{%DIFAUXCMD
\cite{Simon2020}}\hspace{0pt}%DIFAUXCMD
.}}%DIFAUXCMD
\addtocounter{footnote}{-1}%DIFAUXCMD
\DIFdelend \DIFaddbegin \DIFadd{Such states are typically encountered in the context of the (fractional) Quantum Hall effect~\mbox{%DIFAUXCMD
\cite{Moore1991,Read1996,Levin2007,Lee2007,Storni2010,Rezayi2017}}\hspace{0pt}%DIFAUXCMD
}\DIFaddend .
Since fractional quantum Hall states are strongly correlated quantum many body systems, analytic attempts are particularly involved and quantitative statements like those in~\cite{Storni2010} are based on exact diagonalization of small systems or numerical simulations in general.

The ``trivial'' cousins of topologically ordered phases of matter are symmetry protected topological phases of matter which do not \DIFdelbegin \DIFdel{show }\DIFdelend \DIFaddbegin \DIFadd{possess }\DIFaddend long-range entanglement.
\DIFaddbegin \DIFadd{For this reason, MPS simulations provide efficient estimates for the low energy subspace of such systems.
}\DIFaddend The fascination of this topic is -- similar to true topologically ordered phases -- based on ground state degeneracy and exponential localization of the interface excitations.
One famous example of such a system is a symmetry-protected topological insulator, which, like an ordinary insulator, has a bulk energy gap separating the valence from the conduction band.
Unlike ordinary insulators, it hosts gapless surface states which are protected by a set of symmetry constraints.
The nature of topological insulators can be understood by simple non-interacting models, which makes the topic analytically \DIFdelbegin \DIFdel{amendable }\DIFdelend \DIFaddbegin \DIFadd{amenable }\DIFaddend using standard band theory.
This lead to the full classification of symmetry protected topological phases, famously known as the periodic table of topological insulators and superconductors~\cite{Altland1997,Kitaev2009}.

Laboratory experiments based on theoretical predictions remained an exception, until the recent strive of synthetic quantum matter:
engineered systems in which the interactions between constituents are tunable to obtain effectively single-particle and even strongly correlated states of matter.
One of the most prominent examples are ultracold atoms trapped in optical lattices, which provide all kinds of fundamental ``ingredients'' theoreticians like to exploit: magnetic fields~\cite{Lin2009}, spin-orbit coupling~\cite{Lin2011}, mixed particle statistics~\cite{Ferrari2002}, tunable interactions~\cite{Chin2010}, tailored disorder~\cite{Meier2018} and even the realization of $4$ independent \DIFdelbegin \DIFdel{spacial }\DIFdelend \DIFaddbegin \DIFadd{spatial }\DIFaddend coordinates exploiting the concept of synthetic dimensions~\cite{Lohse2018}, just to name a few.
\section{Topological band theory}
\label{sec:topological_band_theory}
To make the \DIFdelbegin \DIFdel{food example }\DIFdelend \DIFaddbegin \DIFadd{appetizer }\DIFaddend in the introduction more rigorous, the Gauss-Bonnet theorem is a beautiful statement on the integral of the Gaussian curvature $K$ over a surface $S$ which defines the Euler characteristic~\cite{Nakahara1990}\DIFdelbegin %DIFDELCMD < \begin{align}
%DIFDELCMD <     \chi = \frac1{2\pi}\int_S K\rd S.
%DIFDELCMD <     \label{eq:gauss_bonnet_theorem}
%DIFDELCMD < \end{align}%%%
\DIFdelend \DIFaddbegin \DIFadd{.
If we focus on the special case of closed surfaces only, the theorem states that
}\begin{align}
    \chi = \frac1{2\pi}\int_S K\rd S\in\mathds Z.
    \label{eq:gauss_bonnet_theorem}
\end{align}\DIFaddend 
If the surface is implicitly described by the regular kernel of a function \DIFdelbegin \DIFdel{$f(x,y,z)$}\DIFdelend \DIFaddbegin \DIFadd{$f(x_1,x_2,x_3)$}\DIFaddend , $K$ is defined through the Hesse matrix ${H_f}_{i,j}=\frac{\partial^2 f}{\partial_{x_i}\partial_{x_j}}$ and gradient \DIFdelbegin \DIFdel{$\bm\nabla f = (\partial_{x}f,\partial_{y}f,\partial_{z}f)^T$~\mbox{%DIFAUXCMD
\cite{Goldman2005}
}\hspace{0pt}%DIFAUXCMD
}%DIFDELCMD < \begin{align}
%DIFDELCMD <     K = -\frac{
%DIFDELCMD <     \det
%DIFDELCMD <     \begin{pmatrix}
%DIFDELCMD <         H_f & \bm\nabla f \\
%DIFDELCMD <         \bm\nabla f^T & 0
%DIFDELCMD <     \end{pmatrix}
%DIFDELCMD <     }{|\bm\nabla f|^4}
%DIFDELCMD <     .
%DIFDELCMD < \end{align}%%%
\DIFdelend \DIFaddbegin \DIFadd{$\bm\nabla f = (\frac{\partial f}{\partial_{x_1}},\frac{\partial f}{\partial_{x_2}},\frac{\partial f}{\partial_{x_3}})$~\mbox{%DIFAUXCMD
\cite{Goldman2005}
}\hspace{0pt}%DIFAUXCMD
}\begin{align}
    K = -\frac{
    \det
    \begin{pmatrix}
        H_f & \bm\nabla f^T \\
        \bm\nabla f & 0
    \end{pmatrix}
    }{|\bm\nabla f|^4}
    .
\end{align}\DIFaddend 
For instance, a three-dimensional sphere of radius $r$ can be described implicitly by $f=\sum_{i=1}^3 x_i^2 - r^2$ and has \DIFdelbegin \DIFdel{as such }\DIFdelend a uniform Gaussian curvature $K=1/r^2$ with Euler characteristic
\begin{align}
    \chi_{S^2} = \frac1{2\pi}\int_0^{2\pi}\rd\phi \int_0^\pi\rd\theta K r^2\sin\theta = 2.
\end{align}
\DIFdelbegin \DIFdel{In general}\DIFdelend \DIFaddbegin \DIFadd{For closed surfaces}\DIFaddend , the Gauss-Bonnet theorem states that the Euler characteristic \DIFdelbegin \DIFdel{is quantized}\DIFdelend \DIFaddbegin \DIFadd{$\chi$ is quantized to integer numbers}\DIFaddend , independent of smooth deformations of the surface $S$, and relates to the genus $g$ (``number of holes'') by $\chi=2-2g$~\cite{Nakahara1990}.
The topological invariants encountered in the next sections are similar, although they characterize more abstract objects.

The classification of topological insulators is possible within the band theory of solids: neglecting interactions and exploiting translational symmetry of crystals defines a periodicity in crystal-momentum space, which we already illustrated in \cref{ch:the_quantization_of_motion_and_fields}.
Each crystal momentum ${\bm k}$ is a good quantum number and linked to the Bloch states defined in a single unit-cell consisting of $s$ individual constituents.
Each Bloch state is an eigenstate of the Bloch Hamiltonian $\hat H(\bm k)$ which can be represented as a $\bm k$-dependent $s\times s$ matrix $H(\bm k)$.
Its ordered spectrum $\varepsilon_s(\bm k)$ define what is known as the band structure.
For insulators, a finite energy gap $\Delta$ separates the occupied valence band states from the empty conduction band states.
Lattice translation symmetry implies the identity $H(\bm k + \bm G) = H(\bm k)$ for any reciprocal lattice vector $\bm G$, which defines the crystal momentum in the periodic Brillouin zone \DIFdelbegin \DIFdel{with }\DIFdelend $\bm k\equiv \bm k+\bm G$ \DIFdelbegin \DIFdel{, which has }\DIFdelend \DIFaddbegin \DIFadd{and therefore }\DIFaddend the topology of a torus.
An insulating band structure can thus be viewed as a mapping from the torus to the space of Bloch Hamiltonians with a finite energy gap~\cite{Kane2013}.

Two insulating systems are called \DIFdelbegin \DIFdel{adiabatically equivalent (or adiabaticallyconnected) , }\DIFdelend \DIFaddbegin \DIFadd{(adiabatically) equivalent }\DIFaddend if their corresponding Hamiltonians \DIFdelbegin \DIFdel{can be }\DIFdelend \DIFaddbegin \DIFadd{are }\DIFaddend adiabatically connected without closing the energy gap.
Consider the special case of an interface where the crystal smoothly interpolates between two distinct insulating phases.
Somewhere along the path the gap necessarily vanishes to not violate the assumption, which implies the presence of zero-energy modes \DIFdelbegin \DIFdel{spacially }\DIFdelend \DIFaddbegin \DIFadd{spatially }\DIFaddend localized around the degeneracy.
This interplay between bulk and surface is a ubiquitous phenomenon called the bulk-boundary correspondence.
The key question we seek to answer in the remaining chapter is: can we find and characterize all equivalence classes of Hamiltonians which are generated from these statements?

One of the key concepts in topological band theory is the Berry phase, for which the construction arises from the intrinsic phase ambiguity of quantum states.
Let \DIFdelbegin \DIFdel{me quickly recap }\DIFdelend \DIFaddbegin \DIFadd{us review }\DIFaddend the main results of the original work, presented in~\cite{Berry1984}.
Imagine a particle evolving along a closed \DIFdelbegin \DIFdel{path ${\bm r}(t)$ }\DIFdelend \DIFaddbegin \DIFadd{parameter path ${\bm \lambda}(0)={\bm \lambda}(T)$ }\DIFaddend in an adiabatic way \DIFaddbegin \DIFadd{($T$ large enough), }\DIFaddend such that the evolution of the state \DIFdelbegin \DIFdel{follows Schrödinger's equation}\DIFdelend \DIFaddbegin \DIFadd{satisfies at all times}\DIFaddend \footnote{Note that \DIFdelbegin \DIFdel{, in this spirit, $\bm r(t)$ }\DIFdelend \DIFaddbegin \DIFadd{$\bm \lambda$ }\DIFaddend enters in the wavefunction as a parameter and thus changes independently of \cref{eq:time_evolution}.}
\DIFdelbegin %DIFDELCMD < \begin{align}
%DIFDELCMD <     \hat H(\bm r)\ket{\psi(\bm r(t))} = \ri\hbar\frac{\partial}{\partial t}\ket{\psi({\bm r(t)})},
%DIFDELCMD <     \label{eq:time_evolution}
%DIFDELCMD < \end{align}%%%
\DIFdelend \DIFaddbegin \begin{align}
    \hat H(\bm \lambda)\ket{\psi(\bm \lambda)} = \ri\hbar\frac{\partial}{\partial t}\ket{\psi({\bm \lambda})},
    \label{eq:time_evolution}
\end{align}\DIFaddend 
and denote \DIFdelbegin \DIFdel{$\ket{n(\bm r)}$ }\DIFdelend \DIFaddbegin \DIFadd{$\ket{n(\bm \lambda)}$ }\DIFaddend the eigenstates with energies \DIFdelbegin \DIFdel{$\varepsilon_n(\bm r)$.
}\DIFdelend \DIFaddbegin \DIFadd{$\varepsilon_n(\bm \lambda)$.
Let us further assume $\bm \lambda\in\mathds R^3$ for simplicity and relax to more general parametrizations in a later discussion.
}\DIFaddend If the initial state is an \DIFdelbegin \DIFdel{eigenstate $\ket{n(\bm r(0))}$, it will evolve according to $\hat H$ and result in $\ket{n(\bm r(t))}=\re^{-\ri/\hbar\int_0^t \varepsilon_n(\bm r(t))\rd t'}\ket{n(\bm r(0))}$}\DIFdelend \DIFaddbegin \DIFadd{energy eigenstate $\ket{n(\bm \lambda(0))}$, it evolves accordingly and results in $\ket{n(\bm \lambda(t))}=\re^{-\ri/\hbar\int_0^t \varepsilon_n(\bm \lambda(t))\rd t'}\ket{n(\bm \lambda(0))}$}\DIFaddend .
However, the eigenvalue equations do not relate the phases of \DIFdelbegin \DIFdel{$\ket{n(\bm r)}$ at different $\bm r$}\DIFdelend \DIFaddbegin \DIFadd{$\ket{n(\bm \lambda)}$ at different $\bm \lambda$}\DIFaddend , and any (smooth) choice is allowed.
A state with ``arbitrary'' phase can thus be written as
\DIFdelbegin %DIFDELCMD < \begin{align}
%DIFDELCMD <     \ket{\psi(t)} = \re^{-\ri\gamma_n(t)}\ket{n(\bm r(t))}.
%DIFDELCMD <     \label{eq:strange_gauge}
%DIFDELCMD < \end{align}%%%
\DIFdelend \DIFaddbegin \begin{align}
    \ket{\psi(t)} = \re^{-\ri\gamma_n(t)}\ket{n(\bm \lambda(t))}.
    \label{eq:strange_gauge}
\end{align}\DIFaddend 
Although \DIFdelbegin \DIFdel{$\gamma$ }\DIFdelend \DIFaddbegin \DIFadd{$\gamma_n$ }\DIFaddend is, in general, non-integrable, the phase is not allowed to fluctuate randomly in time, which is specified by direct substitution of \cref{eq:strange_gauge} in \cref{eq:time_evolution}
\DIFdelbegin %DIFDELCMD < \begin{align}
%DIFDELCMD <     \dot\gamma_n(t) = \varepsilon_n(\bm r(t))/\hbar -\ri \braket{n({\bm r(t)})|\bm\nabla n({\bm r(t)})} \dot{\bm r}(t).
%DIFDELCMD <     \label{eq:phase_evolution}
%DIFDELCMD < \end{align}%%%
\DIFdelend \DIFaddbegin \begin{align}
    \dot\gamma_n(t) = \varepsilon_n(\bm \lambda(t))/\hbar -\ri \braket{n({\bm \lambda(t)})|\bm\nabla n({\bm \lambda(t)})} \dot{\bm \lambda}(t).
    \label{eq:phase_evolution}
\end{align}\DIFaddend 
By defining the Berry connection (sometimes called Berry potential) \DIFdelbegin \DIFdel{$\bm A_n(\bm r) = -\ri \braket{n(\bm r)|\bm\nabla n(\bm r)}$}\DIFdelend \DIFaddbegin \DIFadd{$\bm A_n(\bm \lambda) = -\ri \braket{n(\bm \lambda)|\bm\nabla n(\bm \lambda)}$}\DIFaddend , the allowed phase change of $\ket\psi$ around a closed loop $C$ in \DIFdelbegin \DIFdel{$\bm r$}\DIFdelend \DIFaddbegin \DIFadd{$\bm \lambda$}\DIFaddend -space is \DIFdelbegin \DIFdel{thus given by
}%DIFDELCMD < \begin{align}
%DIFDELCMD <     \ket{\psi(T)} = \re^{-\ri \gamma_n(C)}\re^{-\ri/\hbar\int_0^T\varepsilon_n(\bm r(t))dt}\ket{\psi(0)}
%DIFDELCMD <     ,\quad
%DIFDELCMD <     \\
%DIFDELCMD <     \gamma_n(C) = \oint_C{\bm A_n}\cdot\rd \bm r.
%DIFDELCMD <     \label{eq:phase_change}
%DIFDELCMD < \end{align}%%%
\DIFdelend \DIFaddbegin \DIFadd{given by
}\begin{align}
    \ket{\psi(T)} = \re^{-\ri \gamma_n(C)}\re^{-\ri/\hbar\int_0^T\varepsilon_n(\bm \lambda(t))dt}\ket{\psi(0)}
    ,\quad
\gamma_n(C) = \oint_C{\bm A_n}\cdot\rd \bm \lambda.
    \label{eq:phase_change}
\end{align}\DIFaddend 
Note the transformation of $\bm A_n$ under the gauge transformation \DIFdelbegin \DIFdel{$\ket{n(\bm r)}\rightarrow\re^{\ri\mu(\bm r)}\ket{n(\bm r)}$}\DIFdelend \DIFaddbegin \DIFadd{$\ket{n(\bm \lambda)}\rightarrow\re^{\ri\mu(\bm \lambda)}\ket{n(\bm \lambda)}$}\DIFaddend , i.e.
\DIFdelbegin %DIFDELCMD < \begin{align}
%DIFDELCMD <     \bm A \rightarrow {\bm A} + \bm\nabla\mu(\bm r),
%DIFDELCMD < \end{align}%%%
\DIFdelend \DIFaddbegin \begin{align}
    \bm A \rightarrow {\bm A} + \bm\nabla\mu(\bm \lambda),
\end{align}\DIFaddend 
which is similar to the electromagnetic vector potential.
Normalization of the state implies that $\bm\nabla \braket{n|n} = \braket{\bm\nabla n|n} + \braket{n|\bm\nabla n} = 0$, which shows that $\braket{n|\bm\nabla n}$ is imaginary and thus ${\bm A_n}$ real-valued.
This \DIFdelbegin \DIFdel{also }\DIFdelend demonstrates that $\bm A_n$ is indeed not gauge invariant \DIFaddbegin \DIFadd{(like the electromagnetic vector potential)}\DIFaddend , but the analog of a magnetic flux must be.
In particular, we can derive the Berry flux by rewriting $\gamma_n(C)$ using Stokes's theorem\DIFdelbegin \DIFdel{(here it is assumed that $\bm r\in\mathds R^3$ for simplicity,
a discussion follows later),
}\DIFdelend \DIFaddbegin \DIFadd{,
}\DIFaddend \begin{align}
    \gamma_n(C) = \int_S{\bm\nabla}\times{\bm A_n}\cdot\rd{\bm S} = -\ri\sum_{m\neq n}\int_S\braket{\bm\nabla n|m}\times\braket{m|\bm\nabla n}\cdot \rd{\bm S}
    \label{eq:berry_phase_stokes_theorem}
\end{align}
in which $\rd{\bm S}$ denotes the area element in \DIFdelbegin \DIFdel{${\bm r}$}\DIFdelend \DIFaddbegin \DIFadd{${\bm \lambda}$}\DIFaddend -space, the surface $S$ is spanned by the closed contour $C$, and the excluded terms are justified by $\bm A$ being real-valued.
The off-diagonal matrix elements are given by $\braket{m|{\bm\nabla} n} = \braket{m|\bm\nabla\hat H|n}/(\varepsilon_n-\varepsilon_m)$ ($m\neq n$).
It is thus possible to define the equivalent of a magnetic field, called Berry curvature
\begin{align}
    \bm F_n = -\ri\sum_{m\neq n}\braket{n|\bm\nabla\hat H|m}\times\braket{m|\bm\nabla\hat H|n}/(\varepsilon_m-\varepsilon_n)^2
\end{align}
such that we arrive at the beautiful result
\DIFdelbegin %DIFDELCMD < \begin{align}
%DIFDELCMD <     \gamma_n(C) = \oint_C{\bm A_n}\cdot\rd \bm r = \int_S {\bm F_n}\cdot\rd{\bm S}.
%DIFDELCMD <     \label{eq:berry_phase_stokes_theorem_2}
%DIFDELCMD < \end{align}%%%
\DIFdelend \DIFaddbegin \begin{align}
    \gamma_n(C) = \oint_C{\bm A_n}\cdot\rd \bm \lambda = \int_S {\bm F_n}\cdot\rd{\bm S}.
    \label{eq:berry_phase_stokes_theorem_2}
\end{align}\DIFaddend 
\DIFdelbegin \DIFdel{Note that the final equations derived here, in particular \mbox{%DIFAUXCMD
\cref{eq:phase_change} }\hspace{0pt}%DIFAUXCMD
depend on the path $C$ only.
Starting }\DIFdelend \DIFaddbegin \DIFadd{If we would start }\DIFaddend from an $n$-variable parametrization, \DIFaddbegin \DIFadd{for which }\DIFaddend $C$ \DIFdelbegin \DIFdel{is then promoted from a one-dimensional to }\DIFdelend \DIFaddbegin \DIFadd{forms }\DIFaddend an $n$-dimensional path\DIFdelbegin \DIFdel{and }\DIFdelend \DIFaddbegin \DIFadd{, }\DIFaddend $\gamma_n$ \DIFdelbegin \DIFdel{to }\DIFdelend \DIFaddbegin \DIFadd{can be used to define }\DIFaddend a more general topological invariant, related to the homotopy class of the mapping between parameter and target space of the parametrization\footnote{
    The fundamental generalization of Stokes' theorem is the Stokes-Cartan theorem.
    It states that the integral of a differential form $\omega$ over the surface $\partial M$ of an (orientable) manifold $M$ is equivalent to the integral of the exterior derivative $\rd\omega$ over the full manifold $M$, i.e. $\int_{\partial M}\omega = \int_M\rd\omega$.
    The definitions of \DIFdelbegin \DIFdel{a }\DIFdelend differential \DIFdelbegin \DIFdel{form }\DIFdelend \DIFaddbegin \DIFadd{forms }\DIFaddend $\omega$, \DIFdelbegin \DIFdel{it's }\DIFdelend exterior \DIFdelbegin \DIFdel{derivative }\DIFdelend \DIFaddbegin \DIFadd{derivatives }\DIFaddend $\rd\omega$ and \DIFdelbegin \DIFdel{the surface }\DIFdelend \DIFaddbegin \DIFadd{surfaces }\DIFaddend of \DIFdelbegin \DIFdel{an }\DIFdelend orientable \DIFdelbegin \DIFdel{manifold }\DIFdelend \DIFaddbegin \DIFadd{manifolds }\DIFaddend $\partial M$ are found in many textbooks, e.g.~\cite{Nakahara1990}.
}.

To apply the general equations in a \DIFdelbegin \DIFdel{solid-state }\DIFdelend \DIFaddbegin \DIFadd{more practical }\DIFaddend context, let \DIFdelbegin \DIFdel{me now exploit the previous equations for a generic $2\times 2$ Bloch Hamiltonian spanned by the vector of Pauli matrices $\bm\sigma$ as
}%DIFDELCMD < \begin{align}
%DIFDELCMD <     H(\bm d(\bm k)) = \bm d(\bm k)\cdot\bm\sigma = \varepsilon(\bm k) \hat{\bm d}(\bm k)\cdot\bm\sigma
%DIFDELCMD <     =
%DIFDELCMD <     \varepsilon(\bm k)
%DIFDELCMD <     \begin{pmatrix}
%DIFDELCMD <         \hat{d}_z & \hat{d}_x - \ri \hat{d}_y \\
%DIFDELCMD <         \hat{d}_x + \ri \hat{d}_y & -\hat{d}_z
%DIFDELCMD <     \end{pmatrix},
%DIFDELCMD <     \label{eq:arbitrary_2by2}
%DIFDELCMD < \end{align}%%%
\DIFdelend \DIFaddbegin \DIFadd{us exploit the results for a spin-1/2 in a magnetic field $\bm B$, described by the following Hamiltonian
}\begin{align}
    H(\bm B) = \kappa\hbar\bm B\cdot\hat{\bm s} = \frac{\kappa\hbar B}2 \hat{\bm B}\cdot\bm\sigma
    =
    \frac{\kappa\hbar B}2
    \begin{pmatrix}
        \hat{B}_z & \hat{B}_x - \ri \hat{B}_y \\
        \hat{B}_x + \ri \hat{B}_y & -\hat{B}_z
    \end{pmatrix},
    \label{eq:arbitrary_2by2}
\end{align}\DIFaddend 
in which \DIFdelbegin \DIFdel{$\hat{\bm d}={\bm d}/|\bm d|$ is a vector of unit norm and $\varepsilon_\pm = \pm\varepsilon = \pm|\bm d|$ spans the energy dispersion.
This is the generic Hamiltonian of a translationally invariant tight binding model of a two-site unit cell in arbitrary dimensions $d$.
Note that the path is thus promoted naturally from a one-dimensional to a $d$-dimensional parametrization $C=C(\bm k)$.
For a more concrete example, we refer to the one-dimensional Su-Schrieffer-Heeger chain studied in \mbox{%DIFAUXCMD
\cref{sec:the_SSH_chain}}\hspace{0pt}%DIFAUXCMD
.
A term proportional to the identity, i.e. $d_0\mathbb 1$, can be safely ignored as the eigenvectors do not depend on it.
They are normalized vectors of the Bloch sphere }\DIFdelend \DIFaddbegin \DIFadd{$\kappa$ is a constant involving the gyromagnetic ratio, $\hat{\bm s} = \frac12{\bm \sigma}$ is the vector operator of Pauli matrices, $\hat{\bm B}={\bm B}/B$ is a vector of unit norm and $B = |\bm B|$.
}

\DIFadd{The eigenvectors have eigenvalues $\varepsilon_\pm = \pm k\hbar B/2$ }\DIFaddend and can be parametrized as
\DIFdelbegin %DIFDELCMD < \begin{align}
%DIFDELCMD <     \ket{\pm} = \frac{1}{\sqrt{2}}
%DIFDELCMD <     \begin{pmatrix}
%DIFDELCMD <         \pm \sqrt{1 \pm \hat{d}_z}\re^{+\ri\vartheta/2}\\
%DIFDELCMD <         \phantom\pm\sqrt{1 \mp \hat{d}_z}\re^{-\ri\vartheta/2}
%DIFDELCMD <     \end{pmatrix}
%DIFDELCMD <     ,
%DIFDELCMD <     \quad
%DIFDELCMD <     \vartheta = \arg(\hat{d}_x-\ri \hat{d}_y).
%DIFDELCMD < \end{align}%%%
\DIFdelend \DIFaddbegin \begin{align}
    \ket{\pm} = \frac{1}{\sqrt{2}}
    \begin{pmatrix}
        \pm \sqrt{1 \pm \hat{B}_z}\re^{+\ri\vartheta/2}\\
        \phantom\pm\sqrt{1 \mp \hat{B}_z}\re^{-\ri\vartheta/2}
    \end{pmatrix}
    ,
    \quad
    \vartheta = \arg(\hat{B}_x-\ri \hat{d}_y).
\end{align}\DIFaddend 
Note that the explored parameter space is entirely spanned by the entries of the vector \DIFdelbegin \DIFdel{$\bm d$}\DIFdelend \DIFaddbegin \DIFadd{$\bm B$}\DIFaddend , and as such the Hamiltonian in \cref{eq:arbitrary_2by2} has the property that \DIFdelbegin \DIFdel{$\bm\nabla\hat H \equiv \bm\nabla_{\bm d}\hat H = \bm\sigma$}\DIFdelend \DIFaddbegin \DIFadd{$\bm\nabla\hat H \equiv \bm\nabla_{\bm B}\hat H = \kappa\hbar\hat{\bm s}$}\DIFaddend .
The local Berry curvature evaluates to \DIFdelbegin \DIFdel{${\bm F}_+ = \frac12\bm d/d^3=\frac12\hat{\bm d}/d^2$}\DIFdelend \DIFaddbegin \DIFadd{${\bm F}_+ = \frac12\bm B/B^3=\frac12\hat{\bm B}/B^2$}\DIFaddend .
Note that the curvature has the form of a point source with strength $1/2$ located at the degeneracy located at \DIFdelbegin \DIFdel{$\bm d=0$}\DIFdelend \DIFaddbegin \DIFadd{$\bm B=0$}\DIFaddend .
The surface integral then yields the following fundamental result
\begin{align}
    \gamma_{\pm}(C) = \pm\frac12\Omega(C),
\end{align}
with $\Omega(C)$ the solid angle subtended by the loop $C$ formed by \DIFdelbegin \DIFdel{$\hat {\bm d}$ }\DIFdelend \DIFaddbegin \DIFadd{$\hat {\bm B}$ }\DIFaddend from the degeneracy at \DIFdelbegin \DIFdel{$\bm d=0$}\DIFdelend \DIFaddbegin \DIFadd{$\bm B=0$}\DIFaddend ~\cite{Berry1984}.
\DIFaddbegin 

\DIFaddend Note that the two bands have opposite Berry phases, and in general $\sum_n\gamma_n(C)=0$ \DIFdelbegin \DIFdel{must vanish}\DIFdelend \DIFaddbegin \DIFadd{vanishes}\DIFaddend .
For instance, if the loop is a $2\pi$ rotation of \DIFdelbegin \DIFdel{$\hat{\bm d}$ }\DIFdelend \DIFaddbegin \DIFadd{$\hat{\bm B}$ }\DIFaddend in a plane, the Berry phase \DIFaddbegin \DIFadd{of the two bands }\DIFaddend equals to $\pm\pi$.
The preservation of $C$ forming a great circle, resulting in the quantization of the Berry phase, can be imposed by a set of symmetry constraints, which we detail in an example in \cref{sec:the_SSH_chain}, and elaborate in more depth in \cref{sec:Periodic_table_of_topological_insulators_and_superconductors}.
In case the loop is not forming a great circle, however, the Berry phase takes any value and can thus not be used to define an invariant that is uniquely linked to the properties of the phase.

\DIFdelbegin \DIFdel{If $\bm d(\bm k)$ }\DIFdelend \DIFaddbegin \DIFadd{Incidentally, the Hamiltonian in \mbox{%DIFAUXCMD
\cref{eq:arbitrary_2by2} }\hspace{0pt}%DIFAUXCMD
coincides with the one of translationally invariant tight binding models with two-site unit cell (compare with \mbox{%DIFAUXCMD
\cref{eq:SSH_momentum_space}}\hspace{0pt}%DIFAUXCMD
).
In this case, the parameter path is encoded in a ``pseudo magnetic field'', which is a $d$-dimensional parametrization depending on the lattice dimension, geometry and the crystal momentum $\bm B\equiv\bm B(\bm k)$.
If $\bm B(\bm k)$ }\DIFaddend covers the full Bloch sphere, the Berry phase corresponds to $2\pi$ since the solid angle of the unit-sphere is $4\pi$.
In this case, an integer-valued topological invariant known as the Chern number can be defined~\cite{Nakahara1990}
\begin{align}
    \chi_n = \frac1{2\pi}\int_S {\bm F_n}\cdot\rd{\bm S}.
    \label{eq:chern_number}
\end{align}
Note that \cref{eq:chern_number} shares the same fundamental expression as the Euler characteristic in \cref{eq:gauss_bonnet_theorem}.
$\bm F_n$ can thus be understood as a curvature with similarities to a Gaussian one.
Therefore, the Berry curvature integrated over a closed surface must be an element of $2\pi \mathds Z$, and the quantization of the Chern number \DIFdelbegin \DIFdel{naturally }\DIFdelend extends well beyond the two-band model.

\section{The Su-Schrieffer-Heeger chain}
\label{sec:the_SSH_chain}
Let \DIFdelbegin \DIFdel{me }\DIFdelend \DIFaddbegin \DIFadd{us }\DIFaddend clarify the previous statements by giving a pedagogical example.
Consider the following one-dimensional Bloch Hamiltonian, corresponding to the Su-Schrieffer-Heeger model
\DIFdelbegin \DIFdel{~\mbox{%DIFAUXCMD
\cite{Heeger1988}
}\hspace{0pt}%DIFAUXCMD
}%DIFDELCMD < \begin{align}
%DIFDELCMD <     H_0(k)
%DIFDELCMD <     =
%DIFDELCMD <     \begin{pmatrix}
%DIFDELCMD <         0 & J+J'\re^{-\ri k}\\
%DIFDELCMD <         J+J'\re^{\ri k} & 0
%DIFDELCMD <     \end{pmatrix}
%DIFDELCMD < \end{align}%%%
\DIFdelend \DIFaddbegin \begin{align}
    H_0(k)
    =
    \begin{pmatrix}
        0 & J+J'\re^{-\ri k}\\
        J+J'\re^{\ri k} & 0
    \end{pmatrix}
    \label{eq:SSH_momentum_space}
\end{align}\DIFaddend 
in which the couplings $J,J'\geq0$ are assumed positive.
\DIFdelbegin \DIFdel{The Hamiltonian falls under the category in \ref{eq:arbitrary_2by2}}\DIFdelend \DIFaddbegin \DIFadd{It models (among others) electronic properties of poly\-acetylene with an even number of carbon atoms, which provides the most simple example of a topological insulator.
For details on the derivation, and a careful reduction of interactions, we refer to~\mbox{%DIFAUXCMD
\cite{Heeger1988}}\hspace{0pt}%DIFAUXCMD
.
The Hamiltonian coincides with the one of a spin-1/2 in a magnetic field (compare with \ref{eq:arbitrary_2by2})}\DIFaddend , and we identify
\DIFdelbegin %DIFDELCMD < \begin{align}
%DIFDELCMD <     H_0(k) = {\bm d}\cdot{\bm\sigma},
%DIFDELCMD <     \quad
%DIFDELCMD <     d_x = J+J'\cos(k),
%DIFDELCMD <     \quad
%DIFDELCMD <     d_y = J'\sin(k),
%DIFDELCMD <     \quad
%DIFDELCMD <     d_z = 0.
%DIFDELCMD < \end{align}%%%
\DIFdelend \DIFaddbegin \begin{align}
    H_0(k) = {\bm B}(k)\cdot{\bm\sigma},
    \quad
    B_x(k) = J+J'\cos(k),
    \quad
    B_y(k) = J'\sin(k),
    \quad
    B_z(k) = 0.
\end{align}\DIFaddend 
Note that \DIFdelbegin \DIFdel{in this case, }\DIFdelend the parametrization of the path $C=C(k)$ spanned by the \DIFdelbegin \DIFdel{vector $\hat{\bm d}(k)$ }\DIFdelend \DIFaddbegin \DIFadd{pseudo magnetic field $\hat{\bm B}(k)$ }\DIFaddend is one-dimensional\DIFdelbegin \DIFdel{by definition of $H_0(k)$}\DIFdelend .
Examples of the band dispersion $\varepsilon_\pm(k)=\pm\sqrt{J^2+J'^2+2JJ'\cos(k)}$ for different values of the couplings are plotted in \cref{fig:ssh_dispersion}\DIFaddbegin \DIFadd{.
}\DIFaddend \begin{figure}[ht]
    \centering
    \includegraphics{figures/ssh_dispersion_0.png}
    \includegraphics{figures/cropped_ssh_dispersion_1.png}
    \includegraphics{figures/cropped_ssh_dispersion_2.png}
    \includegraphics{figures/cropped_ssh_dispersion_3.png}
    \caption{Dimensionless band dispersion $\tilde\varepsilon_\pm(k)=\varepsilon_\pm(k)/|J+J'|$ of the SSH model for different real and positive couplings $J'/J$. The band gap \DIFdelbeginFL \DIFdelFL{$\Delta=|J-J'|$ }\DIFdelendFL \DIFaddbeginFL \DIFaddFL{$\Delta=2|J-J'|$ }\DIFaddendFL closes at the point $J'/J=1$ and the (dimensionless) dispersion \DIFdelbeginFL \DIFdelFL{itself }\DIFdelendFL is symmetric with respect to $J'/J\rightarrow J/J'$.}
    \label{fig:ssh_dispersion}
\end{figure}

Let us explore more the topological equivalency we detailed in the previous section.
The valence/conduction band shows a maximum/minimum at $k=\pi$, \DIFdelbegin \DIFdel{which leads to the band gap }\DIFdelend \DIFaddbegin \DIFadd{resulting in a band gap of size }\DIFaddend $\Delta = 2\left|J-J'\right|$.
As long as $J\neq J'$, the band gap $\Delta\neq0$ is preserved, and \DIFdelbegin \DIFdel{, for reasons which are clear in the next sections}\DIFdelend \DIFaddbegin \DIFadd{for reasons explained shortly}\DIFaddend , we call $J>J'$ \DIFdelbegin \DIFdel{the TRI}\DIFdelend \DIFaddbegin \DIFadd{trivial insulator (TRI) }\DIFaddend and $J<J'$ \DIFdelbegin \DIFdel{the TOIphase}\DIFdelend \DIFaddbegin \DIFadd{topological insulator (TOI)}\DIFaddend .
Within each of those phases, all Hamiltonians are topologically equivalent\DIFdelbegin \DIFdel{, which is trivially satisfied by the definition of a smooth path which linearly interpolates the different values of the couplings}\DIFdelend .
It is also possible to define a family of Hamiltonians according to
\begin{align}
    H(t,k) =
    \begin{pmatrix}
        \sin(\pi t) & 1-t + t\re^{-\ri k} \\
        1-t + t\re^{\ri k} & -\sin(\pi t)
    \end{pmatrix},
    \label{eq:hamiltonian_path}
\end{align}
which is equivalent to the Bloch vector
\DIFdelbegin %DIFDELCMD < \begin{align}
%DIFDELCMD <     d_x(t,k) = (1-t)+t\cos(k),
%DIFDELCMD <     \quad
%DIFDELCMD <     d_y(t,k) = t\sin(k),
%DIFDELCMD <     \quad
%DIFDELCMD <     d_z(t,k) = \sin(\pi t),
%DIFDELCMD < \end{align}%%%
\DIFdelend \DIFaddbegin \begin{align}
    B_x(t,k) = (1-t)+t\cos(k),
    \quad
    B_y(t,k) = t\sin(k),
    \quad
    B_z(t,k) = \sin(\pi t).
\end{align}\DIFaddend 
\DIFdelbegin \DIFdel{for which }\DIFdelend \DIFaddbegin \begin{figure}[ht]
    \centering
    \includegraphics[width=0.13\textwidth]{figures/ssh_deformed_normalized_bloch_vector_t0_000.jpg}
    \includegraphics[width=0.13\textwidth]{figures/ssh_deformed_normalized_bloch_vector_t0_125.jpg}
    \includegraphics[width=0.13\textwidth]{figures/ssh_deformed_normalized_bloch_vector_t0_250.jpg}
    \includegraphics[width=0.13\textwidth]{figures/ssh_deformed_normalized_bloch_vector_t0_500.jpg}
    \includegraphics[width=0.13\textwidth]{figures/ssh_deformed_normalized_bloch_vector_t0_750.jpg}
    \includegraphics[width=0.13\textwidth]{figures/ssh_deformed_normalized_bloch_vector_t0_825.jpg}
    \includegraphics[width=0.13\textwidth]{figures/ssh_deformed_normalized_bloch_vector_t1_000.jpg}
    \caption{\DIFaddFL{Depicted paths of $\hat{\bm B}(t,k)={\bm B}/|{\bm B}|$ (see text) as it wraps around the Brillouin zone $k\in(0,2\pi]$ for different $t\in\{0,1/8,1/4,1/2,3/4,7/8,1\}$ (from left to right). The solid angle of $2\pi$ spanned in the TOI phase (i.e. $t=1$) can be smoothly deformed to the trivial case in the TRI phase (i.e. $t=0$).}}
    \label{fig:ssh_deformed}
\end{figure}

\DIFadd{In the equations above, }\DIFaddend $H(0,k)$ \DIFdelbegin \DIFdel{is of the }\DIFdelend \DIFaddbegin \DIFadd{describes a }\DIFaddend TRI and $H(1,k)$ \DIFdelbegin \DIFdel{is of the TOI kind.
In \mbox{%DIFAUXCMD
\cref{fig:ssh_deformed}}\hspace{0pt}%DIFAUXCMD
, the }\DIFdelend \DIFaddbegin \DIFadd{a TOI phase.
The spectrum of $H(t,k)$ is readily solved, and it is easy to verify that $\Delta({t})>0$ for $0\leq t\leq1$.
The }\DIFaddend path of the normalized Bloch vector \DIFdelbegin \DIFdel{$\hat{\bm d}(t,k)$ is depicted .
If we allow the }\DIFdelend \DIFaddbegin \DIFadd{$\hat{\bm B}(t,k)$ for different values of the parametrization $t$ is depicted in \mbox{%DIFAUXCMD
\cref{fig:ssh_deformed}}\hspace{0pt}%DIFAUXCMD
:
If a }\DIFaddend deformation along the $z$-axis \DIFdelbegin \DIFdel{, the cycle that is drawn in the TOI phase can be smoothly deformed to a }\DIFdelend \DIFaddbegin \DIFadd{is allowed, the nontrivial circular path is smoothly connected to a trivial }\DIFaddend point without closing the bulk gap\DIFdelbegin \DIFdel{.
The spectrum of $H(t,k)$ is readily solved and evaluates to $\varepsilon_\pm(k,t)=\pm\sqrt{1-2(1-t)t+2(1-t)t\cos k+\sin(\pi t)}$.
It is easy to see that $\Delta({t})\geq0$ for $0\leq t\leq1$, which demonstrates that }\DIFdelend \DIFaddbegin \DIFadd{, demonstrating the adiabatic connection between }\DIFaddend TOI and TRI phases\DIFdelbegin \DIFdel{are, indeed, topologically equivalent.
}%DIFDELCMD < \begin{figure}[ht]
%DIFDELCMD <     \centering
%DIFDELCMD <     \includegraphics[width=0.13\textwidth]{figures/ssh_deformed_normalized_bloch_vector_t0_000.jpg}
%DIFDELCMD <     \includegraphics[width=0.13\textwidth]{figures/ssh_deformed_normalized_bloch_vector_t0_125.jpg}
%DIFDELCMD <     \includegraphics[width=0.13\textwidth]{figures/ssh_deformed_normalized_bloch_vector_t0_250.jpg}
%DIFDELCMD <     \includegraphics[width=0.13\textwidth]{figures/ssh_deformed_normalized_bloch_vector_t0_500.jpg}
%DIFDELCMD <     \includegraphics[width=0.13\textwidth]{figures/ssh_deformed_normalized_bloch_vector_t0_750.jpg}
%DIFDELCMD <     \includegraphics[width=0.13\textwidth]{figures/ssh_deformed_normalized_bloch_vector_t0_825.jpg}
%DIFDELCMD <     \includegraphics[width=0.13\textwidth]{figures/ssh_deformed_normalized_bloch_vector_t1_000.jpg}
%DIFDELCMD <     %%%
%DIFDELCMD < \caption{%
{%DIFAUXCMD
\DIFdelFL{Depicted paths of $\hat{\bm d}(t,k)={\bm d}/|{\bm d}|$ (see text) as it wraps around the Brillouin zone $k\in(0,2\pi]$ for different $t\in\{0,1/8,1/4,1/2,3/4,7/8,1\}$ (from left to right). The solid angle of $2\pi$ spanned in the TOI phase (i.e. $t=1$) can be smoothly deformed to the trivial case in the TRI phase (i.e. $t=0$).}}
    %DIFAUXCMD
%DIFDELCMD < \label{fig:ssh_deformed}
%DIFDELCMD < \end{figure}
%DIFDELCMD < %%%
\DIFdelend \DIFaddbegin \DIFadd{.
}\DIFaddend 

This statement can be geometrically understood by the fact that the single-valued parametrization \DIFdelbegin \DIFdel{$\bm d(k)$ }\DIFdelend \DIFaddbegin \DIFadd{$\bm B(k)$ }\DIFaddend generates an arc of a circle or a full circle on the Bloch sphere.
In other words, the surface $S$ spanned by $C$ (the white path in \cref{fig:ssh_deformed}) never fully encapsulates the degeneracy, which is the source of ${\bm F}_n$.
As a consequence, $\gamma_n(C)$ is not quantized and adiabatic deformations like \cref{eq:hamiltonian_path} are possible to effectively remove the degeneracy from the surface integral, smoothly connecting all nonzero geometric phases to the trivial one.
\DIFaddbegin \begin{figure}[ht]
    \centering
    \includegraphics{figures/ssh_unnormalized_winding_2.png}
    \includegraphics{figures/ssh_unnormalized_winding_3.png}
    \includegraphics{figures/ssh_unnormalized_winding_4.png}
    \includegraphics{figures/ssh_unnormalized_winding_6.png}
    \caption{\DIFaddFL{The path in $x,y$ plane spanned by the unnormalized Bloch vector $\bm B$, compared with the path of $\hat{\bm B}$ drawn on the Bloch sphere for different values of $J'/J$. The path drawn by $\bm B$ either traverses around the origin or not, resulting in a path spanning a nonzero surface area or a path which retraces itself thus spanning no surface.}}
    \label{fig:ssh_winding_easy}
\end{figure}
\DIFaddend 

If, however, we require to keep $H(t,k)$ entirely off-diagonal\DIFdelbegin \DIFdel{and symmetric, an adiabatic deformation }\DIFdelend \DIFaddbegin \DIFadd{, deformations }\DIFaddend like \cref{eq:hamiltonian_path} with \DIFdelbegin \DIFdel{a term $d_z\neq0$ does }\DIFdelend \DIFaddbegin \DIFadd{terms $B_z\neq0$ do }\DIFaddend not exist.
As a consequence, any \DIFdelbegin \DIFdel{adiabatic deformation }\DIFdelend \DIFaddbegin \DIFadd{interpolation }\DIFaddend from a TOI to a TRI Hamiltonian necessarily crosses the critical point $J=J'$ \DIFdelbegin \DIFdel{for which }\DIFdelend \DIFaddbegin \DIFadd{where }\DIFaddend the bulk gap vanishes.
The Berry phase of such a system is an element of $\pi\mathds Z$, being non-zero \DIFdelbegin \DIFdel{iff. }\DIFdelend \DIFaddbegin \DIFadd{if and only if }\DIFaddend the path $C$ spanned by \DIFdelbegin \DIFdel{$\hat{\bm d}$ defines a surface.
This can again be understood more intuitively from a geometric point of view:
By imposing the off-diagonal structure of the family of Hamiltonians, the degeneracy cannot be removed from a nonzero surfaceintegral (which would indeed require a deformation along $\hat {d}_z$), and as a consequence the Berry phase is quantized.
}%DIFDELCMD < 

%DIFDELCMD < %%%
\DIFdelend \DIFaddbegin \DIFadd{$\hat{\bm B}$ defines the boundary of a nontrivial surface.
}\DIFaddend In analogy to the Chern number, this defines a topological index called the winding number, which corresponds to quantized values only in the equivalence classes of Hamiltonians \DIFdelbegin \DIFdel{which respect the imposed symmetry}\DIFdelend \DIFaddbegin \DIFadd{respecting the imposed ``symmetry'' of being off-diagonal (this is defined more properly in \mbox{%DIFAUXCMD
\cref{sec:Periodic_table_of_topological_insulators_and_superconductors}}\hspace{0pt}%DIFAUXCMD
)}\DIFaddend .
It equals to the Berry phase divided $\pi$, i.e.
\begin{align}
    \mathcal W = |\gamma_n(C)/\pi|,
\end{align}
or, equivalently, to the solid angle spanned by \DIFdelbegin \DIFdel{$\hat{\bm d}$ }\DIFdelend \DIFaddbegin \DIFadd{$\hat{\bm B}$ }\DIFaddend divided $2\pi$, i.e.
\DIFdelbegin %DIFDELCMD < \begin{align}
%DIFDELCMD <     \mathcal W = \frac1{2\pi}\int\rd k\brlr{\hat{\bm d}\times\partial_k\hat{\bm d}}_z.
%DIFDELCMD < \end{align}%%%
\DIFdelend \DIFaddbegin \begin{align}
    \mathcal W = \frac1{2\pi}\int\rd k\brlr{\hat{\bm B}\times\partial_k\hat{\bm B}}_z.
\end{align}\DIFaddend 
Since the winding \DIFdelbegin \DIFdel{counts the number of circles which $\hat{\bm d}$ }\DIFdelend \DIFaddbegin \DIFadd{number counts the amount of full circles $\hat{\bm B}$ }\DIFaddend performs around the origin while passing through the Brillouin zone, it is equal to
\DIFdelbegin %DIFDELCMD < \begin{align}
%DIFDELCMD <     \mathcal W = \frac{1}{2\pi}\int\rd k \arg(\hat{\bm d}_x - \ri \hat{\bm d}_y)
%DIFDELCMD <     =
%DIFDELCMD <     \frac{1}{4\pi\ri} \tr\int\rd k \sigma_z H^{-1}\partial_k H
%DIFDELCMD <     =
%DIFDELCMD <     \frac{1}{4\pi\ri} \tr\int\rd k \sigma_z g^{-1}\partial_k g,
%DIFDELCMD <     \label{eq:winding_number}
%DIFDELCMD < \end{align}%%%
\DIFdelend \DIFaddbegin \begin{align}
    \mathcal W = \frac{1}{2\pi}\int\rd k \arg(\hat{\bm B}_x - \ri \hat{\bm B}_y)
    =
    \frac{1}{4\pi\ri} \tr\int\rd k \sigma_z H^{-1}\partial_k H
    =
    \frac{1}{4\pi\ri} \tr\int\rd k \sigma_z g^{-1}\partial_k g,
    \label{eq:winding_number}
\end{align}\DIFaddend 
in which $g(k)=G(k,\omega=0)$ is the Green's function at zero frequency.

The winding number can most easily \DIFdelbegin \DIFdel{read-out by close observation }\DIFdelend \DIFaddbegin \DIFadd{be identified by close inspection }\DIFaddend of the unnormalized Bloch vector \DIFdelbegin \DIFdel{${\bm d}$, which either traverses }\DIFdelend \DIFaddbegin \DIFadd{${\bm B}$, traversing (or not) }\DIFaddend around the degeneracy\DIFdelbegin \DIFdel{at the origin, or not}\DIFdelend .
This will then translate to \DIFdelbegin \DIFdel{$\hat{\bm d}$ }\DIFdelend \DIFaddbegin \DIFadd{$\hat{\bm B}$ }\DIFaddend drawing circles around the Bloch sphere, or retracing itself, thus spanning no surface (\DIFdelbegin \DIFdel{compare to \mbox{%DIFAUXCMD
\cref{fig:ssh_winding_easy} }\hspace{0pt}%DIFAUXCMD
}\DIFdelend \DIFaddbegin \DIFadd{see \mbox{%DIFAUXCMD
\cref{fig:ssh_winding_easy} }\hspace{0pt}%DIFAUXCMD
and compare plots with the insets}\DIFaddend ).
In particular, the former will be achieved if $J<J'$ ($\mathcal W=1$), while the latter occurs for $J>J'$ ($\mathcal W=0$).
From now on, we call a phase with nonzero winding number topological insulator (TOI) and the other trivial insulator (TRI).
\DIFdelbegin %DIFDELCMD < \begin{figure}[ht]
%DIFDELCMD <     \centering
%DIFDELCMD <     \includegraphics{figures/ssh_unnormalized_winding_2.png}
%DIFDELCMD <     \includegraphics{figures/ssh_unnormalized_winding_3.png}
%DIFDELCMD <     \includegraphics{figures/ssh_unnormalized_winding_4.png}
%DIFDELCMD <     \includegraphics{figures/ssh_unnormalized_winding_6.png}
%DIFDELCMD <     %%%
%DIFDELCMD < \caption{%
{%DIFAUXCMD
\DIFdelFL{The path in $x,y$ plane spanned by the unnormalized Bloch vector $\bm d$, compared with the path of $\hat{\bm d}$ drawn on the Bloch sphere for different values of $J'/J$. The path drawn by $\bm d$ either traverses around the origin or not, resulting in a path spanning a nonzero surface area or a path which retraces itself thus spanning no surface.}}
    %DIFAUXCMD
%DIFDELCMD < \label{fig:ssh_winding_easy}
%DIFDELCMD < \end{figure}
%DIFDELCMD < %%%
\DIFdelend 

To understand the emergence of edge modes, we must obviously introduce an interface of some sort.
First of all, let \DIFdelbegin \DIFdel{me }\DIFdelend \DIFaddbegin \DIFadd{us }\DIFaddend Fourier-transform the Bloch Hamiltonian to find the representation in real space, i.e.
\begin{align}
    \hat H = \sum_x \brlr{J\hat c^\dag_{x,A}\hat c^\pdag_{x,B} + J' \hat c^\dag_{x,B}\hat c^\pdag_{x+1,A}} + \hc,
\end{align}
in which $\hat c_{x,A/B}$ are the annihilators of a spinless fermionic particle at site $x$ and sublattice $A/B$.
An interface \DIFdelbegin \DIFdel{which hosts }\DIFdelend \DIFaddbegin \DIFadd{hosting }\DIFaddend edge states with energy zero, following the logic of the previous section, is readily implemented by two semi-infinite crystals, one with bulk invariant $0$ (e.g. for all negative lattice positions $x<0$), the other with bulk invariant $1$ (for all lattice positions $x\geq0$).
For convenience, let \DIFdelbegin \DIFdel{me here }\DIFdelend \DIFaddbegin \DIFadd{us }\DIFaddend construct this crystal from the two limiting cases (i) $J'=0$ and (ii) $J=0$.
Note that for such a system, the central $A$ site at position $x=0$ does not contribute to the Hamiltonian and hosts a perfectly localized mode of zero energy.
Obviously, this edge mode is preserved if we consider the vacuum in region (i).
Let \DIFdelbegin \DIFdel{me thus }\DIFdelend \DIFaddbegin \DIFadd{us }\DIFaddend simplify the hypothetical crystal by replacing the trivial insulator with the vacuum, and proceed by relaxing the constraint $J=0$ to $J'>J$ on the semi-infinite crystal:
\begin{align}
    \hat H = \sum_{x\geq0} J \hat c^\dag_{x,A}\hat c^\pdag_{x,B} + J' \hat c^\dag_{x,B}\hat c^\pdag_{x+1,A} + \hc
\end{align}
In this case, the presence of an edge zero mode is not particularly obvious and requires further inspection.
One observation is that the fermionic Hamiltonian can be diagonalized by unitary transformations, which requires only a diagonalization of $\hat H = \hat{\bm c}^\dag H \hat{\bm c}$, \DIFdelbegin \DIFdel{in which }\DIFdelend \DIFaddbegin \DIFadd{where }\DIFaddend $\hat{\bm c}=(\hat c_{0,A},\hat c_{0,B},\hat c_{1,A},\hat c_{1,B},\dots)^T$ is the successive vector of all \DIFaddbegin \DIFadd{fermionic }\DIFaddend annihilation operators.
The matrix $H$ is quite sparse and has only alternating off-diagonal elements
\begin{align}
    H =
    \begin{pmatrix}
        0 & J & 0 & 0 &\dots \\
        J & 0 & J' & 0 & \dots \\
        0 & J' & 0 & J & \dots \\
        0 & 0 & J & 0 & \dots \\
        \vdots & \vdots & \vdots & \vdots & \ddots
    \end{pmatrix},
\end{align}
\DIFdelbegin \DIFdel{which makes }\DIFdelend \DIFaddbegin \DIFadd{such that }\DIFaddend an analytic treatment \DIFaddbegin \DIFadd{is }\DIFaddend not too difficult, even if the entries are \DIFdelbegin \DIFdel{spacially }\DIFdelend \DIFaddbegin \DIFadd{spatially }\DIFaddend dependent~\cite{Asboth2016}.
An exact boundary mode can be found by evaluating $hv_L=0$, with a vector $v_L = (a_0,b_0,a_1,b_1,\dots)^T$, resulting in the set of equations
\begin{align}
    J b_0 = 0,
    \quad
    Ja_m + J' a_{m+1} = 0,
    \quad
    J' b_m + J b_{m+1} = 0.
\end{align}
Defining $w=J'/J$, the solution reveals an exponentially localized zero energy mode \DIFdelbegin \DIFdel{which has }\DIFdelend \DIFaddbegin \DIFadd{with }\DIFaddend no support on sublattice $B$, i.e.
\DIFdelbegin %DIFDELCMD < \begin{align}
%DIFDELCMD <     a_k=(-w)^{-k} a_0 \rightarrow |a_k| = \re^{-k/\log w}|a_0|,
%DIFDELCMD <     \quad
%DIFDELCMD <     b_k = 0.
%DIFDELCMD <     \label{eq:boundary_mode}
%DIFDELCMD < \end{align}%%%
\DIFdelend \DIFaddbegin \begin{align}
    a_k=(-w)^{-k} a_0 \rightarrow |a_k| = \re^{-k/\ln w}|a_0|,
    \quad
    b_k = 0.
    \label{eq:boundary_mode}
\end{align}\DIFaddend 
It is also transparent that the state generated by the vector $v_L$ would violate normalization in the TRI phase ($w<1$) and is thus allowed to exist only if the geometric series is convergent.

In case of a finite crystal with $N$ sites, a similar reasoning reveals the existence of two such boundary modes, generated by the vectors $v_L$ and $v_R$.
Similarly to $v_L$, $v_R$ is exponentially localized at the right interface, has no support on sublattice $A$ and decays exponentially into the bulk.
Thus, the two modes share exactly no overlap $v_L^T v_R = 0$.
However, the two modes are not exact eigenstates of the Hamiltonian since each one slightly violates the boundary conditions.
Note that $v_{L/R}^T h v_{L/R} = 0$ and \DIFdelbegin \DIFdel{$v_{L/R}^T h v_{R/L} = (1-w^{-1})^{-1}\re^{-(N-1)/\log w}\re^{\pm\varphi}$ }\DIFdelend \DIFaddbegin \DIFadd{$v_{L/R}^T h v_{R/L} = (1-w^{-1})^{-1}\re^{-(N-1)/\ln w}\re^{\pm\ri\varphi}$ }\DIFaddend with some phase $\varphi$.
This then yields ``state hybridization''\DIFdelbegin \DIFdel{, }\DIFdelend \DIFaddbegin \DIFadd{: }\DIFaddend a pair of orthogonal eigenstates \DIFdelbegin \DIFdel{with energy $\pm(1-w^{-1})^{-1}\re^{-(N-1)/\log w}$ which read $v_\pm = 1/\sqrt2(v_L \pm v_R\re^{\pm\varphi})$}\DIFdelend \DIFaddbegin \DIFadd{$v_\pm = 1/\sqrt2(v_L \pm v_R\re^{\pm\ri\varphi})$ with energy $\pm(1-w^{-1})^{-1}\re^{-(N-1)/\ln w}$}\DIFaddend .
Therefore, the hybridized states are \DIFdelbegin \DIFdel{still }\DIFdelend exponentially localized boundary modes and of ``almost'' zero energy with an \DIFdelbegin \DIFdel{exponential }\DIFdelend \DIFaddbegin \DIFadd{exponentially small }\DIFaddend splitting in the system size.
\section{Periodic table of topological insulators and superconductors}
\label{sec:Periodic_table_of_topological_insulators_and_superconductors}
Let us consider an arbitrary non-interacting system of fermions, described by the particle-number conserving Hamiltonian
\begin{align}
    \hat H = \sum_{i,j}\hat c^\dag_{i}H_{ij}\hat c^\pdag_j = \hat{\bm c}^\dag H \hat{\bm c}^\pdag
\end{align}
with a matrix $H$ dictating possible transition events.
As before, we denote the collection of the fermionic operators as $\hat{\bm c}$.
Note that $i,j$ are generic index labels -- \DIFdelbegin \DIFdel{i.e. it }\DIFdelend \DIFaddbegin \DIFadd{they }\DIFaddend may represent a combination of the position $\bm x_i$ and additional ``internal quantum numbers'' $i\coloneqq(\bm x_i,q_i)$\DIFdelbegin \DIFdel{such as }\DIFdelend \DIFaddbegin \DIFadd{, e.g. }\DIFaddend the sublattice index \DIFaddbegin \DIFadd{$q_i\in \{A,B\}$ }\DIFaddend of the SSH model\DIFdelbegin \DIFdel{for which $q_i\in \{A,B\}$}\DIFdelend .
The system is invariant under a (unitary) symmetry transformation $\hat U$, if it acts on the fermionic annihilation operators as a linear map
\begin{align}
    \hat{\bm c}\rightarrow \hat{\bm c}' = \hat U \hat{\bm c} \hat U^{-1} = U \hat{\bm c},
\end{align}
such that the canonical anticommutation relations and $\hat H$ are preserved, i.e.
\begin{align}
    \hat U\anticommutator{\hat c_i^\pdag,\hat c_j^\dag}\hat U^{-1}
    =
    \anticommutator{\hat c_i^\pdag,\hat c_j^\dag}
    \rightarrow
    U^\dag U = \mathbb1
    ,\quad
    \hat U \hat H \hat U^{-1} = \hat H
    \rightarrow
    U^\dag H U = H
    .
\end{align}
Focussing on the so-called ``Altland-Zirnbauer'' classification scheme, we consider here on-site symmetries only~\cite{Altland1997}.
On-site symmetries, as the name suggests, only act on the ``internal'' degrees of freedom, such that they can be factorized $\hat U=\prod_{\bm x}\hat V_{\bm x}$ with the same $\hat V_{\bm x}$ on all lattice positions.
Please note that $\hat V_{\bm x}$ acts nontrivially only on the ``internal dimension''.

{\it 1. Time reversal symmetry} \DIFdelbegin \DIFdel{Note }\DIFdelend \DIFaddbegin \DIFadd{Recall }\DIFaddend that anti-unitary transformations can be written as a composition of complex conjugation \DIFdelbegin \DIFdel{and a unitary transformation, such that the action of time reversal can be written as}\footnote{\DIFdel{Note that $\hat c\rightarrow \hat c^\dag$ in \mbox{%DIFAUXCMD
\cref{eq:time_reversal_symmetry} }\hspace{0pt}%DIFAUXCMD
would be an equally valid choice, but, as we see later, can be treated as a combination of time reversal and particle hole symmetry.}}
%DIFAUXCMD
\addtocounter{footnote}{-1}%DIFAUXCMD
\DIFdelend \DIFaddbegin \DIFadd{with unitary transformations}\footnote{\DIFadd{Note that $\hat c\rightarrow \hat c^\dag$ in \mbox{%DIFAUXCMD
\cref{eq:time_reversal_symmetry} }\hspace{0pt}%DIFAUXCMD
would be an equally valid choice, but this is treated by convention as a combination of time reversal and particle hole symmetry.}}\DIFadd{, i.e.
}\DIFaddend \begin{align}
    \hat T \hat c_i \hat T^{-1} = {U_T}_{ij} \hat c_j
    ,\quad
    \hat T\ri\hat T^{-1}=-\ri
    ,\quad
    {U_T}^\dag H^* U_T = H.
    \label{eq:time_reversal_symmetry}
\end{align}
Applying the transformation twice results in the condition
\begin{align}
    H = {U_T}^\dag ({U_T}^\dag H^* U_T)^* U_T = ({U_T}^*U_T)^\dag H ({U_T}^* U_T).
\end{align}
Since the first quantized Hamiltonian runs over an irreducible representation space, ${U_T}^* U_T = \re^{\ri\alpha}\mathbb1$ due to Schur's lemma~\cite{Chiu2016}.
Moreover, $U_T$ is a unitary matrix, \DIFdelbegin \DIFdel{thus }\DIFdelend satisfies ${U_T}^*={U_T}^\dag\re^{\ri\alpha}$ (and equally ${{U_T}^\dag}^* = U_T\re^{\ri\alpha}$), and \DIFdelbegin \DIFdel{as such }\DIFdelend $({U_T}^\dag U_T)^* = \re^{-\ri2\alpha}\mathbb1$, for which $\alpha\in\pi\mathds Z$.
This implies that
\begin{align}
    \hat T^2 \hat c_i \hat T^{-2} = {U_T}_{ij}^*{U_T}_{jk} \hat c_k = \re^{\ri\alpha}\hat c_i
\end{align}
maps to itself up to a possible sign.
\DIFdelbegin \DIFdel{Since }\DIFdelend $\hat T$ \DIFdelbegin \DIFdel{is }\DIFdelend \DIFaddbegin \DIFadd{acts }\DIFaddend on-site \DIFdelbegin \DIFdel{, }\DIFdelend \DIFaddbegin \DIFadd{only and therefore }\DIFaddend a generic $n$-body operator $\hat O$ transforms to $\hat T^2\hat O \hat T^{-2} = \re^{\ri\alpha n}\hat O$\DIFdelbegin \DIFdel{and as such
}\DIFdelend \DIFaddbegin \DIFadd{, which leads to
}\DIFaddend \begin{align}
    \hat T^2 = (\pm 1)^{\hat N}\Leftrightarrow {U_T}^* {U_T} = \pm\mathbb1,
\end{align}
with $\hat N = \sum_i \hat c^\dag_i \hat c^\pdag_i$.
In case of $\hat T^2=-1$ (e.g. for systems with an odd number of electrons), this then yields the famous Kramers degeneracy of the eigenvalues~\cite{Schwabl2007}.

{\it 2. Particle-hole symmetry} is unitary and relates creation to annihilation operators, i.e.
\begin{align}
    \hat C \hat c_i \hat C^{-1} = {U_C}^*_{ij} \hat c^\dag_j
    ,\quad
    {U_C}^\dag H^* {U_C} = -H
    .
\end{align}
Such systems are characterized by off-diagonal quadratic Hamiltonians $\tr H=0$.
Repetition of the previous computation yields
\begin{align}
    \hat C^2 = (\pm1)^{\hat N}
    \Leftrightarrow
    {U_C}^* {U_C} = \pm 1.
\end{align}

{\it 3. Chiral symmetry} is the combination of $\hat T$ with $\hat C$ to a so-called chiral symmetry
\begin{align}
    \hat S = \hat T \hat C.
\end{align}
This may lead to a situation \DIFdelbegin \DIFdel{in which }\DIFdelend \DIFaddbegin \DIFadd{where }\DIFaddend both time reversal and particle-hole symmetry do not hold, but chiral symmetry is satisfied.
One finds
\DIFdelbegin %DIFDELCMD < \begin{align}
%DIFDELCMD <     \hat S \hat c_i \hat S^{-1} = (U_CU_T)_{ij}\hat c^\dag_j
%DIFDELCMD <     ,\quad
%DIFDELCMD <     {U_S}^\dag H {U_S} = -H
%DIFDELCMD <     ,\quad
%DIFDELCMD <     {U_S} = {U_C}^* {U_T}^*
%DIFDELCMD < \end{align}%%%
\DIFdelend \DIFaddbegin \begin{align}
    \hat S \hat c_i \hat S^{-1} = (U_CU_T)_{ij}\hat c^\dag_j
    ,\quad
    {U_S}^\dag H {U_S} = -H
    ,\quad
    {U_S} = {U_C}^* {U_T}^*
    ,
\end{align}\DIFaddend 
in which, contrary to the other two symmetries, $U_S^2=\re^{\ri\alpha}\mathbb1$.
This leaves a phase ambiguity \DIFdelbegin \DIFdel{, which }\DIFdelend \DIFaddbegin \DIFadd{that }\DIFaddend is fixed by $U_S\rightarrow U_S\re^{\ri\alpha/2}$ such that the eigenvalues of the chiral operator are gauged to $\pm1$.

The established terminology allows to discuss the general symmetry classification of non-interacting systems.
We obtained so far
\begin{align}
    \begin{array}{l l l l }
        & T^{-1}HT = +H
        ,\quad
        & T = U_T\mathcal K
        ,\quad
        & U_T^*U_T = \pm\mathbb1,\\
& C^{-1}HC = -H
        ,\quad
        & C = U_C\mathcal K
        ,\quad
        & U_C^*U_C = \pm\mathbb1,\\
& S^{-1}HS = -H
        ,\quad
        & S = U_S = U_C^* U^*_T
        ,\qquad
        & U_S^2 = \mathbb1.
    \end{array}
    \label{eq:symmetry_real_space}
\end{align}
The set of symmetries $\hat T$, $\hat C$ and $\hat S$ is exhaustive and spans the ten possible families of Hamiltonians as presented in \cref{tab:symmetry_classes}~\cite{Chiu2016}.
Consider the case of time reversal invariance, there are three distinct classes:
(i) $H$ is not time reversal invariant, denoted by $T=0$, (ii) $H$ is time reversal invariant and $T$ squares to $+\mathbb1$ in which case we write $T=+$ and (iii) $H$ is time reversal invariant and $T$ squares to $-\mathbb1$, denoted by $T=-$.
There are three equivalent ways to characterize the behavior of $H$ under the particle hole symmetry.
Since $\hat S = \hat C\hat T$ in eight of the nine previous possibilities the outcome of chiral symmetry is implied.
However, the Hamiltonian can (or cannot) be chiral invariant in the absence of $C$ and $T$ symmetry, such that there are ten distinct symmetry classes.

\begin{table}
    \centering
    \addtolength{\tabcolsep}{0.25cm}
    \renewcommand{\arraystretch}{1.25} \begin{tabular}{l | c c c | l l l l l l l l l}
        \hline\hline
        Class   & $T$ & $C$ & $S$ & $0$ & $1$ & $2$ & $3$ & $4$ & $5$ & $6$ & $7$ & $8$\\
        \hline
        A       & $0$ & $0$ & $0$ & $\mathds Z$   &               & $\mathds Z$   &               & $\mathds Z$   &               & $\mathds Z$   &               & $\mathds Z$   \\
        AIII    & $0$ & $0$ & $+$ &               & $\mathds Z$   &               & $\mathds Z$   &               & $\mathds Z$   &               & $\mathds Z$   &               \\
        \hline
        AI      & $+$ & $0$ & $0$ & $\mathds Z$   &               &               &               & $2\mathds Z$  &               & $\mathds Z_2$ & $\mathds Z_2$ & $\mathds Z$   \\
        BDI     & $+$ & $+$ & $+$ & $\mathds Z_2$ & $\mathds Z$   &               &               &               & $2\mathds Z$  &               & $\mathds Z_2$ & $\mathds Z_2$ \\
        D       & $0$ & $+$ & $0$ & $\mathds Z_2$ & $\mathds Z_2$ & $\mathds Z$   &               &               &               & $2\mathds Z$  &               & $\mathds Z_2$ \\
        DIII    & $-$ & $+$ & $+$ &               & $\mathds Z_2$ & $\mathds Z_2$ & $\mathds Z$   &               &               &               & $2\mathds Z$  &               \\
        AII     & $-$ & $0$ & $0$ & $2\mathds Z$  &               & $\mathds Z_2$ & $\mathds Z_2$ & $\mathds Z$   &               &               &               & $2\mathds Z$  \\
        CII     & $-$ & $-$ & $+$ &               & $2\mathds Z$  &               & $\mathds Z_2$ & $\mathds Z_2$ & $\mathds Z$   &               &               &               \\
        C       & $0$ & $-$ & $0$ &               &               & $2\mathds Z$  &               & $\mathds Z_2$ & $\mathds Z_2$ & $\mathds Z$   &               &               \\
        CI      & $+$ & $-$ & $+$ &               &               &               & $2\mathds Z$  &               & $\mathds Z_2$ & $\mathds Z_2$ & $\mathds Z$   &               \\
        \hline\hline
    \end{tabular}
    \addtolength{\tabcolsep}{-0.25cm}
    \caption{The ten symmetry classes of topological insulators and superconductors, sorted by time reversal, particle-hole and chiral symmetry~\cite{Altland1997}, and to which values the unitary transformation of the symmetry squares to. A ``$0$'' indicates absence of the corresponding symmetry while $\pm$ represents the outcome of the squared symmetry transformation (see text). The sets $\mathds Z$, $\mathds Z_2$ and $2\mathds Z$ indicate the existence of topological insulators / superconductors for the corresponding Hamiltonian equivalence class defined in a given space dimension (presented from $0$ to $8$) and which values the topological invariant can assume (integer numbers $\mathds Z$, two distinct numbers $\mathds Z_2$ and even integer numbers $2\mathds Z$).}
    \label{tab:symmetry_classes}
\end{table}

Coming back to the adiabatic principle for gapped Hamiltonians, the topological distinction is defined through smooth deformations of the phase diagram by changing its parameters.
If two quantum systems can be smoothly deformed into each other without closing the gap, they are called topologically equivalent.
Trivial insulators (TRI) are then such phases which are adiabatically connected to an atomic insulator, fully characterized by independent and disconnected unit-cells.
All the other equivalence classes are by definition not adiabatically connected to the atomic insulator and thus called topological insulators (TOI)\footnote{The concepts illustrated in this section hold equally for Bogoliubov-de Gennes Hamiltonians, which describe the mean-field theory of a conventional superconductor. Such systems, however, do not make much contact to our published articles, which is why we decided to avoid their notion at this point. Instead, we refer the interested reader to~\cite{Chiu2016,Asboth2016} and the online course \cite{topocondmat}.}.
To use notation of the previous paragraphs, we decouple the space from the internal dimension, i.e. $H_{ij}=H_{q_1,q_2}(\bm x,\bm x')$ and consider translational invariant systems such that $H_{q_1,q_2}(\bm x,\bm x')=H_{q_1,q_2}(\bm x-\bm x')$.
It is then particularly convenient to use the Bloch Hamiltonian description (\DIFdelbegin \DIFdel{$\hat c_q(\bm x)=\sum_{\bm k}\exp(i\bm k\bm x)\hat c_q(\bm k)$}\DIFdelend \DIFaddbegin \DIFadd{$\hat c_{\bm x,q}=\sum_{\bm k}\exp(i\bm k\bm x)\hat c_{{\bm k},q}$}\DIFaddend ), such that the previous Hamiltonian results to
\DIFdelbegin %DIFDELCMD < \begin{align}
%DIFDELCMD <     \hat H = \sum_{\bm k\in {\rm BZ}} \hat c^\dag_q(\bm k) H_{qq'}(\bm k)\hat c^\pdag_{q'}(\bm k),
%DIFDELCMD < \end{align}%%%
\DIFdelend \DIFaddbegin \begin{align}
    \hat H = \sum_{\bm k\in {\rm BZ}} \hat c^\dag_{{\bm k},q} H_{qq'}(\bm k)\hat c^\pdag_{{\bm k},q'},
\end{align}\DIFaddend 
with the crystal momentum $\bm k$ defined in the first Brillouin zone (BZ).
By close inspection of \cref{eq:symmetry_real_space}, the \DIFdelbegin \DIFdel{action }\DIFdelend \DIFaddbegin \DIFadd{actions }\DIFaddend of the symmetries on the Bloch Hamiltonian \DIFdelbegin \DIFdel{is }\DIFdelend \DIFaddbegin \DIFadd{are }\DIFaddend given by
\begin{align}
    \begin{array}{l l l l }
        & TH(\bm k)T^{-1} = +H(-\bm k)
        ,\quad
        & T = U_T\mathcal K
        ,\quad
        & U_T^*U_T = \pm\mathbb1,\\
& CH(\bm k)C^{-1} = -H(-\bm k)
        ,\quad
        & C = U_C\mathcal K
        ,\quad
        & U_C^*U_C = \pm\mathbb1,\\
& SH(\bm k)S^{-1} = -H(\bm k)
        ,\quad
        & S = U_S = U_C^* U^*_T
        ,\qquad
        & U_S^2 = \mathbb1.
    \end{array}
    \label{eq:symmetry_momentum_space}
\end{align}
The result of the classification is called the periodic table of topological insulators (and superconductors)~\cite{Kitaev2009,Qi2008,Ryu2010,Schnyder2008} and presented in \cref{tab:symmetry_classes}.

The table is based on a classification of the vector space of the matrix $\exp(-\ri t H)$, constrained by the presence (or absence) of the symmetries defined by \cref{eq:symmetry_momentum_space}.
For instance, $T=C=S=0$ imposes no additional constraint and the classifying space of the symmetry space $A$ is that of generic unitary matrices of size $N$, i.e. $U(N)$.
This connects to a fundamental work on symmetric space from 1926 by Élie Cartan, and the ten Hamiltonian equivalence classes are in one-to-one correspondence with the ten large families of symmetric spaces~\cite{Heinzner2005}.
Ultimately, it unveils a diagonal structure of the topological index in \cref{tab:symmetry_classes}, in correspondence to the so-called ``Bott clock'', representing a recursive and periodic sequence for the symmetry classes.
In order to see this structure, the table is ordered in a particular manner: the first two rows are spanned by the two classes spanned by complex matrices, whereas the remaining rows represent the spaces spanned by real matrices.
One period of the sequence for the real matrices corresponds to AI$\rightarrow$BDI$\rightarrow$D$\rightarrow$DIII$\rightarrow$AII$\rightarrow$CII$\rightarrow$C$\rightarrow$CI$\rightarrow$AI, in which ``$\rightarrow$'' increases the \DIFdelbegin \DIFdel{spacial }\DIFdelend \DIFaddbegin \DIFadd{spatial }\DIFaddend dimension by one.
The complex matrices form a self-contained cycle A$\rightarrow$AIII$\rightarrow$A.
For a more detailed discussion of the classification of topological insulators and general formulas of their topological index, we refer to a selection of reviews~\cite{Hasan2010,Chiu2016,Cooper2019}.

Although the prescription in terms of Bloch bands breaks down for interacting systems in general, the corresponding many-body definition of topological invariants, e.g. the winding number
\begin{align}
    \mathcal W = \frac{1}{4\pi\ri} \tr\int\rd k \sigma_z g^{-1}\partial_k g,
    \label{eq:winding_number_greens}
\end{align}
extends well beyond the single-particle theory employed in this chapter~\cite{Gurarie2011,Manmana2012}.
Note that it can become arbitrarily complicated to give an operational recipe to compute the invariants in interacting systems.
\DIFdelbegin %DIFDELCMD < 

%DIFDELCMD < %%%
\DIFdel{Our work in \mbox{%DIFAUXCMD
\cref{mcd1} }\hspace{0pt}%DIFAUXCMD
provides a simple tool }\DIFdelend \DIFaddbegin \DIFadd{In \mbox{%DIFAUXCMD
\cref{mcd1} }\hspace{0pt}%DIFAUXCMD
we propose a simple protocol }\DIFaddend to probe the winding number of interacting quasi-one-dimensional systems with chiral symmetry.
The scheme relies on a dynamical measurement of the density distribution followed by a local quench, which can be readily implemented in many platforms of synthetic quantum matter.

Now that the theoretical prerequisites have been detailed, we proceed by presenting our works.
\clearpage{}

\clearpage{}

\begin{partbacktext}
    \part{Results}
    \label{part:results}
\end{partbacktext}
\clearpage{}
\clearpage{}

In this part we present the acquired results \DIFdelbegin \DIFdel{during my PhD }\DIFdelend in a thematic manner.
\DIFdelbegin \DIFdel{All in all, }\DIFdelend I was strongly participated in the publication of five \DIFdelbegin \DIFdel{articles, of which four are }\DIFdelend peer-reviewed \DIFdelbegin \DIFdel{and one is still in the peer-review process}\DIFdelend \DIFaddbegin \DIFadd{articles.
In the following, I list my original contributions}\DIFaddend .

\begin{enumerate}
    \item{\underline{A. Haller}, M. Rizzi and M. Filippone {\it ``Drude weight increase by orbital and repulsive interactions in fermionic ladders''}, published in Phys. Rev. Research 2, 023058 (2020):
    The work was inspired by a discussion between Dr. M. Filippone and Prof. Dr. M. Rizzi. The exact solution of the Creutz model was developed jointly by Prof. Dr. M. Rizzi and me, the effective low-energy description of the Creutz model, the perturbation theory and all numerical simulations were performed by me. The initial draft was prepared by me, which was then extended by all authors.}
    \DIFdelbegin %DIFDELCMD < \item{\underline{A. Haller}, M. Rizzi and M. Burrello, {\it ``The resonant state at filling factor $\nu=1/2$ in chiral fermionic ladders''}, published in New J. Phys. 20, 053007 (2018): The initial observation of effects then linked to the resonant state occurred to me while working on a different project during my M. Sc. studies which was also supervised by Prof. Dr. Matteo Rizzi. All numerical simulations and the data analysis presented in the article were performed during my PhD period by me. Prof. Dr. M. Burrello contributed the second-order RG analysis and prepared the initial draft of the article, which was then extended by all authors.}
%DIFDELCMD <     \item{\underline{A. Haller}, A. S. Matsoukas-Roubeas, Y. Pan, M. Rizzi and M. Burrello, {\it ``Exploring helical phases of matter in bosonic ladders''}, presently in Phys. Rev. Research review process:
%DIFDELCMD <     The article resulted from a follow-up study of our work ``The resonant state at filling factor $\nu=1/2$ in chiral fermionic ladders'' by A. Matsoukas-Roubeas in the scope of his M. Sc. studies supervised by Prof. Dr. M. Burrello. The first version of the RG equations of the integer resonance were derived by A. Matsoukas-Roubeas and Prof. Dr. M. Burrello. Y. Pan aided in the derivation of the low-energy Hamiltonian. The DMRG simulations, the numerical RG solutions and all of the data analysis were performed by me. The manuscript draft was set up by me, which was then extended by Prof. Dr. M. Burrello and Prof. Dr. M. Rizzi.}
%DIFDELCMD <     \item{P. Schmoll, \underline{A. Haller}, M. Rizzi and R. Orús, {\it ``Quantum criticality on a chiral ladder: An SU$(2)$ infinite density matrix renormalization group study''}, published in Phys. Rev. B 99, 205121 (2019):
%DIFDELCMD <     The article was initiated by Prof. Dr. R. Orús in order to benchmark and test the SU$(2)$ framework developed by my PhD colleague P. Schmoll. Our joint work resulted from regular discussions between P. Schmoll and me, during which we developed the exact diagonalization analysis. All the numerical simulations and most parts of the data analysis was performed by P. Schmoll. Part of the numerical results obtained by his SU$(2)$ algorithm were benchmarked against my U$(1)$ MPS algorithm. The block spin coarse graining was performed by Prof. Dr. M. Rizzi. The development of the effective-low energy Hamiltonian using Abelian bosonization and the description of the gapping mechanism was performed by me. Prof. Dr. R. Orús prepared the draft of the manuscript, which was extended by all authors.}
%DIFDELCMD <     %%%
\DIFdelend \DIFaddbegin \item{\underline{A. Haller}, M. Rizzi and M. Burrello, {\it ``The resonant state at filling factor $\nu=1/2$ in chiral fermionic ladders''}, published in New J. Phys. 20, 053007 (2018): The initial observation of effects then linked to the resonant state occurred to me while working on a different project during my M. Sc. studies, supervised by Prof. Dr. Matteo Rizzi. All numerical simulations and the data analysis presented in the article were performed during the PhD by me. Prof. Dr. M. Burrello contributed the second-order RG analysis and prepared the initial draft of the article, which was then extended by all authors.}
    \item{\underline{A. Haller}, A. S. Matsoukas-Roubeas, Y. Pan, M. Rizzi and M. Burrello, {\it ``Exploring helical phases of matter in bosonic ladders''}, accepted in Phys. Rev. Research:
    The article resulted from a follow-up study of our work ``The resonant state at filling factor $\nu=1/2$ in chiral fermionic ladders'' by A. Matsoukas-Roubeas in the scope of his M. Sc. studies supervised by Prof. Dr. M. Burrello. The first version of the RG equations of the integer resonance were derived by A. Matsoukas-Roubeas and Prof. Dr. M. Burrello. Y. Pan aided in the derivation of the low-energy Hamiltonian. The DMRG simulations, the numerical RG solutions and all of the data analysis were performed by me. The manuscript draft was set up by me, which was then extended by Prof. Dr. M. Burrello and Prof. Dr. M. Rizzi.}
    \item{P. Schmoll, \underline{A. Haller}, M. Rizzi and R. Orús, {\it ``Quantum criticality on a chiral ladder: An SU$(2)$ infinite density matrix renormalization group study''}, published in Phys. Rev. B 99, 205121 (2019):
    The article was initiated by Prof. Dr. R. Orús in order to benchmark and test the SU$(2)$ framework developed by my PhD colleague P. Schmoll. Our joint work resulted from regular discussions between P. Schmoll and me, during which we developed the exact diagonalization analysis. All the numerical simulations and most parts of the data analysis was performed by P. Schmoll. Part of the numerical results obtained by his SU$(2)$ algorithm were benchmarked against my U$(1)$ MPS algorithm. The block spin coarse graining was performed by Prof. Dr. M. Rizzi. The development of the effective-low energy Hamiltonian using abelian bosonization and the description of the gapping mechanism was performed by me. Prof. Dr. R. Orús prepared the draft of the manuscript, which was extended by all authors.}
    \DIFaddend \item{\underline{A. Haller}, P. Massignan and M. Rizzi, {\it ``Detecting topology through dynamics in interacting fermionic wires''}, published in Phys. Rev. Research 2, 033200 (2020):
    The work resulted from a discussion between Prof. Dr. P. Massignan and Prof. Dr. M. Rizzi. The numerical simulations, data analysis and the derivation of the effective spin model was performed by me. The equality between the winding number and the mean chiral displacement was derived by all authors with equal contribution. The blueprint for the experimental realization resulted from discussions between Prof. Dr. M. Aidelsburger and Prof. Dr. P. Massignan. I prepared the initial draft, which was then extended by all authors.}
\end{enumerate}

\includepdf[pages={1-},
addtotoc={
    1,chapter,1,Drude weight increase by orbital and repulsive interactions in fermionic ladders,drude_increased1,
1,section,1,Introduction,drude_increased2,
2,section,1,Weak Drude weight renormalization in one dimension,drude_increased3,
3,section,1,Model,drude_increased4,
4,section,1,Perturbation theory,drude_increased5,
6,section,1,Bosonization and connection to Quantum Hall systems,drude_increased6,
7,section,1,Discussion and conclusion,drude_increased7
}
]{Library/drude_increased.pdf}

\includepdf[pages={2-},
addtotoc={
    2,chapter,1,The resonant state at filling factor \texorpdfstring{$\nu=1/2$}{nu=1/2} in chiral fermionic ladders,one_half1,
2,section,1,Introduction,one_half2,
3,section,1,The model,one_half3,
4,section,1,RG analysis,one_half4,
5,section,1,The entanglement properties,one_half5,
6,section,1,The correlations,one_half6,
7,section,1,Conclusions,one_half7
}
]{Library/pretopological_states_fermions.pdf}

\includepdf[pages={1-},
addtotoc={
    1,chapter,1,Exploring helical phases of matter in bosonic ladders,integer1,
2,section,1,The model,integer2,
2,subsection,1,The single-particle physics,integer3,
3,subsection,1,Introducing the interactions,integer4,
3,section,1,Effective low-energy description of the model,integer5,
4,section,1,Renormalization group analysis,integer6,
7,section,1,Numerical results,integer7,
7,subsection,1,Chiral current,integer8,
9,subsection,1,Fluctuations and estimate of the Luttinger parameters,integer9,
10,subsection,1,Correlations,integer10,
11,subsection,1,Dynamics and velocities,integer11,
12,section,1,Conclusions and perspectives,integer12
}
]{Library/pretopological_states_bosons.pdf}

\includepdf[pages={1-},
addtotoc={
    1,chapter,1,Quantum criticality on a chiral ladder: An SU$(2)$ infinite density matrix renormalization group study,chiral1,
1,section,1,Introduction,chiral2,
2,section,1,Chiral ladder,chiral3,
2,subsection,1,Model,chiral4,
3,subsection,1,First intuition with Kadanoff coarse graining,chiral5,
3,subsection,1,Exact diagonalization of small systems,chiral6,
4,subsection,1,Bosonization,chiral7,
6,section,1,Methods,chiral8,
7,section,1,Results,chiral9,
7,subsection,1,Energy convergence,chiral10,
7,subsection,1,Entanglement,chiral11,
8,subsection,1,Correlation functions,chiral12,
9,subsection,1,Entanglement spectrum,chiral13,
9,section,1,Conclusions,chiral14
}
]{Library/spin_chain.pdf}

\includepdf[pages={1-},
addtotoc={
    1,chapter,1,Detecting topology through dynamics in interacting fermionic wires,mcd1,
2,section,1,Mean chiral displacement as a topological marker,mcd2,
2,section,1,Model: Interacting SSH chains,mcd3,
3,section,1,The noninteracting case,mcd4,
3,section,1,The short-ranged case,mcd5,
4,section,1,Effective interacting spin model,mcd6,
4,section,1,The long-ranged case,mcd7,
4,section,1,Experimental blueprint for $\MH_{\rm lr}$,mcd8,
4,section,1,Conclusions,mcd9
}
]{Library/mcd.pdf}
\clearpage{}

\clearpage{}\begin{partbacktext}
    \part{Summary}
\end{partbacktext}
\clearpage{}
\clearpage{}\chapter*{Conclusions and Perspectives}
\addcontentsline{toc}{chapter}{Conclusions and Perspectives}
In general, there is a strong experimental and fundamental interest in the coherent transport properties of correlated systems.
The ability to determine the quantitative behavior of the Drude weight in this context is crucial, as it controls the magnitude of persistent currents.
Our results shed new light on an intriguing effect visible in such strongly correlated quantum systems: We demonstrated how repulsive (attractive) interactions counterintuitively increase (decrease) the mobility of interacting fermions in the presence of orbital effects.
Our statements rely on perturbation theory and bosonization of the interacting Creutz model, for which we focused on the situation with two Fermi points.
We have shown that the Drude weight changes linearly in the coupling parameters of two \DIFdelbegin \DIFdel{orbital- selective }\DIFdelend \DIFaddbegin \DIFadd{orbital-selective }\DIFaddend nearest-neighbor interactions.
Our study clarifies and generalizes the possibility of tuning the Drude weight observed in a previous study using \DIFdelbegin \DIFdel{MPS }\DIFdelend \DIFaddbegin \DIFadd{matrix product states }\DIFaddend and second-order perturbation theory.
The new predictions are furthermore universal: we demonstrated that the counterintuitive renormalization is linked to the topological protection of edge states in the quantum Hall effect, in which the only possible effect for interactions is limited to forward-scattering.
Interesting future directions regard the known zero-frequency anomaly in transport coefficients at finite temperatures: it is known that integrable systems show ballistic finite-temperature transport, despite the presence of strong interactions.
It would be interesting to investigate if the observed effect of the strong Drude renormalization extends to the case of finite temperatures, even if the underlying model breaks integrability.

Particles hopping in ladder geometries and subject to artificial magnetic fluxes \DIFdelbegin \DIFdel{are known to }\DIFdelend generate rich phase diagrams.
For commensurate values of the ratio between the number of fluxes and the number of atoms, helical states with a net chiral current may appear.
The simplest and most evident example of these helical states are the non-interacting Meissner phase for bosons and the helical state at flux $\chi = 2k_F$ for fermions.
It is known, however, that additional strongly correlated helical states originate for suitable values of the filling factor $\nu$ and suitable interactions.
Due to the analogy drawn by the coupled wire construction, some of these helical liquids have an \DIFdelbegin \DIFdel{anology }\DIFdelend \DIFaddbegin \DIFadd{analogy }\DIFaddend to the fractional quantum Hall phases encountered in 2D systems.
The study of fermionic fractional quantum Hall states with even denominators has always been more challenging than the odd denominator cases.
Here we analyzed the resonant state at $\nu=1/2$ in a spin 1/2 fermionic chain.
With a \DIFdelbegin \DIFdel{RG }\DIFdelend \DIFaddbegin \DIFadd{renormalization group }\DIFaddend analysis and extensive \DIFdelbegin \DIFdel{MPS }\DIFdelend \DIFaddbegin \DIFadd{matrix product state }\DIFaddend calculations, we brought compelling evidence that this state is related to the 1D limit of the $K=8$ \DIFdelbegin \DIFdel{FQH }\DIFdelend \DIFaddbegin \DIFadd{fractional quantum Hall }\DIFaddend state and it is generated by a gap in the spin sector of the model.
Our results are relevant for both ultracold atom ladders in a synthetic dimension and nanowires with strong spin–orbit coupling.
In the first case the required interactions may be achieved by exploiting dipolar atoms, like Dy, or orbital Feshbach resonances, in the second case electron–electron interactions play a relevant role in the experimental results and our tight-binding model can describe their interplay with the Zeeman splitting.
We then applied the analytical treatment to that of a two-leg ladder of hardcore bosons at $\nu=1$, the straightforward follow-up of our fermionic study.
We studied its signatures in terms of correlation functions, fluctuations and dynamical evolution.
Our \DIFdelbegin \DIFdel{MPS }\DIFdelend \DIFaddbegin \DIFadd{matrix product state }\DIFaddend simulations show that the strongly correlated $\nu=1$ helical phase can be accessed in systems with contact interactions only, similarly to the bosonic Laughlin-like state at filling factor $\nu = 1/2$.
With respect to the pretopological Laughlin-like state, however, the chiral current and gap signatures we observe are considerably stronger for a broad range of parameters when comparing systems with the same particle density, repulsive interaction and interleg tunneling.
Comparing our findings with the analogous fermionic systems, we also observe that the strongly-correlated phases of bosons can be reached through interactions with a shorter range than their fermionic counterpart, as common for several fractional quantum Hall states.
This intuitively explains also why the signals we detect for bosons at filling $\nu = 1$ are considerably larger than their fermionic counterpart.
In this respect, we find that bosonic systems are more suitable for the experimental characterization of these helical and strongly correlated phases of matter.
The onset of the helical phase is caused by a particular mechanism that allows the interaction to act in the spin sector \DIFdelbegin \DIFdel{of the theory}\DIFdelend \DIFaddbegin \DIFadd{only}\DIFaddend .
This is unusual in case of repulsive interactions only, which normally cause only irrelevant corrections in this sector and lead to relevant operators in the charge sector, instead.
This happens for bosons at the integer filling factor $\nu=1$ and it is analogous to the pairing mechanism determining the appearance of the pretopological $K = 8$ phase in fermionic systems at $\nu=1/2$.
We stress that engineering (strong) pairing mechanisms analogous to the effective interaction is recently at the focus of several proposals, due to its potential for the development of topological and strongly-correlated paired phases of matter.
The combination of artificial gauge potentials, equivalent to a spin-orbit coupling, and a commensurate particle density, can be adopted also in this framework.

We demonstrated that measuring the mean chiral displacement of localized excitations probes the topological properties of chiral 1D many-body systems and signals the presence of symmetry-broken phases.
Furthermore, we presented a readily-feasible experimental setup where it is straightforward to realize and detect interaction-driven transitions to both symmetry-protected chiral order and symmetry-broken long-range order.
It would be interesting to achieve a deeper analytic understanding of the many-body part of the \DIFdelbegin \DIFdel{MCD }\DIFdelend \DIFaddbegin \DIFadd{mean chiral displacement }\DIFaddend for other models, which could lead to a novel way of characterizing phases featuring spontaneous symmetry breaking.
It remains an open question if the \DIFdelbegin \DIFdel{MCD }\DIFdelend \DIFaddbegin \DIFadd{mean chiral displacement }\DIFaddend can distinguish all phases of even richer models.
This could be tested by including long-range hopping processes to the kinetic part of the Hamiltonian and more exotic four-body interactions.
Further intriguing perspectives of this work are a generalization of our protocol to higher dimensions, to other symmetry-protected topological phases and an investigation of the effects of disorder, losses, and temperature on top of interactions.

Synthetic quantum matter will certainly unveil more of the fundamental secrets of our universe, \DIFdelbegin \DIFdel{and }\DIFdelend \DIFaddbegin \DIFadd{they }\DIFaddend might even realize remarkable technological innovations in the years to come\DIFdelbegin \DIFdel{.
Being part of this community makes me proud}\DIFdelend , and I hope that our works provide a \DIFdelbegin \DIFdel{small step in these }\DIFdelend \DIFaddbegin \DIFadd{step towards those }\DIFaddend directions.
\clearpage{}






\printbibliography

\end{document}
