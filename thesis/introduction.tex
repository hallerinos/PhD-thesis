%%%%%%%%%%%%%%%%%%%%%%%%%%%%%%
%!TEX root = thesis.tex
%!TeX spellcheck = en-US,en-DE
%%%%%%%%%%%%%%%%%%%%%%%%%%%%%%
%
\chapter*{Introduction}
\addcontentsline{toc}{chapter}{Introduction}
%
As an undergraduate student studying elementary physics, I always thought of quantum particles as plane waves living in the differential-geometric world of ``first quan\-ti\-zation'' which (at least for me) was both beautiful and frustrating at the same time as the task of solving the most simple problems may become quite involved, if not even impossible.
The modern approach is not a straightforward attempt to solve the real-space wave function by integrating a given Schrödinger equation: It rather sticks to an alternative representation in the algebraic world of ``second quantization'' where the action of operators dictates all physical consequences.
I must admit that I find the nomenclature of ``first'' and ``second'' somewhat confusing, as there is no such thing as two consecutive ways of quantization -- it's just two alternative and consistent ways of describing the same theory.
Ultimately, I've learned to accept this issue as a result of chronology: the original formulation of quantum mechanics is commonly called first quantization, in which the (motion of the) particle is quantized and possible electromagnetic fields or potentials are considered classical, whereas quantized fields have been formulated in the language of second quantization.
However, as we will see soon, the advantage of second quantization manifests itself in a simpler and more efficient way to describe many-body systems such that its development can be seen as the first major cornerstone in the emergence of quantum field theory.

Independent of the formulation, be it first or second, all quantum theories require certain basic concepts:
all quantum states are represented by state vectors $\{|q\rangle\}$ forming a complete basis of the Hilbert space and observables are defined through Hermitian operators acting on that space.
The states are given through a set of good quantum numbers $q$, e.g. for the electron of hydrogen $(n,l,m)$ associated with the total energy, angular momentum and its projection along the primary axis.

In \cref{part:theory}, I review these concepts in more detail.
The first sections, i.e. \cref{sec:creation_and_annihilation_operators,sec:representation_of_generic_operators,sec:periodic_potentials,sec:tight_binding_systems}, are aimed to guide undergraduate students which are new in the field of condensed matter.
To become familar with the basics, I review the tight binding approximation in \cref{sec:tight_binding_systems}, derived from the Kronig-Penney model in the strong potential limit (see \cref{sec:periodic_potentials}).
I then illustrate the treatment of interactions on top of a quadratic/non-interacting Hamiltonian which is a crucial tool in the understanding of quantum matter.
This becomes particularly easy in the concept of Luttinger liquids (see \cref{sec:tomonaga_LL}) as a first example of a quantum field theory satisfying the algebra of a quantum harmonic oscillator.
The key in Luttinger liquids is that the effective theory describing intrinsically interacting microscopic models remains quadratic at the operator level, and as such effectively non-interacting.
I then use the formalism of path integrals (see \cref{sec:properties_of_real_scalar_fields_and_their_correlations}) to derive analytic expressions of the correlation functions.
A common interaction which potentially melts the Luttinger liquid phase is a sine-Gordon term, which I derive in \cref{sec:LL_with_spin}.
In order to treat such higher-order interactions analytically, I introduce the concept of renormalization group theory (see \cref{sec:renormalization_group_theory}), which provides a platform to study if and when the Luttinger liquid basis must be exchanged in favor of the collective many-body basis dictated by the interactions.
One major drawback is that renormalization group theory relies on many simplifications along the way to make the calculation amendable, and microscopic details of the model are often lost in the approximations.
These details are exchanged in favor of ``universality'', an effective theory which describes the physical properties of a family of different microscopic models at long scales.
In this framework, it is thus not intended to provide quantitative predictions for the microscopic theories, such that numerical tools become essential.

A brute force calculation of a full solution scales exponentially with the number of constituents, such that exact diagonalization of interacting models is limited to small system sizes (e.g. $\sim 20$ sites for interacting spin-1/2 models in typical studies).
This stresses the need for more efficient numerical techniques in the study of non-perturbative regimes which cannot be amended analytically.
In this thesis, I mainly use matrix product states (MPS), and I illustrate the main concepts in \cref{ch:matrix_product_states}.
In \cref{sec:tensor_networks,sec:reduced_density_matrix_and_renyi_entropy}, I begin with a general introduction into the field of tensor networks and MPS.
A general discussion of scaling relations of the entanglement entropy is used to explain the effectivity of the MPS Ansatz (see \cref{sec:scaling_relations_of_the_entanglement_entropy}).
As a natural extension to MPS, I then introduce matrix product operators (MPO) in \cref{sec:matrix_product_operators} which are most useful to express the Hamiltonian, and to calculate observables in general.
After the preliminary introductions, I introduce the basic strategy to approximate ground states based on an MPS Ansatz in \cref{sec:variational_ground_state_search}.
This algorithm is quite similar to the (imaginary) time evolution presented in \cref{sec:time_evolution}, which can also be used to simulate finite temperature systems, presented in \cref{sec:finite_temperature}.
I conclude the chapter by a basic introduction into symmetry invariant MPS formulations, discussed in \cref{sec:exploiting_symmetries}.

Perhaps the most interesting behavior of quantum many-body physics is that of so-called emergent phenomena -- e.g. the appearance of quasi-particles which extend the canonical statistical properties of fermions and bosons in low dimensions, or the presence of robust quasi-particles at the boundaries of a topological insulator.
For a basic overview of this topic, I review the classification of non-interacting topological matter in \cref{ch:topological_phases_of_matter}.
The key concept for the full classification is based on topological band theory, presented in \cref{sec:topological_band_theory}, a genuinely non-interacting formulation of symmetry protected topological phases of matter.
The notion of the topological index arises from the Berry phase and the Berry connection, and I recap the original paper.
As an example, the Su-Schrieffer-Heeger chain is then explored in \cref{sec:the_SSH_chain} to apply the introduced concepts.
I conclude the third chapter by explaining the classification of topological insulators and superconductors in \cref{sec:Periodic_table_of_topological_insulators_and_superconductors}.

In \cref{part:results}, I summarize my original contributions to our published works, which are then presented in a thematic manner.
The flexible tuning of transport properties, i.e. the Drude weight, is presented in \cref{drude_increased1}.
The next three works presented in \cref{one_half1,integer1,chiral1} regard the emergence and understanding of strongly correlated helical phases of matter.
In \cref{mcd1}, we propose a dynamical experiment based on the mean chiral displacement to read-out the topological invariant in one-dimensional topological phases protected by chiral symmetry.
This thesis is concluded by a summary and perspectives about interesting future directions.
