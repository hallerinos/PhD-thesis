\usepackage{tikz}
\usepackage{tikz-3dplot}
\usetikzlibrary{shadings,decorations.markings,arrows.meta,quotes,angles,snakes,decorations}
\usepackage{xcolor}
\usepackage{amsmath}
\usepackage{marvosym}

\tikzset{
    %Define standard arrow tip
    >=stealth'
}

\tikzset{
    tst/.style={
      thin, opacity=0.5,
      dashed
    },
    my-arc/.style={
      start angle=0, end angle=360,radius=0.4
    }
}

\tikzset{
    ->-/.style={
        decoration={markings, mark=at position #1 with {\arrow{>}}},
        postaction={decorate}
    },
    scatter/.style={decorate, draw=black,
        decoration={complete sines,amplitude=8pt, segment length=11pt}}
}

\definecolor{blue}{RGB}{92, 96,176}
\definecolor{red}{RGB}{223, 99,140}
\definecolor{green}{RGB}{83,192,97}
\definecolor{yellow}{RGB}{253,216,112}
\definecolor{gray}{RGB}{109,109,109}
\definecolor{lightgray}{RGB}{200,200,200}
\definecolor{orange}{RGB}{239,142,79}

% \newcommand{\bbox}[1]{%
%   \color{red!50}\rlap{\fbox{$\phantom{#1}$}}%
%   \color{black}#1%
% }

\def\cTens#1#2#3 {
    \begin{scope}[shift={#3}]
        	\fill[#1] (0,0) circle (#2);
        	\clip (0,0) circle (#2);
	\shade[outer color=#1, inner color=#1!70] (-#2,#2) circle (1.75*#2);
	\draw[very thin] (0,0) circle (#2);
		\end{scope}
}

\def\rTens#1#2#3#4 {
	\begin{scope}[shift={#4}]
	        	\fill[#1] (-#2/2,-#3/2) rectangle (#2/2,#3/2);
	        	\clip (-#2/2,-#3/2) rectangle (#2/2,#3/2);
		\shade[outer color=#1, inner color=#1!70] (-1.41*#2/2,1.41*#3/2)  circle (1.3*#3/2+1.3*#2/2);
		\draw[very thin] (-#2/2,-#3/2) rectangle (#2/2,#3/2);
	\end{scope}
}

\def\tTens#1#2#3#4 {
	\begin{scope}[shift={#4}]
	        	\fill[#1] (-#2/2,-#3/2) -- (#2/2,-#3/2) -- (0,#3/2) -- (-#2/2,-#3/2);
	        	\clip (-#2/2,-#3/2) -- (#2/2,-#3/2) -- (0,#3/2) -- (-#2/2,-#3/2);
		\shade[outer color=#1, inner color=#1!70] (-1.41*#2/2,1.41*#3/2)  circle (1.3*#3/2+1.3*#2/2);
		\draw[very thin] (-#2/2,-#3/2) -- (#2/2,-#3/2) -- (0,#3/2) -- (-#2/2,-#3/2);
	\end{scope}
}

\def\oTens#1#2#3 {
    \begin{scope}[shift={#3}]
        	\fill[#1] (0,0) circle (#2);
        	\clip (0,0) circle (#2);
	\shade[outer color=#1, inner color=#1!70] (-#2,#2) circle (1.75*#2);
	\draw[very thin] (0,0) circle (#2);
		\end{scope}
}
