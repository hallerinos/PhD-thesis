%%%%%%%%%%%%%%%%%%%%%%%
%!TEX root = thesis.tex
%%%%%%%%%%%%%%%%%%%%%%%
%
\chapter*{Zusammenfassung}
\addcontentsline{toc}{chapter}{Zusammenfassung}
%
In general, there is a strong experimental and fundamental interest in the coherent transport properties of correlated systems.
The ability to determine the quantitative behavior of the Drude weight in this context is crucial, as it controls the magnitude of persistent currents.
Our results shed new light on an intriguing effect visible in such strongly correlated quantum systems: We demonstrated how repulsive (attractive) interactions counterintuitively increase (decrease) the mobility of interacting fermions in the presence of orbital effects.
Our statements rely on perturbation theory and bosonization of the interacting Creutz model, for which we focused on the situation with two Fermi points.
We have shown that the Drude weight changes linearly in the coupling parameters of two orbital- selective nearest-neighbor interactions.
Our study clarifies and generalizes the possibility of tuning the Drude weight observed in a previous study using MPS and second-order perturbation theory.
The new predictions are furthermore universal: we demonstrated that the counterintuitive renormalization is linked to the topological protection of edge states in the quantum Hall effect, in which the only possible effect for interactions is limited to forward-scattering.
Interesting future directions regard the known zero-frequency anomaly in transport coefficients at finite temperatures: it is known that integrable systems show ballistic finite-temperature transport, despite the presence of strong interactions.
It would be interesting to investigate if the observed effect of the strong Drude renormalization extends to the case of finite temperatures, even if the underlying model breaks integrability.

Particles hopping in ladder geometries and subject to artificial magnetic fluxes are known to generate rich phase diagrams.
For commensurate values of the ratio between the number of fluxes and the number of atoms, helical states with a net chiral current may appear.
The simplest and most evident example of these helical states are the non-interacting Meissner phase for bosons and the helical state at flux $\chi = 2k_F$ for fermions.
It is known, however, that additional strongly correlated helical states originate for suitable values of the filling factor $\nu$ and suitable interactions.
Due to the analogy drawn by the coupled wire construction, some of these helical liquids have an anology to the fractional Quantum Hall (FQH) phases encountered in 2D systems.
The study of fermionic FQH state with even denominators has always been more challenging than the odd denominator cases.
Here we analyzed the resonant state at $\nu=1/2$ in a spin 1/2 fermionic chain.
With a RG analysis and extensive MPS calculations, we brought compelling evidence that this state is related to the 1D limit of the $K=8$ FQH state and it is generated by a gap in the spin sector of the model.
Our results are relevant for both ultracold atom ladders in a synthetic dimension and nanowires with strong spin–orbit coupling.
In the first case the required interactions may be achieved by exploiting dipolar atoms, like Dy, or orbital Feshbach resonances, in the second case electron–electron interactions play a relevant role in the experimental results and our tight-binding model can describe their interplay with the Zeeman splitting.
We then applied the analytical treatment to that of a two-leg ladder of hardcore bosons at $\nu=1$, the straightforward follow-up of our fermionic study.
We studied its signatures in terms of correlation functions, fluctuations and dynamical evolution.
Our MPS simulations show that the strongly correlated $\nu=1$ helical phase can be accessed in systems with contact interactions only, similarly to the bosonic Laughlin-like state at filling factor $\nu = 1/2$.
With respect to the pretopological Laughlin-like state, however, the chiral current and gap signatures we observe are considerably stronger for a broad range of parameters when comparing systems with the same particle density, repulsive interaction and interleg tunneling.
Comparing our findings with the analogous fermionic systems, we also observe that the strongly-correlated phases of bosons can be reached through interactions with a shorter range than their fermionic counterpart, as common for several fractional quantum Hall states.
This intuitively explains also why the signals we detect for bosons at filling $\nu = 1$ are considerably larger than their fermionic counterpart.
In this respect, we find that bosonic systems are more suitable for the experimental characterization of these helical and strongly correlated phases of matter.
The onset of the helical phase is caused by a particular mechanism that allows the interaction to act in the spin sector of the theory.
This is unusual in case of repulsive interactions only, which normally cause only irrelevant corrections in this sector and lead to relevant operators in the charge sector, instead.
This happens for bosons at the integer filling factor $\nu=1$ and it is analogous to the pairing mechanism determining the appearance of the pretopological $K = 8$ phase in fermionic systems at $\nu=1/2$.
We stress that engineering (strong) pairing mechanisms analogous to the effective interaction is recently at the focus of several proposals, due to its potential for the develop ment of topological and strongly-correlated paired phases of matter.
The combination of artificial gauge potentials, equivalent to a spin-orbit coupling, and a commensurate particle density, can be adopted also in this framework.

We demonstrated that measuring the mean chiral displacement of localized excitations probes the topological properties of chiral 1D many-body systems and signals the presence of symmetry-broken phases.
Furthermore, we presented a readily-feasible experimental setup where it is straightforward to realize and detect interaction-driven transitions to both symmetry-protected chiral order and symmetry-broken long-range order.
It would be interesting to achieve a deeper analytic understanding of the many-body part of the MCD for other models, which could lead to a novel way of characterizing phases featuring spontaneous symmetry breaking.
It remains an open question if the MCD can distinguish all phases of even richer models.
This could be tested by including long-range hopping processes to the kinetic part of the Hamiltonian and more exotic four-body interactions.
Further intriguing perspectives of this work are a generalization of our protocol to higher dimensions, to other symmetry-protected topological phases and an investigation of the effects of disorder, losses, and temperature on top of interactions.
