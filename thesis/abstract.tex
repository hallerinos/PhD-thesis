%%%%%%%%%%%%%%%%%%%%%%%%%%%%%%
%!TEX root = thesis.tex
%!TeX spellcheck = en-US,en-DE
%%%%%%%%%%%%%%%%%%%%%%%%%%%%%%
%
\chapter*{Abstract}
\addcontentsline{toc}{chapter}{Abstract}
% \thispagestyle{empty}
%
In this thesis, we investigate the effects of interactions in several different quasi one-dimensional ladder models of fermions and bosons, such as those encountered in ongoing experiments of synthetic quantum matter, in particular in setups of ultracold atoms trapped in optical lattices.

In fermionic systems, we found a possibility to flexibly tune the zero frequency component of the conductivity at zero temperature (a.k.a. the Drude weight) by repulsive density-density interactions.
The enhancement of this quantity under repulsive interactions is contradicting the ``common wisdom'' that one-dimensional systems should always show a decreasing Drude weight.
The loophole is given through the slim 2D extension of quasi one-dimensional ladders.
It allows to generate pseudospin polarized band structures which leads to the anomalous behavior of the Drude weight in interacting systems.
Our results are thus relevant for the modification of transport properties of coupled-wire systems in general.
The results are based on perturbation theory, Abelian bosonization, and numerical estimates in the non-perturbative regime through matrix product state (MPS) simulations.
% The analytic concepts necessary to comprehend this paper are illustrated in \cref{sec:periodic_potentials,sec:tight_binding_systems,sec:tomonaga_LL,sec:LL_with_spin}.
% In addition, we use matrix product state (MPS) simulations, which are presented in \cref{ch:matrix_product_states}.

We study the emergence of so-called helical liquids by repulsive interactions and density-assisted hoppings, resulting from a commensurability between particle density and flux generated from complex hopping elements.
Some of these phases have a natural extension to (fractional) Quantum Hall phases in two spatial dimensions, e.g. the Laughlin-like states, and can thus be interpreted as the predecessors of (fractional) topological phases of matter.
The study of strongly correlated helical liquids is thus important for the bottom-up fabrication of (fractional) topological phases in setups of synthetic quantum matter.
For fermions, we study the emergence of an exotic phase located at particle filling factor $\nu=1/2$ which naturally extends to a fractional Quantum Hall phase: the bosonic $K=8$ state.
We then study a similar phase in a ladder of hard-core bosons, located at filling factor $\nu=1$.
In both cases, we derive the phase diagram through renormalization group theory calculations and test our hypotheses through MPS simulations.

A natural quest in the context of topological insulators concerns the measurement of topological invariants, especially in the case of interacting systems, where a priori it does not amount to a band dispersion quantity.
For chiral symmetric systems, we propose a simple dynamical protocol based on the mean chiral displacement which provides a tomography of the topological index and indicates the presence of symmetry-broken phases through a characteristic divergence.
This protocol is tested using dynamical MPS simulations, and we provide a basic blueprint for an experiment of ultracold atoms trapped in optical lattices.
