%!TEX root = thesis.tex
%%%%%%%%%%%%%%%%%%%%%%%
\chapter{The quantization of motion and fields}
\label{ch:the_quantization_of_motion_and_fields}
The present chapter is inspired by a number of excellent books and lectures such as~\cite{AshcroftMermin1978,AltlandSimons2010,BruusFlensberg2004,Czycholl2016,FetterWalecka2003,Giamarchi2003,Rizzi2016,Burrello2020} extending the basic aspects of quantum mechanics to a modern way of understanding quantum field theory in general.
%
%
%%%%%%%%%%%%%%%%%%%%%%%%%%%%%%%%%%%%%%%%%%%%%%%%
\section{Creation and annihilation operators}
\label{sec:creation_and_annihilation_operators}
%%%%%%%%%%%%%%%%%%%%%%%%%%%%%%%%%%%%%%%%%%%%%%%%
%
%
Consider a complete set of quantum numbers $\{\alpha\}$ which label a normalized set of states $\{\ket{\alpha}\}$ spanning the full Hilbert space $\HS^1$ of a generic single particle system described by the (time-independent) Schrödinger equation
\begin{align}
    \hat H \ket{\alpha} = \varepsilon_\alpha\ket{\alpha}.
\end{align}
The single particle wave function $\Psi_\alpha(r)$ of a quantum state occupying $\alpha$ is defined as the inner product of the vector $\ket{\alpha}$ with the real-space covector, i.e.
\begin{align}
    \Psi_\alpha(r) \coloneqq \braket{r|\alpha}.
\end{align}
% \todoil{Comment already on the overcompleteness of the $r$-states, or not?}
It is is thus understood as the coefficients for the basis transform $\ket{\alpha}\rightarrow\ket{r}$, i.e.
\begin{align}
    \ket{\alpha} = \int{\rm d}{r}\, \Psi_\alpha(r) \ket{r}.
\end{align}
According to the basic postulate of quantum mechanics, the wave function of two indistinguishable particles with quantum numbers $\alpha_1$ and $\alpha_2$ is given by the (anti-) symmetrized product
\begin{align}
    \Psi_{\alpha_1,\alpha_2,\nu}(r_1,r_2) = \frac1{\sqrt2}\left(\braket{r_1|\alpha_1}\braket{r_2|\alpha_2} + \nu\braket{r_2|\alpha_1}\braket{r_1|\alpha_2}\right),
\end{align}
depending on the particle statistics upon exchange of their position, i.e. $\nu=\pm1$ for bosons and fermions, respectively.
The two-particle wave function can thus be represented by a more simple inner product
\begin{align}
    \Psi_{\alpha_1,\alpha_2,\nu}(r_1,r_2) = \left(\bra{r_1}\otimes\bra{r_2}\right)\ket{\alpha_1 \alpha_2}_\nu
\end{align}
which is given by the (anti-) symmetric Kronecker product
\begin{align}
    \ket{\alpha_1 \alpha_2}_\nu = \frac1{\sqrt2}\left(\ket{\alpha_1}\otimes\ket{\alpha_2} + \nu\ket{\alpha_2}\otimes\ket{\alpha_1}\right).
\end{align}
In general, the symmetric $N$-particle state vector is an element of the $N$-particle Hilbert space $\HS^N = \bigotimes_{i=1}^N\HS^1$ and reads
\begin{align}
    \ket{\alpha_1,\alpha_2,\dots,\alpha_N}_\nu = \frac1{\sqrt{N!\prod_{\alpha}(n_\alpha!)}}\sum_P \nu^{(1-\sign{P})/2}\ket{\alpha_{P(1)}}\otimes\ket{\alpha_{P(2)}}\otimes\dots\otimes\ket{\alpha_{P(N)}}.
    \label{eq:symmetric_many_body_state}
\end{align}
In the above equation, we assume ordered quantum numbers (e.g. increasing positions along a wire, or increasing energies), denote the total number of particles with quantum number $\alpha$ as $n_\alpha$ and $\sign{P}$ the sign of the permutation $P\in S^N$ of the permutation group [$\sign{P}=\pm1$ if the permutation is even/odd].

The representation in the ordered expression of \cref{eq:symmetric_many_body_state} is not particularly compact since equal values of $\alpha$ may appear $n_\alpha$ times in the $N$-letter long ket -- the occupation number representation removes this redundancy.
The states in this representation are then given by
\begin{align}
    \ket{n_1, n_2, \dots}_\nu = \ket{\underbrace{\alpha_1,\alpha_1,\dots,\alpha_1}_{n_1}, \underbrace{\alpha_2, \alpha_2,\dots, \alpha_2}_{n_2}, \alpha_3, \dots}
\end{align}
and they span the space $\FS^N$ of the (anti-) symmetrized $N$-particle states $\sum_{i} n_i = N$.
Thus, the subset $\FS^N\subset \HS^N$ contains all $N$-particle states which transform according to the basic postulate of quantum mechanics such that any physical state $\ket{\Psi}\in\HS^N$ can be written as a linear superposition of the Fock states
\begin{align}
    \ket{\Psi}_\nu = \sum_{\sum_i n_i = N}C(n_1,n_2,\dots)\ket{n_1,n_2,\dots}_\nu.
\end{align}
The full Fock space $\FS$ is defined as a direct sum of vector spaces with fixed $N$, i.e.
\begin{align}
    \FS = \bigoplus_{N=0}^\infty \FS^N,
\end{align}
including the one-dimensional vacuum space commonly denoted by $\{\ket{0}\}=\FS^0$.

Let us now impose a linear map $\hat a^\dag_i:\FS\rightarrow\FS$ connecting the individual subsets through
\begin{align}
    \hat a^\dag_i\ket{n_1,\dots,n_i,\dots}_\nu = \sqrt{n_i+1}\nu^{\sum_{j<i}n_j}\ket{n_1,\dots,n_i+1,\dots}_\nu
    \label{eq:creation}
\end{align}
in which fermionic states must be understood ${\rm mod}_2$ such that the Pauli exclusion principle is explicitly satisfied: $\hat a^\dag{}^2\ket{0}=\hat a^\dag\ket{1}={\rm mod}_2(1+1)\ket{{\rm mod}_2(1+1)} = 0\ket{0} = 0$.
Notice that through the linear map we can express the canonical basis of any subset $\FS^N\subset\FS$ as
\begin{align}
    \ket{n_1,n_2,\dots}_\nu = \prod_i\frac1{\sqrt{n_i!}}\left(\hat a^\dag_i\right)^{n_i}\ket{0}_\nu,
    \label{eq:Fock_basis}
\end{align}
and as such have a tool which promotes the vacuum to any state of the full Fock space.
\begin{figure}
    \centering
    \subfigure[]{\includegraphics{figures/connected_fock_space_bosons.pdf}}
    \hfil
    \subfigure[]{\includegraphics{figures/connected_fock_space_fermions.pdf}}
    \caption{(a) Subspaces $\FS^N$ of $N$-particle bosonic states $\ket{N}$ characterized by a single quantum number. Adjacent spaces are connected through the linear maps $a^{(\dag)}$ defined in \cref{eq:creation,eq:annihilation}. (b) Subspaces for a fermionic system characterized by a single quantum number.}
    \label{fig:fock_spaces}
\end{figure}
Notice the absence of the phase $\nu$ on the right hand side which is due to the fact that the product is ordered.
For this reason (and in the following), the linear maps $\hat a^\dag_i$ are commonly called creation operators.
Two different linear maps $j<i$ satisfy the following equation
\begin{align}
    \hat a^\dag_i \hat a^\dag_j\ket{n_1,n_2,\dots}_\nu = \nu \hat a^\dag_j\hat a^\dag_i\ket{n_1,n_2,\dots}_\nu,
\end{align}
and thus span the famous (anti-)commutation relation
\begin{align}
    \commutator{\hat a^\dag_i, \hat a^\dag_j}_\nu \coloneqq \hat a^\dag_i \hat a^\dag_j - \nu \hat a^\dag_j \hat a^\dag_i = 0.
\end{align}
From the Hermitian adjoint of the equations before we get the condition
\begin{align}
   \braket{n_1,\dots,n_i,\dots | \hat a^\pdag_i | m_1, \dots, m_i,\dots}_\nu =
   \sqrt{n_i+1}\nu^{\sum_{j<i}n_j} \delta_{n_1,m_1}\dots\delta_{n_i+1,m_i}\dots
\end{align}
and thus
\begin{align}
   \hat a^\pdag_i \ket{n_1, \dots, n_i,\dots}_\nu =
   \sqrt{n_i}\nu^{\sum_{j<i}n_j} \ket{n_1, \dots, n_i-1,\dots}_\nu,
   \label{eq:annihilation}
\end{align}
from which we obtain the algebra relations of the creation/annihilation operators
\begin{align}
    \commutator{\hat a^\pdag_i,\hat a^\dag_j}_\nu = \delta_{i,j},
    \quad
    \commutator{\hat a^\pdag_i,\hat a^\pdag_j}_\nu = 0,
    \quad
    \commutator{\hat a^\dag_i,\hat a^\dag_j}_\nu = 0.
    \label{eq:Heisenberg_algebra}
\end{align}
In reverse,~\cref{eq:Fock_basis} is a consequence of the Stone-von Neumann theorem which states that, given the Heisenberg algebra defined through~\cref{eq:Heisenberg_algebra}, the action of the operators and the representation of the Fock basis is unique (up to unitary transformations)~\cite{Hall2013}.

To conclude this first section, a change of basis $\{\ket\alpha\}\rightarrow\{\ket\beta\}$ yields\footnote{Remember that $\mathbb1 = \sum_\alpha\ket{\alpha}\bra{\alpha}$, $\ket{\beta} = \sum_\alpha\ket{\alpha}\braket{\alpha|\beta}$ and $\ket{\beta}=\hat a^\dag_\beta\ket{0}$, leading to \cref{eq:creation_annihilation_basis_rotation}.} a unitary transformation of the operators
\begin{align}
    \hat a^\dag_\beta = \sum_\alpha \braket{\alpha|\beta}\hat a^\dag_\alpha,
    \quad
    \hat a^\pdag_\beta = \sum_\alpha \braket{\beta|\alpha}\hat a^\pdag_\alpha
    \label{eq:creation_annihilation_basis_rotation}
\end{align}
which requires only a computation of the single-particle matrix elements $\braket{\alpha|\beta}$.
Before we move on to the representation of observables, a word on common notations: Quite often the authors assume a certain particle statistics and operator algebra at the beginning of their work which implies a constant (and thus dropped) subscript $\nu$.
Fermionic annihilation operators are mostly identified through the letter $\hat c$ whereas $\hat b$ often corresponds to bosonic operators.
Furthermore, a common convention identifies the commutator through crotchets
\begin{align}
    \left[\hat A,\hat B\right]\coloneqq\commutator{\hat A,\hat B}_+ = \hat A\hat B - \hat B\hat A
\end{align}
and the anticommutator through curly braces
\begin{align}
    \anticommutator{\hat A,\hat B}\coloneqq\commutator{\hat A,\hat B}_- = \hat A\hat B + \hat B\hat A.
\end{align}
Let me also provide a useful expression to evaluate the commutation relation of operator products recursively
\begin{align}
    \commutator{\hat A, \hat B \hat C}_\pm
    &= \hat A\hat B\hat C \mp \hat B\hat C\hat A + \hat B\hat A\hat C - \hat B\hat A\hat C
    \\
    &= \commutator{\hat A, \hat B}_\pm\hat C \mp \hat B\hat C\hat A \pm \hat B\hat A\hat C
    = \commutator{\hat A, \hat B}_\pm\hat C \mp \hat B\commutator{\hat A,\hat C}.
    \label{eq:recursive_commutation}
\end{align}
In many cases, the sets of quantum numbers are continuous (e.g. position $x$) and as such a sum in \cref{eq:creation_annihilation_basis_rotation} is promoted to an integral expression:
\begin{align}
    \hat a(x) = \sum_\alpha \braket{x | \alpha} \hat a_\alpha,
    \quad
    \hat a_\alpha = \int{\rd x} \braket{\alpha | x} \hat a(x).
\end{align}
This is commonly highlighted through a bracket notation of the continuous quantum number.
In conclusion, using the Fock state representation automatically assures the (anti-)symmetric properties of many-body quantum states in real space.
It provides thus an ideal basis for analytical and numerical simulations of many-body states of matter.
%
%
%%%%%%%%%%%%%%%%%%%%%%%%%%%%%%%%%%%%%%%%%%%%%%%
\section{Representation of generic operators}
\label{sec:representation_of_generic_operators}
%%%%%%%%%%%%%%%%%%%%%%%%%%%%%%%%%%%%%%%%%%%%%%%
Let us start with a general operator acting on a single particle of the full $N$-particle state (usually dubbed ``one-body operator'').
Familiar examples are the momentum or the position operator $\hat x_i$, $\hat p_i$ acting on the $i$th particle, or compositions of these operators like single particle potentials $\hat V(\hat x_i)$.
It is thus not surprising that the general expression of such operators can be given in terms of the particle creation and annihilation operators we introduced in \cref{sec:creation_and_annihilation_operators}.

A one-body operator $\hat o$ diagonal in an arbitrary single-particle basis $\{\alpha_1\}$ ($\hat o = \sum_{\alpha_1} o_{\alpha_1}\ket{\alpha_1}\bra{\alpha_1}$) is trivially extended to the $N$-particle states written in the same basis
\begin{align}
    \hat O_1
    = \sum_{\alpha_i} o_{\alpha_i}\hat a^\dag_{\alpha_i}\hat a^\pdag_{\alpha_i}.
\end{align}
This is most easily understood: one-body operators act on only a single entity of the full set of particles, leaving the others untouched.
In the diagonal basis of $\hat O_1$, the action of $\hat n_{\alpha_i}\coloneqq \hat a^\dag_{\alpha_i}\hat a^\pdag_{\alpha_i}$ just counts the number of particles in the state $\alpha_i$, which is then multiplied with the expectation value of the single-particle operator.
In a more general basis, the one-body operator transforms according to \cref{eq:creation_annihilation_basis_rotation} resulting in
\begin{align}
    \hat O_1 = \sum_{\alpha, \beta}\braket{\alpha | \hat o | \beta} \hat a^\dag_\alpha \hat a^\pdag_\beta = \sum_{\alpha,\beta}o_{\alpha,\beta}\hat a^\dag_\alpha \hat a^\pdag_\beta.
    \label{eq:one_body_operator}
\end{align}
Note that in the above, the indices $\alpha$ and $\beta$ denote a full set of quantum numbers of the many-body system.
It is now straightforward to introduce generic two-body operators $\hat O_2$,
\begin{align}
    \hat O_2 = \sum_{\alpha,\beta,\gamma,\delta}O_{\alpha,\beta,\gamma,\delta}\hat a^\dag_\alpha \hat a^\dag_\beta \hat a^\pdag_\gamma \hat a^\pdag_\delta,
\end{align}
in which the expectation value reads $O_{\alpha,\beta,\gamma,\delta}\coloneqq \braket{\alpha,\beta | \hat o | \gamma,\delta}$.
For example, a generic two-point interaction $\hat V\ket{r_1,\dots,r_N}=1/2\sum_{n\neq m}V(r_n-r_m)\ket{r_1,\dots,r_N}$ in continuous space takes the second quantized form
\begin{align}
    \hat V = \frac12\int{\rd^d r}\int{\rd^d r'}V({\bm r}-{\bm r'})\hat a^\dag({\bm r})\hat a^\dag({\bm r'})\hat a^\pdag({\bm r'})\hat a^\pdag({\bm r}).
    \label{eq:two_point_interaction}
\end{align}
The generalization to generic $M$-body operators is now straightforward
\begin{align}
    \hat O_M = \sum_{\alpha_1,\dots,\alpha_M}\sum_{\beta_1,\dots,\beta_M}O_{\alpha_1,\dots,\alpha_M,\beta_1,\dots,\beta_M}\hat a^\dag_{\alpha_1}\dots \hat a^\dag_{\alpha_M}\hat a^\pdag_{\beta_1}\dots \hat a^\pdag_{\beta_M}.
\end{align}
%
%
%%%%%%%%%%%%%%%%%%%%%%%%%%%%%%%%
\section{Periodic potentials}
\label{sec:periodic_potentials}
%%%%%%%%%%%%%%%%%%%%%%%%%%%%%%%%
To really see when second quantization becomes useful, I go one step back and review basic properties of a single particle moving in a periodic potential which is effectively characterized by a Hamiltonian composed of generic one-body operators of the form of \cref{eq:one_body_operator}.
In particular, the Hamiltonian considered here reads
\begin{align}
    \hat H_0 = \int\rd^d r\, \hat a^\dag({\bm r})\brlr{\frac{\hat{\bm p}^2}{2m}+V_{ae}({\bm r})}\hat a^\pdag({\bm r})
\end{align}
% and later impose generic two-body interactions
% \begin{align}
%     \hat V_{ee} = \frac12\sum_{s,s'}\int\rd^dr\int\rd^dr'\,V_{ee}({\bm r}-{\bm r'})a^\dag_s({\bm r})a^\dag_{s'}({\bm r'})a^\pdag_{s'}({\bm r}')a^\pdag_{s}({\bm r})
% \end{align}
with operators $\hat a^\pdag({\bm r})$ annihilating a particle at position $\bm r$.
The local potential $V_{ae}$ is a collection of $N_a$ local potentials
\begin{align}
    V_{ae}({\bm r}) = \sum_{i=1}^{N_a}v_{ae}({\bm R}_i - {\bm r})
\end{align}
and the value of $v_{ae}$ is determined by the relative distance from the position ${\bm R}_i$.
If the creation operators satisfy the anticommutator algebra in real space
\begin{align}
    \anticommutator{\hat a^\pdag({\bm r}),\hat a^\dag({\bm r'})}=\delta({\bm r}-{\bm r'}),
\end{align}
the Hamiltonian defines a spinless fermionic system embedded in an arbitrary lattice.
A regular (Bravais) lattice structure in $d$ dimensions is in general spanned by $d$ linearly independent (not necessary mutually perpendicular and normalized) vectors ${\bm x}_i$ and can thus be defined as the set
\begin{align}
    \mathcal{R} = \anticommutator{\sum_{i=1}^d n_i {\bm x}_i,\ n_i\in\mathds Z}.
\end{align}
Additional vectors $\{{\bm b}_i\}$ (sometimes called the basis) are used to define the position of the potential minima relative to the points of the Bravais lattice, such that every position in the primitive cell is given by ${\bm R}_j = \sum_i n_i {\bm x}_i + {\bm b}_j$ with integers $n_i\in\mathds Z$.
The beauty of this approach becomes visible as soon as we define the lattice translation operator
\begin{align}
    \hat T_{\bm n} : \psi_\alpha({\bm r}) \mapsto \psi_\alpha\brlr{{\bm r}+T_{\bm n}},\ T_{\bm n}\coloneqq\sum_i n_i {\bm x}_i={\bm n}\underline{\bm x}
\end{align}
which translates every function from one position to a distance parametrized through the $d$-dimensional vector ${\bm n}\in\mathds Z^d$ and the collection of all lattice vectors $\underline{\bm x}\coloneqq ({\bm x}_1,\dots,{\bm x}_d)^T$.
The local potential $V_{ae}({\bm r})$ is by definition invariant under lattice translations, and two-body interactions $V({\bm r}-{\bm r'})$ are clearly invariant under continuous translations, such that the full Hamiltonian $\hat H$ commutes with the translation operator and we can write the eigenfunctions of $\hat H$ as eigenfunctions of $\hat T_{\bm n}$.
The eigenfunctions of all translation operators $\hat T_{\bm n}$ are called Bloch waves and have the following properties\footnote{Note here that a further restriction on the values of the vector ${\bm k}$ arise from imposing boundary conditions. For instance, Born-von Karman boundary conditions imply a periodicity after $L_j$ unit translations $\psi_\alpha({\bm r}+ L_j{\bm R}_j)=\psi_\alpha({\bm r})$, which confines the allowed values of $k_j$ to integer multiples of $\frac{\bm G_j}{L_j}$ in which $\bm G_j$ is the reciprocal vector of ${\bm R}_j$, i.e. ${\bm G}_j {\bm R}_j = 2\pi/a$.}
\begin{align}
    \hat T_{\bm n}\psi_{\alpha}({\bm r}) &= c_{\bm n}\psi_{\alpha}({\bm r})\\
    c_{{\bm n}_1}c_{{\bm n}_2} &= c_{{\bm n}_1+{\bm n}_2} \Rightarrow c_{\bm n} = \re^{{\bm s}\cdot{\bm n}\underline{\bm x}},\ {\bm s}\in\mathds C^d\\
    1=\int_V\rd^dr\,\abs{\psi_{\alpha}({\bm r})}^2 &= \int_V\rd^dr\,\abs{\hat T_{\bm n}\psi_{\alpha}({\bm r})}^2 \Rightarrow 1=\abs{c_{\bm n}}^2 \Rightarrow {\bm s}=\ri{\bm k},\ {\bm k}\in\mathds R^d
\end{align}
It is now possible to phrase Bloch's theorem~\cite{Bloch1929}, which states that the eigenfunctions of a particle moving in periodic potentials assume the simple structure
\begin{align}
    \psi_\alpha({\bm r}) = \psi_{\alpha{\bm k}}({\bm r}) = \re^{\ri{\bm k}\cdot {\bm r}}u_{\alpha{\bm k}}({\bm r}).
    \label{eq:bloch_theorem}
\end{align}
Note that we introduced the translation operators eigenvalue argument $\bm k$ in the list of good quantum numbers.
In particular, $\psi$ is a weighted plane wave in which $u$ inherits the lattice periodicity
\begin{align}
    \hat T_{\bm n} u_{\alpha{\bm k}}({\bm r})
    =
    \re^{-\ri{\bm k}\cdot \brlr{{\bm r}+{\bm n}\underline{\bm x}}}c_{\bm n}\psi_{\alpha{\bm k}}({\bm r})
    =
    \re^{-\ri{\bm k}\cdot {\bm r}}\re^{-\ri{\bm k}\cdot {\bm n}\underline{\bm x}}\re^{\ri{\bm k}\cdot {\bm n}\underline{\bm x}}\psi_{\alpha{\bm k}}({\bm r})
    =
    u_{\alpha{\bm k}}({\bm r}).
\end{align}
% The particular expression of the weights obviously depends on the potential, and obtaining general solutions becomes a hard task even for simple problems\footnote{Except maybe the trivial scenario $V_{ae}({\bm r})={\rm const.}$ which implies $u=1$ in \cref{eq:bloch_theorem}.}.
In general, the wave vector $\bm k$ seems to play the same role as the particle momentum $\bm p$ in the Sommerfeld theory of a free particle.
However, this is not true as is clear from the momentum operator being $\hat{\bm p}=-\ri\hbar\bm \nabla$, and as such
\begin{align}
  \hat{\bm p}\psi_{\alpha{\bm k}}({\bm r}) = \hbar{\bm k}\psi_{\alpha{\bm k}} - \ri\hbar\re^{\ri{\bm k}{\bm r}}\bm\nabla u_{\alpha \bm k}(\bm r).
\end{align}
The vector ${\bm k}$ is thus dubbed crystal momentum, remembering the fact that $\psi_{\alpha{\bm k}}$ is only an eigenstate of momentum if the potential is constant and thus $\bm\nabla V_{ae}=0$.

As we will see in the next section, the Bloch waves can be used to define a complete set of tightly localized ``Wannier'' states, which allows to work with effective ``tight binding approximations'' in which the coupling parameters of the model are implicitly defined through overlap integrals of the Bloch functions.
This way, the many-body phases of the effective model can be studied in a more efficient way without the explicit knowledge of the coupling constants.
Nonetheless, to get an idea of the form of the Bloch functions, let us solve the Schrödinger equation of a particle moving along one dimension in the potential
\begin{align}
    V_{ae}(x) = V_0 \Theta\brlr{b-{\rm mod}_{a}\commutator{x+b}},\ V_0,a,b\in\mathds R_+
    \label{eq:kronig_penney_potential}
\end{align}
such that the nontrivial contours are of size $b$ and displaced by a factor $a$ called the lattice spacing.
This model is called the Kronig-Penney model~\cite{KronigPenney1931} and provides analytic solutions of bound electrons in the limit $b\rightarrow0$ and $V_0\rightarrow\infty$ such that $bV_0={\rm const.}$, which is depicted in \cref{fig:kronig_penney_potential}.
\begin{figure}
    \centering
    \includegraphics{figures/kronig_penney_potential.png}
    \caption{Different contours of the periodic potential \cref{eq:kronig_penney_potential} for $bV_0=1/4$. The gray boxes corresponds to the fixed values $V_0=1/3$ and $b=3/4$, while the blue gradient visualizes the limit $b\rightarrow0$ and $V_0\rightarrow\infty$ while preserving the product $bV_0=P$.}
    \label{fig:kronig_penney_potential}
\end{figure}

% In order to preserve the periodicity of the potential, the system size must be commensurate to the lattice spacing $a$, i.e. $L\in a\mathds N$.
To simplify our problem, we can consider two different regions, i.e. (i) a free particle $-\frac{\hbar^2}{2m}\partial_x^2\psi_{\rm (i)}(x) = E\psi_{\rm (i)}(x)$ and (ii) a particle moving in a constant potential $-\frac{\hbar^2}{2m}\partial_x^2\psi_{\rm (ii)}(x) = (E-V_0)\psi_{\rm (ii)}(x)$.
In order to read out the Bloch weights I conveniently factor a crystal momentum from the wave function
\begin{align}
    \psi_{\alpha k}^{\rm (i)}(x) = \re^{\ri kx}u_{\alpha k}^{\rm (i)}(x),
    \
    \psi_{\beta k}^{\rm (ii)}(x) = \re^{\ri kx}u_{\beta k}^{\rm (ii)}(x),
    \\
    u_{\alpha k}^{\rm (i)}(x) = A\re^{(\tau\alpha-\ri k) x} + A'\re^{-(\tau\alpha+\ri k) x},
    \
    u_{\beta k}^{\rm (ii)}(x) = B\re^{(\tau'\beta-\ri k) x} + B'\re^{-(\tau'\beta+\ri k) x},
\end{align}
with $\hbar^2\alpha^2 = 2m\abs{E}$, $\hbar^2\beta^2 = 2m\abs{(E-V_0)}$, $\tau^2=-\sign{E}$ and $\tau'^2=-\sign{E-V_0}$.
Plane waves are thus obtained in case of $\tau=\tau'=\ri$ are obtained for $E>V_0$.
The wave functions are supposed to be smooth at the boundaries\footnote{I hereby use the notation of limits from above or below, i.e. $f(0^\pm)=\lim_{\epsilon\rightarrow0}f(\pm\epsilon)$ for $\epsilon>0$.}, which results in
\begin{align}
    % \psi_{\alpha k}^{\rm (i)}(0^+)=\psi_{\beta k}^{\rm (ii)}(0^-),
    % \quad
    % {\psi_{\alpha k}^{\rm (i)}}'(0^+)={\psi_{\beta k}^{\rm (ii)}}'(0^-),
    % \\
    \psi_{\alpha k}^{\rm (i)}(-b+0^-)=\psi_{\beta k}^{\rm (ii)}(-b+0^+),
    \quad
    {\psi_{\alpha k}^{\rm (i)}}'(-b+0^-)={\psi_{\beta k}^{\rm (ii)}}'(-b+0^+),
\end{align}
and the Bloch functions inherit the potential's periodicity
\begin{align}
    u_{\alpha k}^{\rm (i)}(a+0^+) = u_{\alpha k}^{\rm (ii)}(0^-),
    \quad
    u_{\alpha k}^{\rm (i)}{}'(a+0^+) = u_{\alpha k}^{\rm (ii)}{}'(0^-).
\end{align}
In summary, the following matrix equation is derived
\begin{align}
    &\qquad M =\\
    &\begin{pmatrix}
        1 & 1 & -1 & -1\\
        %
        \tau\alpha & -\tau\alpha & -\tau'\beta & \tau'\beta\\
        %
        \re^{(\tau \alpha-\ri k)(a-b)}  & \re^{-(\tau \alpha+\ri k) (a-b)} &
        -\re^{-(\tau'\beta -\ri k)b}     & -\re^{ (\tau'\beta+\ri k)b} \\
        %
        (\tau \alpha-\ri k)\re^{(\tau \alpha-\ri k)(a-b)}  & -(\tau \alpha+\ri k)\re^{-(\tau \alpha+\ri k) (a-b)} &
        -(\tau'\beta -\ri k)\re^{-(\tau'\beta -\ri k)b}     & (\tau'\beta+\ri k)\re^{ (\tau'\beta+\ri k)b} \nonumber\\
    \end{pmatrix},
\end{align}
satisfying $M(A,A',B,B')^T=0$.
For nontrivial results, the determinant of $M$ should be equal to zero, which is satisfied for solutions of the transcendental equation
\begin{align}
    \cos(ka)
    =
    \cosh(\alpha\tau(a-b))\cosh(b\beta\tau')
    +
    \frac{\alpha^2\tau^2+\beta^2\tau'^2}{2\alpha\beta\tau\tau'}\sinh(\alpha\tau(a-b))\sinh(b\beta\tau).
    \label{eq:kronig_penney_transcendental_equation}
\end{align}
To approximate the right hand side in the aforementioned limits, application of
\begin{align}
    b\rightarrow0,
    \quad
    V_0\rightarrow\infty,
    \quad
    bV_0 = {\rm const.}
    \\
    \Rightarrow
    b\beta^2\rightarrow 2mbV_0/\hbar^2,
    \quad
    \cosh(b\beta\tau')\rightarrow 1,
    \quad
    \sinh(b\beta\tau')\rightarrow b\beta\tau'
\end{align}
provides an exact rewriting of \cref{eq:kronig_penney_transcendental_equation} in the limit of narrow and strong periodic potentials\footnote{This limit is actually equivalent to a potential composed by delta-distributions.}
\begin{align}
    f(\alpha a) \coloneqq \cosh(\alpha\tau a) + \tau'^2\frac{P}{\alpha\tau a}\sinh(\alpha\tau a),
    \quad
    P \coloneqq mabV_0/\hbar^2.
    \label{eq:kronig_penney_transcendental_equation_approx}
\end{align}
If we assume bound states between the potential wells $0<E<V_0$, the signs become $\tau^2=-1$ and $\tau'^2=+1$, such that \cref{eq:kronig_penney_transcendental_equation_approx} evaluates to
\begin{align}
    f(\alpha a) = \cos(\alpha a) + \frac{P}{\alpha a}\sin(\alpha a).
    \label{eq:kronig_penney_transcendental_equation_approx_bound}
\end{align}
For nonzero $P$, the right-hand-side of \cref{eq:kronig_penney_transcendental_equation_approx_bound} is not bound to the interval $[-1,+1]$ spanned by the left-hand-side of \cref{eq:kronig_penney_transcendental_equation}, and establishes values of $\alpha$ (thus, the square-root of the energy $E$) for which no (real) momentum exists (see \cref{fig:kronig_penney_dispersion} (a)).
Such energies are called forbidden and allow for a first understanding of band-gaps induced by the nonzero lattice potential $V_0>0$.
\begin{figure}
    \centering
    \subfigure[]{\includegraphics{figures/kronig_penney_transcendental.png}}
    \subfigure[]{\includegraphics{figures/kronig_penney_dispersion_2.png}}
    \subfigure[]{\includegraphics{figures/kronig_penney_dispersion_1.png}}
    \caption{
    (a) Plot of the right-hand-side of the transcendental equation. Solutions do not exist in the red regions.
    (b)-(c) Shapes of the dispersion relation $(\alpha a/\pi)^2$ versus crystal momentum $ka/\pi$ given by \cref{eq:kronig_penney_transcendental_equation_approx_bound}.
    Different opacity (transparent to colors) represent increasing values of $P\in\{0.1,1,5,20,50,1000\}$.
    (c) The properties of the wave functions allow to uniquely relate the crystal momentum $ka$ to an energy $\alpha a$ in which the limit of free electrons [i.e. $P\rightarrow0$] is pronounced.}
    \label{fig:kronig_penney_dispersion}
\end{figure}

If $P=0$, we are left with the energy-momentum relation of free electrons $k=\alpha$ and thus $E_0={\hbar^2k^2}/({2m})$.
If we approach $P\rightarrow\infty$, the allowed energies are formed by the roots of $\sin(\alpha a)$, i.e.
\begin{align}
    E_{\infty,n_b}=(\hbar n_b\pi)^2/(2ma^2),
    \label{eq:kronig_penney_energy_tb}
\end{align}
for which the level spacing is given by the squares of integer numbers ($n_b=1,2,\dots$).
The intermediate regimes $0<P<\infty$ can be solved by numerical evaluation of \cref{eq:kronig_penney_transcendental_equation_approx_bound} and are plotted in \cref{fig:kronig_penney_dispersion} (a).
Straightforward evaluation of $\arccos(f)$ yields the so-called ``reduced zone scheme'' displayed in \cref{fig:kronig_penney_dispersion} (b).

To get an analytic understanding of the wave functions in the above limit, let's focus on the region without potential (i) [remember that one is interested in the limit $b\rightarrow0$]
\begin{align}
    \psi_{\alpha k}^{\rm (i)}(x) = \re^{\ri kx}u_{\alpha k}^{\rm (i)}(x),
    \quad
    u_{\alpha k}^{\rm (i)}(x) = A\re^{\ri(\alpha-k) x} + A'\re^{-\ri(\alpha+k) x},
    \quad
    u_{\alpha k}^{\rm (i)}(x+na)=u_{\alpha k}^{\rm (i)}(x)
    \label{eq:kronig_penney_wavefunctions}
\end{align}
in which the prefactors are related by\footnote{An additional constraint for $AA^*$ is found by requiring normalization of the wave functions, which allows to compute expectation values. However, it is not needed for the purpose of this section and I refer to \cite{KronigPenney1931}.}
\begin{align}
    A' = -A\frac{1-\re^{\ri(\alpha-k)a}}{1-\re^{-\ri(\alpha+k)a}}.
    \label{eq:kronig_penney_constants}
\end{align}

Starting from \cref{eq:kronig_penney_transcendental_equation_approx_bound}, we see that for any solution $\alpha a$ satisfying $f(\alpha a)=\cos(k a)$ there is an infinite number of symmetric points in momentum space which fulfill the same equation, i.e. (a) $ka + 2n\pi$ and (b) $-ka+2n\pi$.
Close inspection of \cref{eq:kronig_penney_wavefunctions,eq:kronig_penney_constants} reveals that a transformation according to (a) leaves the wave function invariant, while (b) maps it to the solutions of $-ka$, $-\alpha a$.

In other words, (a) is merely a shift by a unit reciprocal vector and (b) flips the direction of the propagating wave, hence establishes a mirror symmetry at the $ka=0$ axis.
The properties (a) and (b) allow to define the ``unfolding'' of the reduced zone scheme:
without loss of generality, a crystal momentum $n_b\pi\leq ka<(n_b+1)\pi$ can be uniquely connected to an energy value $n_b\pi\leq \alpha a<(n_b+1)\pi$ by introducing a band index $(n_b=0,1,\dots)$.
Here, the limit of free particles ($P=0$) is readily restored, since \cref{eq:kronig_penney_constants} vanishes for $\alpha a = ka$ (except for the special points $ka=n_b\pi$).

At the special points $ka=n_b\pi$, \cref{eq:kronig_penney_constants} is actually ill defined and evaluates to the two viable solutions $A'=\pm A$, which can be understood from the two non-commuting limits
\begin{align}
    -\lim_{\alpha a \searrow n_b\pi}\frac{1-\re^{\ri(\alpha a-n_b\pi))}}{1-\re^{-\ri(\alpha a+n_b\pi)}}
    =
    -\lim_{\epsilon\searrow0}\frac{1-\re^{+\ri\epsilon}}{1-\re^{-\ri\epsilon}}
    =
    \lim_{\epsilon\searrow0}\re^{+\ri\epsilon}\frac{1-\re^{-\ri\epsilon}}{1-\re^{-\ri\epsilon}}
    =
    +1,
    \\
    -\lim_{k a \searrow n_b\pi}\frac{1-\re^{\ri(n_b\pi - ka))}}{1-\re^{-\ri(n_b\pi+ka)}}
    =
    -\lim_{k a \searrow n_b\pi}\frac{1-\re^{-\ri(n_b\pi + ka))}}{1-\re^{-\ri(n_b\pi+ka)}}
    =
    -1.
\end{align}
The two allowed eigenfunctions are standing waves $\propto \cos(kx),\sin(kx)$.
By reformulating the special points in terms of the particles' de Broglie wavelength $\lambda=2\pi/k$, the formation of standing waves can be understood as a residual effect of (constructive) Bragg reflection on a periodic grid structure.
Standing waves are obtained if the particle's de Broglie wavelength and the spacing of the potential satisfy $2a=n_b\lambda$.

In the limit $P\rightarrow\infty$, the energy $E_{\infty,n_b}$ assumes the discrete values in \cref{eq:kronig_penney_energy_tb} resulting in motionless eigenstates -- the resulting waves are tightly bound to the potential minimum.
If we relax this limit a bit, i.e. $P\gg 1$, a reasonable approximation of the lowest band curvature is obtained by a first order Taylor series of \cref{eq:kronig_penney_transcendental_equation_approx_bound} resulting in the typical dispersion relation for tight binding systems, i.e.
\begin{align}
    E_{P\gg1,1}\approx t_0 + t_1 \cos(k a),
    \quad
    t_0 = + E_{\infty,1} - \frac{\pi^2 \hbar^4}{a^3 m^2 b V_0},
    \quad
    t_1 = -\frac{\pi^2\hbar^4}{a^3 m^2 bV_0}.
    \label{eq:kronig_penney_tight_binding_dispersion}
\end{align}
This concludes the pedagogic recap of the Kronig-Penney model, and I proceed by giving a more generic recipe to solve arbitrary periodic potentials.

In order to tackle generic potentials, let us expand the Schrödinger equation in reciprocal space through the following identities\footnote{The reciprocal space provides a way to Fourier-transform as the functions $\re^{\ri {\bm G r}}$ form a basis on the primitive cell of the real lattice over the square-integrable functions. In particular, the functions satisfy the orthogonality equation $\delta_{\bm G, \bm G'}=\frac1{V}\int\rd^dr\,\re^{\ri({\bm G-\bm G'}){\bm r}}$.}
\begin{align}
    V_{ae}({\bm r}) = \sum_{\bm G}V_{ae}{}_{\bm G}\re^{\ri {\bm G r}},
    \quad
    V_{ae}{}_{\bm G} = \frac1{V}\int\rd^dr\,V_{ae}({\bm r})\re^{-\ri {\bm G r}},
    \\
    u_{n{\bm k}}({\bm r}) = \sum_{\bm G}u_{n{\bm k}}{}_{\bm G}\re^{\ri {\bm G r}},
    \quad
    u_{n{\bm k}}{}_{\bm G} = \frac1{V}\int\rd^dr\,u_{n{\bm k}}({\bm r})\re^{-\ri {\bm G r}},
\end{align}
in which $V$ is the volume of the primitive unit cell.
Straightforward evaluation yields the algebraic eigenvalue problem for the unknown functions $u_{n{\bm k}}{}_{\bm G}$
\begin{align}
    \frac{\hbar^2}{2m}({\bm G}+{\bm k})^2u_{n{\bm k}}{}_{\bm G}+\sum_{\bm G'}V_{ae}{}_{\bm G-\bm G'}u_{n{\bm k}}{}_{\bm G'} = E_{n{\bm k}}.
    \label{eq:periodic_lattices_numerics}
\end{align}
The dimension of the linear equation is infinite due to the sum over all reciprocal lattice vectors and has to be truncated if one wants to solve the equation numerically (the convergence of such a truncation has to be carefully checked).
Clearly, strongly confined potentials such as delta functions studied in the Kronig-Penney model are particularly bad candidates to solve numerically through evaluation of \cref{eq:periodic_lattices_numerics} because the resulting matrix equation will not be sparse and any truncation will result in a significant error.
%
%
%%%%%%%%%%%%%%%%%%%%%%%%%%%%%%%%%%
\section{Tight binding systems}
\label{sec:tight_binding_systems}
%%%%%%%%%%%%%%%%%%%%%%%%%%%%%%%%%%
%
%
The systems considered here are those of tightly bound constituents to the lattice centers.
Such types can be found in traditional solid state scenarios where the nuclei are well separated beyond the typical Bohr radius of the valence electrons, in setups of ultracold atoms trapped in optical lattices~\cite{Bloch2008}, in photonic waveguides~\cite{Lu2014} or polaritons~\cite{Amo2016}.
In all our works, we cover theoretical aspects of interacting tight binding models which can be experimentally realized in a variety of different setups.
For this reason, it will be useful to review briefly how these models are motivated from first principles, and how they can be understood in second quantization.

\begin{figure}
    \centering
    \subfigure[]{\includegraphics{figures/wannier1_1.png}}
    \subfigure[]{\includegraphics{figures/wannier1_10.png}}
    \subfigure[]{\includegraphics{figures/wannier1_100.png}}\\
    \subfigure[]{\includegraphics{figures/wannier2_1.png}}
    \subfigure[]{\includegraphics{figures/wannier2_10.png}}
    \subfigure[]{\includegraphics{figures/wannier2_100.png}}\\
    \caption{Example Wannier functions of the first two bands for a periodic potential of the form $V_{ae}(x)=V_0\cos(2\pi/ax)$ for different potential depth. The integration of the Bloch states was performed by assuming a periodicity over $L=50$ lattice translations.}
    \label{fig:tight_binding_wanniers}
\end{figure}

For this purpose, we assume the problem of the single-particle Hamiltonian $\hat H_0$ to be fully solved, such that the Bloch states $\psi_{\alpha{\bm k}}$ diagonalize $\hat H_0$ and have energy eigenvalues $\varepsilon_{\alpha{\bm k}}$.
This allows to introduce the Wannier basis -- a localized basis composed of the Bloch states and defined as
\begin{align}
    \ket{w_{\alpha{\bm R}}} = \frac1{\sqrt N}\sum_{\bm k}\re^{-\ri{\bm k}{\bm R}}\ket{\psi_{\alpha{\bm k}}}.
    \label{eq:wannier_states}
\end{align}
Note that every momentum-resolved Bloch function can be multiplied with a complex phase without changing its properties.
This naturally provides a gauge freedom to optimize the Wannier function's properties -- for instance, the construction of a maximally localized basis~\cite{Marzari2012}.
Without going into detail about optimizing Wannier functions, I want to present a basic visualization of these localized states.
For this purpose, let us consider a cosine periodic potential
\begin{align}
    V_{ae}(x) = V_0\cos\brlr{\frac{2\pi}a x}
\end{align}
which yields a particular easy (a tridiagonal Toeplitz) matrix equation for the Bloch vectors presented in \cref{eq:periodic_lattices_numerics}.
To obtain the Wannier functions, we gauge every Bloch function to be purely real at $x=0$, resulting in the examples displayed in \cref{fig:tight_binding_wanniers}.

The existence of localized Wannier states is translated to the language of second quantization by the notion that transformations between a Bloch and Wannier basis are always unitary.
Hence, annihilation and creation operators of Wannier and Bloch states are set in relation by
\begin{align}
    \hat a^\dag_{\alpha{\bm R}} = \frac1{\sqrt N}\sum_{\bm k}\re^{-\ri{\bm k}{\bm R}}\hat a^\dag_{\alpha{\bm k}},
    \quad
    \hat a^\dag_{\alpha{\bm k}} = \frac1{\sqrt N}\sum_{\bm R}\re^{+\ri{\bm k}{\bm R}}\hat a^\dag_{\alpha{\bm R}}.
    \label{eq:wannier_states_2}
\end{align}
Since the non-interacting Hamiltonian is diagonal in the Bloch basis, they are the eigenfunctions with energies $\varepsilon_{\alpha{\bm k}} = \int\rd^dr\, \psi_{\alpha{\bm k}}^* \hat H_0 \psi_{\alpha{\bm k}} = \frac1N\sum_{\bm R,\bm R'}\re^{\ri\bm k(\bm R'-\bm R)}\int\rd^dr\, w_{\alpha{\bm R}}^* \hat H_0 w_{\alpha{\bm R'}}$.
In particular, the Hamiltonian is readily cast into the localized basis according to
\begin{align}
    \hat H_0
    =
    \varepsilon_{\alpha{\bm k}}\hat a^\dag_{\alpha{\bm k}}\hat a^\pdag_{\alpha{\bm k}}
    \overset{\text{\cref{eq:wannier_states_2}}}{=}
    \frac1N
    \re^{\ri{\bm k}\brlr{{\bm R}-{\bm R'}}}
    \varepsilon_{\alpha{\bm k}}
    \hat a^\dag_{\alpha{\bm R}}\hat a^\pdag_{\alpha{\bm R'}}
    =
    T_{\alpha,i,j}
    \hat a^\dag_{\alpha{\bm R}_i}\hat a^\pdag_{\alpha{\bm R}_j},
    \label{eq:tight_binding_hamiltonian}
\end{align}
for which I conveniently use the sum convention.
The matrix $T_{\alpha}$ contains all amplitudes of transition processes between two lattice centers, to be determined through the dispersion relation of the $\alpha$-band
\begin{align}
    T_{\alpha,i,j} \coloneqq \frac1N\sum_{\bm k}\re^{\ri{\bm k}\brlr{{\bm R}_i-{\bm R}_j}}\varepsilon_{\alpha{\bm k}} = \frac1N\sum_{\bm k}\re^{\ri{\bm k}\brlr{{\bm R}_i-{\bm R}_j}}\int\rd^dr\, \psi_{\alpha{\bm k}}^* \hat H_0 \psi_{\alpha{\bm k}}.
\end{align}
On an intuitive level, the tight binding Hamiltonian denotes quantum particles hopping between lattice sites, connected through the matrix elements $T_{\alpha,i,j}$.
Without quantitative computations of the transition probabilities, we can fix the matrix to $T_{\alpha,i,j}=\sum_rt_{\alpha r}\delta_{i-r,j} + \hc$ with some constants $t_{\alpha r}$.
The relation \cref{eq:tight_binding_hamiltonian} has strong implications on the analytic form of the dispersion relation $\varepsilon_{\alpha{\bm k}}$ -- it is fully determined through the geometry of the crystal lattice.
For instance, one-dimensional lattices with single atom unit cells and lattice spacing $a$ have the particularly easy solution
\begin{align}
    \varepsilon_{\alpha k} = \sum_r 2t_{\alpha r}\cos(r ka).
    \label{eq:1D_tight_binding_dispersion}
\end{align}
Note that this equation is consistent with the Kronig-Penney dispersion relation in the tight-binding limit, presented in \cref{eq:kronig_penney_tight_binding_dispersion}.
In higher dimensions, the evaluation of the dispersion relation may become lengthy, but remains always analytic.

Oftentimes, a single-band approximation is established to fix (and drop) the band index $\alpha$.
This situation may be achieved in case the bottom band is sufficiently separated from the second.
In the Kronig-Penney model, this corresponds to $V_0\gg \hbar^2/(mab)$ with particle mass $m$, lattice constant $a$ and potential width $b$.
For the emergence and discussion of such an approximation in case of ultracold atoms trapped in optical lattices, I refer to~\cite{Jaksch1998,Bloch2008,Buechler2010,Mazza2012}.
Note that the single band approximation naturally fails do describe effects like orbital-selective Mott transitions which require at least two inequivalent bands~\cite{Anisimov2002,vanDongen2005}.

A corresponding two-body interaction $\hat V$ in the single-band approximation reads
\begin{align}
    \hat V = \frac12\sum_{i,i',j,j'}V_{i,i',j,j'}\hat a^\dag_{{\bm R}_i}\hat a^\dag_{{\bm R}_{i'}}\hat a^\pdag_{{\bm R}_j}\hat a^\pdag_{{\bm R}_{j'}}.
\end{align}
The matrix elements of the interaction are given by the integral expressions
\begin{align}
    V_{i,i',j,j'} = \int\rd^dr\int\rd^dr'\,w^*_{{\bm R}_i}({\bm r})w^*_{{\bm R}_{i'}}({\bm r}')w_{{\bm R}_j}({\bm r}')w_{{\bm R}_{j'}}({\bm r})V({\bm r-\bm r'}).
    \label{eq:two_body_interaction_transition_rates}
\end{align}
The explicit determination of the matrix elements requires knowledge of the form of the Wannier states which is an active field of research on its own.
However, these states are localized and as such the transition rate integrals are short-ranged in most cases.
In one dimension, it is often sufficient to account for transitions and interactions up to nearest neighbors
\begin{align}
    T_{i,j} \approx t_0\delta_{i,j} + t_1\delta_{i-1,j}+t_1^*\delta_{i+1,j},
    \quad
    V_{i,i',j,j'} \approx U\delta_{i,j'}\delta_{i',j}\delta_{i,i'} + V\delta_{i,j'}\delta_{i',j}(\delta_{i-1,i'}+\delta_{i+1,i'}).
\end{align}
Note that in case the particle has some flavor (e.g. spin), the kinetic terms of $\hat H_0$ typically act in a diagonal manner.
On the contrary, the two-body interaction allows for both intra-flavor and inter-flavor scattering processes.

We are at liberty to slightly simplify the notation and arrive at the family of single-orbital Hubbard-type models, written in second quantization
\begin{align}
    \hat H_{\rm Hubbard} = \sum_{i,s}\brlr{t_1 \hat a^\dag_{i,s}\hat a^\pdag_{j+1,s} + \hc} + \frac U2\sum_{i,s,s'}\hat n_{i,s}\hat n_{i,s'} + V\sum_{i,s,s'}\hat n_{i,s}\hat n_{i+1,s'}.
    \label{eq:hubbard_hamiltonian}
\end{align}
The operator $\hat a_{i,s}$ annihilates a Wannier state of flavor $s$, localized around the $i$'th lattice position.
In most cases, $t_1 = -t$ with $t>0$, which is also physically motivated to confine the minimum of the dispersion to $k=0$.
The validity of the single-band approximation for real materials requires small interaction amplitudes compared to the band gap between the first and second Bloch band.

If the matrix $T$ defines a lattice grid with loops, complex transition amplitudes give rise to nontrivial fluxes penetrating the lattice which may alter the underlying physics.
The famous Peierls' substitution approximates the effect of a uniform magnetic field, in which case the Hamiltonian follows the principle of minimal coupling
\begin{align}
    \hat H_0 = \frac{\hat{\bm p}^2}{2m}+V_{ae}(\bm r)\rightarrow \frac{\brlr{\hat{\bm p} - q\bm A(\bm r)}^2}{2m}+V_{ae}(\bm r),
\end{align}
and $q$ denotes the electric charge of the particle.
Consider now the rescaled Wannier functions
\begin{align}
    w'_{\alpha\bm R} = \re^{\ri\frac q\hbar G_{\bm R}}w_{\alpha\bm R},
    \quad
    \psi'_{\alpha\bm k} = \frac1N\sum_{\bm R}\re^{\ri\bm k\bm R}w'_{\alpha\bm R}
\end{align}
in which $G_{\bm R}(\bm r) = \int_{\bm R}^{\bm r}\bm A(\bm r')\cdot\rd \bm r'$ is a lattice-position and position dependent phase.
In certain situations, this provides a neat substitution to treat the troublesome $\hat{\bm p}\bm A$ term.
In fact, $\bm\nabla G_{\bm R}(\bm r) = \bm A(\bm r) + \int_0^1\rd\lambda\lambda (\bm r-\bm R)\times\bm B(\bm R+\lambda(\bm r-\bm R))$ with magnetic field $\bm B=\bm\nabla\times\bm A$~\cite{Luttinger1951}.
To observe the influence of the magnetic field, the transition elements between different Wannier states have to be computed, i.e.
\begin{align}
    % \varepsilon_{\alpha\bm k}
    % =
    % \frac1N\sum_{\bm R,\bm R'}\re^{\ri\bm k(\bm R'-\bm R)}
    % \int\rd^d r
    %     w_{\alpha\bm R}^*
    %     \re^{-\ri\frac q\hbar\int_{\bm R}^{\bm r}\bm A(\bm r')\cdot\rd \bm r'}
    %     \brlr{\frac{\brlr{\hat{\bm p}-q\bm A}^2}{2m}+V_{ae}}
    %     w_{\alpha \bm R'}
    %     \re^{\ri\frac q\hbar\int_{\bm R'}^{\bm r}\bm A(\bm r')\cdot\rd \bm r'}
    % \\
    % =
    % \frac1N\sum_{\bm R,\bm R'}\re^{-\ri\frac q\hbar\int_{\bm R}^{\bm R'}\bm A(\bm r')\cdot\rd \bm r'}\re^{\ri\bm k(\bm R'-\bm R)}
    % \int\rd^d r
    %     \re^{\ri\frac q\hbar\Phi_{\bm R,\bm R'}(\bm r)}
    %     w_{\alpha\bm R}^*
    %     \brlr{\frac{\brlr{\hat{\bm p}-q\bm A+q\bm\nabla G_{\bm R'}}^2}{2m}+V_{ae}}
    %     w_{\alpha \bm R'}
    T_{\alpha,i,j}
    =
    \int\rd^d r
        w_{\alpha\bm R_i}^*
        \re^{-\ri\frac q\hbar G_{\bm R_i}}
        \brlr{\frac{\brlr{\hat{\bm p}-q\bm A}^2}{2m}+V_{ae}}
        w_{\alpha \bm R'}
        \re^{\ri\frac q\hbar G_{\bm R_j}}
    \\
    =
    \re^{-\ri\frac q\hbar\int_{\bm R_i}^{\bm R_j}\bm A(\bm r')\cdot\rd \bm r'}
    \int\rd^d r
        \re^{\ri\frac q\hbar\Phi_{\bm R_i,\bm R_j}(\bm r)}
        w_{\alpha\bm R_i}^*
        \brlr{\frac{\brlr{\hat{\bm p}-q\bm A+q\bm\nabla G_{\bm R_j}}^2}{2m}+V_{ae}}
        w_{\alpha \bm R_j},
\end{align}
in which $\Phi_{\bm R_i,\bm R_j}(\bm r) = \oint_{\bm R_j\rightarrow\bm r\rightarrow\bm R_i}\bm A(\bm r')\cdot\rd \bm r'$ is the flux through the plaquette spanned by $\bm R_i, \bm R_j$ and $\bm r$.
% Now the locality of the Wannier functions is exploited, and the variation of the vector potential on the atomic scale is such that $\nabla G_{\bm R'}(\bm r)\approx\bm A(\bm r)$, which leads to
% \begin{align}
%     \varepsilon_{\alpha\bm k}
%     \approx
%     \frac1N\sum_{\bm R,\bm R'}\re^{\ri\frac q\hbar\int_{\bm R'}^{\bm R}\bm A(\bm r')\cdot\rd \bm r'}\re^{\ri\bm k(\bm R'-\bm R)}
%     \int\rd^d r
%         \re^{\ri\frac q\hbar\Phi_{\bm R,\bm R'}(\bm r)}
%         w_{\alpha\bm R}^*
%         \hat H_0
%         w_{\alpha \bm R'}
%     .
% \end{align}
Expanding to first order in $B=|\bm B|$, the transition elements are dressed with a complex phase factor and result to the famously known Peierls approximation
\begin{align}
    T_{\alpha,i,j}'=\re^{\ri\frac q\hbar\int_{\bm R_j}^{\bm R_i}\bm A(\bm r')\cdot\rd\bm r'}T_{\alpha,i,j}+\mathcal O(B).
\end{align}
The transition elements correspond to the change in the Berry phase on the lattice scale~(for a discussion and implications of this effect, see \cref{ch:topological_phases_of_matter}).
Note that the amplitudes of the transitions are altered as well, which must be considered in the implementation of real experiments~\cite{Alexandrov1991}.

Perhaps the most successful description of electrons in solids is band theory -- based on screened many-body interactions described by effective one-body potentials.
This results in a form of \cref{eq:hubbard_hamiltonian} without density-density interactions such that a diagonalization of the hopping matrix $T$ solves the problem.
However, due to its intrinsic single-particle character, band theory cannot reliably capture truly many-body features such as band magnetism or Mott-to-metal-insulator transitions.
This motivates the study of Hubbard-type models: despite being a brutal simplification of the true two-body interaction, they are the simplest systems that provide an explanation of such interaction-driven features.
Despite its apparent innocence, there is no universal treatment of Hubbard models in general.
In one spatial dimension, the Hamiltonian can be solved analytically using the Bethe Ansatz and thus falls in the category of being ``integrable''~\cite{Essler2005}.
In general, Bethe Ansatz integrability is a fragile property and even slight perturbations will break it.
The models we study in the next chapters, albeit closely related to the one-dimensional Hubbard model, do not categorize as integrable due to the presence of additional terms that break some of its fundamental symmetries (e.g. the conservation of the particle flavor).
This motivates the use of alternative methods to study the properties of interacting tight binding models.
One of the most prominent and useful analytic concepts in one dimension is the theory of Luttinger liquids (paired with renormalization group theory) which approximates the microscopic model in the low temperature limit.
%
%
%%%%%%%%%%%%%%%%%%%%%%%%%%%%%%%%%%%%
\section{Tomonaga-Luttinger liquids}
\label{sec:tomonaga_LL}
%%%%%%%%%%%%%%%%%%%%%%%%%%%%%%%%%%%%
Tomonaga-Luttinger liquids are gapless interacting states of matter appearing in many one-dimensional quantum systems.
Their understanding was formalized at the beginning of the '80s by Duncan Haldane~\cite{Haldane1981} through the application of a technique called Abelian bosonization.
It was understood that the low-lying excitations of these fermionic interacting models can be approximated with free bosons.
In this section, I give a basic overview of the mapping from spinless fermions to bosonic particle-hole excitations in the vicinity of the Fermi points.

The main interest here is to achieve an effective model in one dimension capturing the relevant degrees of freedom at low temperatures and low energies.
For that purpose, let me start with the free fermionic 1D Hamiltonian $\hat H_0$ denoted by the following (diagonal) representation in momentum space
\begin{align}
    \hat H_0 = \sum_k \frac{(\hbar k)^2}{2m}\hat n_k
    \label{eq:hamiltonian_free_particles}
\end{align}
with dispersion relation $\varepsilon_k =\frac{k^2}{2m}$ depicted in \cref{fig:1D_quadratic_dispersion}.
\begin{figure}
    \centering
    \includegraphics{figures/1D_quadratic_dispersion.png}
    \caption{Quadratic dispersion relation with approximations close to the Fermi energy $\varepsilon_F$.}
    \label{fig:1D_quadratic_dispersion}
\end{figure}
Close to the Fermi energy $\varepsilon_F\coloneqq\varepsilon_{\pm k_F}$, one may linearize the free dispersion to obtain a system of two different species
\begin{align}
    \hat H_0
    &\approx \sum_k \hbar^2\brlr{\varepsilon_F \pm \frac{k_F}{m}(k\mp k_F)}\hat n_k
    \\
    &= \sum_q \hbar^2\brlr{\varepsilon_F + \frac{k_F}{m}q\brlr{\hat n_{q+k_F}-\hat n_{q-k_F}}}
    \approx \sum_{q} v_F \hbar q \brlr{\hat c^\dag_{q,R}\hat c^\pdag_{q,R} - \hat c^\dag_{q,L}\hat c^\pdag_{q,L}},
    \label{eq:dispersion_linearization}
\end{align}
which implies a restriction of $q$ to a small window $|q|<\Gamma\ll k_F$ beyond which \cref{eq:dispersion_linearization} is considered invalid.
Note the introduction of the so-called right and left operators $\hat c_{R/L,q}$ which annihilate particles propagating to the left/right with Fermi velocity $\pm v_F=\hbar k_F/m$.

For the next part, it is convenient to investigate the local density in momentum space, i.e.
\begin{align}
    \hat n(x) = \hat c^\dag_x \hat c^\pdag_x = \frac1L\sum_{k,q}\re^{-\ri x q}\hat c^\dag_{k+q}\hat c^\pdag_{k}.
    \label{eq:local_density}
\end{align}
$\hat n(x)$ thus creates a superposition of particle-hole plane waves with characteristic wavelength $q^{-1}$.
The bare creation of particle-hole pairs is defined through the density operator
\begin{align}
    \hat \rho_{-q}\coloneqq \sum_k \hat c^\dag_{k-q}\hat c^\pdag_{k}.
\end{align}
Note further that $\hat\rho_q^\dag = \hat\rho_{-q}$.
By confining the theory close to the Fermi points, there are only two different classes of particle-hole excitations with $q\approx 0$ and $q\approx\pm2k_F$.
The long-wavelength excitations $q\approx0$ are particle-hole pairs of the same species (left or right movers), and excitations of $q\approx\pm2k_F$ are particle-hole pairs of a mixture of the two.
This implies drastic consequences for the relevant action of operators, which we will see in the following.
Let us note here that the density operator of the left/right species $\hat\rho_{\tau,q}$, $\tau\in\{L,R\}$ applied to a Fermi sea creates stable particle-hole excitations (i.e. particles and holes propagate with the same velocity $\pm v_F$) and can thus be used to construct a complete basis of the subspace $\FS^N$ -- for this rather dry discussion, I refer to~\cite{vonDelft1998}.
The consequences of the approximation in \cref{eq:dispersion_linearization} is easily understood for the single-particle operators
\begin{align}
    \hat c^\dag_x = \frac1{\sqrt L}\sum_k \re^{-\ri k x}\hat c^\dag_k \approx \frac1{\sqrt L}\sum_{|q|<\Gamma}\re^{-\ri (q+k_F) x}\hat c^\dag_{q,R} + \re^{-\ri (q-k_F) x}\hat c^\dag_{q,L}
    \label{eq:confinement_creation_annihilation}
\end{align}
which is then used to find the local density
\begin{align}
    \hat n(x)
    &\approx \frac1L\sum_{q,q'}\brlr{\re^{-\ri (q+k_F) x}\hat c^\dag_{R, q} + \re^{\ri (k_F-q) x}\hat c^\dag_{L, q}}\brlr{\re^{\ri (k_F+q') x}\hat c^\pdag_{R, q'} + \re^{-\ri (k_F-q') x}\hat c^\pdag_{L, q'}},
    \\
    &= \hat \rho_R(x) + \hat \rho_L(x) + \re^{-2\ri k_Fx}\hat c^\dag_{R}(x)\hat c^\pdag_{L}(x) + \re^{2\ri k_Fx}\hat c^\dag_{L}(x)\hat c^\pdag_{R}(x).
    \label{eq:local_density_approximation}
\end{align}
The first two terms correspond to the $q\approx0$ part of the density, and scattering occurs on the same side of the dispersion relation.
The last two terms scatter right with left movers and transfer particles from one side to the other, which appear at $q\approx\pm2k_F$.

We now turn to an arbitrary two-body interaction of the form~\cref{eq:two_point_interaction} which reads
\begin{align}
    \hat V
    % &= \frac12\int\rd x'\int\rd x V(x'-x)c^\dag(x')c^\dag(x)c^\pdag(x)c^\pdag(x')
    % \\
    &= \frac12\int\rd r\int\rd x V(r)\hat c^\dag(r+x)\hat c^\dag(x)\hat c^\pdag(x)\hat c^\pdag(r+x),
    \\
    &= \frac1{2L^2}\sum_{kk'll'}\int\rd r\int\rd x V(r)\re^{-\ri r(k-l')}\re^{-\ri x(k-l'+k'-l)}\hat c^\dag_k\hat c^\dag_{k'}\hat c^\pdag_l\hat c^\pdag_{l'},
    % \\
    % &= \frac1{2L}\sum_{kk'lq}\int\rd r V(r)\re^{-\ri rq}\delta_{l,k'+q}c^\dag_kc^\dag_{k'}c^\pdag_lc^\pdag_{k-q}
    \\
    &= \frac1{2L}\sum_{kk'q}V(q)\hat c^\dag_k\hat c^\dag_{k'}\hat c^\pdag_{k'+q}\hat c^\pdag_{k-q}
    = \frac1{2L}\sum_{q}V(q)\hat\rho_q\hat\rho^\dag_{q} - \mu.
    \label{eq:two_body_interaction_momentum_space}
\end{align}
The last term is just a constant $\mu = \frac N{2L}\sum_qV(q)$ and can thus be neglected.
By imposing that relevant contributions act close to the Fermi energy involving only momenta in the interval $|k|\in[k_F-\Gamma,k_F+\Gamma]$, we can split the sum in two contributions, one involving scattering processes at small and the other scattering at large momenta
\begin{align}
    \hat V \approx \frac1{2L}\brlr{\sum_{q\approx0}V(q)\hat\rho_{q}\hat\rho^\dag_{q} + \sum_{q\approx2k_F}V(q)\hat\rho_{q}\hat\rho^\dag_{q}}.
\end{align}
% For the study presented in \cref{drude_increased1}, it is important to note that the amplitude of the two different $q\approx0$ processes, denoted by $V(q\approx0)$, is equal and independent of the form of the interaction.
An intuitive classification of the scattering processes is given in~\cref{fig:scattering_processes}.
\begin{figure}
    \centering
    \subfigure[]{\includegraphics[width=0.328\textwidth]{figures/g4_right.pdf}}
    \subfigure[]{\includegraphics[width=0.328\textwidth]{figures/g4_left.pdf}}
    \subfigure[]{\includegraphics[width=0.328\textwidth]{figures/g2.pdf}}
    \subfigure[]{\includegraphics[width=0.328\textwidth]{figures/g1.pdf}}
    \subfigure[]{\includegraphics[width=0.328\textwidth]{figures/g13.pdf}}
    \subfigure[]{\includegraphics[width=0.328\textwidth]{figures/g22.pdf}}
    \caption{Relevant scattering processes of a generic density-density interaction in one-dimensional quantum systems. (a)/(b) The depicted scattering is commonly referred to as ``forward scattering'' $g_4$ process ($4$ right/left operators, $q\approx0$), (c) as ``backscattering'' $g_2$ process (containing $2$ pairs of right and left operators, $q\approx0$) and (d) as $g_1$ process with momentum transfer $q\approx 2k_F$. Other possible scatterings like the ones depicted in (e) and (f) require the existence of high-energy excitations and are thus exponentially suppressed at low temperatures.}
    \label{fig:scattering_processes}
\end{figure}
This simple argumentation allows to consider only the most relevant processes at low temperatures/energies, i.e. those presented in panels (a) - (d).
We will call those processes forward scattering
\begin{align}
    g_4\sum_{k,k',q}\hat c^\dag_{k+q,\tau}\hat c^\dag_{k'-q,\tau}\hat c^\pdag_{k',\tau}\hat c^\pdag_{k,\tau} = g_4 \sum_q\hat\rho^\pdag_{q,\tau}\hat\rho^\dag_{q,\tau},
\end{align}
backscattering
\begin{align}
    g_2\sum_{k,k',q}\hat c^\dag_{k+q,\tau}\hat c^\dag_{k'-q,\overline\tau}\hat c^\pdag_{k',\overline\tau}\hat c^\pdag_{k,\tau} = g_2\sum_q \hat\rho^\pdag_{q,\tau}\hat\rho^\dag_{q,\overline\tau}
\end{align}
and $g_1$ scattering
\begin{align}
    g_1\sum_{k,k',q} \hat c^\dag_{k+q,\tau}\hat c^\dag_{k'-q,\overline\tau}\hat c^\pdag_{k',\tau}\hat c^\pdag_{k,\overline\tau}.
\end{align}
Note that we are safely discarding possible constants on the right hand side of the previous equations.
The $g_1$ scattering of spinless fermions can be rewritten according to
\begin{align}
    \sum_{k,k',q}V(q)
    \hat c^\dag_{k+q,\tau}\hat c^\dag_{k'-q,\overline\tau}\hat c^\pdag_{k',\tau}\hat c^\pdag_{k,\overline\tau}
    =
    \sum_{k,k',q,q'}V(q) \delta_{q,q'-k+k'}
    \hat c^\dag_{k'+q',\tau}\hat c^\dag_{k-q',\overline\tau}\hat c^\pdag_{k',\tau}\hat c^\pdag_{k,\overline\tau}
    \\
    =
    -\sum_{k,k',q}V(k'-k+q)\hat c^\dag_{k+q,\tau}\hat c^\dag_{k'-q,\overline\tau}\hat c^\pdag_{k',\overline\tau}\hat c^\pdag_{k,\tau}
    \approx -V(2k_F)\sum_q\hat\rho^\pdag_{q,\tau}\hat\rho^\dag_{q,\overline\tau},
    \label{eq:umklapp_backscattering_equivalence}
\end{align}
which proofs that $g_1$ terms are up to a sign equivalent to backscattering terms in case of indiscernible particles.
Interestingly, the $2k_F$ components of forward scatterings can be reordered in the same manner\footnote{This is interesting because one is inclined to neglect the $2k_F$ forward scattering processes at first sight. The reordering demonstrates that $q\approx0$ are equivalent to $2k_F$ scatterings such that both have to be accounted for on equal grounds.}, such that we find the identity $g_4 = g_2 \approx V(0)-V(2k_F)$.
% Umklapp terms cannot be expressed in the density operators of the moving modes $\hat\rho_{q,\tau}$ and, for this reason, are discarded here\footnote{Later, we will see that the Umklapp term in the spinless case are ``marginal'' and can be disregarded under certain conditions.}.
Other processes like those depicted in \cref{fig:scattering_processes} (e) and (f) require the existence of high-energy excitations and can thus be neglected for the effective low temperature theory developed here.
In summary, we can express the interaction in terms of the density-density operators as
\begin{align}
    \hat V \approx \frac{\hbar}{2L}\sum_{q,\tau}\brlr{g_4\hat\rho^\pdag_{q,\tau}\hat\rho^\dag_{q,\tau} + g_2\hat\rho^\pdag_{q,\tau}\hat\rho^\dag_{q,\overline\tau}}
    =
    \frac{\hbar}{L}\sum_{q>0}
    \begin{pmatrix}
        \hat\rho^\pdag_{q,R} & \hat\rho^\pdag_{q,L}
    \end{pmatrix}
    \begin{pmatrix}
        g_4 & g_2 \\
        g_2 & g_4
    \end{pmatrix}
    \begin{pmatrix}
        \hat\rho_{-q,R} \\ \hat\rho_{-q,L}
    \end{pmatrix}
    .
    \label{eq:interaction_densities}
\end{align}
To proceed further, it will be useful to compute the commutation of the density operators
\begin{align}
    \commutator{\hat\rho_{q,\tau},\hat\rho_{q',\tau'}}
    =
    \delta_{\tau,\tau'}\sum_k\brlr{\hat c^\dag_{k+q,\tau}\hat c^\pdag_{k-q',\tau}-\hat c^\dag_{k+q+q',\tau}\hat c^\pdag_{k,\tau}}
    \approx
    -\sigma_\tau\delta_{\tau,\tau'}\delta_{q,-q'}\frac{qL}{2\pi}
    \label{eq:chiral_density_commutation}
\end{align}
in which $\sigma_\tau=\pm1$ for $\tau=R/L$, respectively, and the right hand side is obtained through a projection on the ground state\footnote{This is a reasonable approximation for small interactions only (as a result from a first order perturbative expansion).
For a more thorough discussion of \cref{eq:chiral_density_commutation}, see~\cite{Giamarchi2003}.
Although this approximation may turn out to be too rough for real systems, it allows to obtain the Luttinger liquid field theory.}.
We are now at liberty to define canonical bosonic operators representing the interaction degrees of freedom for $q>0$
\begin{align}
    \hat b^\dag_{+q} \coloneqq \sqrt\frac{2\pi}{qL}\hat\rho_{-q,L},
    \quad
    \hat b^\pdag_{+q} \coloneqq \sqrt\frac{2\pi}{qL}\hat\rho_{+q,L},
    \\
    \hat b^\dag_{-q} \coloneqq \sqrt\frac{2\pi}{qL}\hat\rho_{+q,R},
    \quad
    \hat b^\pdag_{-q} \coloneqq \sqrt\frac{2\pi}{qL}\hat\rho_{-q,R},
    \label{eq:canonical_bosonic_operators}
\end{align}
that satisfy the commutation relation $\commutator{\hat b_q^\pdag,\hat b_{q'}^\dag} = \delta_{q,q'}$.
Using \cref{eq:canonical_bosonic_operators} results in a familiar expression for the interaction written in \cref{eq:interaction_densities}, i.e.
\begin{align}
    \hat V \approx\sum_{q>0}\frac{\hbar q}{2\pi}
    \begin{pmatrix}
        \hat b_q & \hat b^\dag_{-q}
    \end{pmatrix}
    \begin{pmatrix}
        g_4 & g_2 \\
        g_2 & g_4
    \end{pmatrix}
    \begin{pmatrix}
        \hat b^\dag_q \\ \hat b_{-q}
    \end{pmatrix}
    .
    \label{eq:quadratic_interactions}
\end{align}
The interaction, originally quartic in the fermionic degrees of freedom, can be cast into a (quadratic) sum of bosonic operators which are the low-energy excitations of the original model.
All that is left to do is to cast the kinetic term into this new basis, which is actually a quite lengthy calculation if we were to approach it by brute-force.
There is an indirect reasoning through Schur's lemma~\cite{schur_neue_1905}: if two operators $\hat H$ and $\hat H'$ have identical commutation relations with all $\{c^\pdag_\alpha,c^\dag_\alpha\}$, then the two operators are equal up to an overall constant, which in this case can be interpreted as a chemical potential.
One can easily compute the commutator of the mover-density with the kinetic Hamiltonian
\begin{align}
    \commutator{\hat H_0, \hat\rho_{q,\tau}}
    =
    \sum_{p,\tau'}v_F\sigma_{\tau'}\hbar \commutator{\hat n_{p,\tau'},\hat\rho_{q,\tau}}
    =
    \sigma_\tau v_F \hbar q \hat\rho_{q,\tau}
\end{align}
and find its equivalent expression in bosonic degrees of freedom to be
\begin{align}
    \hat H_0' = \frac{\hbar \pi v_F}L\sum_{q,\tau}\hat\rho_{q,\tau}\hat\rho_{-q,\tau} = \frac{2\hbar\pi v_F}L\sum_{q>0,\tau}\hat\rho_{q,\tau}\hat\rho_{-q,\tau},
\end{align}
which can be verified through evaluation of
\begin{align}
    \commutator{\hat H_0',\hat\rho_{q,\tau}}
    \overset{\text{\cref{eq:recursive_commutation}}}{=}
    -\frac{\hbar \pi v_F}L
    \sum_{p,\tau'}
    \brlr{
    \commutator{\hat\rho_{q,\tau},\hat\rho_{p,\tau'}}\hat\rho_{-p,\tau'}
    -
    \hat\rho_{p,\tau'}\commutator{\hat\rho_{q,\tau},\hat\rho_{-p,\tau'}}
    }
    \overset{\text{\cref{eq:chiral_density_commutation}}}{=}
    \sigma_\tau v_F \hbar q\hat\rho_{q,\tau}
\end{align}
and thus we conclude our previous statement $\hat H_0' = \hat H_0 + \mu$ with an irrelevant constant $\mu$.
The effective low energy Hamiltonian containing kinetic and interaction energy satisfies the following matrix equation
\begin{align}
    \hat H = \hat H_0 + \hat V \approx
    \sum_{q>0}\frac{\hbar q}{2\pi}
    \begin{pmatrix}
        \hat b_q & \hat b^\dag_{-q}
    \end{pmatrix}
    \begin{pmatrix}
        2\pi v_F + g_4 & g_2 \\
        g_2 & 2\pi v_F + g_4
    \end{pmatrix}
    \begin{pmatrix}
        \hat b^\dag_q \\ \hat b_{-q}
    \end{pmatrix}
    .
    \label{eq:luttinger_hamiltonian_nondiagonal}
\end{align}
As a final step, we want to find the spectrum of the previous Hamiltonian through a basis transformation $\hat B_q = T \hat B'_q$\footnote{The matrix coupling the dot product of the operator spinors $\hat B_q$ is independent on the momentum $q$ and as such the basis transformation $T$ will not depend on $q$ either.} with $\hat B_q = (\hat b_q^\dag, \hat b_{-q})^T$ such that
\begin{align}
    \hat H = \sum_{q>0}\hbar q\hat B^\dag_q H \hat B^\pdag_q = \sum_{q>0}\hbar q\hat B'^\dag_q T^\dag H T \hat B'^\pdag_q
\end{align}
is in its diagonal form.
Naive (unitary) rotations do not preserve the commutators of the spinor $\hat B$, defined through
\begin{align}
    \commutator{\hat B_{q,i}^\pdag,\hat B^\dag_{q,j}} = \commutator{\hat B_{q,i}'^\pdag,\hat B'^\dag_{q,j}} = (-\sigma_z)_{i,j},
\end{align}
which imposes an additional constraint on the transformation $T$ according to
\begin{align}
    T^\dag\sigma_z T = \sigma_z,
    \quad
    T^\dag = \sigma_zT^{-1}\sigma_z.
\end{align}
We thus find the similarity relation between the original and the rotated basis according to
\begin{align}
    H'=T^\dag HT =\sigma_z T^{-1}\sigma_z H T,
\end{align}
which allows us to solve the eigenvalue equation of $\sigma_z H$ without knowing the explicit form of $T$.
Since $\sigma_zH$ has vanishing trace, its spectrum is symmetric $\pm E$ and we arrive at the appealing result $H' = u\mathbb1$ with the $2\times2$ unit matrix $\mathbb1$ and scalar eigenvalue
\begin{align}
    u = \frac1{2\pi}\sqrt{\brlr{2\pi v_F + g_4}^2 - g_2^2},
\end{align}
leading to the identity
\begin{align}
    \hat H = \sum_{q>0} u\hbar q \hat B'^\dag_q \hat B'^\pdag_q = \sum_{q>0} \hbar \omega_q \hat B'^\dag_q \hat B'^\pdag_q = \sum_q \hbar  \omega_q \hat b'^\dag_q\hat b'^\pdag_q.
\end{align}
Note that we succeeded to rewrite the original problem as a sum of decoupled harmonic oscillators with frequencies $\omega_q\coloneqq u|q|$.
Assuming two conjugate fields $\commutator{\hat \phi_{k},\hat \pi_{k'}} = \ri\pi\hbar\delta_{k,k'}$, the substitution of ladder operators,
\begin{align}
    \hat b'_q = \sqrt{\frac{mw_q}{2\pi\hbar}}\brlr{\hat\phi_q + \frac{\ri}{mw_q}\hat\pi_{-q}},
    \quad
    \hat b'^\dag_q = \sqrt{\frac{mw_q}{2\pi\hbar}}\brlr{\hat\phi_{-q} - \frac{\ri}{mw_q}\hat\pi_{q}},
\end{align}
brings the diagonalized Hamiltonian to the more traditional form
\begin{align}
    \hat H = \sum_q \frac{\hat\pi_q\hat\pi_{-q}}{2\pi m} + \frac{m\omega_q^2}{2\pi}\hat\phi_q\hat\phi_{-q}
    =
    \int\frac{\rd x}{2\pi}\, \brlr{\hat\pi^2/m + {Da^2}(\partial_x\hat\phi)^2}.
\end{align}
In the above, I neglected a constant term ($-\sum_q\omega_q/2\pi$) and introduced an effective spring constant through the velocity and lattice spacing $D=mu^2/a^2$.

The field $\partial_x\hat\phi$ can thus be interpreted as the ``position offset'' from the equilibrium, which in this case translates to density fluctuations around the equilibrium density $n=k_F/\pi$.
I thus make the connection
\begin{align}
    \int\rd x\hat n = \frac1\pi\brlr{k_Fx - \hat\phi(x)},
\end{align}
which captures the non-oscillatory behavior of \cref{eq:local_density_approximation}.
$\hat\pi$ is the canonical momentum of the density fluctuations, i.e., a charge current.
Moving to dimensionless units (by imposing $\hbar=c=1$ and requiring a unit mass) and rescaling the fields $\hat\phi\rightarrow \frac1{\sqrt{uK}}\hat\phi$, $\hat\pi\rightarrow \sqrt{uK}\hat\pi$ results in the standard Luttinger liquid Hamiltonian
\begin{align}
    \hat H = \int\frac{\rd x}{2\pi}\, \brlr{uK\hat\pi^2 + \frac uK(\partial_x\hat\phi)^2},
    \label{eq:ll_hamiltonian}
\end{align}
in which the couplings are encoded in the dimensionless Luttinger parameters
\begin{align}
    u = v_F\sqrt{\brlr{1 + y_4}^2 - y_2^2},
    \quad
    K = \sqrt{\frac{1+y_4-y_2}{1+y_4+y_2}},
    \quad
    y_4=g_4/(2\pi v_F),
    \quad
    y_2=g_2/(2\pi v_F).
\end{align}
On an intuitive level (given the demonstrated equivalence to the quantum harmonic oscillator), the relation to the local densities is now straightforward:
\begin{align}
    \partial_x\hat\phi(x)=-\pi[\hat\rho_R(x)+\hat\rho_L(x)],
    \quad
    \hat\pi(x)\coloneqq\partial_x\hat\theta=\pi[\hat\rho_R(x)-\hat\rho_L(x)].
\end{align}
Note that the chosen prefactor $\pm\pi$ is only a matter of convention and convenience.
A rigorous derivation of the field operators in terms of the local creation and annihilation operators is presented in several excellent books on the topic, e.g.~\cite{Bruus2004,Giamarchi2003,Gogolin2004}.

These identities, combined with \cref{eq:local_density_approximation} are the basic ingredient to implement the concept of Luttinger liquids to a given (density-density) interaction and find the model-specific coupling constants $g_{4/2}$ on a practical level.
The relation to the annihilation operators (in the continuum) is maybe less obvious,
\begin{align}
    \hat c_{R/L}(x) = \frac1{\sqrt{2\pi\alpha}}\exp\brlr{\pm\ri k_F x\mp\ri\hat \phi(x)}\re^{\ri\hat\theta(x)}\hat \eta_{R/L},
    \label{eq:bosonization_identity}
\end{align}
in which the operator identity is to be understood in the limit $\alpha\rightarrow0$, corresponding to a regularized momentum cutoff~\cite{Bruus2004}.
Note that the field operators commute with the total density, and as such the so-called Klein factors $\hat\eta_{R/L}$ are introduced to connect different Fock spaces.

On a more intuitive level, the identity above shares a fundamental equivalence to the famous Jordan-Wigner transformation~\cite{Jordan1928}.
The exact identity
\begin{align}
    \hat c^\pdag_x=\exp\brlr{\ri\pi\sum_{x'<x}\hat \sigma^+_{x'}\hat \sigma^-_{x'}}\hat \sigma^-_x
    ,\quad
    \anticommutator{\hat c^\dag_x,\hat c^\pdag_{x'}}=\delta_{x,x'}
    ,\quad
    \commutator{\hat \sigma^+_x,\hat \sigma^-_{x'\neq x}}=0
    ,
\end{align}
establishes the equality between spinless fermions and spin-1/2 on a lattice.
Preservation of the (anti-)commutation relation is ensured by the exponential factor, the so-called Jordan-Wigner string.
This allows to interpret $\pm\pi\int\rd x\hat n(x) = \pm k_F x\mp\hat\phi(x)$ as the argument of the Jordan-Wigner string.
The Jordan-Wigner transformation is regularly used to (i) find representations of fermionic operators in terms of spin matrices (see \cref{eq:jordan_wigner_trafo}), and (ii) to treat one-dimensional interacting spin models in the fermionic language, in which case they become exactly solvable under some circumstances~\cite{Lieb1961}.
In this sense bosonization is more powerful and extends to a universal treatment of interacting one-dimensional fermions and bosons.
For a more elaborate discussion and the bosonization of bosons see e.g.~\cite{Cazalilla2004}.
% The identity \cref{eq:bosonization_identity} allows to reduce the complexity of any correlation function
% \begin{align}
%     C = \braket{\re^{\ri \sum_j \brlr{a_j \phi(x_j) + b_j \theta(x_j)}}} = \re^{-\frac12\braket{\brlr{\sum_j a_j\phi(x_j)}^2+\brlr{\sum_j b_j\theta(x_j)}^2}}
% \end{align}
% using the average expressions of the two dual bosonic fields
% \begin{align}
%     \braket{\phi(x)\phi(0)} \approx -\frac K2\log(x),
%     \quad
%     \braket{\theta(x)\theta(0)} \approx -\frac 1{2K}\log(x),
%     \label{eq:LL_fields_correlations}
% \end{align}
% as derived from Green's functions of the massless Klein-Gordon theory~\cite{AltlandSimons2010}.

Note that the results presented here are actually independent of the original starting point, i.e. \cref{eq:hamiltonian_free_particles}.
Therefore, the fermi velocity $v_F$ can be replaced by arbitrary group velocities evaluated on the Fermi surface $v_F=(\partial_k\varepsilon_k)_{k=k_F}$.
Significant deviations are then expected in case of flat bands (to be more precise, in case of strong interactions comparable with the bandwidth of the kinetic Hamiltonian), in which case all scattering processes of \cref{fig:scattering_processes} have to be considered.
An attempt which takes into account such curvature effects solves many of the related issues~\cite{Imambekov2009}.

The evaluation of all scattering processes is a tough task and regularly neglected, which then leads to expected quantitative deviations.
However, from a qualitative viewpoint, the neglected scatterings (whenever they are irrelevant in the sense of not opening energy gaps, see \cref{sec:renormalization_group_theory}) change only the effective couplings such that physical consequences predicted from the Luttinger liquid theory (like the asymptotic decay of correlation functions) remain valid.
%
%
%%%%%%%%%%%%%%%%%%%%%%%%%%%%%%%%%%%%%%%%%%%%%%%%%%%%%%%%%%%%%%%%%%%
\section{Properties of real scalar fields and their correlations}
\label{sec:properties_of_real_scalar_fields_and_their_correlations}
%%%%%%%%%%%%%%%%%%%%%%%%%%%%%%%%%%%%%%%%%%%%%%%%%%%%%%%%%%%%%%%%%%%
For the evaluation of correlation functions and the later employed renormalization group theory presented in \cref{sec:renormalization_group_theory}, it is beneficial to recap shortly the language of path integrals (which Feynman invented as a graduate student).
During the process of derivation, the Luttinger parameter $K$ is absorbed in the definition of the fields and appears again in the discussion at the end of this section.
The starting point is thus a Hamiltonian that is separable in the fields, i.e. $\hat H(\hat\phi,\hat\pi) = \hat H(\hat \phi) + \hat H(\hat \pi)$.

The goal is to evaluate the so-called time evolution kernel
\begin{align}
    U(x_f,x_i,T) = \braket{\phi(x_f,T)|\phi(x_i,t_i=0)}=\braket{\phi_f|\re^{-\ri\hat H(\hat\phi,\hat\pi) T}|\phi_i}
\end{align}
in which $\ket{\phi}$ is the simultaneous eigenstate of the field operator $\hat\phi(x)\ket{\phi}=\phi(x)\ket{\phi}$.
Instead of taking the full exponential, one may split the time integration into $N$ parts
\begin{align}
    \re^{-\ri\hat HT}=\brlr{\re^{-\ri\hat H\Delta t}}^N
    ,
    \quad
    \Delta t = \frac TN
    ,
\end{align}
and then insert a complete set of states in between each of the factors, i.e. $\mathbb 1=\int\rd\bm\phi\ket{\bm\phi}\bra{\bm\phi}$
\footnote{
    The identity is understood as a continuum limit of the discretized fields $\hat\phi(x,t)\rightarrow{\hat{\bm\phi}}(t)$ in which the vector $\hat{\bm\phi}=(\hat\phi_1,\hat\phi_2,\dots,\hat\phi_{N_L})^T$ contains the lattice support of the continuum field operator $\hat\phi(x_j,t)=\hat\phi_j(t)$ such that $x_j=ja$ and $a=L/N_L$ (same for $\hat\pi(x,t)\rightarrow\hat{\bm\pi}(t)$).
    As such, one obtains a finite set of quantum numbers $\ket{\bm\phi(t)}=\ket{\phi_1(t),\dots,\phi_{N_L}(t)}$ for which each $\ket{\phi_j(t)}$ encodes a ``position'' state and $\ket{\pi_j}$ it's conjugate ``momentum'' state.
    From this analog picture, and for all practical purposes, one may borrow the identities $\braket{\bm\phi(t)|\bm\pi(t')}=\frac1{2\pi}\exp\brlr{\ri\bm\phi(t)\cdot\bm\pi(t')}$ and $\mathbb1=\brlr{\prod_{j=1}^{N_L}\int\rd\phi_j}\ket{\phi_1,\dots,\phi_j}\bra{\phi_1,\dots,\phi_j}=\int\rd\bm\phi\ket{\bm\phi}\bra{\bm\phi}=$ (same for $\hat{\bm\pi}$).
    The continuum is then restored at the end of the calculation by the simultaneous limits $a\rightarrow0$ and $N_L\rightarrow\infty$ such that $aN_L=L$ is preserved at all times.
    To keep this in mind, I now proceed by using bold letters in case of explicit discretization of the corresponding continuum field.
    For a more universal derivation of the many-body path integral using coherent state representations, I refer to \cite{AltlandSimons2010}.
}.
The insertion of this set of states then corresponds to unconstrained integrations over smaller time slices of the full time evolution which in this sense already assumes the form of a path integral, i.e.
\begin{align}
    U(x_f,x_i,T) =
    \int{\rd{\bm\phi}_{N-1}\rd{\bm\phi}_{N-2}\dots\rd{\bm\phi}_2,\rd{\bm\phi}_1}
    \braket{\bm\phi_f|\re^{-\ri\hat H\Delta t}|\bm\phi_{N-1}}
    \braket{\bm\phi_{{N-1}}|\re^{-\ri\hat H\Delta t}|\bm\phi_{{N-2}}}
    \dots
    \nonumber\\
    \braket{\bm\phi_{2}|\re^{-\ri\hat H\Delta t}|\bm\phi_{1}}
    \braket{\bm\phi_{1}|\re^{-\ri\hat H\Delta t}|\bm\phi_i}.
\end{align}
The integration variables $\{{\bm\phi}_{N-1},\dots,{\bm\phi}_1\}$ can be viewed as the field configuration at times $n\Delta t$ and in this sense, the previous expression assumes already an integral over paths.
Defining the transfer ``matrix'' $T_{{\bm\phi}',{\bm\phi}}=\braket{{\bm\phi}'|\re^{-\ri\hat H\Delta t}|{\bm\phi}}$, the time evolution kernel reads
\begin{align}
    U(x_f,x_i,T) =
    \int{\rd{\bm\phi}_{N-1}\rd{\bm\phi}_{N-2}\dots\rd{\bm\phi}_2\rd{\bm\phi}_1}
    T_{{\bm\phi}_f,{\bm\phi}_{{N-1}}}
    T_{{\bm\phi}_{{N-1}},{\bm\phi}_{{N-2}}}
    \dots
    T_{{\bm\phi}_{{2}},{\bm\phi}_{{1}}}
    T_{{\bm\phi}_{{1}},{\bm\phi}_i}.
\end{align}
For the next steps I need the Baker-Campbell-Hausdorff formula
\begin{align}
    \re^{\hat A+\hat B} = \re^{\hat A + \hat B + \frac12\commutator{\hat A, \hat B}+\dots}
\end{align}
which for a pair of canonically conjugate operators is exact at second order.
The separability of the Hamiltonian $\hat H = \hat H(\bm\phi,\bm\pi) = \hat H(\bm\phi)+\hat H(\bm\pi)$ is exploited to expand the exponential to
\begin{align}
    \re^{-\ri\hat H\Delta t}
    % = \re^{-\ri\Delta t\hat H(\bm\phi)}\re^{-\ri\Delta t\hat H(\bm\pi)}+\mathcal O(\Delta t^2)
    = \re^{-\ri\Delta t\hat H(\bm\pi)}\re^{-\ri\Delta t\hat H(\bm\phi)}+\mathcal O(\Delta t^2).
\end{align}
The error in the above is well under control by taking $N$ sufficiently large, and vanishes in the continuum limit.
Insertion of $\mathbb1=\int\rd\bm\pi\ket{\bm\pi}\bra{\bm\pi}$ between the two exponentials yields the approximate form of the transfer matrix,
\begin{align}
    T_{{\bm\phi}',{\bm\phi}} =
    \int\rd\bm\pi
    \braket{{\bm\phi}'|\re^{-\ri\Delta t\hat H(\bm\pi)}\ket{\bm\pi}\bra{\bm\pi}\re^{-\ri\Delta t\hat H(\bm\phi)}|{\bm\phi}} + \mathcal O(\Delta t^2),
    \\
    % =
    % \int\rd\bm\pi\braket{{\bm\phi}'|\bm\pi}\braket{\bm\pi|{\bm\phi}}\re^{-\ri\Delta t H(\frac{\bm\phi+\bm\phi'}2,\bm\pi)} + \mathcal O(\Delta t^2)
    =
    \int\frac{\rd\bm\pi}{2\pi}\re^{\ri\bm\pi^T(\bm\phi'-\bm\phi)}\re^{-\ri\Delta t H(\bm\phi,\bm\pi)} + \mathcal O(\Delta t^2).
\end{align}
Note that we succeeded to replace the operator form of the Hamiltonian by its scalar eigenvalues.
The full time evolution kernel then evaluates to
\begin{align}
    U(x_f,x_i,T) =
    \int_{{\bm\phi_N}_i}^{{\bm\phi_N}_f}
    {\rd\{\bm\phi\}}\int{\rd\{\bm\pi\}}
    \re^{\ri\Delta t\sum_{n=1}^{N}\brlr{\bm\pi_n^T\frac{\bm\phi_{n}-\bm\phi_{n-1}}{\Delta t}-H(\bm\phi_{n-1},\bm\pi_n)}},
    \label{eq:path_integral_pre_integration}
\end{align}
where I conveniently use $\bm\phi_{0/N}\coloneqq \bm\phi_{i/f}$, $\rd\{\bm\phi\}=\prod_{n=1}^{N-1}\rd\bm\phi_n$ and $\rd\{\bm\pi\}=\prod_{n=1}^N\rd\bm\pi_n/2\pi$.
Note that the argument in the exponential clearly corresponds to a discretized version of the Lagrangian.
Now I exploit the fact that $H(\bm\phi,\bm\pi)=au/(2\pi)\bm\pi^T\bm\pi + V(\bm\phi)$ is separable in a quadratic kinetic and a potential term, which allows to perform a Gaussian integration of the momentum fields.
A Gaussian integral of two real vectors $\bm v$, $\bm j$ and a real and invertible matrix $\bm A$ is
\begin{align}
    \int\rd\bm v\re^{-\frac12\bm v^T\bm A\bm v + \bm j^T\bm v} = \frac{(2\pi)^{{\rm rank}(\bm A)/2}}{\sqrt{\det\bm A}}\re^{\frac12\bm j^T\bm A^{-1}\bm j}.
    \label{eq:multi_gaussian}
\end{align}
The argument of \cref{eq:path_integral_pre_integration} is purely imaginary, and as such the integral is convergent with a regularization $\Delta t\rightarrow\Delta t(1+\ri\epsilon)$ by taking the limit $\epsilon\rightarrow0$.
More importantly, the (arbitrary!) vector $\bm j$ acts as a ``source'' term to define an expectation value.
Consider the derivative $\partial_{j_m}\partial_{j_n}|_{\bm j=0}$ acting on the gaussian, it will pull the $m,n$ component of $\bm A^{-1}$ up to an overall constant, i.e.
\begin{align}
    \partial^2_{j_m,j_n}\int\rd\bm v\re^{-\frac12\bm v^T\bm A\bm v + \bm j^T \bm v}\Bigg|_{\bm j=0}
    =
    \int\rd\bm v\re^{-\frac12\bm v^T\bm A\bm v}v_mv_n
    =
    \frac{(2\pi)^{{\rm rank}(\bm A)/2}}{\sqrt{\det(\bm A)}}A^{-1}_{mn}
    \\
    \longrightarrow \braket{v_mv_n}\coloneqq (2\pi)^{-{\rm rank}(\bm A)/2}\sqrt{\det(\bm A)}\int\rd\bm v\re^{-\frac12\bm v^T\bm A\bm v}v_mv_n = A_{mn}^{-1}.
    \label{eq:multi_source}
\end{align}
This suggests an interpretation of the Gaussian weight as a probability distribution.
Iteration of the differentiation operation four times yields $\braket{v_mv_nv_qv_p}=A^{-1}_{mn}A^{-1}_{qp}+A^{-1}_{mq}A^{-1}_{np}+A^{-1}_{mp}A^{-1}_{nq}$ and one obtains the general formula
\begin{align}
    \braket{v_{i_1}v_{i_2}\dots v_{i_{2n}}} =
    \sum_{\tiny\begin{array}{c}\text{pairings of}\\ \{i_1,\dots,i_{2n}\}\end{array}}
    A^{-1}_{i_{k_1}i_{k_2}}\dots A^{-1}_{i_{k_{2n-1}}i_{k_{2n}}}.
\end{align}
In the continuum limit, the set $\bm v$ translates to a function $v(x)$ and the matrix $\bm A$ is replaced by the propagator $A(x,x')$.
Therefore, the natural generalization becomes
\begin{align}
    \int\rD[v]\re^{-\frac12\int\rd x\rd x' v(x)A(x,x')v(x')+\int\rd x j(x) v(x)}
    \propto
    (\det A)^{-1/2}\re^{\frac12\int\rd x\rd x' j(x) A^{-1}(x,x')j(x')}.
    \label{eq:field_source}
\end{align}
% Before we proceed, I must first comment a bit about the form of \cref{eq:field_source}.
When the variables entering the Gaussian integration were discrete, the interpretation of the determinant was straightforward.
In the present case, one must interpret $A$ as a Hermitian operator with an infinite set of eigenvalues, and $\det A$ denotes the product over this infinite set.
Although the constant of proportionality $(2\pi)^{{\rm rank}(\bm A)/2}$ is formally divergent in the continuum limit, it does not affect the averages (by definition), i.e.
\begin{align}
    \braket{v(x)v(x')} = A^{-1}(x,x').
\end{align}

Finally, application of \cref{eq:multi_gaussian} on \cref{eq:path_integral_pre_integration} and then taking the continuum limits yields the path integral
\begin{align}
    U(x_f,x_i,T)
    =
    \int_{\phi(x_i)}^{\phi(x_f)}{\rD[\phi]}
    \re^{\ri\int_0^T\rd tL[\phi,\dot\phi]}
    ,\quad
    L = \int\frac{\rd x}{2\pi u}(\partial_t\phi)^2 - V(\phi).
    \label{eq:kg_lagrangian}
\end{align}
Since constants of proportionality do not enter in the expectation values by definition, they are absorbed into the definition of $\rD$ (which is given implicitly here).
The integration covers all paths through the classical coordinate space spanned by $\phi$ which begin and end at the same initial and final points $\phi_i$ and $\phi_f$, respectively.
Each path is weighted by the corresponding classical action -- note the absence of hats which would denote a quantum mechanical operator.
However, quantum mechanics is still at full presence, as the integration is not restricted to solutions of the classical equations of motion.

The standard definition of the partition function is given as $Z=\tr(\exp(-\beta \hat H))$, and the trace amounts to a summation over all possible configurations of the system.
This makes apparent the interpretation of $t=-\ri \beta$, resulting in the ``Euclidean'' action such that the path integral becomes a quantum mechanical partition function
\begin{align}
    Z = \tr(\exp(-\beta \hat H)) = \int\rD[\phi]\re^{-S[\phi]}
    ,\quad
    S = -\int\rd\tau\brlr{\frac1{2\pi u}\int\rd x(\partial_\tau\phi)^2+V(\phi)}.
\end{align}

We now impose the Luttinger liquid potential, i.e. $V(\phi) = \frac u{2\pi}\int\rd x(\partial_x\phi)^2$, and proceed with the action of the Luttinger liquid, i.e.
\begin{align}
    S[\phi]=-\frac1{2\pi}\int\rd\tau\rd x\brlr{ \frac{(\partial_\tau\phi)^2}{u} + u(\partial_x\phi)^2}.
\end{align}
The partition function is readily recast into frequency and momentum space (using the notation ${\bm k}=(\omega, k)$).
In addition, I introduce an additional source term similar to \cref{eq:field_source},
\begin{align}
    Z[\eta] = \int\rD_\phi \re^{-S[\phi]} = \int\rD_\phi \exp\brlr{-\int\frac{\rd k\rd\omega}{2\pi}\phi(-{\bm k})\commutator{\frac{w^2+(uk)^2}{u}}\phi({\bm k}) +\int\rd^2k\eta(-{\bm k})\phi({\bm k})},
    \label{eq:full_partition_function}
\end{align}
with the definition of the Green's function $G^{-1}({\bm k})=\frac{\omega^2+(uk)^2}{\pi}$.
One arrives at
\begin{align}
    \braket{\phi({\bm k})\phi(-{\bm k})} = G({\bm k})
    ,\quad
    \braket{\phi(x,\tau)\phi(x',\tau')} = G(x,\tau,x',\tau').
    \label{eq:KG_greens_functions_equality}
\end{align}
A subsequent Fourier transformation, combined with the definition of the polar coordinates ${\bm p}=(\omega/u,-k)$, ${\bm r} = (u(\tau-\tau'), x-x')$, $r = \sqrt{(x-x')^2+u^2(\tau-\tau')^2}$ yields an analytic expression of the correlation function
\begin{align}
    \braket{\phi(x,\tau)\phi(x',\tau')}
    = \int\frac{\rd k\rd\omega}{4\pi^2}G({\bm k})\re^{\ri\omega(\tau-\tau')-\ri k(x-x')}
    = \int_0^\infty\rd p\, p\int_0^{2\pi}\frac{\rd\alpha}{4\pi}\frac{\re^{\ri pr\cos\alpha}}{p^2}
    \\
    = \int_0^\infty\rd p\frac{J_0(pr)}{2p}
    \approx
    \int_{\Lambda_{\rm min}}^{\Lambda_{\rm max}}\rd p \frac{J_0(pr)}{2p}
    \approx
    -\frac14\log\brlr{\frac{r^2+\Lambda_{\rm max}^{-2}}{\Lambda_{\rm min}^{-2}}},
    \label{eq:kg_correlations_approximation}
\end{align}
with $J_0$ the Bessel function of the first kind.
The first approximation considers the fact that the Green's function should be the effective description of a system on a lattice which provides momentum cutoffs $\Lambda_{\rm max} = 2\pi/a$ and $\Lambda_{\rm min} = 2\pi/L$.
The second approximation considers a smooth upper momentum cutoff rather than the sharp lattice limit, which is discussed in more detail in \cref{sec:renormalization_group_theory}.

To apply the statements for the interacting theory, only the fields must be rescaled $\phi\rightarrow\phi/\sqrt K$ and $\theta\rightarrow\theta\sqrt K$.
In conclusion, the correlations for the interacting theory satisfy
\begin{align}
    \frac1K\braket{\phi(x)\phi(x')} = \braket{\phi'(x)\phi'(x')} \approx -\frac12\log(|x-x'|).
    \label{eq:greens_1}
\end{align}
The Lagrangian of the $\theta$ field is the same as for the $\phi$ field, thus the same correlations hold, i.e.
\begin{align}
    K\braket{\theta(x)\theta(x')} = \braket{\theta'(x)\theta'(x')} \approx -\frac12\log(|x-x'|).
    \label{eq:greens_2}
\end{align}
%
%
%%%%%%%%%%%%%%%%%%%%%%%%%%%%%%%%%%%%%
\section{Luttinger liquids with spin}
\label{sec:LL_with_spin}
%%%%%%%%%%%%%%%%%%%%%%%%%%%%%%%%%%%%%
To proceed further to the case of spinful models, the kinetic Hamiltonian is assumed to be decoupled, i.e.
\begin{align}
    \hat H_0 = \hat H_{0,\uparrow}+\hat H_{0,\downarrow}
    = \frac{2\hbar\pi v_F}L\sum_{q>0,\tau\in\{L,R\},s\in\{\uparrow,\downarrow\}}\hat\rho^\pdag_{q,\tau,s}\hat\rho^\dag_{q,\tau,s},
\end{align}
which allows to introduce a pair of conjugate fields $(\hat \phi_s,\hat \pi_s)$ for each spin flavor $s$\footnote{In our research, the spin conservation is explicitly broken in most cases. The analytic treatment employed in \cref{one_half1,integer1,chiral1} starts from a decoupled system, which is understood as two independent wires each hosting a Luttinger liquid. We then study the ``most relevant'' emergent terms resulting from (perturbatively) coupling the two wires in the presence of interactions. For the definition of relevant and irrelevant terms, see \cref{sec:renormalization_group_theory}.}.
The scattering processes $g_4$ and $g_2$ can be generalized in a straightforward manner, and the interaction is approximated with
\begin{align}
  \hat V \approx
  \sum_{q,\tau,s}
  g_{4\parallel}\hat\rho^\pdag_{q,\tau,s}\hat\rho^\dag_{q,\tau,s}
  +
  g_{4\perp}\hat\rho^\pdag_{q,\tau,s}\hat\rho^\dag_{q,\tau,\overline s}
  +
  g_{2\parallel}\hat\rho^\pdag_{q,\tau,s}\hat\rho^\dag_{q,\overline\tau,s}
  +
  g_{2\perp}\hat\rho^\pdag_{q,\tau,s}\hat\rho^\dag_{q,\overline\tau,\overline s}.
\end{align}
To account for the scattering among different spins, it is customary to write the model in terms of a charge and spin degree of freedom, defined as
\begin{align}
    \hat f_+=\frac1{\sqrt2}\brlr{\hat f_\uparrow + \hat f_\downarrow},
    \quad
    \hat f_-=\frac1{\sqrt2}\brlr{\hat f_\uparrow - \hat f_\downarrow},
    \quad
    f\in\{\theta,\phi\}.
\end{align}
This rotation is trivial on the level of the kinetic Hamiltonian and
\begin{align}
    \hat H_0 = \frac{2\hbar\pi v_F}L\sum_{q>0,\tau,s\in\{+,-\}}\hat\rho^\pdag_{q,\tau,s}\hat\rho^\dag_{q,\tau,s}.
\end{align}
Things are different for the forward and backscattering processes $g_{2/4}$.
In the spinful scenario, it is necessary to distinguish between intra-spin and inter-spin scattering, such that
\begin{align}
  \sum_{s\in\{\uparrow,\downarrow\}}
  \brlr{
  g_{4\parallel}\hat\rho^\pdag_{q,\tau,s}\hat\rho^\dag_{q,\tau,s}
  +
  g_{4\perp}\hat\rho^\pdag_{q,\tau,s}\hat\rho^\dag_{q,\tau,\overline s}
  +
  g_{2\parallel}\hat\rho^\pdag_{q,\tau,s}\hat\rho^\dag_{q,\overline\tau,s}
  +
  g_{2\perp}\hat\rho^\pdag_{q,\tau,s}\hat\rho^\dag_{q,\overline\tau,\overline s}
  }
  \\
  =
  \frac12
  \sum_{i\in\{2,4\}}
  \brlr{
  [g_{i\parallel}+g_{i\perp}]\hat\rho^\pdag_{q,\tau,+}\hat\rho^\dag_{q,\tau,+}
  +
  [g_{i\parallel}-g_{i\perp}]\hat\rho^\pdag_{q,\tau,-}\hat\rho^\dag_{q,\tau,-}
  }.
\end{align}
The $g_1$ terms require special attention in this case:
While the inter-spin terms are again equivalent to backscattering terms (up to a sign, see \cref{eq:umklapp_backscattering_equivalence}), the $g_{1\perp}$ term is
\begin{align}
    g_{1\perp}\sum_{s\in\{\uparrow,\downarrow\}}\hat c^\dag_{L,s}(x)\hat c^\pdag_{R,s}(x)\hat c^\dag_{R,\overline s}(x)\hat c^\pdag_{L,\overline s}(x)
    =
    g_{1\perp}\sum_{s\in\{\uparrow,\downarrow\}}\re^{-2\ri\hat \phi_s(x)}\re^{2\ri\hat \phi_{\overline s}(x)}
    =
    \frac{g_{1\perp}}{2\pi^2\alpha^2}\cos(2\sqrt2\hat \phi_-(x))
\end{align}
and therefore, the total Hamiltonian is of the form (again, setting $\hbar=1$)
\begin{align}
    \hat H = \sum_{s\in\{+,-\}}\int\frac{\rd x}{2\pi}\brlr{u_s K_s\hat\pi_s^2 + \frac{u_s}{K_s}(\partial_x\hat\phi_s)^2}
    +
    \frac{2g_{1\perp}}{(2\pi\alpha)^2}\int\rd x\cos(2\sqrt2\hat\phi_-(x))
    \label{eq:ll_hamiltonian_spin}
\end{align}
with coupling constants
\begin{align}
    u_sK_s = v_F(1+y_{4s}-y_{2s}),
    \quad
    \frac{u_s}{K_s} = v_F(1+y_{4s}+y_{2s}),
    \\
    y_{is} = \frac{g_{i\parallel}+sg_{i\perp}}{2\pi v_F},
    \quad
    i\in\{2,4\},
    \quad
    s\in\{+,-\}.
\end{align}
The first observation regards the separation of the Hamiltonian into two quasi-independent sectors (the overall particle number must still be conserved), namely those of charge and spin excitations propagating with different velocities.
This is famously known as spin-charge separation and a feature of all one-dimensional metallic systems.
It describes the fractionalization of particles into two distinct quasiparticles, called spinons with zero charge and spin 1/2 and chargeons with charge minus one and no spin.
As is clear from the discussion above, the two sectors remain independent and therefore move in general with different velocities~\cite{Tomonaga1950,Luttinger1963,Haldane1981,Kim2006}.

The second observation regards the presence of a non-quadratic term in the spin sector of the Hamiltonian.
In general, $\hat O_{\rm sG} = g\int\rd x\cos(\beta\hat f)$
are called sine-Gordon terms of the field $\hat f$, with $\beta$ an arbitrary number, and are potentially ``relevant'' for a Luttinger liquid~\cite{Giamarchi2003,Gogolin2004,AltlandSimons2010}.
The ``relevancy'' of such terms is determined by its flow under Wilsonian renormalization group theory which I perform explicitly in \cref{sec:renormalization_group_theory}.
If $g$ is large enough (depending on $\beta$), it drives the system to an ordered phase by ``pinning'' its argument's eigenvalues to semiclassical minima $\beta f=(2n+1)\pi$.
In the context of \cref{eq:ll_hamiltonian_spin}, it opens a gap in the spin sector and establishes an ordered spin density associated to the argument of $\hat O_{\rm sG}$.
%
%
%%%%%%%%%%%%%%%%%%%%%%%%%%%%%%%%%%%%%%%%
\section{Renormalization group theory}
\label{sec:renormalization_group_theory}
%%%%%%%%%%%%%%%%%%%%%%%%%%%%%%%%%%%%%%%%
In \cref{sec:LL_with_spin}, I motivated an interaction $\hat O_{\rm sG}$ which is typically encountered in field theories that may drive the gapless Luttinger liquid to a different phase of matter.
This interaction is commonly used to describe a particular class of phase transitions from a gapless to a gapped system, called Berezinskii-Kosterlitz-Thouless (BKT) transition.
It is peculiar in the sense that it does not involve any spontaneous symmetry breaking and can thus be considered as an example of a topological phase transition.
Traditionally, the BKT transition was first encountered in 2D classical systems, but due to the close analogy between (1+1)D quantum field theories and classical statistical mechanics in two spatial dimensions, it appears naturally in one-dimensional quantum systems as well.
The emergence of the sine-Gordon type field theory from a 2D classical model requires a rather involved treatment of the classical partition function, and I refer to~\cite{AltlandSimons2010} for details on the derivation.

I now proceed by explaining the basic renormalization group (RG) analysis of the sine-Gordon model, based on~\cite{Gogolin2004}.
The action of the sine-Gordon model reads $S = S_0 + S_I$, with
\begin{align}
    S_0 = \frac1{2\pi}\int\rd x\rd\tau\brlr{\frac1{uK}(\partial_\tau\phi)^2 + \frac uK (\partial_x\phi)^2}
    ,\quad
    S_I = g\int\rd x\rd\tau\cos(\beta\phi)
    .
\end{align}
Remember that the action resulted as the continuum limit of a lattice model with well defined Brillouin zone, which connects the long wavelength with the small crystal momentum components of $\phi$.
The idea of renormalization is to let the system flow towards larger distances by sequentially integrating the short-distance (fast) components of the fields by partial path integration and representing the result in terms of an effective model for the long-wavelength field.
The final step is then to recover the original size of the momentum shell by a rescaling of lengthscales.
For technical convenience, it is best to assume a circular cutoff constraint $|{\bm k}|<\Lambda\sim2\pi/a$.
If the model is of a sine-Gordon type, the effective model will have the same form as the original one (up to a rescaled set of coupling constants and extra ``irrelevant'' terms which tend to disappear), from which the so-called ``RG-flow equations'' are derived.

Let's start by splitting the fields into the aforementioned slow and fast components, where the ``fast'' components are contained in an infinitesimal momentum shell $\rd\Lambda=\Lambda-\Lambda'$
\begin{align}
    \phi_\Lambda({\bm x}) = \phi_{\Lambda'}({\bm x}) + h({\bm x}),
    \quad
    \phi_{\Lambda'}({\bm x})\coloneqq\frac1{\sqrt L}\sum_{k<\Lambda'}\re^{\ri {\bm k}{\bm x}}\phi_{\bm k},
    \quad
    h({\bm x}) \coloneqq \frac1{\sqrt L}\sum_{\Lambda'<k<\Lambda}\re^{\ri {\bm k}{\bm x}}\phi_{\bm k}.
\end{align}
We will see shortly, that instead of using $\rd\Lambda$ as a measure of the flow it is more useful to consider the quantity $\brlr{\Lambda'/\Lambda}^\alpha=1-\alpha\frac{\rd\Lambda}\Lambda + O(\rd\Lambda^2) = 1-\alpha\rd l + O(\rd l^2)$ with $\rd l =\frac{\rd\Lambda}\Lambda = \rd\log\Lambda$.
Note that these definitions imply $\Lambda'=\Lambda\re^{-l}$, such that the momentum shell flows exponentially fast to smaller values.
The Luttinger liquid part of the Euclidian action is linear under such decomposition and therefore
\begin{align}
    Z_\Lambda
    = \int\rD\phi_{\Lambda'}\rD h \re^{-S_0[\phi_{\Lambda'}]-S_0[h]-S_I[\phi_{\Lambda'}({\bm x}) + h({\bm x})]}
    = Z_h\int\rD\phi_{\Lambda'}\re^{-S_0[\phi_{\Lambda'}]}\braket{\re^{-S_I[\phi_{\Lambda'}({\bm x}) + h({\bm x})]}}_h,
\end{align}
which results in the definition of an effective action
\begin{align}
    S_{\rm eff}[\phi_{\Lambda'}] = S_0[\phi_{\Lambda'}] - \log\braket{\re^{-S_I[\phi_{\Lambda'}+h]}}_h.
\end{align}
The analytic evaluation of the expectation on the right hand side requires further simplifications -- one prominent possibility is the perturbative expansion
\begin{align}
    S^{(2)}_{\rm eff}[\phi_{\Lambda'}] = S_0[\phi_{\Lambda'}] + \braket{S_I[\phi_{\Lambda'}+h]}_h - \frac12\braket{S^2_I[\phi_{\Lambda'}+h]}_{h,{\rm conn.}} + O(g^3)
\end{align}
in which $\braket{F^2}_{\rm conn.} = \braket{F^2}-\braket{F}^2$ denotes the ``connected'' part of the expectation value.
The renormalization group approach carried out in the following is thus reliable for small renormalized couplings $g\ll \frac u{2\pi K}$.

The first order term can be rewritten as
\begin{align}
    \braket{S_I[\phi_{\Lambda'}+h]}_h = g\int\rd^2x\braket{\cos(\beta\anticommutator{\phi_{\Lambda'}+h})}_h
    = \frac g2\int\rd^2x\brlr{\re^{\ri\beta \phi_{\Lambda'}}\braket{\re^{\ri\beta h}}_h+\re^{-\ri\beta \phi_{\Lambda'}}\braket{\re^{-\ri\beta h}}_h},
\end{align}
in which the expectation value of the exponential can be rewritten by using
\begin{align}
    \braket{\re^{\ri\sum_kb_k\phi(x_k)}} = \re^{-\frac12\sum_{k,k'}b_kb_{k'}\braket{\phi(x_k)\phi(x_{k'})}}.
    \label{eq:expectation_value_exponential_fields}
\end{align}
In this case the correlations are finite up to first order in $\rd\Lambda$, i.e.
\begin{align}
    \braket{h^2}_h = K\int_{\Lambda-\rd\Lambda}^\Lambda\rd p\frac{J_0(0)}{2p} = \frac K2\frac{\rd\Lambda}\Lambda = \frac{K\rd l}2.
    \label{eq:rg_hsq}
\end{align}
This way, we can easily expand the expectation value from above to
\begin{align}
    \braket{S_I[\phi_{\Lambda'}+h]}_h
    = g\int\rd^2x\anticommutator{\cos(\beta \phi_{\Lambda'})\re^{-\frac{K\beta^2}4\rd l}}
    \\
    = g\brlr{1 - \frac{K\beta^2}4\rd l}\int\rd^2x\cos(\beta \phi_{\Lambda'}).
\end{align}
As a next step, we would like to recover the size of the original momentum shell by $|{\bm k}|\rightarrow |{\bm k}'|=\frac{\Lambda}{\Lambda'}|{\bm k}| = (1+\rd l)|{\bm k}|+O(\rd l^2)$.
In order to preserve the Fourier transform, it is a necessity to keep the dot product ${\bm k'}{\bm x'}$ invariant under the flow $\rd l$.
Therefore, space-time must be rescaled in the opposite manner to crystal momentum, i.e. $|{\bm x}|\rightarrow|{\bm x'}|=(1-\rd l)|{\bm x}|$.
Differentials are thus transformed according to the equality
\begin{align}
    \rd^2x = \rd^2 x'(1-\rd l)^{-2} = \rd^2 x'(1+2\rd l) + O(\rd l^2).
\end{align}
Due to the presence of gradients, the Gaussian part $S_0$ is left invariant and we arrive at the first-order approximation of the effective action
\begin{align}
    S^{(1)}_{\rm eff}[\phi_\Lambda]=S_0[\phi_\Lambda] + g'\int\rd^2x\cos(\beta \phi_{\Lambda}),
    \quad
    g' = g\brlr{1 + \commutator{2-D_g}\rd l},
    \quad
    D_g=\frac{K\beta^2}4.
    \label{eq:bkt_first_order}
\end{align}
In the first-order approximation, the so-called scaling dimension $D_g$ appears and is associated with the coupling $g$ of the perturbation $\cos(\beta\phi)$.
From \cref{eq:bkt_first_order} the differential equation of the coupling constant under the renormalization flow $\rd l$ is identified as
\begin{align}
    \frac{\rd g}{\rd l} = \frac{g'-g}{\rd l} = (2-D_g)g.
\end{align}
Its solutions are $g(l) = g_0\re^{(2-D_g)l}$, where $g_0\coloneqq g(0)$ is the initial condition of the coupling before the RG flow.
It is thus evident that the flow of the coupling is fully determined by the scaling dimension: if $D_g>2$, then $g(l)\rightarrow0$ vanishes exponentially fast and the coupling is dubbed ``irrelevant''.
If however $D_g<2$, then $g(l)\rightarrow\infty$ and the interaction is called ``relevant''.
Due to the perturbative character of this study, one must however consider that $g(l)$ is not allowed to exceed the initial values of the Gaussian part $g(l)<g^*$ with $g^*=\min(uK,u/K)$.
This value is thus related to an energy scale of the microscopic system, e.g. the bandwidth of the non-interacting model.
An upper stop value for the flow $l^*$ is thus given by $g^*\coloneqq g(l^*)$ and one may assume that the system is driven sufficiently away from the Luttinger liquid critical point, characterized by the scale invariant action $S_0$.

An estimate for the scaling of the correlation length is obtained by
\begin{align}
    \xi(l) \propto a(l) \propto \Lambda^{-1}(l) = \Lambda^{-1}_0\re^{l},
    \label{eq:correlation_length_rg_flow}
\end{align}
in which $a(l)$ is the renormalized lattice spacing, which naturally scales like the inverse of the momentum shell, i.e. $\Lambda\sim2\pi/a$.
In conclusion, the energy gap associated with the correlation length $\xi^{-1}$ scales like the momentum shell itself, i.e. $\Delta(l)\propto\xi(l)^{-1}\propto\re^{-l}$.
By combining the solutions $\Delta^*$ with $g^*$, one can thus estimate the size of the gap induced by the relevant sine-Gordon operator~\cite{Gogolin2004}, i.e.
\begin{align}
    {\Delta^*}\propto\re^{-l^*}=\brlr{\frac{g_0}{g^*}}^{1/(2-D_g)}.
\end{align}
Note that the estimate is given in arbitrary units and as such makes no quantitative prediction.
Instead, it predicts a qualitative behavior -- the smaller the scaling dimension $D_g$ the larger the related gap will be.
This is in agreement with the previous discussion of the relevancy of the sine-Gordon interaction for $D_g<2$: if the scaling dimension evolves to smaller values, the associated operator becomes ``more relevant''.

Now for the second order correction -- the first term in the connected correlator evaluates to
\begin{align}
    \braket{S^2_I[\phi_{\Lambda'}+h]}=\frac{g^2}2\re^{-\beta^2\braket{h^2}}\int\rd^2x\rd^2x'
    \left(
        \cos\commutator{\phi_{\Lambda'}({\bm x})+\phi_{\Lambda'}({\bm x'})}\re^{-\beta^2\braket{h({\bm x})h({\bm x'})}}
        \right.\\
        \left.+
        \cos\commutator{\phi_{\Lambda'}({\bm x})-\phi_{\Lambda'}({\bm x'})}\re^{+\beta^2\braket{h({\bm x})h({\bm x'})}}
    \right)
    \label{eq:second_order_rg_term1}
\end{align}
and contains the decaying two-point correlations of the $h$-fields.
For convenience, let me proceed with the assumption that the two-point correlations are of the form $\braket{h({\bm x})h({\bm x'})} = \frac K2 C(r=|{\bm x}-{\bm x'}|)\rd l$ such that the action is rewritten to
\begin{align}
    \braket{S^2_I[\phi_{\Lambda'}+h]}=\frac{g^2}2\brlr{1 - \frac{K\beta^2}2\rd l}\int\rd^2x\rd^2x'
    \left(
        \cos\commutator{\beta\phi_{\Lambda'}({\bm x})+\beta\phi_{\Lambda'}({\bm x'})}\anticommutator{1-\frac{K\beta^2}2C(|{\bm x}-{\bm x'}|)\rd l}
        \right.\\
        \left.+
        \cos\commutator{\beta\phi_{\Lambda'}({\bm x})-\beta\phi_{\Lambda'}({\bm x'})}\anticommutator{1+\frac{K\beta^2}2C(|{\bm x}-{\bm x'}|)\rd l}
    \right).
\end{align}
The disconnected part of the correlation function is evaluated using the identity $2\cos(a)\cos(b)=\cos(a+b)+\cos(a-b)$ and $(1-\alpha\rd l)^2 = 1-2\alpha\rd l + O(\rd l^2)$ and reads
\begin{align}
    \braket{S_I[\phi_{\Lambda'}+h]}_h^2
    = g^2\brlr{1 - \frac{K\beta^2}4\rd l}^2\int\rd^2x\rd^2x'\cos(\beta \phi_{\Lambda'}({\bm x}))\cos(\beta \phi_{\Lambda'}({\bm x'}))
    \\
    = \frac{g^2}2\brlr{1 - \frac{K\beta^2}2\rd l}\int\rd^2x\rd^2x'
        \left(
            \cos\commutator{\beta \phi_{\Lambda'}({\bm x})+\beta \phi_{\Lambda'}({\bm x'})}
            +
            \cos\commutator{\beta \phi_{\Lambda'}({\bm x})-\beta \phi_{\Lambda'}({\bm x'})}
        \right).
\end{align}
Therefore, the $C$-independent terms cancel in the connected part of the correlation function and the final result is
\begin{align}
    -\frac12\braket{S^2_I[\phi_{\Lambda'}+h]}_{h,{\rm conn.}} = \frac{g^2K\beta^2}8\rd l\int\rd^2x\rd^2x'
    C(|{\bm x}-{\bm x'}|)
    \left(
        \cos\commutator{\beta\phi_{\Lambda'}({\bm x})+\beta\phi_{\Lambda'}({\bm x'})}
        \right.
        \\
        \left.-
        \cos\commutator{\beta\phi_{\Lambda'}({\bm x})-\beta\phi_{\Lambda'}({\bm x'})}
    \right)
    % \\
    % \approx
    % \frac{\gamma g^2K\beta^2}8\rd l\int\rd^2x
    % \left(
    %     \cos\commutator{2\beta\phi_{\Lambda'}({\bm x})}
    %     % \right.\\
    %     % \left.
    %     -\cos\commutator{\beta\partial_{\bm x}\phi_{\Lambda'}({\bm x})}
    % \right).
    % \label{eq:RG_second_order_approximation_1}
\end{align}
% In case $C(|{\bm x}-{\bm x'}|)$ is sufficiently short-ranged, the arguments in the $\cos$ of \cref{eq:rg_int_1,eq:rg_int_2} can be approximated as indicated in \cref{eq:RG_second_order_approximation_1} using a non-universal constant $\gamma$.

To proceed in solving the integral, the function $C$ needs to be discussed further.
For a sharp momentum cutoff, one obtains from \cref{eq:kg_correlations_approximation} $C(r) = J_0(\Lambda r)$ with asymptotic expression $J_0(\Lambda r)\approx\sqrt{2/(\pi \Lambda r)}\cos(\Lambda r-\pi/4)$.
This function has a long algebraic tail and is thus not a sharp function in $r$ (see also \cref{fig:rg_cutoff}).
The origin of this long tail resides in the choice of the momentum cutoff in the integration scheme.
Recall that we aim to integrate an infinitesimal shell of large momenta, which allows a certain degree of freedom in the form of the shell itself.
The momentum integration can be rewritten as
\begin{align}
    \int_{0}^{\Lambda}\rd p \rightarrow \int_{0}^\infty\rd p f_n(p,\Lambda),
    \quad
    f_n(p,\Lambda) = \frac{\Lambda^n}{p^n+\Lambda^n},
    \quad
    n\in\mathds N,
    \label{eq:integral_cutoff}
\end{align}
which implements a smooth cutoff around $\Lambda$.
The sharp situation is recovered for $n\rightarrow\infty$ (see \cref{fig:rg_cutoff}).
% \footnote{
    % Without going into full detail, the smooth cutoff is the key ingredient to find the logarithmic expression of the propagator, with the general result
    % \begin{align}
        % \int_{\Lambda_{\rm min}}^{\Lambda_{\rm max}}\rd p\frac{J_0(pr)}{2p}\rightarrow\int_{\Lambda_{\rm min}}^{\infty}\rd p\frac{J_0(pr)}{2p}f_n(p,\Lambda_{\rm max})\approx-\frac14\log\brlr{\frac{(\Lambda_{\rm max}^{-n}+r^n)^{\frac 2n}}{\Lambda_{\rm min}^{-2}}}.
    % \end{align}
    % In the above approximation, one first splits the integral into two parts, one ranging from the lower cutoff $\Lambda_{\rm min}$ to $1/r$, and the second from $1/r$ to infinity.
    % The asymptotic part of the integral decays algebraically as a function of $\Lambda_{\rm max}r$, it thus bears a small weight of the full integral and vanishes in the limit $\Lambda_{\rm max}\rightarrow\infty$.
    % The main weight of the integral resides in $[\Lambda_{\rm min},1/r]$, for which $J_0\approx1$ if the lower cutoff is chosen small enough.
    % This then yields the approximated propagator, with the special case $n=2$ used in~\cref{eq:kg_correlations_approximation}.
% }.
In practice, the integration of the fast modes in \cref{eq:kg_correlations_approximation} evaluates to
\begin{align}
    \braket{h({\bm x}),h(\bm x')}_h = \frac K2\int_{\Lambda'}^\Lambda\rd p\frac{J_0(p r)}{p} = \frac K2\int_{0}^\Lambda\rd p\frac{J_0(p r)}{p}
    -\frac K2\int_{0}^{\Lambda'}\rd p\frac{J_0(p r)}{p}
    \\
    \longrightarrow
    \frac K2\int_{0}^\infty\rd p J_0(p r)p^{n-1}\brlr{\frac1{p^n+\Lambda'^n}-\frac1{p^n+\Lambda^n}}
    \\
    =
    \frac K2\int_{0}^\infty\rd p J_0(p r)p^{n-1}\frac{n\Lambda^n}{\left(\Lambda^n+p^n\right)^2}\rd l + O(\rd l^2),
    \label{eq:rg_hxhxpr}
\end{align}
leading to the modified function $C_n$ which depends on the smoothness $n$, i.e.
\begin{align}
    C_n(r)
    =
    \int_{0}^\infty\rd p J_0(p r)p^{n-1}\frac{n\Lambda^n}{\left(\Lambda^n+p^n\right)^2}.
    \label{eq:rg_cn_def}
\end{align}
\begin{figure}
    \centering
    \subfigure[]{\includegraphics{figures/cutoff_function.png}}
    \subfigure[]{\includegraphics{figures/rg_correlations.png}}
    \subfigure[]{\includegraphics{figures/rg_correlations_log.png}}
    \caption{Panel (a) shows the chosen cutoff $f_n(p,\Lambda)$ with $n\in\{2,4,6,8,\infty\}$ for the integral expression in \cref{eq:integral_cutoff}, which results in various approximations of $C(r)$ for altering $n$ plotted in (b). Panel (c) highlights the exponentially sharp function $C_2(r)=\Lambda rK_1(r\Lambda)$ compared to the sharp cutoff result $C_\infty(r)=J_0(\Lambda r)$.}
    \label{fig:rg_cutoff}
\end{figure}
% At this point it is worth mentioning that the correlation functions of the slow fields are
% \begin{align}
%     \braket{\phi_\Lambda({\bm x}),\phi_\Lambda(\bm x')} = \frac K2\int_{0}^\infty\rd p J_0(p r)p^{n-1}\brlr{\frac1{p^n+\Lambda^n_{\rm min}}-\frac1{p^n+\Lambda^n_{\rm max}}}
% \end{align}
% which, in case of $n=2$, yield the appealing result
% \begin{align}
%     \braket{\phi_\Lambda({\bm x}),\phi_\Lambda(\bm x')} = \frac K2 \brlr{K_0(\Lambda_{\rm min} r) - K_0(\Lambda_{\rm max} r)}.
%     \label{eq:rg_slow_fields_approximation}
% \end{align}
For the special case $n=2$, the integral of the $h$-fields evaluates to $C_2(r)=\Lambda r K_1(\Lambda r)$ that decays exponentially fast (see \cref{fig:rg_cutoff}(c)).
In particular, the function follows the asymptotic decay $z K_1(z) \sim \sqrt{\pi z/2}\exp(-z)$ and is already negligible for $z=1$, i.e. $K_1(1)\approx0.0062$.
Therefore, the integration of $C_2(r)$ can be confined to a small interval $r<\alpha$ where $\alpha\sim2\pi/\Lambda=a$ is a small length scale comparable with the lattice spacing $a$.

In order to utilize the strong confinement of $C_2$, it is beneficial to introduce relative coordinates ${\bm R} = 1/2({\bm x}+{\bm x'})$ and ${\bm r} = {\bm x}-{\bm x'}$.
The integral can then be rewritten as
\begin{align}
    -\frac12\braket{S^2_I[\phi_{\Lambda'}+h]}_{h,{\rm conn.}}
    =
    \frac{g^2K\beta^2}8\rd l\int\rd^2R\int\rd^2r
    C_2(r)
    \left(
        \cos\commutator{\beta\phi_{\Lambda'}({\bm R}+{\bm r})+\beta\phi_{\Lambda'}({\bm R}-{\bm r})}
        \right.
        \label{eq:rg_int_1}\\
        \left.-
        \cos\commutator{\beta\phi_{\Lambda'}({\bm R}+{\bm r})-\beta\phi_{\Lambda'}({\bm R}-{\bm r})}
    \right)
    \label{eq:rg_int_2}
    \\
    \approx
    \frac{g^2K\beta^2}{8}\rd l\int\rd^2R\rd^2rC_2(r)
    \left(
        \cos\commutator{2\beta\phi_{\Lambda'}({\bm R})}
        -
        \cos\commutator{\beta\partial_{\bm R}\phi_{\Lambda'}({\bm R}){\bm r})}
    \right).
    \label{eq:rg_int_rel_coords}
\end{align}

The first $\cos$ term (\cref{eq:rg_int_1}) did not exist in the original Hamiltonian\footnote{
    Note that this term generation is continuous, and to account for it, we should start from a more generic interaction containing all the higher harmonics, i.e. $S_{I'} = S_I + \sum_{j=1}^\infty \tilde g_j\int\rd x\rd\tau\cos(2j\beta\phi)$.
    The first-order corrections of the Euclidian action then yield a coupled system of differential equations in which the amplitudes of less relevant operators depend on those of more relevant ones (but {\it not} vice-versa).
    This allows for a practical and well-justified simplification of dropping the sum in the previous expression, because the dominant coupling is always independent of the amplitudes of less-relevant operators.
} -- it is a new sine-Gordon type interaction with larger scaling dimension $D_{\tilde g} = K\beta^2$ compared to $D_g=K\beta^2/4$.
The operator associated to $\tilde g$ is thus less relevant than the original term and can be disregarded.
The second $\cos$ term in (\cref{eq:rg_int_2}) yields a renormalization of the quadratic part,
\begin{align}
    -\frac12\braket{S^2_I[\phi_{\Lambda'}+h]}_{h,{\rm conn.}}
    \approx
    \frac{\alpha^4 g^2K\beta^4}{16u^2}\rd l\int\rd x\rd\tau\frac1{u^2}(\partial_
    \tau\phi_{\Lambda'})^2+(\partial_x\phi_{\Lambda'})^2.
    \label{eq:RG_second_order_approximation}
\end{align}
In the above, I neglect a constant and gave for granted the harmonic approximation and integration of $\bm r$ with $r=\sqrt{(x-x')^2+u^2(\tau-\tau')^2}$ in polar coordinates.
The constant $\alpha^4/u^2$ is determined by the cutoff of the integrals over the relative coordinates, and $\alpha=\mathcal O(a)$ is on the order of the lattice spacing.
In summary, we obtain the effective action
\begin{align}
    S_{\rm eff}^{(2)}[\phi_\Lambda] = \frac1{2\pi}\int\rd\brlr{ x\rd\tau\frac{1}{uK'}(\partial_\tau\phi_\Lambda)^2 + \frac{u}{K'}(\partial_x\phi_\Lambda)^2} +  g'\int\rd^2 x\cos(\beta\phi_\Lambda),
\end{align}
which is self-similar to the original action up to the renormalized couplings
\begin{align}
    \frac1{uK'}=\frac1{uK}+\frac{\alpha^4 g^2 K\beta^4\pi}{8 u^4}\rd l,
    \quad
    \frac u{K'}=\frac u{K}+\frac{\alpha^4 g^2 K\beta^4\pi}{8u^2}\rd l,
    \quad
    g' = g\brlr{1 + \commutator{2-\frac{\beta^2 K}4}\rd l},
    \\
    \Rightarrow
    K' = \brlr{\commutator{\frac{1}{uK}+\frac{\alpha^4 g K\beta^4\pi}{8u^4}}\commutator{\frac{u}{K}+\frac{\alpha^4 g K\beta^4\pi}{8u^2}}}^{-1/2}
    =
    K-\frac{\pi \alpha^4 \beta^4 g^2 K^3}{8u^2}\rd l + O(\rd l^2).
\end{align}

This concludes the derivation of the so-called RG flow which is described by the system of differential equations we just derived
\begin{align}
    \frac{\rd K}{\rd l} = -\frac{\pi \alpha^4 \beta^4 g^2 K^3}{8u^2},
    \quad
    \frac{\rd g}{\rd l} = g\brlr{2-\frac{\beta^2K}4}.
\end{align}
Viable estimates are already obtained far from the phase transitions, and the second order approximation provides more insights close to the phase transition at $D_g=2$, in particular at $K=8/\beta^2$.
Equivalent results are obtained for the dual field by replacing $K\rightarrow K^{-1}$.
It is beneficial to define the flow equations close to the critical point, e.g. through the parameters
\begin{align}
    x = \frac{\beta^2 K}4 - 2,
    \quad
    y = 4g\sqrt{\frac{\pi\alpha^4}{u^2}},
\end{align}
which dictate the modified RG equations
\begin{align}
    \frac{\rd x}{\rd l} = -\frac{y^2}8(x+2)^3,
    \quad
    \frac{\rd y}{\rd l} = - xy.
\end{align}
This set of variables is particularly useful to investigate the equations in proximity of $x=0$.
Note that $(x+2)^3\approx8$, such that $\rd x/\rd l = -y^2$ and $\rd y/\rd l=-xy$ describes the effective RG flow close to the Luttinger liquid fix point at $x=0$ and $y=0$, which are known as the ``Kosterlitz-Thouless equations''~\cite{Kosterlitz1974,Kosterlitz1973}.

\begin{figure}
    \centering
    \subfigure[]{\includegraphics{figures/BKT_RG_flow1.png}}
    \subfigure[]{\includegraphics{figures/BKT_RG_flow2.png}}
    \subfigure[]{\includegraphics{figures/BKT_RG_flow3.png}}
    \caption{Plot of the modified RG equations far (a) and close (b) to the critical point $x=0$. Panel (b) close to the transition at $x=0$ shows four different scenarios for $g>0$: (i) $x>0$ and $y<x$ drives the coupling $y\rightarrow0$. This is the regime in which $g$ is irrelevant such that the system remains in the Luttinger liquid phase. (ii) $x>0$ and $y=x$ describes the critical BKT line which flows to the critical fix point $x=y=0$. (iii) $y>x$, the system always flows to $y\rightarrow\infty$ independently of $x$. (iv) $x<0$, the system flows to $y\rightarrow\infty$. Panel (c) contrasts the results of the first order RG approximation which neglects the flow of the Luttinger parameter $K$.}
    \label{fig:bkt_flow_equations}
\end{figure}

We obtain four scenarios (assuming positive couplings $y>0$ in general):

(i) If $x>0$, and $y<x$, the system flows towards $y=0$ and a finite value of $x$ at which the flow ends.
This is a situation in which the interaction is irrelevant and the coupling vanishes.
Note that a vanishing coupling $y$ implies that the flow ends, and the system thus assumes a fix point.
Although the interaction is irrelevant and as such the system remains in a Luttinger liquid phase, it ``renormalizes'' the effective Luttinger liquid parameter during the flow.
The effective interaction is then given by the final value of $x$ (thus $K$) at which the flow terminates.

(ii) If $x>0$ and $y=x$ (taken in proximity of $x$ and $y$ small), the trajectory follows a critical BKT line to the fix point $x=y=0$.
The effective Luttinger parameter assumes the ``marginal'' value $K=8/\beta^2$ for which the scaling dimension $D_g=2$.

(iii) In case of $y>|x|$, the coupling $y$ first flows towards smaller couplings, but then towards $y\rightarrow\infty$, independently on the value of $x$.
This corresponds to a case in which the interaction becomes dominating.
The Luttinger liquid description breaks down and the system ends up in a gapped phase, characterized by the interaction.

(iv) For $x<0$ and $y<|x|$, the system flows towards $y\rightarrow\infty$, which is also a gapped phase.
The dominance of $y$ is entirely determined by the initial value of $x$, and even an infinitesimal interaction amplitude results in the breakdown of the Luttinger liquid description and the formation of an energy gap.

To characterize the trajectories of the flow chart, one must notice that $x\rd x = y\rd y$ in the proximity of $x=0$, which describes the conserved quantity $c = x^2-y^2$.
The constant $c$ thus describes hyperbolas cutting the $x$ and $y$ axis for $c>0$ and $c<0$, respectively (cf. \cref{fig:bkt_flow_equations} (b)).
We focus now on the critical line $c=0, x>0$ separating relevant and irrelevant $y$.
Assume now that the microscopic Hamiltonian is described by an internal parameter $\lambda>0$ which drives the system to the BKT critical line at $\lambda_c$.
Assume further that $y$ is irrelevant for $\lambda<\lambda_c$, and relevant for $\lambda>\lambda_c$.
There are two cases:

(a) For $c>0$ and $x>0$, $y$ is irrelevant and the trajectory flows to the fix point at $x_{\rm eff} = \lim_{l\rightarrow\infty} x(l)$ and $y=0$.
We can thus linearize $c=b^2(\lambda_c-\lambda)$ in the vicinity of the critical line, in which $\lambda$ encodes an internal parameter of the microscopic Hamiltonian that drives the phase transition (see previous paragraph).
Correlations of the fields $\phi/\theta$ still decay logarithmically and the coupling is being renormalized to the asymptotic value $K_{\rm eff} = \frac4{\beta^2}\brlr{b\sqrt{\lambda_c-\lambda}+2}$, which describes a square-root singularity close to the critical point.

(b) For $c<0$ and $x>0$, $y$ is relevant and pins the field configuration to the semiclassical minima of the sine-Gordon interaction.
The differential equation can be recast to
\begin{align}
    \frac{\rd x}{\rd l} = -y^2 = c-x^2 = -(|c|+x^2)
    \Rightarrow
    \int_{x(0)}^{x(l)}\rd x(x^2+|c|)^{-1} = -l
    .
\end{align}
The solution is thus given by $x(l) = \sqrt{|c|}\tan(\arctan(x(0)/\sqrt{|c|})-l\sqrt{|c|})$ and the integration must be terminated when $|x(l^*)|\sim 1$ (for which the perturbative calculation is invalid), i.e.
\begin{align}
    l^*=\frac{\pi}{2\sqrt{|c|}}+\frac{\arctan\brlr{\frac{x(0)}{\sqrt{|c|}}}}{\sqrt{|c|}}\approx \frac{\pi}{2b\sqrt{\lambda-\lambda_c}}.
\end{align}
An estimate of the correlation length in the system after the RG flow is given by \cref{eq:correlation_length_rg_flow}, and evaluates to $\xi\propto a\exp(l^*)=a\exp[{{\pi}/({2b\sqrt{\lambda-\lambda_c}})}]$.
Any attempt to realize systems close to the phase transition thus requires system lengths in excess of the exponentially divergent correlation length.
The requirement for numerical techniques is thus two-fold: resolving the phase transition requires excessively long systems, and must capture a divergent correlation length.
One simulation technique which was successfully applied for such tough systems is the density matrix renormalization group (DMRG) based on matrix product states (MPS), which I formalize in \cref{ch:matrix_product_states}.
This is to say that the bottleneck of MPS to be efficient is based on simulating systems with a short correlation length (see \cref{sec:scaling_relations_of_the_entanglement_entropy}).
However, pushing the limits of state-of-the-art MPS algorithms and CPU's overcomes the related issues and allows to extract quantitative results with great success, even if the correlation length is large.

In \cref{one_half1,integer1}, we encounter various types of sine-Gordon terms embedded in a two-component Luttinger liquid.
Among those which are potentially relevant, we characterize the phase diagram by a second order RG analysis equivalent to the example detailed in this section.
In general, the resulting RG equations are not the simple two-variable BKT equations and they form a higher dimensional flow chart.
We thus rely on numerical integration of coupled differential equations through Runge-Kutta methods to determine the relevancy of operators.
For more quantitative estimates, we use MPS simulations.
